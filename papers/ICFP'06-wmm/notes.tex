\documentclass[11pt]{article}
\usepackage[latin1]{inputenc}
\usepackage{xspace} % To get the right spacings in front of : and so on
\usepackage[english]{babel}
\usepackage{abbrevs}
\usepackage{subfigure}
\usepackage{ifthen}
\usepackage{coqdoc}
\usepackage{bussproofs}
\usepackage{amsmath}
%\usepackage{concmath}
\usepackage{url}
\usepackage{fullpage}
\usepackage{qsymbols}
\usepackage{coqdoc}

\newboolean{defineTheoremFr}
\newboolean{defineTheoremEn}
\setboolean{defineTheoremFr}{false}
\setboolean{defineTheoremEn}{false}
\newtheorem{theorem}{Th�or�me}[section]
\newtheorem{lemma}[theorem]{Lemme}
\newtheorem{fact}[theorem]{Fait}
\newtheorem{proposition}[theorem]{Proposition}
\newtheorem{definition}[theorem]{D�finition}
\newtheorem{example}[theorem]{Exemple}
\newtheorem{remark}[theorem]{Remarque}
\newtheorem{corrolary}[theorem]{Corrolaire}

\makeatletter

\newcommand{\UR}[2]{\RightLabel{\rulelabel{#1}}\UIC{#2}}
\newcommand{\URL}[2]{\LeftLabel{\rulelabel{#1}}\UIC{#2}}

\newboolean{displayLabels}
\setboolean{displayLabels}{true}

\newcommand{\LeftRuleLabel}[1]{
  \@ifnotmtarg{#1}
  {\ifthenelse{\boolean{displayLabels}}{\LeftLabel{\rulelabel{#1}}}{}}
}

\newcommand{\UAX}[4]{\AXC{#2}
  \LeftRuleLabel{#1}
  \@ifnotmtarg{#4}{\RightLabel{#4}}
  \UIC{#3}}

\newcommand{\BAX}[5]{\AXC{#2}\AXC{#3}
  \LeftRuleLabel{#1}
  \@ifnotmtarg{#5}{\RightLabel{#5}}
  \BIC{#4}}

\newcommand{\TAX}[6]{\AXC{#2}\AXC{#3}\AXC{#4}
  \LeftRuleLabel{#1}
  \@ifnotmtarg{#6}{\RightLabel{#6}}
  \TIC{#5}}

\newcommand{\QAX}[7]{\AXC{#2}\noLine\UIC{#3}\AXC{#4}\AXC{#5}
  \LeftRuleLabel{#1}
  \@ifnotmtarg{#7}{\RightLabel{#7}}
  \TIC{#6}}

\newcommand{\BR}[2]{\RightLabel{\rulelabel{#1}}\BIC{#2}}
\newcommand{\BRL}[2]{\LeftLabel{\rulelabel{#1}}\BIC{#2}}

\makeatother

%%% Local Variables: 
%%% mode: latex
%%% TeX-master: t
%%% End: 

%\def\ifstreq#1#2{\ifEqString{#1}{#2}}

\def\impsub{\rightslice}

\def\judgewf{\vdash_{wf}}
\def\judgetyped{\vdash}
\def\judgetypea{\vdash_{\bullet}}
\def\judgetypei{\vdash_{\Box}}
\def\typed{\judgetyped}
\def\typedwf{\judgewf}
\def\typea{\judgetypea}
\def\typeawf{\judgetypea_{wf}}
\def\typei{\judgetypei}
\def\typeiwf{\judgetypei_{wf}}
\def\type{\typed}
\def\typewf{\typedwf}
\def\judgesubd{\vdash}
\def\judgesuba{\vdash_{\bullet}}
\def\judgesubi{\vdash_{\Box}}
\def\subtd{\judgesubd}
\def\subta{\judgesuba}
\def\subti{\judgesubi}
\def\subt{\subtd}
\def\subd{\rightslice}
\def\suba{\rightslice_{\bullet}}
\def\subi{\rightslice_{\Box}}
\def\sub{\subd}

\def\judgerw{~{\mbox{$"~>"$}}~}
\def\wf{\judgewf}

\newcommand{\matht}[1]{\text{{\tt #1}}}

\def\even{\matht{even}}
\def\odd{\matht{odd}}
\def\Set{\matht{Set}}
\def\Prop{\matht{Prop}}
\def\Type{\matht{Type}}
\def\ps{\emph{predicate subtyping}}

\def\WfAtomRule{Wf-Atom}
\def\WfVarRule{Wf-Var}
\def\PropSetRule{PropSet}
\def\TypeRule{Type}
\def\VarRule{Var}
\def\ProdRule{Prod}
\def\AbsRule{Abs}
\def\AppRule{App}
\def\LetInRule{Let-In}
\def\SigmaRule{Sigma}
\def\SumRule{Sum}
\def\LetSumRule{Let-Sum}
\def\SumInfRule{Sum-Inf}
\def\SumDepRule{Sum-Dep}
\def\SubsetRule{Subset}
\def\LetSubRule{Let-Sub}
\def\SubsumRule{Subsumption}
\def\ConvRule{Conv}

\def\SubReflRule{Sub-Refl}
\def\SubTransRule{Sub-Trans} 
\def\SubConvRule{Sub-Conv}
\def\SubProdRule{Sub-Prod}
\def\SubSigmaRule{Sub-Sigma}
\def\SubLeftRule{Sub-Left}
\def\SubRightRule{Sub-Right}
\def\SubProofRule{Sub-Proof}
\def\SubSubRule{Sub-Subset}
\def\SubTransRule{Sub-Trans}

\def\ifstreq#1#2{\def\testa{#1}\def\testb{#2}\ifx\testa\testb }

\def\inductionon#1{
  \ifstreq{#1}{typing-decl}{Par induction sur la d�rivation de typage.}
  \else\ifstreq{#1}{typing-algo}{Par induction sur la d�rivation de
    typage dans le syst�me algorithmique.}
  \else\ifstreq{#1}{typing-impl}{Par induction sur la d�rivation de
    typage.}
  \else\ifstreq{#1}{subtyping-decl}{Par induction sur la d�rivation de
    sous-typage.}
  \else\ifstreq{#1}{subtyping-algo}{Par induction sur la d�rivation de
    sous-typage dans le syst�me algorithmique.}
  \else\ifstreq{#1}{subtyping-impl}{Par induction sur la d�rivation de
    sous-typage.}
  \fi
}

\newenvironment{induction}[1][text=\empty]{
  \if#1\empty\else\inductionon{#1}
  \begin{list}{Unset default item}{}}
  {\end{list}}

%% Should be able to work with \else...
\newcommand{\inductionrule}[1]
{\ifstreq{#1}{WfAtom}{\WfAtomRule}
\fi\ifstreq{#1}{WfVar}\WfVarRule
\fi\ifstreq{#1}{PropSet}\PropSetRule
\fi\ifstreq{#1}{Type}\TypeRule
\fi\ifstreq{#1}{Var}\VarRule
\fi\ifstreq{#1}{Prod}\ProdRule
\fi\ifstreq{#1}{Abs}\AbsRule
\fi\ifstreq{#1}{LetIn}\LetInRule
\fi\ifstreq{#1}{Sigma}\SigmaRule
\fi\ifstreq{#1}{Sum}\SumRule
\fi\ifstreq{#1}{LetSum}\LetSumRule
\fi\ifstreq{#1}{App}\AppRule
\fi\ifstreq{#1}{SumInf}\SumInfRule
\fi\ifstreq{#1}{SumDep}\SumDepRule
\fi\ifstreq{#1}{LetSub}\LetSubRule
\fi\ifstreq{#1}{Subsum}\SubsumRule
\fi\ifstreq{#1}{Conv}\ConvRule
\fi\ifstreq{#1}{Subset}\SubsetRule

\fi\ifstreq{#1}{SubRefl}\SubReflRule
\fi\ifstreq{#1}{SubTrans}\SubTransRule
\fi\ifstreq{#1}{SubConv}\SubConvRule
\fi\ifstreq{#1}{SubProd}\SubProdRule
\fi\ifstreq{#1}{SubSigma}\SubSigmaRule
\fi\ifstreq{#1}{SubLeft}\SubLeftRule
\fi\ifstreq{#1}{SubRight}\SubRightRule
\fi\ifstreq{#1}{SubProof}\SubProofRule
\fi\ifstreq{#1}{SubSub}\SubSubRule
\fi}

\newcommand{\rulename}[1]{{\bf \inductionrule{#1}}}

\def\indrule{\rulename}

\def\case#1{\item[- \rulename{#1} :]}
\newcommand{\casetwo}[2]{\item[- \rulename{#1}, \rulename{#2} :]}
\def\casethree#1#2#3{\item[- \rulename{#1}, \rulename{#2},
  \rulename{#3} :]}
\def\casefour#1#2#3#4{\item[- \rulename{#1}, \rulename{#2},
  \rulename{#3}, \rulename{#4} :]}

\newcommand{\rname}[1]{{\bf \rulename{#1}}}
\newcommand{\rulelabel}[1]{{\bf (\rulename[#1])}}

%%% Local Variables: 
%%% mode: latex
%%% TeX-master: "subset-typing"
%%% LaTeX-command: "TEXINPUTS=\"style:$TEXINPUTS\" latex"
%%% End: 

\EnableBpAbbreviations
\def\TPOSR{{\sc tposr}\xspace}
\def\tactic#1{\texttt{#1}}

\def\coqmodule#1{\texttt{#1}}

\newcommand{\WfEmptyFull}[1]{
  \UAX{WfEmpty}
  {}
  {$\seq []$}
  {}
}  
\newcommand{\WfEmpty}[1][\Gamma]{\WfEmptyFull{#1}}
  
\newcommand{\WfVarFull}[4]{
  \UAX{WfVar}
  {$\tgen{#1}{#2}{#3}$}
  {$\wf #1, #4 : #2$}
  {$#3 `: \setproptype{} `^{} #4 `; #1$}
}
\newcommand{\WfVar}[1][\Gamma]{\WfVarFull{#1}{A}{s}{x}}

\newcommand{\PropSetFull}[2]{
  \UAX{PropSet}
  {$\wf #1$}
  {$\tgen{#1}{#2}{\Type}$}
  {$#2 `: \setprop$} 
}
\newcommand{\PropSet}[1][\Gamma]{\PropSetFull{#1}{s}}


\newcommand{\TypeTypeFull}[1]{
  \UAX{Type}
  {$\wf #1$}
  {$\tgen{#1}{\Type(i)}{\Type(i + 1)}$}
  {$i `: `N$}
}
\newcommand{\TypeType}[1][\Gamma]{\TypeTypeFull{#1}}

\newcommand{\VarFull}[3]{
  \BAX{Var}
  {$\wf #1$}
  {$#2 : #3 `: #1$}
  {$\tgen{#1}{#2}{#3}$}
  {}
}
\newcommand{\Var}[1][\Gamma]{\VarFull{#1}{x}{A}} 

\newcommand{\ProdFull}[7]{
  \BAX{Prod}
  {$\tgen{#1}{#2}{#3}$}
  {$\tgen{#1, #4 : #2}{#5}{#6}$}
  {$\tgen{#1}{\Pi #4 : #2.#5}{#7}$}
  {$(#3, #6, #7) `: \mathcal{R} `^{} #4 `; #1$}
}
\newcommand{\Prod}[1][\Gamma]{\ProdFull{#1}{T}{s_1}{x}{U}{s_2}{s_3}}

\newcommand{\AbsFull}[6]{
  \BAX{Abs}
  {$\tgen{#1}{\Pi #2 : #3.#4}{#5}$}
  {$\tgen{#1, #2 : #3}{#6}{#4}$}
  {$\tgen{#1}{\lambda #2 : #3. #6}{\Pi #2 : #3.#4}$}
  {$#2 `; #1$}
}

\newcommand{\Abs}[1][\Gamma]{\AbsFull{#1}{x}{T}{U}{s}{M}}

 \newcommand{\AppFull}[6]{
  \BAX{App}
  {$\tgen{#1}{#2}{\Pi #3 : #4. #5}$}
  {$\tgen{#1}{#6}{#4}$}
  {$\tgen{#1}{(#2 #6)}{#5 [ #6 / #3 ]}$}
  {$$}
}

\newcommand{\App}[1][\Gamma]{\AppFull{#1}{f}{x}{V}{W}{u}}

\newcommand{\SigmaRFull}[5]{
  \BAX{Sigma}
  {$\tgen{#1}{#2}{#3}$}
  {$\tgen{#1, #4 : #2}{#5}{#3}$}
  {$\tgen{#1}{\Sigma #4 : #2.#5}{#3}$}
  {$#3 `: \{ \Prop, \Set \} `^{} #4 `; #1$} 
}
\newcommand{\SigmaR}[1][\Gamma]{\SigmaRFull{#1}{T}{s}{x}{U}}

\newcommand{\SumDepFull}[7][\Gamma]{
  \TAX{SumDep}
  {$\tgen{#1}{\Sigma #2 : #5. #6}{#7}$}
  {$\tgen{#1}{#3}{#5}$}
  {$\tgen{#1}{#4}{#6[#3 / #2]}$}
  {$\tgen{#1}{\pair{\Sigma #2 : #5.#6}{#3}{#4}}{\Sigma #2 : #5.#6}$}
  {}
}

\newcommand{\SumDep}[1][\Gamma]{\SumDepFull[#1]{x}{t}{u}{T}{U}{s}}
 
\newcommand{\PiLeftFull}[5][\Gamma]{
  \UAX{PiLeft}
  {$\tgen{#1}{#2}{`S #3 : #4.#5}$}
  {$\tgen{#1}{\pi_1~#2}{#4}$}
  {}
}
\newcommand{\PiLeft}[1][\Gamma]{\PiLeftFull[#1]{t}{x}{T}{U}}

\newcommand{\PiRightFull}[5][\Gamma]{
  \UAX{PiRight}
  {$\tgen{#1}{#2}{`S #3 : #4.#5}$}
  {$\tgen{#1}{\pi_2~#2}{#5[\pi_1~#2/#3]}$}
  {}
}
\newcommand{\PiRight}[1][\Gamma]{\PiRightFull[#1]{t}{x}{T}{U}}


\newcommand{\SubsetFull}[4]{
  \BAX{Subset}
  {$\tgen{#1}{#3}{\Set}$}
  {$\tgen{#1, #2 : #3}{#4}{\Prop}$}
  {$\tgen{#1}{\mysubset{#2}{#3}{#4}}{\Set}$}
  {$#2 `; #1$}
}
\newcommand{\SubsetR}[1][\Gamma]{\SubsetFull{#1}{x}{U}{P}}


\newcommand{\SubsumFull}[5]{
  \TAX{Subsum}
  {$\tgen{#1}{#4}{#5}$}
  {$\tgen{#1}{#2}{#3}$}
  {$#5 \sub #2$} % #1 \subt 
  {$\tgen{#1}{#4}{#2}$}
  {}
}
\newcommand{\Subsum}[1][\Gamma]{\SubsumFull{#1}{T}{s}{t}{U}} 

\def\Coerce{\Subsum}

\newcommand{\ConvFull}[5]{
  \TAX{Conv}
  {$\tgen{#1}{#2}{#3}$}
  {$\tgen{#1}{#4}{#5}$}
  {$#5 \eqbr #2$}
  {$\tgen{#1}{#4}{#2}$}
  {}
}
\newcommand{\Conv}[1][\Gamma]{\ConvFull{#1}{T}{s}{t}{U}}

\newcommand{\typedRules}{
  \begin{center}
    \def\seq{\typed}
    \def\fCenter{\wf}
    \WfEmpty \DP\quad
    \WfVar \DP
    
    \def\fCenter{\typed}
    \vspace{\infvspace}
    \PropSet\DP

    \vspace{\infvspace}
    \Var\DP
    
    \vspace{\infvspace}
    \Prod\DP
    
    \vspace{\infvspace}
    \Abs\DP

    \vspace{\infvspace}
    \App\DP

    \vspace{\infvspace}
    \SigmaR\DP

    \vspace{\infvspace}
    \SumDep\DP
    
    \vspace{\infvspace}
    \PiLeft\DP
    \quad
    \PiRight\DP

    \vspace{\infvspace}
    \SubsetR\DP

    \vspace{\infvspace}
    \Subsum\DP
      
  \end{center}
}

\def\typedFig
{
\begin{figure}[tb]
  \typedRules
  \caption{Calcul de coercion par pr�dicats - version d�clarative}
  \label{fig:typing-decl-rules}
\end{figure}
}

%%% Local Variables: 
%%% mode: latex
%%% TeX-master: "subset-typing"
%%% LaTeX-command: "TEXINPUTS=\"style:$TEXINPUTS\" latex"
%%% End: 

\def\SubConv{
  \UAX{SubConv}
  {$T \eqbi U$}
  {$T \sub U$}
  {} 
}

\def\SubRefl{
  \UAX{SubRefl}
  {}
  {$S \seq S$}
  {}
}

\def\SubTrans{
\BAX{SubTrans}
{$S \seq T$}
{$T \seq U$}
{$S \seq U$}
{}
} 

\def\SubProd{
\BAX{SubProd}
{$U \seq T$} %"<|-|>"
{$V \seq W$}
{$\Pi x : T.V \seq \Pi x : U.W$}
{}
}

\def\SubSigma{
  \BAX{SubSigma}
  {$T \seq U$}
  {$V \seq W$}
  {$\Sigma x : T. V \seq \Sigma y : U. W$}
  {}
}

\def\SubProof{
  \UAX{SubProof}
  {$U \seq V$}
  {$U \seq \subset{x}{V}{P}$}
  {}
}

\def\SubSub{
  \UAX{SubSub}
  {$U \seq V$}
  {$\subset{x}{U}{P} \seq V$}
  {}
}

\def\subtdFig{
\begin{figure}[ht]
  \begin{center}
    \def\fCenter{\subd}
    
    \vspace{\infvspace}
    \SubConv\DP

    \vspace{\infvspace}
    \SubTrans\DP

    \vspace{\infvspace}
    \SubProd\DP

    \vspace{\infvspace}
    \SubSigma\DP
    
    \vspace{\infvspace}
    \SubProof\DP
    
    \vspace{\infvspace}
    \SubSub\DP
    
  \end{center}
  \caption{Coercion par pr�dicats - version d�clarative}
  \label{fig:subtyping-decl-rules}
\end{figure}
}

\def\subtdShort{
\begin{figure}[ht]
  \begin{center}
    \def\fCenter{\subd}
    \SubConv\DP
    \SubProof\DP
    \SubSub\DP

    \vspace{1cm}
    \SubProd\DP
    \SubSigma\DP
    %\SubTrans\DP
  \end{center}
  \caption{Coercion par pr�dicats - version d�clarative}
  \label{fig:subtyping-decl-rules-short}
\end{figure}
}

%%% Local Variables: 
%%% mode: latex
%%% TeX-master: "subset-typing"
%%% LaTeX-command: "TEXINPUTS=\"style:$TEXINPUTS\" latex"
%%% End: 


\title{\Russell's Metatheoretic Study - Notes}
\date{\today}
\author{Matthieu Sozeau \\
  LRI - Paris Sud - XI University \\
  sozeau@lri.fr}
\begin{document}
\maketitle
\begin{abstract}

We are working on the formalization of \Russell{}'s type theory in the
\Coq{} proof assistant \cite{sozeau:coq/Russell/meta}.  The type system
of \Russell{} is based on the Calculus of Constructions with
$\Sigma$-types (dependent sums), extended by an equivalence on types
which subsumes $\beta$-conversion. The extension permits to identify
types and subsets based on them in a manner similar to the
\emph{Predicate Subtyping} feature of \PVS{}.

We are aiming at a complete proof of \Russell{}'s metatheoretic properties
(structural properties, Subject Reduction, maybe Strong Normalization),
the refining steps which led us to the algorithmic system and the
corresponding typing algorithm and also the correctness of an interpretation
from \Russell{} to the Calculus of Inductive Constructions with
metavariables.

We started the development using the formalization of the Calculus of
Constructions by Bruno Barras \cite{Barras96a}.  We kept the standard de
Bruijn encoding for variable bindings and defined our judgements using
\emph{dependent} inductive predicates.  This alone causes some problems
for the faithful formalization of the paper results.  The proofs offer
several other technical difficulties including:
\begin{itemize}
\item Elimination of transitivity in a system with an \emph{untyped} type
  conversion relation.

\item Subject Reduction, which is \emph{not} directly provable for the
  declarative system. Here we adapted the technique developed by
  Robin Adams \cite{adams:PTSEQ}. It includes a new term algebra, with
  associated reduction operations and a new type system
  for which we have to prove metatheoretic properties.

\item Correction of the interpretation: the target system include
  metavariables, introducing a second, \emph{unusual} kind of variable binding.
\end{itemize}

\end{abstract}

% - Works only with $``>=$ coq-v8.2.

% Some topics of interest:
% \begin{itemize}
% \item Work on a dependency analyser for terms.
% \item Problem of mutual induction, conjunction of goals, generating the
%   principle and applying it.
% \item Problem of induction on partially instanciated goals, see McBride.
% \item Induction on depths:
%   \begin{itemize}
%   \item Adding a parameter on judgements.
%   \item Adding a parameter and going in \Set.
%   \end{itemize}
% \item Useful tactics: \tactic{subst}, \tactic{eauto}, \tactic{generalize dependent}.
% \item Reusability of proofs (multiple systems).
% \end{itemize}

\section*{Organization of the \Coq proof}
The proof is developed in the \coqmodule{Lambda} namespace. At the root
we have definition of the term language \coqmodule{Lambda.Terms} which
is a lambda calculus with dependent products, sums and a distinguished
subset type with no introduction or elimination constructs. 
In \coqmodule{LiftSubst} we have lemmas on substitution and lifting.
The definitions and proofs about
reduction and conversion including Church-Rosser for $\beta\pi$ reduction
are in the remaining modules.

The first type system is \coqmodule{Lambda.CCSum}, the Calculus of
Constructions enriched with dependent sums. All the metatheory up to
Subject Reduction and Unicity has been done for this system, simply
adapting the work of Bruno Barras.

Then we have two versions of the \Russell type system.
The \coqmodule{Lambda.JRussell} directory contains a definition of
Russell with judgemental equality and a coercion relation with
transitivity and symetry built-in. It is the most declarative, and thus
the clearest presentation of \Russell, however we can't prove Subject
Reduction for this system directly. We proved basic metatheory for this
system up to validity.

The modules in \coqmodule{Lambda.Russell} define a more algorithmical
version of the \Russell type system. This version uses and untyped
conversion relation and an already refined coercion algorithm (without
transitivity).

\section*{The proof of Subject Reduction}
Directly proving Subject Reduction for our system is not possible
because of the enriched equivalence we use. Indeed, we use a typed
equivalence relation, hence we cannot use the same style of proof as the
one for the Calculus of Constructions. In the later case, when proving
subject reduction at the application case, with reduction being $\beta$, we have:
\begin{prooftree}
  \AXC{$`G \type (\lambda x : A.M) : \Pi x : B.C$}
  \AXC{$`G \type N : B$}
  \BIC{$`G \type (\lambda x : A.M)~N : C[N/x]$}
\end{prooftree}

\def\redonebp{"->"_{\beta\pi}}
\def\redbp{"->>"_{\beta\pi}}
\def\redpbp{"->"_{\beta\pi}^+}

%Hence by induction hypothesis we get:
%$`A t', (\lambda x : A. M) \redonebp t' "=>" `G \type t' : \Pi x : B.C$ and
%$`A N', N \redonebp N' "=>" `G \type N' : B$.
We need to prove $`G \type M[N/x] : C[N/x]$.

By inversion for dependent products, we have $(s, s') `: \PTSaxioms$ so that
\begin{eqnarray*}
  `G & \type & A : s \\
  `G, x : A & \type & D : s' \\
  `G, x : A & \type & M : D
\end{eqnarray*}
and $\Pi x : B.C \eqbp \Pi x : A.D$ ($`G \type \Pi x : B.C = \Pi x : A.D
: s'$ for typed equivalence). By the Church-Rosser theorem for $\eqbp$ we
get $A \eqbp B$ and $C \eqbp D$. Hence we have $`G \type N : A$ and $`G, x
: A \type M : C$ by conversion. By substitution we can then derive
$`G \type M[N/x] : C[N/x]$.

\begin{paragraph}{}
What goes wrong in the typed version ? We would need to prove the
following lemma (injectivity of pi):
\[`G \type \Pi x : A.B = \Pi x : C.D : s' "=>" `E s, 
`G \type A = C : s `^ `G, x : A \type B = D : s `^ (s, s') `: \PTSaxioms\] However, this is not
provable as we can use a transitivity rule in the definition of typed
equivalence. Here the premise can have the form
$`G \type \Pi x : A.B = X_0 `^ \ldots `^ X_n = \Pi x : C.D : s$ and we
can't say anything about the $X$'s yet, in particular we cannot prove
that they reduce to products.
Hence we need to devise a new, equivalent system for which subject
reduction is provable. We can do that using a typed parallel one step
reduction relation (\TPOSR) on \emph{labelled} terms which generates the typed
equivalence relation on the original terms. 
The formalization as a reduction relation, along with its proof of
Church-Rosser eliminates transitivity and permits to prove Uniqueness of
Types and Injectivity of products (and sums in our case). It is then possible
to prove Subject Reduction for this system and transpose this result to
the judgemental equality system by virtue of the equivalence of the two
systems.
\end{paragraph}



\subsection*{Proof sketch}
The technique of using \TPOSR is due to Robin Adams work on proving the
equivalence of judgemental equality (or typed equivalence) and untyped
conversion presentations of Pure Type Systems \cite{adams:PTSEQ}. We
extended this technique to an enriched equivalence relation which we
call coercion (denoted by $\sub$). Coercion includes the usual equivalence relation
generated by the reduction rules of the system ($\beta$ and $\pi$ in our
case) and extends it with custom equality rules. In the \Russell system,
we added the following rules:

\begin{center}
\SubProof\DP
\vspace{1em}

\SubSub\DP
\end{center}

The following module is the definition of the \Russell system with
judgemental equality in \Coq.

\input{JRussell.Types.v}


\def\lapp#1#2{\text{app}_{#1}(#2)}
\def\lpi#1#2{\text{pi}_{#1}(#2)}
\def\rel{\mathcal{R}}



The proof works as follows: 

\subsubsection*{\TPOSR system}
We define a \TPOSR system on labelled terms. Here follows the
corresponding \Coq definitions (\coqmodule{Lambda.TPOSR.Types}).

\vspace{1em}

\input{TPOSR.Types.v}

We add labels at applications and projections. 
Applications are annoted with the codomain of their function, for example, in
$\lapp{(x)B}{M~N}$, we suppose that $M$ can be typed with a product
type of the form $\Pi x : \_. B$. Projections are annotated with their
full domain, e.g in $\lpi{`S x : A.B}{t}$, we suppose that $t$ can be
typed with the sum type $`S x : A.B$. The reduction associated with
labelled terms is the same as for unlabelled ones, plus reduction on
labels. On the other hand, in the \TPOSR relation, we allow coercion
of type labels at will. Indeed an application of type $B$ can be seen
as an object of type $B'$ if they are coercible. 

We prove the basic metatheory of the system, including thinning
 (\coqmodule{Lambda.TPOSR.Thinning}), substituion
(\coqmodule{Lambda.TPOSR.Substitution}) and
substituion of \TPOSR derivations
(\coqmodule{Lambda.TPOSR.SubstitutionTPOSR}). Most proofs use mutual
induction on the typing, well-formedness of contexts, equivalence
relation and coercion relation derivations and are
straightforward. 
However, much care must be taken in the staging of
proofs. We denote by $`G \type J$ any of the four judgements and use
$\rel$ to range over $"->"$ (reduction), $\simeq$ (equivalence) and $\sub$ (coercion).
We were able to complete these proofs by following this order:
\begin{enumerate}
\item First, proving left reflexivity:
  If $`G \type t "->" t' : T$ then $`G \type t "->" t : T$
\item Then a preliminary form of context coercion:
  If $`G, x : A, `D \type J$ and $`G \type A \sub B : s$ 
  with $`G \type A "->" A : s$ and $`G \type B "->" B : s$ then
  $`G, x : B, `D \type J$.
\item Then substitution: if $`G, x : U, `D \type J$ and $`G \type u
  "->" u : U$ then $`G, `D[u/x] \type J[u/x]$.
\item Then a preliminary form of substitution of \TPOSR derivations:
  If $`G, x : U, `D \type t~\rel~t' : T$ and $`G \type u "->" u' : U$
  with $`G \type u' "->" u' : U$ 
  then $`G, `D[u/x] \type t[u/x]~\rel~t'[u'/x] : T[u/x]$.
\item Then we can prove right reflexivity:
  If $`G \type t~\rel~t' : T$ then $`G \type t'~\rel~t' : T$
\end{enumerate}
   
After proving right reflexivity we can remove the side conditions in the
statements of context coercion and substitution.

Once done with this we can prove generation lemmas for the \TPOSR judgement.
They are used to prove validity: if $`G \type t "->" t : T$ then $T = s$
for $s `: \PTSsorts$ or there exists $s `: \PTSsorts$ so that $G \type T
"->" T : s$. Once we have validity we just need to prove we have
functionality of types to get uniqueness of types. Functionality says
that types with equivalent components are equivalent. For example,
functionality of pi is the property:
if $`G \type A \simeq B : s1$ and $`G, x : A \type C \simeq D : s2$ then
$`G \type \Pi x : A.C \simeq \Pi x : B.D : s2$.

We can then prove uniqueness of types: $`G \type t "->"~? : T$ and $`G
\type t "->"~? : U$ then $T = U = \Type$ or there exists a sort $s$ so
that $`G \type T \sub U : s$. This is were labels are needed, indeed at
the application case, in the original system we have the derivations:

\begin{prooftree}
  \AXC{$`G \type M : \Pi x : A. C$}
  \AXC{$`G \type N : A$}
  \BIC{$`G \type M~N "->"~? : C[N/x]$}
\end{prooftree}
\begin{prooftree}
  \AXC{$`G \type M : \Pi x : B. D$}
  \AXC{$`G \type N : B$}
  \BIC{$`G \type M~N "->"~? : D[N/x]$}
\end{prooftree}

By induction hypothesis we have $`G \type \Pi x : A.C \sub \Pi x : B.D :
s$ for some s. However we cannot deduce yet that it implies $`G, x : B
\type C \sub D : s$, as this requires injectivity of products which we
can't prove, just like for the subject reduction proof. If we add labels
we have:

\begin{prooftree}
  \AXC{$`G \type M : \Pi x : A. C$}
  \AXC{$`G \type N : A$}
  \BIC{$`G \type \lapp{(x)E}{M~N} "->"~? : C[N/x]$}
\end{prooftree}
\begin{prooftree}
  \AXC{$`G \type M : \Pi x : B. D$}
  \AXC{$`G \type N : B$}
  \BIC{$`G \type \lapp{(x)E}{M~N} "->"~? : D[N/x]$}
\end{prooftree}

Generation gives us the addidional hypothesis that $`G \type E[N/x] \sub
C[N/x] : s$ and $`G \type E[N/x] \sub D[N/x]$, hence we are able to
complete the proof.
The same technique permits to prove uniqueness of types for projections.

Now that we have uniqueness of types, we can prove a Church-Rosser
property for \TPOSR:
If $`G \type t "->" u : U$ and $`G \type t "->" v : V$ then there exists
$t'$ so that
\begin{eqnarray*}
  `G \type u "->" t' : U \\
  `G \type u "->" t' : V \\
  `G \type v "->" t' : U \\
  `G \type v "->" t' : V
\end{eqnarray*}

The proof is by induction on the sum of the depth of the two derivations 
(\coqmodule{Lambda.TPOSR.\-ChurchRosserDepth}).

As a corrolary we get (\coqmodule{Lambda.TPOSR.ChurchRosser}):
If $`G \type t \simeq u : s$ then there exists $x$ so that $`G \type t
\redpbp x : s$ and $`G \type u \redpbp x : s$.

It is then easy to prove injectivity of products and sums using a
similar argument as the one for untyped conversion : if $`G \type \Pi x :
A. C \simeq \Pi x : B. D : s$ then there exists $x$, $`G \type \Pi x : A. C
\redpbp x : s$ and $`G \type \Pi x : B.D \redpbp x : s$. So we can
derive that $x `= \Pi x : E.F$ for some $E, F$ and there exists $s'$
so that $`G \type A \simeq E \simeq C : s1$, and $`G, x : B \type C
\simeq F \simeq D : s$.

The story doesn't ends here for injectivity as we have enriched the
equivalence with a coercion system, so we need to prove a similar
property for this system. Again this is not a trivial matter because we
have a transitivity rule in this system which causes the same problem as
for the usual equivalence. What we need to do is prove elimination of
transitivity for the coercion system. The proof is again by induction on
the depths of the two coercion derivations from which we build one 
(\coqmodule{Lambda.TPOSR.CoercionDepth}).
Then we are able to prove injectivity of products and sums with respect
to coercion (\coqmodule{Lambda.TPOSR.Injectivity}):
If $`G \type \Pi x : A. C \sub \Pi x : B. D : s$ then there exists $s'$
so that $`G \type B \sub C : s'$ and $`G, x : B \type C \sub D : s$.

Once we have injectivity for our type constructors we can easily prove
subject reduction for the \TPOSR system.

\subsubsection*{Translation to and from \Russell}
To prove the equivalence of the \TPOSR system and the original one, we
need to show properties on the unlabelling of terms (denoted by the
$|\_|$ function) (\coqmodule{Lambda.TPOSR.Unlab}). In this module, apart from easy
properties of commutation of substitution, lifting and unlabelling,
we show that labelled reductions entail reductions on the unlabelled terms and vice-versa:
If $|t'| \redonebp u$, then there exists $u'$, $|u'| = u$ and $t'
\redonebp u'$. 

Then the easy way, from \TPOSR to \Russell
(\coqmodule{Lambda.Meta.TPOSR\_JRussell}, \coqmodule{Lambda.Meta.\-TPOSR\_Russell}), is proved by a simple induction
on derivations:
If $`G \type t "->" t' : T$ then $|`G| \type |t| : |T|$ and $|`G| \type |t'| : |T|$.

The other way needs another lemma: stated simply, we need to show
that if two typable labelled terms have syntaxically equal translates, then they
have a common reduct and coercible types
(\coqmodule{Lambda.TPOSR.UnlabConv}). Once armed with this result
and its extension to contexts we can prove that unlabelling is complete
with respect to the original system (\coqmodule{Lambda.Meta.Russell\_TPOSR}), that is:
If $`G \type t : T$ then there exists $`G'$, $t'$ and $T'$ so that
$|`G'| = `G$, $|t'| = t$ and $T' = |T|$, with $|`G'| \type |t'| : |T'|$.

It is then possible to prove subject reduction for the original system:
If $`G \type t : T$ and $t \redonebp u$ then $`G \type u : T$ (\coqmodule{Lambda.Meta.SubjectReduction}).

First we use the last lemma to go in \TPOSR, then we translate the
unlabelled reduction ta labelled one: $t' \redonebp u'$ with $|u'| = u$.
We need just apply subject reduction for \TPOSR and go back to \Russell.

\bibliography{bib-joehurd,barras,pvs-bib,bcp,Luo,biblio-subset,%
cparent,demons,demons2,demons3}
\bibliographystyle{acm}

%% \renewcommand{\thefootnote}{}
%% \footnotetext{This article was typeset in \LaTeX~using the
%%   \texttt{Concrete Math} font}

\end{document}

%%% Local Variables: 
%%% mode: latex
%%% LaTeX-command: "x=pdf; TEXINPUTS=\"..:../style:../figures:../../proof/latex:\" ${pdfx}latex"
%%% TeX-master: t
%%% End: 
