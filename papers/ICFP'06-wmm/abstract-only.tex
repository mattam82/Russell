\documentclass{article}
\usepackage[latin1]{inputenc}
\usepackage{xspace} % To get the right spacings in front of : and so on
\usepackage[english]{babel}
\usepackage{abbrevs}
\usepackage{subfigure}
\usepackage{ifthen}
\usepackage{coqdoc}
\usepackage{bussproofs}
\usepackage{pgf,pgfarrows,pgfnodes}
%\usepackage{concmath}
%\usepackage{ccfonts,eulervm}
 \usepackage{concmath}
\usepackage{url} \usepackage{fullpage}

\title{\Russell's Metatheoretic Study - Abstract}
\date{\today}
\author{Matthieu Sozeau \\
  LRI -- Paris Sud XI University \\
  \url{sozeau@lri.fr} -- \url{http://www.lri.fr/~sozeau}}
\begin{document}
\maketitle

We are working on the formalization of \Russell{}'s type theory in the
\Coq{} proof assistant \cite{sozeau:coq/Russell/meta}.  The type system
of \Russell{} is based on the Calculus of Constructions with
$\Sigma$-types (dependent sums), extended by an equivalence on types
which subsumes $\beta$-conversion. The extension permits to identify
types and subsets based on them in a manner similar to the
\emph{Predicate Subtyping} feature of \PVS{}.

We are aiming at a complete proof of \Russell{}'s metatheoretic properties
(structural properties, Subject Reduction, maybe Strong Normalization),
the refining steps which led us to the algorithmic system and the
corresponding typing algorithm and also the correctness of an interpretation
from \Russell{} to the Calculus of Inductive Constructions with
metavariables.

We started the development using the formalization of the Calculus of
Constructions by Bruno Barras \cite{Barras96a}.  We kept the standard de
Bruijn encoding for variable bindings and defined our judgements using
\emph{dependent} inductive predicates.  This alone causes some problems
for the faithful formalization of the paper results.  The proofs offer
several other technical difficulties including:
\begin{itemize}
\item Elimination of transitivity in a system with an \emph{untyped} type
  conversion relation.

\item Subject Reduction, which is \emph{not} directly provable for the
  declarative system. Here we adapted the technique developed by
  Robin Adams \cite{adams:PTSEQ}. It includes a new term algebra, with
  associated reduction operations and a new type system
  for which we have to prove metatheoretic properties.

\item Correction of the interpretation: the target system has
  metavariables, introducing a second, \emph{unusual} kind of variable
  binding.
\end{itemize}

We will present these difficulties and our recipes for solving them in
the \Coq environment. Additionally, we will identify more general
problems arising in \Coq proofs and suggest possible solutions. 

{\bf Disclaimer}: this is a work in
progress ; to date we have proved elimination of transitivity and
subject reduction should follow soon.

\bibliography{bib-joehurd,barras,pvs-bib,bcp,Luo,biblio-subset,%
cparent,demons,demons2,demons3}
\bibliographystyle{acm}

%% \renewcommand{\thefootnote}{}
%% \footnotetext{This article was typeset in \LaTeX~using the
%%   \texttt{Concrete Math} font}

\end{document}

%%% Local Variables: 
%%% mode: latex
%%% LaTeX-command: "x=pdf; TEXINPUTS=\"..:../style:../figures:\" ${pdfx}latex"
%%% TeX-master: t
%%% End: 
