\chapter{Exemples}
\begin{figure}[ht]
\begin{verbatim}
(* Subtac ne g�re pas encore les notations de Coq *)
Definition neq (A : Type) (x y : A) : Prop := x <> y.
Definition div_prop (a b q r : nat) := a = (b * q) + r /\ r < b. 
Definition lt_ge_dec (x y : nat) : { x < y } + { x >= y }.
Proof.
  intros ; elim (le_lt_dec y x) ; intros ; auto with arith.
Defined.

Recursive program mydiv (a : nat) { well_founded lt a lt_wf } : { b : nat | neq nat b O } ->
  [ q : nat ] { r : nat | div_prop a b q r } :=
  fun { y : nat | neq nat y O } =>
    if lt_ge_dec a y
    then (q := O, a : { r : nat | div_prop a y q r })
    else let (q', r) = mydiv (minus a y) y in 
        (q := S q', r : { r : nat | div_prop a y q r }).

(* Dans Coq, mydiv aura le type:
forall a : nat, forall b : { b : nat | b <> 0 },
 { q : nat & { r : nat | div_prop a (proj1_sig b) q r } } *)

(* Obligations de preuves engendr�es *)
(* Hypoth�ses communes: *)
a : nat
mydiv : (n : nat) n < a -> forall b : { b : nat | b <> 0 },
 { q : nat & { r : nat | div_prop n (proj1_sig b) q r } }
y : { b : nat | b <> 0 }

(* (q := 0, a ...)
[ H : a < proj1_sig y,
|- div_prop a (proj1_sig y) 0 a]

(* Argument de r�cursion *)
[H : a >= proj1_sig y |- a - proj1_sig y < a]

(* (q := S q', r) *)
[ H : a >= proj1_sig y
  q' : nat
  r : { r : nat | div_prop (a - proj1_sig y) (proj1_sig y) q' r }
|- div_prop a (proj1_sig y) (S q')  (proj1_sig r)]
\end{verbatim}
  \caption{La division euclidienne avec \Subtac}
  \label{fig:euclid-subtac}
\end{figure}

\begin{figure}[ht]
\begin{verbatim}
Definition div : forall a : nat, forall b : nat, 
  b <> 0 -> { q : nat & { r : nat | r < b /\ a = b * q + r } }.
Proof.
intros a ; pattern a ; apply lt_wf_rec ; intros. (* R�cursion *)
elim (lt_ge_dec n b). (* If then else *)
intros. (* Premi�re branche *)
(* Structure du terme *)
refine (existS _ 0 _) ; refine (exist _ n _) ; refine (conj _ _) ;
[ assumption | rewrite mult_0_r ; rewrite plus_0_l ; reflexivity ]. (* Preuve *)
(* Seconde branche *)
intros ; assert (n - b < n). (* Preuve pour l'appel *)
apply lt_minus ; [ apply (ge_le _ _ b0) | apply (nat_neq_0_gt_0 b H0) ].
induction (H (n - b) H1 b H0). (* Appel r�cursif *)
induction p ; induction p. (* Destruction du r�sultat *)
refine (existS _ (S x) _) ; refine (exist _ x0 _). (* Structure du terme *)
(* Preuve *)
split.
assumption.
pose (eq_plus_eq _ _ H3 b).
assert (n - b + b = n) ; try omega.
rewrite <- H4 ; rewrite e ; rewrite plus_comm ; rewrite plus_assoc.
replace (b + b * x) with (b * S x).
reflexivity.
rewrite mult_comm ; simpl ; pattern (x * b) ; rewrite mult_comm.
reflexivity.
Qed.
\end{verbatim}
  \caption{Script de preuve de la division euclidienne}
  \label{fig:euclid-script}
\end{figure}

%%% Local Variables: 
%%% mode: latex
%%% TeX-master: "subset-typing"
%%% End: 
