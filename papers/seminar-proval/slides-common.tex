\usepackage[T1]{fontenc}
\usepackage{concrete}

\mode<all>
{
\setbeamerfont{block title}{size={}}
\usefonttheme{serif}
\usefonttheme{professionalfonts}
\setbeamercovered{invisible}
\useoutertheme[subsection=false]{smoothbars}
\useinnertheme[shadow=true]{rounded}
\usecolortheme{orchid}
\usecolortheme{whale}
\setbeamertemplate{navigation symbols}{}
\setbeamertemplate{mini frames}[box]
\setbeamertemplate{itemize items}[triangle]
\setbeamertemplate{enumerate items}[square]

}

\usepackage[latin1]{inputenc}
\usepackage{xspace} % To get the right spacings in front of : and so on
\usepackage[english]{babel}
\usepackage{abbrevs}
\usepackage{subfigure}
\usepackage{ifthen}
\usepackage{mycoqdoc}
\usepackage{bussproofs}
\usepackage{pgf}
\usepackage{colortbl}
\usepackage{xcolor}

\def\shaded#1{{\color{black!50}#1}}

\def\nat{`N}
\def\app{\mathbin{+\!\!\!+}}

% macro for typesetting keywords
\renewcommand{\coqdockw}[1]{\texttt{#1}}

\EnableBpAbbreviations
\def\fCenter{\vdash}
\def\seq{\fCenter}

\hypersetup{
  pdftitle = Programing in Coq,
  pdfauthor = Matthieu Sozeau,
  pdfsubject = Theoretical Computer Science
 } 

\def\destml{\mlkw{dest}}
\newcommand{\eqrec}[6]{\texttt{eq\_rec}~#1~#2~#3~#4~#5~#6}
\newcommand{\eqrect}[6]{\texttt{eq\_rect}~#1~#2~#3~#4~#5~#6}

\newcommand{\termgrammar}
{$\begin{array}{lcl}
    `a & \Coloneqq & x \\
    & | & \funml{}~x~:~`t "=>" `a \\
    & | & `a~`a \\ 
    & | & `a~`t \\
    & | & \text{\emph{constante}} \\
% & | & (`a,~`a) \\
    & | & (x \coloneqq `a,~`a~: `t) \\
    & | & \letml~x = `a ~\inml~`a \\
    & | & \letml~(x, y) = `a ~\inml~ `a

%    & | & \ifml~`a~\thenml~`a~\elseml~`a
  \end{array}$}

\newcommand{\typegrammarOrig}
{$\begin{array}{lcl}
    `t & \Coloneqq & x \\
    & | & `t "->" `t \\
    & | & `t~`t \\
    & | & \text{\emph{constante}} \\
    & | & `t * `t \\
    & | & \Sigma x : `t. `t \\
    & | & \lambda{}~x~:~s "=>" `t \\
    & | & `t~`a \\
    & | & \Pi x : `t. `t \\
    & | & \subset{x}{`t}{`t}
  \end{array}$}

\newcommand{\typegrammar}
{$\begin{array}{lcl}
    `t & \Coloneqq & x \\
    & | & `t~`t \\
    & | & `t~`a \\
    & | & \lambda{}~x~:~`t "=>" `t \\
    & | & \Pi x : `t. `t \\
    & | & \Sigma x : `t. `t \\
%    & | & `t * `t \\
    & | & \subset{x}{`t}{`t} \\
    & | & \Set \\
    & | & \Prop \\
    & | & \Type \\
    & | & \text{\emph{constante}} 
%    & & \\
%    \text{{\tt Inductive}} & \Coloneqq & ident
  \end{array}$}

\newboolean{defineTheoremEn}
\newboolean{defineTheoremFr}
\setboolean{defineTheoremEn}{true}
\setboolean{defineTheoremFr}{false}
\newtheorem{theorem}{Th�or�me}[section]
\newtheorem{lemma}[theorem]{Lemme}
\newtheorem{fact}[theorem]{Fait}
\newtheorem{proposition}[theorem]{Proposition}
\newtheorem{definition}[theorem]{D�finition}
\newtheorem{example}[theorem]{Exemple}
\newtheorem{remark}[theorem]{Remarque}
\newtheorem{corrolary}[theorem]{Corrolaire}

\makeatletter

\newcommand{\UR}[2]{\RightLabel{\rulelabel{#1}}\UIC{#2}}
\newcommand{\URL}[2]{\LeftLabel{\rulelabel{#1}}\UIC{#2}}

\newboolean{displayLabels}
\setboolean{displayLabels}{true}

\newcommand{\LeftRuleLabel}[1]{
  \@ifnotmtarg{#1}
  {\ifthenelse{\boolean{displayLabels}}{\LeftLabel{\rulelabel{#1}}}{}}
}

\newcommand{\UAX}[4]{\AXC{#2}
  \LeftRuleLabel{#1}
  \@ifnotmtarg{#4}{\RightLabel{#4}}
  \UIC{#3}}

\newcommand{\BAX}[5]{\AXC{#2}\AXC{#3}
  \LeftRuleLabel{#1}
  \@ifnotmtarg{#5}{\RightLabel{#5}}
  \BIC{#4}}

\newcommand{\TAX}[6]{\AXC{#2}\AXC{#3}\AXC{#4}
  \LeftRuleLabel{#1}
  \@ifnotmtarg{#6}{\RightLabel{#6}}
  \TIC{#5}}

\newcommand{\QAX}[7]{\AXC{#2}\noLine\UIC{#3}\AXC{#4}\AXC{#5}
  \LeftRuleLabel{#1}
  \@ifnotmtarg{#7}{\RightLabel{#7}}
  \TIC{#6}}

\newcommand{\BR}[2]{\RightLabel{\rulelabel{#1}}\BIC{#2}}
\newcommand{\BRL}[2]{\LeftLabel{\rulelabel{#1}}\BIC{#2}}

\makeatother

%%% Local Variables: 
%%% mode: latex
%%% TeX-master: t
%%% End: 

\setboolean{displayLabels}{true}

%\def\ifstreq#1#2{\ifEqString{#1}{#2}}

\def\impsub{\rightslice}

\def\judgewf{\vdash_{wf}}
\def\judgetyped{\vdash}
\def\judgetypea{\vdash_{\bullet}}
\def\judgetypei{\vdash_{\Box}}
\def\typed{\judgetyped}
\def\typedwf{\judgewf}
\def\typea{\judgetypea}
\def\typeawf{\judgetypea_{wf}}
\def\typei{\judgetypei}
\def\typeiwf{\judgetypei_{wf}}
\def\type{\typed}
\def\typewf{\typedwf}
\def\judgesubd{\vdash}
\def\judgesuba{\vdash_{\bullet}}
\def\judgesubi{\vdash_{\Box}}
\def\subtd{\judgesubd}
\def\subta{\judgesuba}
\def\subti{\judgesubi}
\def\subt{\subtd}
\def\subd{\rightslice}
\def\suba{\rightslice_{\bullet}}
\def\subi{\rightslice_{\Box}}
\def\sub{\subd}

\def\judgerw{~{\mbox{$"~>"$}}~}
\def\wf{\judgewf}

\newcommand{\matht}[1]{\text{{\tt #1}}}

\def\even{\matht{even}}
\def\odd{\matht{odd}}
\def\Set{\matht{Set}}
\def\Prop{\matht{Prop}}
\def\Type{\matht{Type}}
\def\ps{\emph{predicate subtyping}}

\def\WfAtomRule{Wf-Atom}
\def\WfVarRule{Wf-Var}
\def\PropSetRule{PropSet}
\def\TypeRule{Type}
\def\VarRule{Var}
\def\ProdRule{Prod}
\def\AbsRule{Abs}
\def\AppRule{App}
\def\LetInRule{Let-In}
\def\SigmaRule{Sigma}
\def\SumRule{Sum}
\def\LetSumRule{Let-Sum}
\def\SumInfRule{Sum-Inf}
\def\SumDepRule{Sum-Dep}
\def\SubsetRule{Subset}
\def\LetSubRule{Let-Sub}
\def\SubsumRule{Subsumption}
\def\ConvRule{Conv}

\def\SubReflRule{Sub-Refl}
\def\SubTransRule{Sub-Trans} 
\def\SubConvRule{Sub-Conv}
\def\SubProdRule{Sub-Prod}
\def\SubSigmaRule{Sub-Sigma}
\def\SubLeftRule{Sub-Left}
\def\SubRightRule{Sub-Right}
\def\SubProofRule{Sub-Proof}
\def\SubSubRule{Sub-Subset}
\def\SubTransRule{Sub-Trans}

\def\ifstreq#1#2{\def\testa{#1}\def\testb{#2}\ifx\testa\testb }

\def\inductionon#1{
  \ifstreq{#1}{typing-decl}{Par induction sur la d�rivation de typage.}
  \else\ifstreq{#1}{typing-algo}{Par induction sur la d�rivation de
    typage dans le syst�me algorithmique.}
  \else\ifstreq{#1}{typing-impl}{Par induction sur la d�rivation de
    typage.}
  \else\ifstreq{#1}{subtyping-decl}{Par induction sur la d�rivation de
    sous-typage.}
  \else\ifstreq{#1}{subtyping-algo}{Par induction sur la d�rivation de
    sous-typage dans le syst�me algorithmique.}
  \else\ifstreq{#1}{subtyping-impl}{Par induction sur la d�rivation de
    sous-typage.}
  \fi
}

\newenvironment{induction}[1][text=\empty]{
  \if#1\empty\else\inductionon{#1}
  \begin{list}{Unset default item}{}}
  {\end{list}}

%% Should be able to work with \else...
\newcommand{\inductionrule}[1]
{\ifstreq{#1}{WfAtom}{\WfAtomRule}
\fi\ifstreq{#1}{WfVar}\WfVarRule
\fi\ifstreq{#1}{PropSet}\PropSetRule
\fi\ifstreq{#1}{Type}\TypeRule
\fi\ifstreq{#1}{Var}\VarRule
\fi\ifstreq{#1}{Prod}\ProdRule
\fi\ifstreq{#1}{Abs}\AbsRule
\fi\ifstreq{#1}{LetIn}\LetInRule
\fi\ifstreq{#1}{Sigma}\SigmaRule
\fi\ifstreq{#1}{Sum}\SumRule
\fi\ifstreq{#1}{LetSum}\LetSumRule
\fi\ifstreq{#1}{App}\AppRule
\fi\ifstreq{#1}{SumInf}\SumInfRule
\fi\ifstreq{#1}{SumDep}\SumDepRule
\fi\ifstreq{#1}{LetSub}\LetSubRule
\fi\ifstreq{#1}{Subsum}\SubsumRule
\fi\ifstreq{#1}{Conv}\ConvRule
\fi\ifstreq{#1}{Subset}\SubsetRule

\fi\ifstreq{#1}{SubRefl}\SubReflRule
\fi\ifstreq{#1}{SubTrans}\SubTransRule
\fi\ifstreq{#1}{SubConv}\SubConvRule
\fi\ifstreq{#1}{SubProd}\SubProdRule
\fi\ifstreq{#1}{SubSigma}\SubSigmaRule
\fi\ifstreq{#1}{SubLeft}\SubLeftRule
\fi\ifstreq{#1}{SubRight}\SubRightRule
\fi\ifstreq{#1}{SubProof}\SubProofRule
\fi\ifstreq{#1}{SubSub}\SubSubRule
\fi}

\newcommand{\rulename}[1]{{\bf \inductionrule{#1}}}

\def\indrule{\rulename}

\def\case#1{\item[- \rulename{#1} :]}
\newcommand{\casetwo}[2]{\item[- \rulename{#1}, \rulename{#2} :]}
\def\casethree#1#2#3{\item[- \rulename{#1}, \rulename{#2},
  \rulename{#3} :]}
\def\casefour#1#2#3#4{\item[- \rulename{#1}, \rulename{#2},
  \rulename{#3}, \rulename{#4} :]}

\newcommand{\rname}[1]{{\bf \rulename{#1}}}
\newcommand{\rulelabel}[1]{{\bf (\rulename[#1])}}

%%% Local Variables: 
%%% mode: latex
%%% TeX-master: "subset-typing"
%%% LaTeX-command: "TEXINPUTS=\"style:$TEXINPUTS\" latex"
%%% End: 

\def\infvspace{0.5cm}

\newcommand{\WfEmptyFull}[1]{
  \UAX{WfEmpty}
  {}
  {$\seq []$}
  {}
}  
\newcommand{\WfEmpty}[1][\Gamma]{\WfEmptyFull{#1}}
  
\newcommand{\WfVarFull}[4]{
  \UAX{WfVar}
  {$\tgen{#1}{#2}{#3}$}
  {$\wf #1, #4 : #2$}
  {$#3 `: \setproptype{} `^{} #4 `; #1$}
}
\newcommand{\WfVar}[1][\Gamma]{\WfVarFull{#1}{A}{s}{x}}

\newcommand{\PropSetFull}[2]{
  \UAX{PropSet}
  {$\wf #1$}
  {$\tgen{#1}{#2}{\Type}$}
  {$#2 `: \setprop$} 
}
\newcommand{\PropSet}[1][\Gamma]{\PropSetFull{#1}{s}}


\newcommand{\TypeTypeFull}[1]{
  \UAX{Type}
  {$\wf #1$}
  {$\tgen{#1}{\Type(i)}{\Type(i + 1)}$}
  {$i `: `N$}
}
\newcommand{\TypeType}[1][\Gamma]{\TypeTypeFull{#1}}

\newcommand{\VarFull}[3]{
  \BAX{Var}
  {$\wf #1$}
  {$#2 : #3 `: #1$}
  {$\tgen{#1}{#2}{#3}$}
  {}
}
\newcommand{\Var}[1][\Gamma]{\VarFull{#1}{x}{A}} 

\newcommand{\ProdFull}[7]{
  \BAX{Prod}
  {$\tgen{#1}{#2}{#3}$}
  {$\tgen{#1, #4 : #2}{#5}{#6}$}
  {$\tgen{#1}{\Pi #4 : #2.#5}{#7}$}
  {$(#3, #6, #7) `: \mathcal{R} `^{} #4 `; #1$}
}
\newcommand{\Prod}[1][\Gamma]{\ProdFull{#1}{T}{s_1}{x}{U}{s_2}{s_3}}

\newcommand{\AbsFull}[6]{
  \BAX{Abs}
  {$\tgen{#1}{\Pi #2 : #3.#4}{#5}$}
  {$\tgen{#1, #2 : #3}{#6}{#4}$}
  {$\tgen{#1}{\lambda #2 : #3. #6}{\Pi #2 : #3.#4}$}
  {$#2 `; #1$}
}

\newcommand{\Abs}[1][\Gamma]{\AbsFull{#1}{x}{T}{U}{s}{M}}

 \newcommand{\AppFull}[6]{
  \BAX{App}
  {$\tgen{#1}{#2}{\Pi #3 : #4. #5}$}
  {$\tgen{#1}{#6}{#4}$}
  {$\tgen{#1}{(#2 #6)}{#5 [ #6 / #3 ]}$}
  {$$}
}

\newcommand{\App}[1][\Gamma]{\AppFull{#1}{f}{x}{V}{W}{u}}

\newcommand{\SigmaRFull}[5]{
  \BAX{Sigma}
  {$\tgen{#1}{#2}{#3}$}
  {$\tgen{#1, #4 : #2}{#5}{#3}$}
  {$\tgen{#1}{\Sigma #4 : #2.#5}{#3}$}
  {$#3 `: \{ \Prop, \Set \} `^{} #4 `; #1$} 
}
\newcommand{\SigmaR}[1][\Gamma]{\SigmaRFull{#1}{T}{s}{x}{U}}

\newcommand{\SumDepFull}[7][\Gamma]{
  \TAX{SumDep}
  {$\tgen{#1}{\Sigma #2 : #5. #6}{#7}$}
  {$\tgen{#1}{#3}{#5}$}
  {$\tgen{#1}{#4}{#6[#3 / #2]}$}
  {$\tgen{#1}{\pair{\Sigma #2 : #5.#6}{#3}{#4}}{\Sigma #2 : #5.#6}$}
  {}
}

\newcommand{\SumDep}[1][\Gamma]{\SumDepFull[#1]{x}{t}{u}{T}{U}{s}}
 
\newcommand{\PiLeftFull}[5][\Gamma]{
  \UAX{PiLeft}
  {$\tgen{#1}{#2}{`S #3 : #4.#5}$}
  {$\tgen{#1}{\pi_1~#2}{#4}$}
  {}
}
\newcommand{\PiLeft}[1][\Gamma]{\PiLeftFull[#1]{t}{x}{T}{U}}

\newcommand{\PiRightFull}[5][\Gamma]{
  \UAX{PiRight}
  {$\tgen{#1}{#2}{`S #3 : #4.#5}$}
  {$\tgen{#1}{\pi_2~#2}{#5[\pi_1~#2/#3]}$}
  {}
}
\newcommand{\PiRight}[1][\Gamma]{\PiRightFull[#1]{t}{x}{T}{U}}


\newcommand{\SubsetFull}[4]{
  \BAX{Subset}
  {$\tgen{#1}{#3}{\Set}$}
  {$\tgen{#1, #2 : #3}{#4}{\Prop}$}
  {$\tgen{#1}{\mysubset{#2}{#3}{#4}}{\Set}$}
  {$#2 `; #1$}
}
\newcommand{\SubsetR}[1][\Gamma]{\SubsetFull{#1}{x}{U}{P}}


\newcommand{\SubsumFull}[5]{
  \TAX{Subsum}
  {$\tgen{#1}{#4}{#5}$}
  {$\tgen{#1}{#2}{#3}$}
  {$#5 \sub #2$} % #1 \subt 
  {$\tgen{#1}{#4}{#2}$}
  {}
}
\newcommand{\Subsum}[1][\Gamma]{\SubsumFull{#1}{T}{s}{t}{U}} 

\def\Coerce{\Subsum}

\newcommand{\ConvFull}[5]{
  \TAX{Conv}
  {$\tgen{#1}{#2}{#3}$}
  {$\tgen{#1}{#4}{#5}$}
  {$#5 \eqbr #2$}
  {$\tgen{#1}{#4}{#2}$}
  {}
}
\newcommand{\Conv}[1][\Gamma]{\ConvFull{#1}{T}{s}{t}{U}}

\newcommand{\typedRules}{
  \begin{center}
    \def\seq{\typed}
    \def\fCenter{\wf}
    \WfEmpty \DP\quad
    \WfVar \DP
    
    \def\fCenter{\typed}
    \vspace{\infvspace}
    \PropSet\DP

    \vspace{\infvspace}
    \Var\DP
    
    \vspace{\infvspace}
    \Prod\DP
    
    \vspace{\infvspace}
    \Abs\DP

    \vspace{\infvspace}
    \App\DP

    \vspace{\infvspace}
    \SigmaR\DP

    \vspace{\infvspace}
    \SumDep\DP
    
    \vspace{\infvspace}
    \PiLeft\DP
    \quad
    \PiRight\DP

    \vspace{\infvspace}
    \SubsetR\DP

    \vspace{\infvspace}
    \Subsum\DP
      
  \end{center}
}

\def\typedFig
{
\begin{figure}[tb]
  \typedRules
  \caption{Calcul de coercion par pr�dicats - version d�clarative}
  \label{fig:typing-decl-rules}
\end{figure}
}

%%% Local Variables: 
%%% mode: latex
%%% TeX-master: "subset-typing"
%%% LaTeX-command: "TEXINPUTS=\"style:$TEXINPUTS\" latex"
%%% End: 

\def\SubConv{
  \UAX{SubConv}
  {$T \eqbi U$}
  {$T \sub U$}
  {} 
}

\def\SubRefl{
  \UAX{SubRefl}
  {}
  {$S \seq S$}
  {}
}

\def\SubTrans{
\BAX{SubTrans}
{$S \seq T$}
{$T \seq U$}
{$S \seq U$}
{}
} 

\def\SubProd{
\BAX{SubProd}
{$U \seq T$} %"<|-|>"
{$V \seq W$}
{$\Pi x : T.V \seq \Pi x : U.W$}
{}
}

\def\SubSigma{
  \BAX{SubSigma}
  {$T \seq U$}
  {$V \seq W$}
  {$\Sigma x : T. V \seq \Sigma y : U. W$}
  {}
}

\def\SubProof{
  \UAX{SubProof}
  {$U \seq V$}
  {$U \seq \subset{x}{V}{P}$}
  {}
}

\def\SubSub{
  \UAX{SubSub}
  {$U \seq V$}
  {$\subset{x}{U}{P} \seq V$}
  {}
}

\def\subtdFig{
\begin{figure}[ht]
  \begin{center}
    \def\fCenter{\subd}
    
    \vspace{\infvspace}
    \SubConv\DP

    \vspace{\infvspace}
    \SubTrans\DP

    \vspace{\infvspace}
    \SubProd\DP

    \vspace{\infvspace}
    \SubSigma\DP
    
    \vspace{\infvspace}
    \SubProof\DP
    
    \vspace{\infvspace}
    \SubSub\DP
    
  \end{center}
  \caption{Coercion par pr�dicats - version d�clarative}
  \label{fig:subtyping-decl-rules}
\end{figure}
}

\def\subtdShort{
\begin{figure}[ht]
  \begin{center}
    \def\fCenter{\subd}
    \SubConv\DP
    \SubProof\DP
    \SubSub\DP

    \vspace{1cm}
    \SubProd\DP
    \SubSigma\DP
    %\SubTrans\DP
  \end{center}
  \caption{Coercion par pr�dicats - version d�clarative}
  \label{fig:subtyping-decl-rules-short}
\end{figure}
}

%%% Local Variables: 
%%% mode: latex
%%% TeX-master: "subset-typing"
%%% LaTeX-command: "TEXINPUTS=\"style:$TEXINPUTS\" latex"
%%% End: 



\def\AppA{
\TAX{App}
{$\talgo{`G}{f}{T} \quad \mualgo(T) = \Pi x : V. W : s$}
{$\talgo{`G}{u}{U} \quad \talgo{`G}{U, V}{s'}$}
{$\subalgo{`G}{U}{V} $}
{$\talgo{`G}{(f u)}{W [ u / x ]}$}
{}
}

\def\LetSumA{
  \TAX{LetSum}
  {$`G \seq t : S$}
  {$\mualgo(S) = `S x : T. U $}
  {$`G, x : T, u : U \seq v : V $}
  {$`G \seq \letml~(x, u) = t~\inml~v : V$}
  {$x,y `; `G$}
}

\def\PiLeftA{
  \BAX{PiLeft}
  {$\tgen{`G}{t}{S}$}
  {$\mualgo(S) = `S x : T.U$}
  {$\tgen{`G}{\pi_1~t}{T}$}
  {}
}

\def\PiRightA{
  \BAX{PiRight}
  {$\tgen{`G}{t}{S}$}
  {$\mualgo(S) = `S x : T.U$}
  {$\tgen{`G}{\pi_2~t}{U[\pi_1~t/x]}$}
  {}
}


\def\SumInfA{
  \TAX{SumInf}
  {$`G \seq t : T $}
  {$`G \seq u : U $}
  {$`G \seq \Sigma \_ : T.U : s$}
  {$`G \seq (t, u) : \Sigma \_ : T.U$}
  {}
}

\def\SumDepAold{
  \QAX{SumDep}
  {$`G \seq t : T$}
  {$`G \seq u : U'$}
  {$`G \seq \Sigma x : T.U : s$}
  {$`G \seq U' : s \quad U' \suba U[t/x]$}
  {$`G \seq (x \coloneqq~t, u : U) : \Sigma x : T.U$}
  {}
}

\def\SumDepA{
  \QAX{SumDep}
  {$`G \seq t : T$}
  {$`G \seq u : U'$}
  {$`G \seq \Sigma x : T.U : s$}
  {$`G \seq U' : s \quad U' \suba U[t/x]$}
  {$`G \seq \pair{\Sigma x : T.U}{t}{u} : \Sigma x : T.U$}
  {}
}

\def\typeaFig{
\begin{figure}[t]
  \begin{center}
    \def\fCenter{\wf}
    \def\type{\typea}
    \def\subt{\subta}
    \def\sub{\suba}
    
    \WfEmpty\DP
    \quad
    \WfVar\DP
    
    \def\fCenter{\typea}
    \vspace{\infvspace}
    \PropSet\DP
%    \quad
%    \TypeType\DP
    
    \vspace{\infvspace}
    \Var\DP
    
    \vspace{\infvspace}
    \Prod\DP
    
    \vspace{\infvspace}
    \Abs\DP

    \vspace{\infvspace}
    \AppA\DP

%    \vspace{\infvspace}
%    \LetIn\DP
    
    \vspace{\infvspace}
    \SigmaR\DP

%    \vspace{\infvspace}
%    \SumInfA\DP

    \vspace{\infvspace}
    \SumDepA\DP
    
    \vspace{\infvspace}
    \PiLeftA\DP
    \quad
    \PiRightA\DP
    %\LetSumA\DP
    
    \vspace{\infvspace}
    \Subset\DP
  \end{center}
  \caption{Calcul de coercion par pr�dicats - version algorithmique}
  \label{fig:typing-algo-rules}
\end{figure}
}

\def\typemuaFig{
  \begin{figure}[ht]
    \begin{eqnarray*}
      \mualgo'(\mysubset{x}{U}{P}) & "=>" & \mualgo'(\downarrow{U}) \\
      \mualgo'(x)                & "=>" & x \\
      \\
      \mualgo(x) & "=>" & \mualgo' (\downarrow{x})
    \end{eqnarray*}
    \caption{D�finition de $\mualgo$}
    \label{fig:mualgo-definition}
\end{figure}
}

%%% Local Variables: 
%%% mode: latex
%%% TeX-master: "subset-typing"
%%% LaTeX-command: "TEXINPUTS=\"style:$TEXINPUTS\" latex"
%%% End: 

%% \def\SubProd{
%%   \BAX{Sub-Prod}
%%   {$`G \seq U \sub T$} %"<|-|>"
%%   {$`G, x : U \seq V \sub W$}
%%   {$`G \seq \Pi x : T.V \sub \Pi x : U.W$}
%%   {}
%% }

%% \def\SubSigma{
%%   \BAX{Sub-Sigma}
%%   {$`G \seq T \sub U$}
%%   {$`G \seq V \sub W$}
%%   {$`G \seq \Sigma x : T. V \sub \Sigma x : U. W$}
%%   {}
%% }

\def\SubHnf{
  \UAX{SubHnf}
  {$\hnf{T}~\sub \hnf{U}$}
  {$T \sub U$}
  {} 
}

\def\SubLeft{
  \BAX{Sub-Left}
  {$`G \seq U \sub V$}
  {$`G \type P : \Pi x : U. \Prop$}
  {$`G \seq \subset{x}{U}{P} \sub V$}
  {}
}
    
\def\SubRight{
  \UAX{Sub-Right}
  {$`G \seq T \sub U$}
                                %{$`G \judgetype h : P~p$}
  {$`G \seq T \sub \subset{x}{U}{P}$}
  {}
}

\def\subtaFig{
\begin{figure}[ht]
  \begin{center}
    \def\fCenter{\suba}
    \def\type{\typea}
    \def\sub{\suba}

    \SubConv\DP
    \quad
    \SubHnf\DP
    \vspace{\infvspace}

    \SubProd\DP

    \vspace{\infvspace}
    \SubSigma\DP
    
    \vspace{\infvspace}
    \SubProof\DP

    \vspace{\infvspace}    
    \SubSub\DP
  \end{center}
  \caption{Coercion par pr�dicats - version algorithmique}
  \label{fig:subtyping-algo-rules}
\end{figure}
}

%%% Local Variables: 
%%% mode: latex
%%% TeX-master: "subset-typing"
%%% LaTeX-command: "TEXINPUTS=\"style:$TEXINPUTS\" latex"
%%% End: 


\def\typec{\vdash_{CCI}}

\newcommand{\timpl}[6]{#1 \typea #2 : #3 "~>" #4 \typec #5 : #6}
\newcommand{\subimpl}[4]{#1 \typec #2 : #3 \sub #4}

\def\muterm{\mu_{\mbox{term}}}

\def\WfAtomI{
  \UAX{Wf-Atom}
  {}
  {$\wf [] "~>"~\wf []$}
  {}
}  
  
\def\WfVarI{
  \UAX{Wf-Var}
  {$\timpl{`G}{A}{s}{`G'}{A'}{s}$}
  {$\wf `G, x : A "~>"~\wf `G', x : A'$}
  {$s `: \{ \Set, \Prop, \Type(i) \}$}
}

\def\PropSetI{
  \UAX{PropSet}
  {$\wf `G "~>" ~\wf `G'$}
  {$\timpl{`G}{s}{\Type(0)}{`G'}{s}{\Type(0)}$}
  {$s `: \{ \Prop, \Set \}$} 
}

\def\TypeTypeI{
  \UAX{Type}
  {$\wf `G "~>" \wf `G$}
  {$\timpl{`G}{\Type(i)}{\Type(i + 1)}{`G'}{\Type(i)}{\Type(i + 1)}$}
  {}
}

\def\VarI{
  \BAX{Var}
  {$\wf `G "~>"~\wf `G'$}
  {$x : A `: `G "~>" x : A' `: `G'$}
  {$\timpl{`G}{x}{A}{`G'}{x}{A'}$}
  {}
}

\def\ProdI{
  \BAX{Prod}
  {$\timpl{`G}{T}{s1}{`G'}{T'}{s1}$}
  {$\timpl{`G, x : T}{U}{s2}{`G', x : T'}{U'}{s2}$}
  {$\timpl{`G}{\Pi x : T.U}{s2}{`G'}{\Pi x : T'.U'}{s2}$}
  {$(s1, s2) `: \text{{\sc Sort}}$}
}

\def\AbsI{
  \BAX{Abs}
  {$\timpl{`G}{\Pi x : T. U}{s}{`G'}{\Pi x : T'. U'}{s}$}
  {$\timpl{`G, x : T}{M}{U}{`G', x : T'}{M'}{U'}$}
  {$\timpl{`G}{\lambda x : T. M}{\Pi x : T.U}
    {`G'}{\lambda x : T'. M'}{\Pi x : T'.U'}$}
  {}
}

\def\AppI{
  \QAX{App}
  {$\timpl{`G}{f}{T}{`G'}{f'}{T'}$}
  {$\mu~T' `= (\pi, \Pi x : V'. W')$}
  {$\timpl{`G}{u}{U}{`G'}{u'}{U'}$}
  {$\subimpl{`G'}{c}{U'}{V'}$}
  {$\timpl{`G}{f u}{W[u/x]}{`G'}{(\pi~f')~(c~u')}{W'[ c~u' / x ]}$}
  {}
}

\def\LetInI{
  \BAX{LetIn}
  {$\timpl{`G}{t}{T}{`G'}{t'}{T'}$}
  {$\timpl{`G, x : T}{v}{V}{`G', x : T'}{v'}{V'}$}
  {$\timpl{`G}{\letml x = t \inml v}{V[t / x]}
    {`G'}{\letml x = t' \inml v'}{V'[t' / x]}$}
  {}
}

\def\SigmaRI{
  \BAX{Sigma}
  {$\timpl{`G}{T}{s1}{`G'}{T'}{s1}$}
  {$\timpl{`G, x : T}{U}{s2}{`G', x : T'}{U'}{s2}$}
  {$\timpl{`G}{\Sigma x : T.U}{s2}{`G'}{\Sigma x : T'.U'}{s2} $}
  {$(s1, s2) `: \matht{Sort}$}
}


\def\SumInfI{
  \TAXWide{SumInf}
  {$\timpl{`G}{t}{T}{`G'}{t'}{T'}$}
  {$\timpl{`G}{u}{U}{`G'}{u'}{U'}$}
  {$\timpl{`G}{\Sigma \_ : T.U}{s}{`G'}{\Sigma \_ : T'.U'}{s}$}
  {$\timpl{`G}{(t, u)}{\Sigma \_ : T.U}{`G'}{(t', u')}{\Sigma \_ : T'.U'}$}
  {}
}

\def\SumDepI{
  \TAXWide{SumDep}
  {$\timpl{`G}{t}{T}{`G'}{t'}{T'}$}
  {$\timpl{`G}{u}{U[t/x]}{`G'}{u'}{U'[t'/x]}$}
  {$\timpl{`G}{\Sigma x : T.U}{s}{`G'}{\Sigma x : T'.U'}{s}$}
  {$\timpl{`G}{(t, u : U)}{\Sigma x : T.U}{`G'}{(t', u')}{\Sigma x : T'.U'}$}
  {}
}
 
\def\LetSumI{
  \BAX{LetSum}
  {$\timpl{`G}{t}{\Sigma x : T. U}{`G'}{t'}{\Sigma x : T'.U'}$}
  {$\timpl{`G, x : T, u : U}{v}{V}{`G, x : T', u : U'}{v'}{V'}$}
  {$\timpl{`G}{\letml (x, u) = t \inml v}{V}
    {`G'}{\letml (x, u) = t' \inml v'}{V'}$}
  {}
}

\def\SubsetI{
  \BAX{Subset}
  {$\timpl{`G}{U}{\Type}{`G'}{U'}{\Type}$}
  {$\timpl{`G, x : U}{P}{\Prop}{`G', x : U'}{P'}{\Prop} $}
  {$\timpl{`G}{\subset{x}{U}{P}}{\Type}{`G'}{\subset{x}{U'}{P'}}{\Type}$}
  {$$}
}

\def\marginleft{0em}

\def\typeiFig{
\begin{figure}[ht]
  \begin{center}
    \def\fCenter{\wf}
    \def\type{\typec}
    
    \WfAtomI\DP

    \vspace{\infvspace}
    \WfVarI\DP
    
    \vspace{\infvspace}
    \PropSetI\DP
    
    \vspace{\infvspace}
    \VarI\DP
    
    \vspace{\infvspace}
    \ProdI\DP
    
    \vspace{\infvspace}
    \AbsI\DP
    
    \vspace{\infvspace}
    \AppI\DP

    \vspace{\infvspace}
    \LetInI\DP
    
    \vspace{\infvspace}
    \SigmaRI\DP

    \vspace{\infvspace}
    \SumInfI\DP

    \vspace{\infvspace}
    \SumDepI\DP
    
    \vspace{\infvspace}
    \LetSumI\DP
    
    \vspace{\infvspace}
    \SubsetI\DP
  \end{center}
  \label{typing-impl-rules}
  \caption{R�ecriture du typage vers \Coq}
\end{figure}
}

\def\typemuiFig{
\begin{figure}
  \begin{eqnarray*}
    (f, \subset{x}{U}{P}))   & "=>" & \letml (f, t) = \muterm~(f, U) \inml (f `o
    \pi_1, t) \\
    x                        & "=>"  & x
  \end{eqnarray*}
  \label{muimpl-definition}
  \caption{D�finition de $\muterm$}
\end{figure}
}

%%% Local Variables: 
%%% mode: latex
%%% TeX-master: "subset-typing"
%%% LaTeX-command: "TEXINPUTS=\"style:$TEXINPUTS\" latex"
%%% End: 

\def\ctxdot{\text{\textbullet}}

\newcommand{\SubConvI}[1][\Gamma]{
\UAX{SubConv}
{$T \eqbr U$}
{$\subimpl{#1}{\ctxdot}{T}{U}$}
{$T = \hnf{T} `^ T "/=" \Pi, \Sigma, \{|\} `^ U = \hnf{U}$}
}

\newcommand{\SubConvIs}[1][\Gamma]{
\UAX{SubConv}
{$T \eqbr U$}
{$\subimpl{#1}{\ctxdot}{T}{U}$}
{} 
}

\newcommand{\SubHnfI}[1][\Gamma]{
\UAX{SubHnf}
{$\subimpl{#1}{c}{\hnf{T}}{\hnf{U}}$}
{$\subimpl{#1}{c}{T}{U}$}
{$T "/=" \hnf{T} `V U "/=" \hnf{U}$} 
}

\newcommand{\SubProdI}[1][\Gamma]{%
  \BAX{SubProd}
  {$\subimpl{#1}{c_1}{U}{T} : s$}
  {$\subimpl{#1, x : U}{c_2}{V}{W}$}  
  {$\subimpl{#1}{\lambda x : \ip{U}{#1}. c_2[\ctxdot~(c_1[x])]}
    {\Pi x : T.V}{\Pi x : U.W}$}
  {}
}

\newcommand{\SubSigmaI}[1][\Gamma]{%
\BAX{SubSigma}
{$\subimpl{#1}{c_1}{T}{U}$}
{$\subimpl{#1, x : T}{c_2}{V}{W}$}
{$\subimpl{#1}{\pair{\ip{\Sigma x : U.W}{`G}}{c_1[\pi_1~\ctxdot]}{c_2[\pi_2~\ctxdot][\pi_1~\ctxdot/x]}}
  {\Sigma x : T. V}{\Sigma x : U. W}$}
{}
}

\newcommand{\SubSubI}[1][\Gamma]{%
\UAX{SubSub}
{$\subimpl{#1}{c}{U}{T}$}
{$\subimpl{#1}{c[\eltpit~\ctxdot]}{\mysubset{x}{U}{P}}{T}$}
{$T = \hnf{T} `^ \hnf{T} "/=" \{|\}$}
}

\newcommand{\SubSubIs}[1][\Gamma]{%
\UAX{SubSub}
{$\subimpl{#1}{c}{U}{T}$}
{$\subimpl{#1}{c[\eltpit~\ctxdot]}{\mysubset{x}{U}{P}}{T}$}
{}
}
   
\newcommand{\SubProofI}[1][\Gamma]{%
  \BAX{SubProof}
  {$\subimpl{#1}{c}{T}{U}$}
  {$\timpl{#1}{\mysubset{x}{U}{P}}{\Set}$}
  {$\subimpl{#1}
    {\elt{\ip{U}{#1}}{\ip{\lambda x : U.P}{#1}}{c}
      {\ex{\ip{P}{#1, x : U}[c/x]}}}
  {T}{\mysubset{x}{U}{P}}$}
{$T = \hnf{T}$}
}

\def\subtiFig{
\begin{figure}[ht]
  \begin{center}
    \def\fCenter{\subtd}

    \vspace{\infvspace}
    \SubConvI\DP
    
    \vspace{\infvspace}
    \SubHnfI\DP

    \vspace{\infvspace}
    \SubProdI\DP

    \vspace{\infvspace}
    \SubSigmaI\DP
    
    \vspace{\infvspace}
    \SubSubI\DP
    
    \vspace{\infvspace}
    \SubProofI\DP
    
  \end{center}
  \caption{R��criture de la coercion vers \CCI}
  \label{fig:coerce-impl-rules}
\end{figure}
}

\def\subtisc{
  \begin{center}
    \def\fCenter{\subti}
    \SubConvIs\DP
    \quad
    \SubHnfI\DP

    \vspace{\infvspace}
    \SubProdI\DP
    
    \vspace{\infvspace}
    \SubSigmaI\DP

    \vspace{\infvspace}
    \SubSubIs\DP

    \vspace{\infvspace}
    \SubProofI\DP
  \end{center}}

\def\subtis{
\begin{figure}[ht]
  \subtisc
  \caption{R��criture de la coercion vers \CCI}
  \label{fig:coerce-impl-rules-short}
\end{figure}
}

%%% Local Variables: 
%%% mode: latex
%%% TeX-master: "subset-typing"
%%% LaTeX-command: "TEXINPUTS=\"style:.:\" latex"
%%% End: 


\newqsymbol{"="}{\triangleq}

% [Short Paper Title] (optional, use only with long paper titles)
\title{\Program{}ing in \Coq}

\author[Matthieu Sozeau]
{{\sc Matthieu Sozeau} \\ under the direction of {\sc Christine Paulin-Mohring}}


\def\LRI{\name{LRI}}
\def\INRIAFuturs{\name{INRIA Futurs}}
\def\Proval{\name{ProVal}}
\def\Logical{\name{LogiCal}}

\institute[]
{
  \LRI{}, Univ. Paris-Sud - \Demons{} Team \& \INRIAFuturs{} - \Proval{} Project
}

\date % (optional, should be abbreviation of conference name)
{\Demons-\Proval-\Logical seminar \\
9 march 2007}

\subject{Theoretical Computer Science}

\pgfdeclareimage[height=0.5cm]{ups-logo}{../figures/ups-logo}
\pgfdeclareimage[height=0.5cm]{lri-logo}{../figures/lri-logo}
\pgfdeclareimage[height=0.5cm]{inria-logo}{../figures/inria-logo}

\def\imgheight{7.5cm}

\pgfdeclareimage[height=\imgheight]{term}{figures/bigpicterm}
\pgfdeclareimage[height=\imgheight]{simple}{figures/bigpicsimple}
\pgfdeclareimage[height=\imgheight]{simplecoq}{figures/bigpicsimplecoq}
\pgfdeclareimage[height=\imgheight]{depone}{figures/bigpicdepone}
\pgfdeclareimage[height=\imgheight]{deponeobl}{figures/bigpicdeponeobl}
\pgfdeclareimage[height=\imgheight]{deponecoq}{figures/bigpicdeponecoq}
\pgfdeclareimage[height=\imgheight]{deponeextr}{figures/bigpicdeponeextr}
% \pgfdeclareimage[height=\imgheight]{dep}{figures/bigpicdep}
% \pgfdeclareimage[height=\imgheight]{depobl}{figures/bigpicobl}
% \pgfdeclareimage[height=\imgheight]{depoblterm}{figures/bigpicoblterm}
\pgfdeclareimage[height=\imgheight]{depextr}{figures/bigpicextr}

\pgfdeclareimage[height=3cm]{lambda-cube}{../figures/lambda-cube}

\pgfdeclareimage[interpolate=true,height=8cm]{subtac-syntax}{capturesyntax}
\pgfdeclareimage[interpolate=true,height=8cm]{myhd-proof}{capturemyhd-proof}
\pgfdeclareimage[interpolate=true,height=8cm]{myhd-extract}{capturemyhd-extract}
\pgfdeclareimage[interpolate=true,height=8cm]{mytail-extract}{capturemytail-extract}
\pgfdeclareimage[interpolate=true,height=8cm]{mytest-good}{capturemytest-ok}
\pgfdeclareimage[interpolate=true,height=8cm]{mytest-good-extract}{capturemytest-ok-extract}
\pgfdeclareimage[interpolate=true,height=8cm]{mytest-bad}{capturemytest-bad}

\pgfdeclareimage[interpolate=true,height=8cm]{append-proof}{captureappend-proof}
\pgfdeclareimage[interpolate=true,height=8cm]{append-extract}{captureappend-extract}
\pgfdeclareimage[interpolate=true,height=8cm]{append-app}{captureappend-app}

\def\typea{\vdash}
\def\suba{\rightslice}

\newcommand{\indfig}{
  \begin{center}
    \AXC{$`G, {(\tele{x_i : X_i `~ y_i : Y_i})}_{0}^{j-1} \vdash ?_j : x_j :
      X_j `~ y_j : Y_j \quad `A j `: [0..i]$}
    \UIC{$`G \vdash c(\tele{?_i}) : I~\tele{x_i} \suba I~\tele{y_i} : s$}
    \DP    
  \end{center}
}   

\def\coqeuclid{
\coqdocindent{0.50em}
\coqdockw{Inductive} \coqdocid{diveucl} \coqdocid{a} \coqdocid{b} : \coqdockw{Set} :=\coqdoceol
\coqdocindent{1.50em}
\coqdocid{divex} : \alert{\ensuremath{\forall} \coqdocid{q} \coqdocid{r},
\coqdocid{b} > \coqdocid{r} \ensuremath{\to} \coqdocid{a} =
\coqdocid{q} \ensuremath{\times} \coqdocid{b} + \coqdocid{r}}
\ensuremath{\to} \coqdocid{diveucl} \coqdocid{a}
\coqdocid{b}.\coqdoceol

\medskip
\coqdocindent{0.50em}
\coqdockw{Lemma} \coqdocid{eucl\_dev} : \ensuremath{\forall} \coqdocid{n}, \coqdocid{n} > 0 \ensuremath{\to} \ensuremath{\forall} \coqdocid{m}:\coqdocid{nat}, \coqdocid{diveucl} \coqdocid{m} \coqdocid{n}.\coqdoceol
\noindent
\coqdocindent{0.50em}
\coqdockw{Proof}.\coqdoceol
\noindent
\coqdocindent{1.50em}
\coqdocid{intros} \coqdocid{b} \coqdocid{H} \coqdocid{a}; \coqdocid{pattern} \coqdocid{a} \coqdockw{in} \ensuremath{\vdash} \ensuremath{\times}; \coqdocid{apply} \coqdocid{gt\_wf\_rec}; \coqdocid{intros} \coqdocid{n} \coqdocid{H0}.\coqdoceol
\noindent
\coqdocindent{1.50em}
\coqdocid{elim} (\coqdocid{le\_gt\_dec} \coqdocid{b} \coqdocid{n}).\coqdoceol
\noindent
\coqdocindent{1.50em}
\coqdocid{intro} \coqdocid{lebn}.\coqdoceol
\noindent
\coqdocindent{1.50em}
\coqdocid{elim} (\coqdocid{H0} (\coqdocid{n} - \coqdocid{b})); \coqdocid{auto} \coqdockw{with} \coqdocid{arith}.\coqdoceol
\noindent
\coqdocindent{1.50em}
\coqdocid{intros} \coqdocid{q} \coqdocid{r} \coqdocid{g} \coqdocid{e}.\coqdoceol
\noindent
\coqdocindent{1.50em}
\coqdocid{apply} \coqdocid{divex} \coqdockw{with} (\coqdocid{S} \coqdocid{q}) \coqdocid{r}; \coqdocid{simpl} \coqdockw{in} \ensuremath{\vdash} \ensuremath{\times}; \coqdocid{auto} \coqdockw{with} \coqdocid{arith}.\coqdoceol
\noindent
\coqdocindent{1.50em}
\coqdocid{elim} \coqdocid{plus\_assoc}.\coqdoceol
\noindent
\coqdocindent{1.50em}
\coqdocid{elim} \coqdocid{e}; \coqdocid{auto} \coqdockw{with} \coqdocid{arith}.\coqdoceol
\noindent
\coqdocindent{1.50em}
\coqdocid{intros} \coqdocid{gtbn}.\coqdoceol
\noindent
\coqdocindent{1.50em}
\coqdocid{apply} \coqdocid{divex} \coqdockw{with} 0 \coqdocid{n}; \coqdocid{simpl} \coqdockw{in} \ensuremath{\vdash} \ensuremath{\times}; \coqdocid{auto} \coqdockw{with} \coqdocid{arith}.\coqdoceol
\noindent
\coqdocindent{0.50em}
\coqdockw{Qed}.\coqdoceol
}

\begin{document}

\begin{frame}[plain]
  \titlepage
  \vfill\hfill
  \pgfuseimage{ups-logo}
  ~
  \pgfuseimage{lri-logo}
  ~
  \pgfuseimage{inria-logo}
\end{frame}

\begin{frame}[plain]
  \frametitle{The Big Picture}
  \only<1>{\pgfuseimage{term}}%
  \only<2>{\pgfuseimage{simple}}%
  \only<3>{\pgfuseimage{simplecoq}}\only<4>{\pgfuseimage{depone}}\only<5>{\pgfuseimage{deponeobl}}\only<6>{\pgfuseimage{deponecoq}}\only<7>{\pgfuseimage{deponeextr}}%
  \only<8>{\coqeuclid{}}%
  \only<9>{\pgfuseimage{depextr}}
\end{frame}

\frame{\tableofcontents}

\section{The idea}

\subsection{A simple idea}
\begin{frame}
  \frametitle{A simple idea}
  
  \begin{block}{Definition}
    $\{ x : T `| P \}$ is the set of objects of set $T$ verifying property $P$. 
  \end{block}
  
  \begin{itemize}
  \item Useful for specifying, widely used in mathematics ;    
  \item Links object and property.
  \end{itemize}
  \pause
  \begin{block}{Adapting the idea} 
    \begin{center}
      \BAX{}{$t : T$}
      {$\uncover<3->{\alert{p} : }P[t/x]$}
      {$\uncover<3->{(}t\uncover<3->{, \alert{p})} : \mysubset{x}{T}{P}$}
      {}\DP\quad      
      \UAX{}{$t : \mysubset{x}{T}{P}$}
      {$\uncover<3->{\alert{\texttt{proj}}~}t : T$}
      {}
      \DP
    \end{center}
  \end{block}  
\end{frame}

\subsection{From \PVS to \Coq}
\begin{frame}
  \frametitle{From ``\ps''\ldots}
  
  \begin{block}{\PVS{}}
    \vspace{0.7em}
    \begin{itemize}
    \item Specialized typing algorithm for subset types, generating
      \emph{Type-checking conditions}.    
      \begin{tabular}{lcll}
        $t : \mysubset{x}{T}{P}$ & used as & $t : T$
        & ok \\
        $t : T$ & used as & $t : \mysubset{x}{T}{P}$ 
        & if $P[t/x]$
      \end{tabular}
      \pause
      
    \item[{\bf +}] Practical success ; \pause
    \item[{\bf --}] No strong safety guarantee in \PVS{}.
    \end{itemize}  
  \end{block}
%   \pause 
%   \begin{block}{\Coq{}}
%     Use coercions to explicit the equivalence:
%     \begin{tabular}{lclcl}
%       $t : \mysubset{x}{T}{P}$ & \only<4>{$"<=>"$} \only<5->{$"->"$} & $t : T$
%       & \only<5->{$"~>"$} & \only<5->{$\pi_1~t$ \\
%         $t : T$ & $"->"$ & $t : \mysubset{x}{T}{P}$ 
%         & $"~>"$ & $\sref{elt}~{t}~(\alert<3->{? : P[t/x]})$}
%     \end{tabular}
%   \end{block}
\end{frame}

\begin{frame}[t]
  \frametitle{\ldots to Subset coercions}
  
  \begin{enumerate}
  \item<1-> A property-irrelevant language (\lng{}) with \alert{decidable} typing ;
  \item<2-> A total traduction to \Coq{} terms with holes ;
  \item<3-> A mechanism to turn the holes into proof obligations and
    manage them.
  \end{enumerate}
  
  \begin{center}
    \UAX{}{$`G \typea t : \mysubset{x}{T}{P}$}
    {$`G \typea \uncover<2->{\texttt{proj}~}t : T$}
    {}\DP

    \vspace{0.5cm}
    \TAX{}{$`G \typea t : T$}
    {$`G, x : T \type P : \Prop$}
    {\uncover<3>{$`G \type p : P[t/x]$}}
    {$`G \type \uncover<2->{(}t\uncover<2->{, }\uncover<2->{{\only<2>{?}\only<3>{p}})} : \mysubset{x}{T}{P}$}
    {\uncover<2>{$`G \type ? : P[t/x]$}}\DP
  \end{center}
  
%   \begin{center}
%   \begin{tabular}{lclcl}
%     $t : \mysubset{x}{T}{P}$ & \only<1>{$"<=>"$} \only<2->{$"->"$} & $t : T$
%     & \only<2->{$"~>"$} & \only<2->{$\pi_1~t$ \\
%       $t : T$ & $"->"$ & $t : \mysubset{x}{T}{P}$ 
%       & $"~>"$ & $\sref{elt}~{t}~(\alert<3->{? : P[t/x]})$}
%   \end{tabular}
% \end{center}

\end{frame}

\section{Theoretical development}
\frame<beamer>{\tableofcontents[currentsection]}

\subsection{\lng{}}

\begin{frame}
  \frametitle{\lng{} syntax}
  
  $\begin{array}{lcl}
    x & `: & \mathcal{V} \\
    \\
    s, t, u, v & \Coloneqq & x \\
    & | & \Set \\
    & | & \Prop \\
    & | & \Type \\
\pause
    & | & \lambda x : s. t \\
    & | & s~t \\
    & | & \Pi x : s. t \\
\pause
    & | & \pair{\Sigma x : s.t}{u}{v} \\
    & | & \pi_1~s `| \pi_2~s \\
    & | & \Sigma x : s. t \\
\pause
    & | & \mysubset{x}{s}{t} 
  \end{array}$  
\end{frame}

\begin{frame}[t,label=typingdecl]
  \frametitle{\lng{} typing $\typed$ and coercion $\subd$}
  
  Calculus of Constructions with
  \typenvd
  \setboolean{displayLabels}{false}
  \begin{center}
    \vspace{0.25cm}
    \only<1>{\Conv\DP}
    \only<2->{\Coerce\DP
      \def\fCenter{\subd}
      \SubConv\DP}
    \uncover<3->{
    \vspace{0.5cm}    
    \SubTrans\DP
    
    \vspace{0.25cm}
    \only<4->{\BAX{SubSub}
      {$`G \type U \sub V : \Set$}
      {$`G, x : U \type P : \Prop$}
      {$`G \type \mysubset{x}{U}{P} \sub V : \Set$}
      {}\DP
      \vspace{0.25cm}
      \BAX{SubProof}
      {$`G \type U \sub V : \Set$}
      {\alert{$`G, x : V \type P : \Prop$}}
      {$`G \type U \sub \mysubset{x}{V}{P} : \Set$}
      {}\DP}

    \vspace{0.25cm}
    \only<5-6>{
      \AXC{$`G \type 0 : \nat$}
      \AXC{$`G \type \nat~\sub \mysubset{x}{\nat}{x \neq 0} : \Set$}
      \BIC{$`G \type 0 : \mysubset{x}{\nat}{x \neq 0}$}
      \only<6>{\UIC{$`G \type \alert{? : 0 \neq 0}$}}
      \DP
    }
    \only<7->{\SubProd\DP}}
  % \SubSigma\DP}
  \end{center}

\end{frame}

\begin{frame}
  \frametitle{Results}

  \begin{theorem}[Decidability of type checking and type inference]
    $`G \type t : T$ is decidable. 
  \end{theorem}  
  
  \uncover<2->{
    \begin{lemma}[Elimination of transitivity]
      If $T \sub U `^ U \sub V$ then $T \sub V$.
    \end{lemma}

  \begin{theorem}[Subject Reduction]
    If $`G \type t : T$ and $t "->"_{\beta\pi} t'$ then $`G \type t' : T$.
  \end{theorem}}
  
  \begin{prooftree}
    \TAX{App}
    {$\tgen{`G}{f}{T} \quad \subgen{`G}{T}{\Pi x : A.B}{s}$}
    {$\tgen{`G}{e}{E}$}
    {$\subgen{`G}{E}{A}{s'}$}
    {$\tgen{`G}{(f~e)}{B [ e / x ]}$}
    {}
  \end{prooftree}
  
\end{frame}

\def\typec{\vdash_{?}}
\renewcommand{\subimpl}[4]{#1 \typec #2 : #3 \sub #4}

\newcommand{\blue}[1]{{\color{blue}#1}}

\subsection{Interpretation in \Coq{}}
\frame<beamer>{\tableofcontents[currentsubsection]}

\setboolean{displayLabels}{false}
\begin{frame}
  \frametitle{From \lng{} to \Coq}
  
  \begin{block}{The target system : \CIC{} with metavariables}
    \begin{center}
      \BAX{}{$`G \typec t : T$}
      {$`G \typec p : P[t/x]$}
      {$`G \typec \elt{T}{P}{t}{p} : \mysubset{x}{T}{P}$}
      {}\DP
      
      \vspace{0.25cm}
      \UAX{}{$`G \typec t : \mysubset{x}{T}{P}$}
      {$`G \typec \eltpit~t : T$}
      {}
      \DP
      \quad
      \UAX{}{$`G \typec t : \mysubset{x}{T}{P}$}
      {$`G \typec \eltpip~t : P[\eltpit~t/x]$}
      {}
      \DP

      \vspace{0.25cm}
      \UAX{}
      {$`G \typec P : \Prop$}
      {$`G \typec ?_{P} : P$}
      {}\DP
    \end{center}
  \end{block}

  We build an interpretation $\ip{\_}{`G}$ from \Russell to \CICq
  terms.
  \pause  
  \begin{block}{Our goal}
    \begin{center}
      If $`G \type t : T$ then $\ipG{`G} \typec \ip{t}{`G} : \ip{T}{`G}$.
    \end{center}
  \end{block}
\end{frame}

\begin{frame}[t]
  \frametitle{Deriving explicit coercions}
      
  \begin{block}{Interpretation of coercions}
    \begin{center}
      If $\subgen{`G}{T}{U}{s}$ then $`G \typec c[\ctxdot] : T \sub U$ which implies
      $\ipG{`G}, x : \ip{T}{`G} \typec \blue{c}[x] : \ip{U}{`G}$.
    \end{center}
  \end{block}
  
  \uncover<2->{
    \begin{definition}
      \begin{center}
        \def\fCenter{\sub}    
        \UAX{SubConv}
        {$T \eqbr U$}
        {$\subimpl{`G}{\uncover<3->{\blue{\ctxdot}}}{T}{U}$}
        {}\DP
        
        \vspace{0.25cm}
        \uncover<4->{$\subimpl{`G}{\uncover<5->{\blue{\eltpit~\ctxdot}}}{\mysubset{x}{T}{P}}{T}$}
        
        \vspace{0.25cm}
        \uncover<6->{
          {$\subimpl{`G}
            {\uncover<7->{\blue{\elt{\_}{\_}{\ctxdot}
                {\ex{\ip{P}{`G, x : T}[\ctxdot/x]}}}}}
            {T}{\mysubset{x}{T}{P}}$}}
      \end{center}
    \end{definition}
  
    \uncover<8->{
      \begin{example}      
        \begin{prooftree}
          \AXC{$`G \typec 0 : \nat$}
          \AXC{$`G \typec \elt{\_}{\_}{\ctxdot}{?_{(x \neq 0)[\ctxdot/x]}} : \nat \sub \mysubset{x}{\nat}{x \neq 0}$}
          \BIC{$`G \typec %(\elt{\_}{\_}{\ctxdot}{?_{(x \neq
                          %0)[\ctxdot/x]}})[0] =
            \elt{\_}{\_}{0}{?_{\alert{0 \neq 0}}} : \mysubset{x}{\nat}{x \neq 0}$}
        \end{prooftree}
      \end{example}}
  }
\end{frame}

\begin{frame}
  \frametitle{Interpretation of terms}
    
  \begin{example}[Application]
    \setboolean{displayLabels}{false}
    \typenva
    \begin{center}
      \TAX{App}
      {$\tgen{`G}{f}{T} \quad \subgen{`G}{T}{\Pi x : V. W}{s}$}
      {$\tgen{`G}{u}{U}$} % \quad \tgen{#1}{#9, #5}{#7'}$}
      {$\subgen{`G}{U}{V}{s'}$}
      {$\tgen{`G}{(f~u)}{W[u/x]}$}
      {}
      \DP
    \end{center}
    
    \typenvi
    \[\begin{array}{lcll}
      \ip{f~u}{`G} 
      & = & \letml~\pi = \coerce{`G}{T}{(\Pi x : V.W)}~\inml & \\
      & & \letml~c = \coerce{`G}{U}{V}~\inml & \\
      & & (\pi[\ip{f}{`G}])~(c[\ip{u}{`G}]) & \\
    \end{array}\]    
  \end{example}
  \begin{theorem}[Soundness]
    If $`G \typea t : T$ then $\ipG{`G} \typec \ip{t}{`G} : \ip{T}{`G}$.
  \end{theorem}
\end{frame}


\begin{frame}
  \frametitle{Theoretical matters \dots}
  
  $\typec$'s equational theory:
  
  \[\begin{array}{llcll}
    \shaded{(\beta)} & \shaded{(\lambda x : X.e)~v} & \shaded{`=} & \shaded{e[v/x]} & \\
    \shaded{(\pi_i)} & \shaded{\pi_i~\pair{T}{e_1}{e_2}} & \shaded{`=} & \shaded{e_i} & \\
    \shaded{(\sigma_i)} & \shaded{\sigma_i~(\elt{E}{P}{e_1}{e_2})} & \shaded{`=} & \shaded{e_i} & \\
    (\eta) & (\lambda x : X.e~x) & `= & e & \text{if $x `; FV(e)$} \\ % et $e : \Pi x : X.Y$} \\
    (\text{\SP}) % & \pair{\Sigma x : X.Y}{\pi_1~e}{\pi_2~e} & `= & e & \\ % \text{if $e : \Sigma x : X. Y$} \\
    & \elt{E}{P}{(\eltpit~e)}{(\eltpip~e)} & `= & e & \\%\text{if $e : \mysubset{x}{E}{P}$} \\
    \uncover<2->{\alert{(\sigma)} & \elt{E}{P}{t}{p} & `= & \elt{E}{P}{t'}{p'} & \text{if $t
        `= t'$}}
  \end{array}\]

  \uncover<2->{$"=>"$ \alert{Proof Irrelevance}}

  \uncover<3->{
  \begin{block}{\dots have practical effects}
    Difficulty to reason on code: $f (\elt{T}{P}{x}{p_1}) \not`= f (\elt{T}{P}{x}{p_2})$.
  \end{block}}
\end{frame}

\subsection{Inductive types}
\def\vector#1{\texttt{vec}~#1}

\begin{frame}
  \frametitle{Inductive types}
  
  \begin{block}{A natural extension: Reindexing}
    Let $I~\tele{x}$ be an inductive type with \emph{real} arguments
    $\tele{x}$.
    We can coerce any object of this type to an object of
    $I~\tele{y}$ provided $\tele{x} = \tele{y}$.
  \end{block}
  
  \pause
  \begin{center}
    \begin{prooftree}
      \AXC{$`G \type v : \vector{(m + 0)}$}
      \AXC{$`G \type \vector{(m + 0)} \suba \vector{m} : \Set$}
      \BIC{$`G \type v : \vector{m}$}
    \end{prooftree}
  \end{center}

  \pause
  \small{
    \begin{center}
      \begin{prooftree}
        \AXC{$\dots$}
        \AXC{$`G \typec ?_{m+0=m} : m + 0 = m$}
        \UIC{$`G \typec \eqrec{`N}{(m+0)}{\vector{}}{\ctxdot}{m}{?_{m+0 = m}} : \vector{(m + 0)} \suba \vector{m}$}
        \BIC{$`G \typec \eqrec{`N}{(m+0)}{\vector{}}{v}{m}{?_{m+0 = m}} : \vector{m}$}
      \end{prooftree}
      
      $\texttt{eq\_rec} : `A (A : \Set) (x : A) (P : A "->" \Set),
      P~x "->" `A y, x = y "->" P~y$
    \end{center}
  }

\end{frame}

\section{\Program}
\frame<beamer>{\tableofcontents[currentsection]}

\subsection{Architecture}
\begin{frame}[t]
  \frametitle{The \Program vernacular}
  
  \begin{block}{Architecture}
    Wrap around \Coq{}'s vernacular commands (\texttt{Definition},
    \texttt{Fixpoint}, \texttt{Lemma}, \ldots).
    
    \begin{enumerate}
    \item<2-> Use the \Coq{} parser.
%     \item<3-> Typecheck $`G^{\circ} \type t : T$ and generate
%       $\ipG{`G^{\circ}} \typec \ip{t}{`G^{\circ}} : \ip{T}{`G^{\circ}}$ ;
    \item<3-> Typecheck $`G \type t : T$ and generate
      $\ipG{`G} \typec \ip{t}{`G} : \ip{T}{`G}$ ;
    \item<4-> Interactive proving of obligations ;
    \item<5-> Final definition.
    \end{enumerate}
  \end{block}
  
  \only<2->{
    \[\uncover<2-4>{\texttt{Program}~}\texttt{Definition}~f~:
    \uncover<3->{\llbracket} T \uncover<3->{\rrbracket_{`G}} := 
    \uncover<3->{\llbracket} t
    \uncover<3->{\rrbracket_{`G}}\uncover<4->{ + \text{ obligations}}.\]    
    \uncover<6->{
      \begin{remark}[Restriction]
        We assume $`G \typecci \ip{T}{`G} : s$. % $`G^{\circ} = `G$ and
      \end{remark}}}
  
\end{frame}

\subsection{Hello world}
\begin{frame}
  \frametitle{Hello world: Euclidian division}

  \begin{center}
    \centering{\Huge DEMO}
  \end{center}
\end{frame}

\subsection{Extensions}
\frame<beamer>{\tableofcontents[currentsubsection]}


\begin{frame}
  \frametitle{Pattern-matching revisited}

  From patterns to equations:
  \[ \begin{array}{lcl}
    \matchml~\uncover<2->{\nu{}(}e\uncover<2->{)~\asml~t} & \returnml & \uncover<2->{t = \nu{(e)} "->"} T~\withml \\
    `| p_1 & "=>" & \uncover<2->{\funml{}~H : p_1 = \nu{(e)} "=>"} t_1 \\
    & \vdots{} & \\
    `| p_n & "=>" & \uncover<2->{\funml{}~H : p_n = \nu{(e)} "=>"} t_n \\
    \enml & & \\
    \uncover<2->{(\sref{refl\_equal}~\nu{(e)})} & &
  \end{array}\]

  where $\nu(e)$ coerces $e$ to an inductive object.
  \uncover<3->{
  \begin{block}{Further refinements}
    \begin{itemize}
    \item Generalized to dependent inductive types ;
    \item Each branch typed only once ;
    \item Add inequalities for intersecting patterns.
    \end{itemize}
  \end{block}}
  
\end{frame}

\begin{frame}
  \frametitle{Sugar}

  \begin{block}{Obligations}
    Unresolved implicits ($\_$) are turned into obligations.
  \end{block}
  
  \begin{block}{Bang}
    $! "=" (\texttt{False\_rect}~\_~\_)$ where $\texttt{False\_rect} : `A
    A : \Type, \texttt{False} "->" A$. It corresponds to \ML's
    \texttt{assert(false)}.
  \end{block}
  
  \begin{example}
    $\matchml~0~\withml~0 "=>" 0 `| n "=>" !~\enml$
  \end{example}
  
%   \pause
%   \begin{block}{Destruction}
%     Let $\destml~t~\asml~p~\inml~e "=" \matchml~t~\asml~p "=>" e~\enml$.
%     $p$ can be an arbitrary pattern.
%   \end{block}
  

\end{frame}

\begin{frame}
  \frametitle{Recursion}

  Support for well-founded recursion (and measures).
  \[ \texttt{Program}~\Fixpoint~f~(a : nat)~\{ \mlkw{wf}~lt~a \} : \nat := b \]
  
  \uncover<2->{
    \[\begin{array}{lcl}
      a & : & \nat \\
      f & : & \{ x : \nat `| x < a \} "->" \nat \\
      \hline
      b & : & \nat
    \end{array}\]
  }
\end{frame}

% \subsection{Safe lists}
% \begin{frame}[t]
%   \frametitle{\Program: The list example}
%   \vspace{-0.75em}
%   \begin{center}%
%     \only<1>{\pgfuseimage{subtac-syntax}}%
%     \only<2>{\pgfuseimage{myhd-proof}}%
%     \only<3>{\pgfuseimage{myhd-extract}}%
%     \only<4>{\pgfuseimage{mytail-extract}}%
%     \only<5>{\pgfuseimage{mytest-good}}%
%     \only<6>{\pgfuseimage{mytest-bad}}%
%     \only<7>{\pgfuseimage{append-proof}}%
%     \only<8>{\pgfuseimage{append-extract}}%
%     \only<9>{\pgfuseimage{append-app}}%
%   \end{center}
% \end{frame}

\subsection{Finger Trees}
\frame<beamer>{\tableofcontents[currentsubsection]}
\begin{frame}[t]
  \frametitle{Digits}
  \scriptsize{
  \coqdockw{Section} \coqdocid{Digit}.\coqdoceol
\coqdocindent{1.00em}
\coqdockw{Variable} \coqdocid{A} : \coqdockw{Type}.\coqdoceol


\medskip

\coqdocindent{1.00em}
\coqdockw{Inductive} \coqdocid{digit} : \coqdockw{Type} := \coqdoceol
\coqdocindent{1.00em}
\coqor  \coqdocid{One} : \coqdocid{A} \ensuremath{\to} \coqdocid{digit}\coqdoceol
\coqdocindent{1.00em}
\coqor  \coqdocid{Two} : \coqdocid{A} \ensuremath{\to} \coqdocid{A} \ensuremath{\to} \coqdocid{digit}\coqdoceol
\coqdocindent{1.00em}
\coqor  \coqdocid{Three} : \coqdocid{A} \ensuremath{\to} \coqdocid{A} \ensuremath{\to} \coqdocid{A} \ensuremath{\to} \coqdocid{digit}\coqdoceol
\coqdocindent{1.00em}
\coqor  \coqdocid{Four} : \coqdocid{A} \ensuremath{\to} \coqdocid{A} \ensuremath{\to} \coqdocid{A} \ensuremath{\to} \coqdocid{A} \ensuremath{\to} \coqdocid{digit}.\coqdoceol

\medskip

\coqdocindent{1.00em}
\coqdockw{Definition} \coqdocid{full} \coqdocid{x} := \coqdockw{match} \coqdocid{x} \coqdockw{with} \coqdocid{Four} \coqdocid{\_} \coqdocid{\_} \coqdocid{\_} \coqdocid{\_} \ensuremath{\Rightarrow} \coqdocid{True} \coqor  \coqdocid{\_} \ensuremath{\Rightarrow} \coqdocid{False}  \coqdockw{end}.\coqdoceol
% \coqdocindent{1.00em}
% \coqdockw{Definition} \coqdocid{single} (\coqdocid{x} : \coqdocid{digit}) := \coqdockw{match} \coqdocid{x} \coqdockw{with} \coqdocid{One} \coqdocid{\_} \ensuremath{\Rightarrow} \coqdocid{True} \coqor  \coqdocid{\_} \ensuremath{\Rightarrow} \coqdocid{False} \coqdockw{end}.\coqdoceol
\pause

\medskip
\coqdocindent{1.00em}
\coqdockw{Program Definition} \coqdocid{add\_digit\_left} (\coqdocid{a} : \coqdocid{A}) (\coqdocid{d} : \coqdocid{digit} \coqor  \ensuremath{\lnot} \coqdocid{full} \coqdocid{d}) : \coqdocid{digit} :=\coqdoceol
\coqdocindent{2.00em}
\coqdockw{match} \coqdocid{d} \coqdockw{with}\coqdoceol
\coqdocindent{3.00em}
\coqor  \coqdocid{One} \coqdocid{x} \ensuremath{\Rightarrow} \coqdocid{Two} \coqdocid{a} \coqdocid{x}\coqdoceol
\coqdocindent{3.00em}
\coqor  \coqdocid{Two} \coqdocid{x} \coqdocid{y} \ensuremath{\Rightarrow} \coqdocid{Three} \coqdocid{a} \coqdocid{x} \coqdocid{y}\coqdoceol
\coqdocindent{3.00em}
\coqor  \coqdocid{Three} \coqdocid{x} \coqdocid{y} \coqdocid{z} \ensuremath{\Rightarrow} \coqdocid{Four} \coqdocid{a} \coqdocid{x} \coqdocid{y} \coqdocid{z}\coqdoceol
\coqdocindent{3.00em}
\coqor  \coqdocid{Four} \coqdocid{\_} \coqdocid{\_} \coqdocid{\_} \coqdocid{\_} \ensuremath{\Rightarrow} !\coqdoceol
\coqdocindent{2.00em}
\coqdockw{end}.\coqdoceol
\coqdocindent{1.00em}
\coqdockw{Next} \coqdockw{Obligation}.\coqdoceol
\noindent
\coqdocindent{2.00em}
\coqdocid{intros} ; \coqdocid{simpl} \coqdockw{in} \coqdocid{n} ; \coqdocid{auto}.\coqdoceol
\noindent
\coqdocindent{1.00em}
\coqdockw{Qed}.\coqdoceol

}


\end{frame}

\begin{frame}[t]
  \frametitle{Monoids \& Nodes}
\scriptsize{
\coqdocindent{1.00em}
\coqdockw{Variable} \coqdocid{v} : \coqdockw{Type}.\coqdoceol
\coqdocindent{1.00em}
\coqdockw{Variable} \coqdocid{mono} : \coqdocid{monoid} \coqdocid{v}.\coqdoceol
\medskip
\coqdocindent{1.00em}
\coqdockw{Notation} "'$\varepsilon$'" := \coqdocid{mempty} \coqdocid{mono}.\coqdoceol
\coqdocindent{1.00em}
\coqdockw{Infix} "$\cdot$" := \coqdocid{mappend} \coqdocid{mono} (\coqdocid{right} \coqdocid{associativity}, \coqdocid{at} \coqdocid{level} 20).\coqdoceol
\medskip
\coqdocindent{1.00em}
\coqdockw{Program Definition} \coqdocid{listMonoid} (\coqdocid{A} :
\coqdockw{Type}) := \coqdoceol
\coqdocindent{2.00em}
@\coqdocid{mkMonoid} (\coqdocid{list} \coqdocid{A}) \coqdocid{nil} (@\coqdocid{app} \coqdocid{A}) \coqdocid{\_} \coqdocid{\_} \coqdocid{\_}.\coqdoceol
\medskip
\pause
\coqdocindent{1.00em}
\coqdockw{Section} \coqdocid{Nodes}.\coqdoceol
\coqdocindent{2.00em}
\coqdockw{Variable} \coqdocid{A} : \coqdockw{Type}.\coqdoceol
\coqdocindent{2.00em}
\coqdockw{Variable} \coqdocid{measure} : \coqdocid{A} \ensuremath{\to} \coqdocid{v}.\coqdoceol
\coqdocindent{2.00em}
\coqdockw{Notation} "'\ensuremath{\coqdoublebar}' \coqdocid{x} '\ensuremath{\coqdoublebar}'" $\coloneqq$ (\coqdocid{measure} \coqdocid{x}).\coqdoceol

\medskip
\coqdocindent{2.00em}
\coqdockw{Inductive} \coqdocid{node} : \coqdockw{Type} $\coloneqq$\coqdoceol
\coqdocindent{2.00em}
\coqor  \coqdocid{Node2} : \ensuremath{\forall} \coqdocid{x} \coqdocid{y}, \{ \coqdocid{s} : \coqdocid{v} \coqor  \coqdocid{s} = $\coqdoublebar$ \coqdocid{x} $\coqdoublebar$ $\cdot$ $\coqdoublebar$ \coqdocid{y} $\coqdoublebar$ \} \ensuremath{\to} \coqdocid{node}\coqdoceol
\coqdocindent{2.00em}
\coqor  \coqdocid{Node3} : \ensuremath{\forall} \coqdocid{x} \coqdocid{y} \coqdocid{z}, \{ \coqdocid{s} : \coqdocid{v} \coqor  \coqdocid{s} = $\coqdoublebar$ \coqdocid{x} $\coqdoublebar$ $\cdot$ $\coqdoublebar$ \coqdocid{y} $\coqdoublebar$ $\cdot$ $\coqdoublebar$ \coqdocid{z} $\coqdoublebar$ \} \ensuremath{\to} \coqdocid{node}.\coqdoceol
\pause
\medskip
\coqdocindent{2.00em}
\coqdockw{Program Definition} \coqdocid{node2} (\coqdocid{x} \coqdocid{y} : \coqdocid{A}) : \coqdocid{node} $\coloneqq$ \coqdoceol
\coqdocindent{3.00em}
\coqdocid{Node2} \coqdocid{x} \coqdocid{y} ($\coqdoublebar$ \coqdocid{x}
$\coqdoublebar$ $\cdot$ $\coqdoublebar$ \coqdocid{y}
$\coqdoublebar$).\coqdoceol
\medskip
\coqdocindent{2.00em}
\coqdockw{Program Definition} \coqdocid{node\_measure} (\coqdocid{n} : \coqdocid{node}) : \coqdocid{v} $\coloneqq$\coqdoceol
\coqdocindent{3.00em}
\coqdockw{match} \coqdocid{n} \coqdockw{with} \coqor  \coqdocid{Node2} \coqdocid{\_} \coqdocid{\_} \coqdocid{s} \ensuremath{\Rightarrow} \coqdocid{s} \coqor  \coqdocid{Node3} \coqdocid{\_} \coqdocid{\_} \coqdocid{\_} \coqdocid{s} \ensuremath{\Rightarrow} \coqdocid{s} \coqdockw{end}.\coqdoceol
}

\end{frame}

\begin{frame}[t]
  \frametitle{Dependent Finger Trees}
  \scriptsize{

\medskip
\coqdocindent{1.00em}
\coqdockw{Inductive} \coqdocid{fingertree} (\coqdocid{A} : \coqdockw{Type}) (\coqdocid{measure} : \coqdocid{A} \ensuremath{\to} \coqdocid{v}) : \coqdocid{v} \ensuremath{\to} \coqdockw{Type} $\coloneqq$\coqdoceol
\coqdocindent{1.00em}
\coqor  \coqdocid{Empty} : \coqdocid{fingertree} \coqdocid{measure} \alert{$\varepsilon$}\coqdoceol
\coqdocindent{1.00em}
\coqor  \coqdocid{Single} : \ensuremath{\forall} \coqdocid{x} : \coqdocid{A}, \coqdocid{fingertree} \coqdocid{measure} (\alert{\coqdocid{measure} \coqdocid{x}})\coqdoceol
\coqdocindent{1.00em}
\coqor  \coqdocid{Deep} : \ensuremath{\forall} (\coqdocid{l} : \coqdocid{digit} \coqdocid{A}) (\coqdocid{m} : \coqdocid{v}),\coqdoceol
\coqdocindent{2.00em}
@\coqdocid{fingertree} (\alert{\coqdocid{node} \coqdocid{measure}})
\coqdocid{nodeMeasure} \coqdocid{m} \ensuremath{\to} \ensuremath{\forall} \coqdocid{r} : \coqdocid{digit} \coqdocid{A}, \coqdoceol
\coqdocindent{2.00em}
\coqdocid{fingertree} \coqdocid{measure} (\alert{\coqdocid{digit\_measure} \coqdocid{measure} \coqdocid{l} $\cdot$ \coqdocid{m} $\cdot$ \coqdocid{digit\_measure} \coqdocid{measure} \coqdocid{r}}).\coqdoceol

\pause
\medskip
\coqdocindent{2.00em}
\coqdockw{Program Fixpoint} \coqdocid{add\_left} \coqdocid{A} (\coqdocid{measure} : \coqdocid{A} \ensuremath{\to} \coqdocid{v}) (\alert{\coqdocid{a}} : \coqdocid{A}) (\alert{\coqdocid{s}} : \coqdocid{v})\coqdoceol
\coqdocindent{3.00em}
(\coqdocid{t} : \coqdocid{fingertree} \coqdocid{measure} \coqdocid{s})
\{\coqdocid{struct} \coqdocid{t}\} : \coqdoceol
\coqdocindent{3.00em}
\coqdocid{fingertree} \coqdocid{measure} (\alert{\coqdocid{measure} \coqdocid{a} $\cdot$ \coqdocid{s}}) $\coloneqq$\coqdoceol
\coqdocindent{3.00em}
\coqdockw{match} \coqdocid{t} \coqdockw{with}\coqdoceol
\coqdocindent{4.00em}
\coqor  \coqdocid{Empty} \ensuremath{\Rightarrow} \coqdocid{Single} \coqdocid{a}\coqdoceol
\coqdocindent{4.00em}
\coqor  \coqdocid{Single} \coqdocid{b} \ensuremath{\Rightarrow} \coqdocid{Deep} (\coqdocid{One} \coqdocid{a}) \coqdocid{Empty} (\coqdocid{One} \coqdocid{b})\coqdoceol
\coqdocindent{4.00em}
\coqor  \coqdocid{Deep} \coqdocid{pr} \coqdocid{st'} \coqdocid{t'} \coqdocid{sf} \ensuremath{\Rightarrow} \coqdoceol
\coqdocindent{5.00em}
\coqdockw{match} \coqdocid{pr} \coqdockw{with}\coqdoceol
\coqdocindent{6.00em}
\coqor  \coqdocid{Four} \coqdocid{b} \coqdocid{c} \coqdocid{d} \coqdocid{e} \ensuremath{\Rightarrow} \coqdockw{let} \coqdocid{sub} $\coloneqq$ \coqdocid{add\_left} (\coqdocid{node3} \coqdocid{measure} \coqdocid{c} \coqdocid{d} \coqdocid{e}) \coqdocid{t'} \coqdockw{in}\coqdoceol
\coqdocindent{7.00em}
\coqdocid{Deep} (\coqdocid{Two} \coqdocid{a} \coqdocid{b}) \coqdocid{sub} \coqdocid{sf}\coqdoceol
\coqdocindent{6.00em}
\coqor  \coqdocid{x} \ensuremath{\Rightarrow} \coqdocid{Deep} (\alert{\coqdocid{add\_digit\_left} \coqdocid{a} \coqdocid{pr}}) \coqdocid{t'} \coqdocid{sf}\coqdoceol
\coqdocindent{5.00em}
\coqdockw{end}\coqdoceol
\coqdocindent{3.00em}
\coqdockw{end}.\coqdoceol

}

\end{frame}

\begin{frame}
  \frametitle{Dependent Finger Trees - cont'd}

  \begin{block}{Two hundred lines, one hundred obligations concatenation}
    \coqdocindent{1.00em}
    \coqdockw{Definition} \coqdocid{app} (\coqdocid{A} : \coqdockw{Type}) (\coqdocid{measure} : \coqdocid{A} \ensuremath{\to} \coqdocid{v}) \coqdoceol
    \coqdocindent{2.00em}
    (\coqdocid{xs} : \coqdocid{v}) (\coqdocid{x} : \coqdocid{fingertree} \coqdocid{measure} \coqdocid{xs})\coqdoceol
    \coqdocindent{2.00em}
    (\coqdocid{ys} : \coqdocid{v}) (\coqdocid{y} : \coqdocid{fingertree} \coqdocid{measure} \coqdocid{ys}) : \coqdoceol
    \coqdocindent{2.00em}
    \coqdocid{fingertree} \coqdocid{measure} (\coqdocid{xs} $\cdot$
    \coqdocid{ys}).\coqdoceol
  \end{block}
  
  Splitting, views can be defined in a similar way.

\end{frame}

\begin{frame}
  \frametitle{A glimpse of Finger Trees in \Coq}
  
  \begin{block}{The development}
    \begin{itemize}
    \item Certified implementation of Finger Trees. 
      Certified implementation of sequences built on top of Finger Trees.
    \item $\sim$ 1200 lines of specification, $\sim$ 1400 of proof, mostly unchanged
      code.
    \item Extracts to \Haskell and \ML (uses dependent records).
    \end{itemize}
  \end{block}
  \pause
  \begin{block}{Conclusions}
    \begin{itemize}
    \item \Program scales ;
    \item Need more language technology, e.g: overloading ;
    \item Subset types arise naturally ;
    \item Dependent types are a powerfull specification tool ;
    \item Some difficulties with reasonning and computing.
    \end{itemize}   
  \end{block}
  
\end{frame}

\section{Conclusion}
\begin{frame}
  \frametitle{Conclusion}
  
  \begin{block}{Our contributions}
    \begin{itemize}
    \item A more \alert{flexible} programming language, (almost) \alert{conservative} over
      \CIC, \alert{integrated} with the existing environment and a
      formal \alert{justification} of ``\ps{}''.
    \item A tool to make \alert{programming} in \Coq using the \alert{full} language
      possible, which can \alert{effectively} be used for non-trivial
      developments.
    \end{itemize}
  \end{block}
  
  \begin{block}{Future work}
    Reasonning support through tactics, implementation of
    proof-irrelevance in \Coq's kernel.
  \end{block}
  
\end{frame}

\begin{frame}
  \frametitle{The End}
  \centering{
    \begin{center}
      \url{http://www.lri.fr/~sozeau/research/russell.en.html}
    \end{center} 
  }
\end{frame}

\begin{frame}
  \frametitle{Why eta rules and proof irrelevance ?}
  
  Let $A,B : \Type$, $c : A \suba B$, $d : B \suba A$. 
  \pause
  
  Let $P : A "->" \Prop$ with $p : \Pi x : A, P~x$ and $q : \Pi x : B,
  P~x$.
  
  \pause
  Consider: $p~x =_{(P~x)} q~x$ in context $`G "=" x : A$. 
  
  By $\ip{\_}{`G}$: $\ip{P}{`G} "=" P$, $\ip{p}{`G} "=" p$ and
  $\ip{q}{`G} "=" q : \alert{\Pi x : B, P~d[x]}$, \pause 
  
  hence: $\ip{p~x =_{(P~x)} q~x}{`G} "=" p~x =_{(P~x)} \alert{q~c[x]}$.

  We have $p~x : P~x$ but $q~c[x] : P~d[c[x]]$, so $x `= d[c[x]]$ is
  necessary.
  This means eta rules for all constructs are needed.
  \pause
  
  Now: $c "=" \elt{A}{P}{\cdot}{?_{P[\cdot/x]}} : A \suba
  \mysubset{x}{A}{P}$ and $d "=" \eltpit{\cdot} : \mysubset{x}{A}{P}
  \suba A$, so $d[c[x]] "=" x$ but what about $c[d[x]]$ ? 
  \pause

  $c[d[x]] "=" \elt{A}{P}{(\eltpit~x)}{?_{P[\eltpit x/x]}} \alert{\not`=}
  x$ except if PI is included in $`=$.
  
\end{frame}



\begin{frame}
  \frametitle{Dependent Inductive coercion rule}
  \indfig
\end{frame}

\begin{frame}
  \frametitle{Splitting nodes}
  
  \scriptsize{\coqdocindent{1.00em}
\coqdockw{Program Definition} \coqdocid{splitNode} (\coqdocid{p} : \coqdocid{v} \ensuremath{\to} \coqdocid{bool}) (\coqdocid{i} : \coqdocid{v}) (\coqdocid{n} : \coqdocid{node} \coqdocid{measure}) : \coqdoceol
\coqdocindent{1.00em}
\{ \coqdocid{s} : \coqdocid{split} (\coqdockw{fun} \coqdocid{A} \ensuremath{\Rightarrow} \coqdocid{option} (\coqdocid{digit} \coqdocid{A})) \coqdocid{A} \coqor  \coqdockw{let} (\coqdocid{l}, \coqdocid{x}, \coqdocid{r}) $\coloneqq$ \coqdocid{s} \coqdockw{in}\coqdoceol
\coqdocindent{2.00em}
\coqdockw{let} \coqdocid{ls} $\coloneqq$ \coqdocid{option\_digit\_measure} \coqdocid{measure} \coqdocid{l} \coqdockw{in}\coqdoceol
\coqdocindent{2.00em}
\coqdockw{let} \coqdocid{rs} $\coloneqq$ \coqdocid{option\_digit\_measure} \coqdocid{measure} \coqdocid{r} \coqdockw{in}\coqdoceol
\coqdocindent{2.00em}
\coqdocid{node\_measure} \coqdocid{n} = \coqdocid{ls} $\cdot$ $\coqdoublebar$ \coqdocid{x} $\coqdoublebar$ $\cdot$ \coqdocid{rs} \ensuremath{\land}\coqdoceol
\coqdocindent{2.00em}
\coqdocid{node\_to\_list} \coqdocid{n} = \coqdocid{option\_digit\_to\_list} \coqdocid{l} $\app$ [x] $\app$ \coqdocid{option\_digit\_to\_list} \coqdocid{r} \ensuremath{\land}\coqdoceol
\coqdocindent{2.00em}
(\coqdocid{l} = \coqdocid{None} \ensuremath{\lor} \coqdocid{p} (\coqdocid{i} $\cdot$ \coqdocid{ls}) = \coqdocid{false}) \ensuremath{\land}\coqdoceol
\coqdocindent{2.00em}
(\coqdocid{r} = \coqdocid{None} \ensuremath{\lor} \coqdocid{p}
(\coqdocid{i} $\cdot$ \coqdocid{ls} $\cdot$ $\coqdoublebar$ \coqdocid{x}
$\coqdoublebar$) = \coqdocid{true})\} $\coloneqq$ $\ldots$}
\end{frame}

\begin{frame}
  \frametitle{Instanciation: sequences}
  \begin{block}{The monoid and measure: executable semantics}
    \small{\coqdockw{Definition} \coqdocid{below} \coqdocid{i} := \{ \coqdocid{x} : \coqdocid{nat} | \coqdocid{x} $<$ \coqdocid{i} \}.\coqdoceol
    \coqdockw{Definition} \coqdocid{v} := \{ \coqdocid{i} : \coqdocid{nat} \& (\coqdocid{below} \coqdocid{i} \ensuremath{\rightarrow} \coqdocid{A}) \}.\coqdoceol
    \coqdockw{Program Definition} $\varepsilon$ : \coqdocid{v} := 0 $\Yleft$ (\coqdockw{fun} \coqdocid{\_} \ensuremath{\Rightarrow} !).\coqdoceol
    \coqdockw{Program Definition} \coqdocid{append} (\coqdocid{xs} \coqdocid{ys} : \coqdocid{v}) : \coqdocid{v} := \coqdoceol
    \coqdocindent{1.00em}
    \coqdockw{let} (\coqdocid{n}, \coqdocid{fx}) := \coqdocid{xs} \coqdockw{in}
    \coqdockw{let} (\coqdocid{m}, \coqdocid{fy}) := \coqdocid{ys} \coqdockw{in}\coqdoceol
    \coqdocindent{2.00em}
    (\coqdocid{n} + \coqdocid{m}) $\Yleft$ \coqdoceol
    \coqdocindent{3.00em}
    (\coqdockw{fun} \coqdocid{i} \ensuremath{\Rightarrow} \coqdockw{if}
    \coqdocid{lt\_ge\_dec} \coqdocid{i} \coqdocid{n} \coqdockw{then}
    \coqdocid{fx} \coqdocid{i} \coqdockw{else} \coqdocid{fy}
    (\coqdocid{i} - \coqdocid{n})).\coqdoceol
    \coqdockw{Program Definition} \coqdocid{seqMonoid} := @\coqdocid{mkMonoid} \coqdocid{v} $\varepsilon$ \coqdocid{append} \coqdocid{\_} \coqdocid{\_} \coqdocid{\_}.\coqdoceol
    \coqdockw{Program Definition} \coqdocid{measure} (\coqdocid{x} : \coqdocid{A}) : \coqdocid{v} := 1 $\Yleft$ (\coqdockw{fun} \coqdocid{\_} \ensuremath{\Rightarrow} \coqdocid{x}).\coqdoceol
  }
  \end{block}

  \begin{block}{Specialization}
    \small{\coqdockw{Definition} \coqdocid{seq} (\coqdocid{x} : \coqdocid{v}) := \coqdocid{fingertree} \coqdocid{seqMonoid} \coqdocid{measure} \coqdocid{x}.\coqdoceol
    \coqdockw{Program Fixpoint} \coqdocid{make} (\coqdocid{i} : \coqdocid{nat}) (\coqdocid{v} : \coqdocid{A}) \{ \coqdocid{struct} \coqdocid{i} \} : \coqdocid{seq} (\coqdocid{i} $\Yleft$ (\coqdockw{fun} \coqdocid{\_} \ensuremath{\Rightarrow} \coqdocid{v})) := \coqdoceol
    \coqdockw{Program Definition} \coqdocid{set} (\coqdocid{i} : \coqdocid{nat}) (\coqdocid{m} : \coqdocid{below} \coqdocid{i} \ensuremath{\rightarrow} \coqdocid{A}) (\coqdocid{x} : \coqdocid{seq} (\coqdocid{i} $\Yleft$ \coqdocid{m})) \coqdoceol
    \coqdocindent{1.00em}
    (\coqdocid{j} : \coqdocid{below} \coqdocid{i}) (\coqdocid{value} : \coqdocid{A}) \coqdoceol
    \coqdocindent{1.00em}
    :  \coqdocid{seq} (\coqdocid{i} $\Yleft$ (\coqdockw{fun} \coqdocid{idx} : \coqdocid{below} \coqdocid{i} \ensuremath{\Rightarrow} \coqdockw{if} \coqdocid{eq\_nat\_dec} \coqdocid{idx} \coqdocid{j} \coqdockw{then} \coqdocid{value} \coqdockw{else} \coqdocid{m} \coqdocid{idx})) := \coqdoceol
    \coqdockw{Program Definition} \coqdocid{get} (\coqdocid{i} : \coqdocid{nat}) (\coqdocid{m} : \coqdocid{below} \coqdocid{i} \ensuremath{\rightarrow} \coqdocid{A}) (\coqdocid{x} : \coqdocid{seq} (\coqdocid{i} $\Yleft$ \coqdocid{m})) \coqdoceol
    \coqdocindent{1.00em}
    (\coqdocid{j} : \coqdocid{below} \coqdocid{i}) : \{ \coqdocid{value} : \coqdocid{A} | \coqdocid{m} \coqdocid{j} = \coqdocid{value} \} :=\coqdoceol}
  \end{block}
\end{frame}

\end{document}

%%% Local Variables: 
%%% mode: latex
%%% TeX-master: "slides"
%%% LaTeX-command: "x=pdf; TEXINPUTS=\"~/research/publication/styles:..:../style:../figures:\" ${pdfx}latex"
%%% TeX-PDF-mode: t
%%% End: 
