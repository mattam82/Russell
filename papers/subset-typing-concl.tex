\chapter{Conclusion}
Nous avons d�velopp� un langage de programmation plus souple que le
langage de \Coq{} mais conservant sa richesse d'expression (types
d�pendants). Il permet de d�coupler la description algorithmique de la
v�rification. La correction des termes engendr�s est
garantie par le syst�me sous-jacent qui offre ensuite la possibilit�
d'extraire un programme correct par construction dans un langage de type
\ML. D'autre part, cette m�thode s'int�gre bien dans l'environnement
\Coq{} et ouvre la voie � la r�alisation de travaux plus complexes par
des utilisateurs non-experts. Cela constitue la premi�re �tape vers un
environnement de programmation s�re utilisable dans \Coq. 

\section*{Travaux futurs}
Nous avons de multiples directions dans lesquelles �tendre notre
syst�me. Le support des inductifs, de la r�cursion interne, des univers cumulatifs
et des d�finitions dans les contextes semblent des probl�mes techniques
abordables. 
La possibilit� d'utiliser \lng{} dans
d'autres parties de \Coq{} (�nonc�s de lemmes, termes de tactiques) est
aussi un objectif int�ressant. Modifier \Coq{} pour qu'il g�re mieux les
existentielles est aussi un probl�me crucial dont nous avons fait
l'exp�rience durant le d�veloppement de \Subtac{}.

%%% Local Variables: 
%%% mode: latex
%%% TeX-master: "subset-typing"
%%% LaTeX-command: "TEXINPUTS=\"style:$TEXINPUTS\" latex"
%%% End: 
