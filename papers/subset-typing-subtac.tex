\newpage
\section{La tactique \Subtac}
Nous avons d\'evelopp\'e la tactique \Subtac{} disponible dans la version
\CVS{}~courante de \Coq{} (\url{http://coq.inria.fr}). Elle permet de
cr\'eer un programme, le typer et g\'en\'erer un terme incomplet
correspondant (voir annexe \ref{fig:euclid-subtac}). 

\subsection{Existentielles}
La g\'en\'eration des buts correspondant aux variables existentielles et la
formation du terme final devaient originellement \^etre laiss\'ees \`a la
tactique \Refine~et au syst\`eme de gestion des existentielles de \Coq. Certaines limitations 
dans l'impl\'ementation du raffinement (le m\'ecanisme permettant de manipuler
des termes ``\`a trous'') nous ont emp\^ech\'e d'utiliser \Refine. En
particulier, la gestion des d\'efinitions r\'ecursives et la pr\'esence de
variables existentielles dans les types d'autres existentielles
 n'\'etaient pas support\'ees. En
cons\'equence, nous avons d\'evelopp\'e une nouvelle tactique \Coq{}
permettant de g\'erer les termes avec existentielles de fa\c con plus
g\'en\'erale. 

\subsubsection{La tactique \eterm}
L'id\'ee de d\'epart de la tactique \Refine{} est de prendre un terme \`a
trous et d'en faire une traduction en une s\'equence de tactiques. Par
exemple, lorsque \Refine{} rencontre une abstraction, il fait une
introduction, lorsque c'est un cast, on applique l'identit\'e et ainsi de
suite. Intuitivement, la s\'equence de tactiques engendr\'ee va construire
le terme de d\'epart implicitement. 

La tactique \eterm{} fonctionne diff\'erement. \`A partir d'un terme $t$ contenant des existentielles, \eterm{} va g\'en\'eraliser le terme par
rapport \`a celles-ci, cr\'eant ainsi un objet $\lambda ex_1 : T_1, \ldots, ex_n :
T_n, t[?_1 := ex_1, \ldots, ?_n := ex_n]$. 
Habituellement, on propose $t$ comme habitant d'un type $T$ donn\'e (le
but), par exemple on peut proposer $\lambda x : \nat.x$ comme preuve du
but $\nat "->" \nat$.
Plut\^ot que de donner directement $t$, on applique le nouveau
terme, et \Coq{} va automatiquement nous demander d'instancier les
arguments $ex_1 \ldots ex_n$ correspondants aux existentielles du terme 
$t$. Cette technique permet d'avoir des d\'ependances entre existentielles
(par exemple, $ex_1$ peut apparaitre dans tout les types $T_2 \ldots
T_n$) et de ne pas reposer sur la gestion des existentielles de \Coq{}
qui n'est pas tr\`es flexible \`a l'heure actuelle.

Il nous faut nous pencher un peu plus avant sur la g\'en\'eralisation des
existentielles pour comprendre le m\'ecanisme d'\eterm.
Puisqu'on veut pouvoir avoir des d\'ependances entres les $n$
existentielles d'un termes et qu'on s\'erialise celles-ci en un produit
$n$-aire, il nous faut \^etre tr\`es attentifs \`a l'ordre dans lequel on
g\'en\'eralise les variables existentielles. Si $?_3 : T_3$ o\`u $T_3$
r\'ef\'erence $?_4$, il faut que l'existentielle $ex_4$ apparaisse
\emph{avant} $ex_3$ dans notre produit. Il est toujours possible de
trouver un ordre compatible avec ces d\'ependances puisqu'il est
impossible de cr\'eer un cycle o\`u $?_i$ r\'ef\'erencerait $?_j$ et vice-versa
(ceci est assur\'e par le caract\`ere fonctionnel des objets impl\'ementants
les existentielles dans \Coq). La tactique actuelle est cod\'ee avec
l'hypoth\`ese que toute existentielle $?_i$ ne d\'epend pas des
existentielles d'indice sup\'erieur \`a $i$. Il est cependant envisagable de
r\'eecrire tout terme contenant des existentielles comme un terme
\'equivalent avec des indices respectant cet ordre.

\subsection{Traitement de la r\'ecursion}
Lorsque l'on d\'eveloppe un programme r\'ecursif dans un syst\`eme tel que
\Coq, on est forc\'e de fournir une preuve de terminaison de son
algorithme. Pour cela, on montre g\'en\'eralement qu'on a un ordre bien
fond\'e sur le type de l'argument de r\'ecursion et que chaque appel respecte
cet ordre. Nous avons ajout\'e des facilit\'es d'\'ecriture de fonctions
r\'ecursives \`a notre langage ; on ajoute les existentielles
correspondant aux preuves que l'ordre est bien fond\'e ou qu'il est bien
respect\'e par les termes. Ainsi lors du raffinement on obtient naturellement
les buts correspondants \`a prouver.

\subsection{Traitement des inductifs}
Notre langage ne prend pas encore en compte les d\'efinitions inductives g\'en\'erales.
Au-del\`a du traitement des types sous-ensemble, on a un support minimal
pour les inductifs \`a deux constructeurs qui correspondent \`a des bool\'eens
annot\'es par des propri\'et\'es logiques (voir traitement de la
conditionnelle figure \ref{fig:euclid-subtac}). A long terme on devrait
pouvoir traiter les inductifs dans $\Set$ pr\'edicatif, qui ne peuvent 
embarquer des propositions qu'en utilisant des types sous-ensemble avec
le m\^eme m\'ecanisme de coercion et conserver l'inf\'erence.

%%% Local Variables: 
%%% mode: latex
%%% TeX-master: "subset-typing"
%%% LaTeX-command: "TEXINPUTS=\"style:$TEXINPUTS\" latex"
%%% End: 
