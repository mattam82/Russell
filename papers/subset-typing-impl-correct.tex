\paragraph{Substitutivité}
\setboolean{displayLabels}{false}
Le lemme qui pose problème est la substitutivité de l'interprétation.
On l'énonce de la façon suivante:

\begin{lemma}[Coercion de sortes]
  \label{subi-sorts}
  Si $s \subi T$ ou $T \subi s$ où $s `: \setproptype$ alors $T = s$.
\end{lemma}
\begin{proof}
  \TODO{}
\end{proof}

\def\iu{\ip{u}{\iG}}
\def\tux{\ip{t[u/x]}{`G, `D[u/x]}}
\def\cux{[c~\ip{u}{`G}/x]}
\def\tcux{\ip{t}{\ipG{`G, x : V, `D}}[c~\ip{u}{`G}/x]}
\def\GD{`G, x : V, `D}
\def\Gr{`G, `D[u/x]}
\def\iGD{\ipG{`G, x : V, `D}}
\def\iGr{\ipG{`G, `D[u/x]}}


Il nous faut tout d'abord un lemme supplémentaire sur la substitution
dans le système algorithmique:

\begin{lemma}[Substitutivité du typage algorithmique avec coercion]
  \label{subst-coerce-algo}
  \[\left.\begin{array}{l}
      `G \typea u : U \\
      \GD \typea t : T \\ 
      U \suba V
    \end{array}
\right\} "=>"    
`G, `D[u/x] \typea t[u/x] : T' \suba T[u/x]
    \]
\end{lemma}

\begin{proof}
  Par induction sur la dérivation de $\GD \typea t : T$.

  \typenva
  \begin{induction}
    \case{Var} On a deux cas:
    \begin{itemize}
    \item Si $y = x$, alors $T = V$. On a $t[u/x] = u$ donc $T' =
      U$. Par hypothèse $U \suba V$ donc $T' \suba T[u/x]$ triviallement
      ($x$ n'apparait pas dans $V$).

    \item Sinon, $t[u/x] = y$ donc $T' = T[u/x]$ et on a $T' \suba T[u/x]$
      par réflexivité de la coercion.
    \end{itemize}
    

    \case{Prod}
    Par induction, $\Gr \typea T[u/x] : s_1' \suba s_1[u/x]$ et $\Gr, x
    : T[u/x]
    \typea U[u/x] : s_2' \suba s_2[u/x]$. Or $s_1[u/x] = s_1$ et
    $s_2[u/x] = s_2$. On en déduit par \irule{Prod}: 
    $\Gr \typea (\Pi x : T.U)[u/x] : s_3 \suba s_3[u/x]$.

    \case{Abs}
    On a: 
    \begin{prooftree}
      \Abs
    \end{prooftree}
    
    Par induction, $\Gr \typea (\Pi y : T.U)[u/x] : s' \suba s[u/x]$ et
    $\Gr, y : T[u/x] \typea M[u/x] : U' \suba U[u/x]$.
    Par le même raisonnement que précedemment, $s' = s[u/x] = s$.
    On peut appliquer \irule{Abs} pour obtenir:
    $\Gr \typea (\lambda y : T.M)[u/x] : \Pi y : T[u/x].U'$. Comme $U'
    \suba U[u/x]$ on a $\Pi y : T[u/x].U' \suba \Pi y : T[u/x].U[u/x]$
    et la propriété est vérifiée.

    \case{App}
    On a: 
    \begin{prooftree}
      \TAX{App}
      {$\GD \seq f : F \quad \mualgo(F) = \Pi y : A. B$}
      {$\GD \seq e : E \quad \GD \seq E, A : s$}
      {$E \sub A $}
      {$\GD \seq (f~e) : B[ e / y ]$}
      {}
    \end{prooftree}
    
    Par induction, $\Gr \typea f[u/x] : F' \suba F[u/x]$ et $\Gr \typea
    e[u/x] : E' \suba E[u/x]$ et $\Gr \typea E[u/x], A[u/x] : s$.
    Comme $\mualgo(F) = \Pi y : A.B$, $\mualgo(F[u/x]) = \Pi y :
    A[u/x].B[u/x]$ et par transitivité de la coercion, $F' \suba \Pi y :
    A[u/x].B[u/x]$. On en déduit que $\mualgo(F') = \Pi y : A'.B'$ avec
    $A[u/x] \suba A'$ et $B' \suba B[u/x]$.
    Par substitutivité de la coercion, $E \suba A$ implique $E[u/x]
    \suba A[u/x]$. On applique la transitivité pour obtenir $E' \suba
    A'$. Par définition, si $E' \suba E[u/x]$ et $\Gr \typea E[u/x] : s$
    alors $\Gr \typea E' : s$. De même pour $A'$.  

    On peut donc appliquer \irule{App}:    
    \begin{prooftree}
      \TAX{App}
      {$\GD \seq f[u/x] : F' \quad \mualgo(F') = \Pi y : A'. B'$}
      {$\GD \seq e[u/x] : E' \quad \GD \seq E', A' : s$}
      {$E' \sub A' $}
      {$\GD \seq (f~e)[u/x] : B'[ e[u/x] / y ]$}
      {}
    \end{prooftree}
    
    Par substitutivité de la coercion, $B' \suba B[u/x]$ implique
    $B'[e[u/x]/y] \suba B[u/x][e[u/x]/y]$. Or $B[u/x][e[u/x]/y] =
    B[e/y][u/x]$ car $y$ n'apparait pas dans $u$. On a donc bien
    $T' = B'[e[u/x]/y] \suba T[u/x] = B[e/y][u/x]$.
     
    \item[] Les autres cas passent aisément par induction.
  \end{induction}
  

\end{proof}



\begin{lemma}[Substitution et coercion]
  Si $`G \typea u : U$ et $`G \typec c : U \suba V$ alors
  \[\GD \typea t : T "=>"
  \begin{array}{l}  
    `E!`a, \Gr \typec `a : \typeafn{\Gr}{t[u/x]} \suba T[u/x],\\
    `a \ip{t[u/x]}{\Gr} \eqbr \ip{t}{\GD}\cux
  \end{array}\]
  et
  \[\left.\begin{array}{l}
      \GD \typea T, T' : s \\
      \GD \typec d : T \suba T'
    \end{array}
  \right\} "=>" \Gr \typec d\cux : T[u/x] \suba T'[u/x]\]
\end{lemma}


\begin{proof}
  Ici, $\typeafn{\Gr}{t[u/x]}$ est l'égal du $T'$ du lemme précédent, il
  est donc uniquement déterminé par l'algorithme de typage ce qui
  entraine que $`a$ est unique.
  
  On peut remarquer que si $T$ est une sorte $s$ alors
  $\typeafn{\Gr}{t[u/x]} = s$ et $`a$ est l'identité. On en déduit que
  \[\ip{t[u/x]}{\Gr}\cux \eqbr \ip{t}\GD\cux\] On utilisera ce fait à
  plusieurs reprises.

  On omet les environement passés à $\sref{coerce}$ s'ils sont évidents.
  On procède par induction mutuelle sur la dérivation de $`G, x : V, `D
  \typea t : T$ et la dérivation de $\GD \typec d : T \suba T'$.
  \begin{induction}
    \case{Var} 
    On a:
    \typenva
    \begin{prooftree}
      \BAX{Var}
      {$\wf \GD$}
      {$y : T `: \GD$}
      {$\GD \seq y : T$}
      {}
    \end{prooftree}    
    \typenvi

    \begin{itemize}
    \item Si $x "/=" y$ alors $T' = (`G, `D[u/x])(y) = T[u/x]$. Il
      existe une unique coercion $`a : T[u/x] \suba T[u/x]$, l'identité.   
      Par définition de l'interprétation, on doit montrer
      $`a~y \eqbr y$, trivial.

    \item Sinon, $t[u/x] = u$ et donc $T' = U$. 
      On a $T = V$, $x `; V$ par hypothèse.
      On peut prendre $`a = c : U \suba V$ car $T' = U$ et $T[u/x] = V$.
      On peut vérifier:
      \begin{eqnarray*}
        c~\ip{x[u/x]}{\Gr} & = & c~\ip{u}{\Gr} \\
        & = & \ip{x}{\GD}\cux
      \end{eqnarray*}
      
    \end{itemize}
    
    \case{App}\quad
    \typenva
    \begin{prooftree}
      \TAX{App}
      {$`G \seq f : F \quad \mualgo(F) = \Pi y : A. B$}
      {$`G \seq e : E \quad `G \seq E, A : s$}
      {$E \sub A $}
      {$`G \seq (f~e) : B [ e / y ]$}
      {}
    \end{prooftree}
    \typenvi

    Par induction:
    \[\Gr \typea f[u/x] : F' `^
    `E `a_f : F' \suba F[u/x],
    `a_f~\ip{f[u/x]}{\Gr} \eqbr \ip{f}{\GD}\cux\]
    \[\Gr \typea e[u/x] : E' `^
    `E `a_e : E' \suba E[u/x],
    `a_e~\ip{e[u/x]}{\Gr} \eqbr \ip{e}{\GD}\cux\]
    
    \def\afe{`a}    
    Par définition de la substitution et de l'interprétation, 
    \begin{eqnarray*}
      \ip{(f~e)}{\GD}
      & = & (((\pi_F~\ip{f}{\GD})~(c_e~\ip{e}{\GD})) \\
      \text{ où } & & \\
      \pi_F & = & \sref{coerce}~F~(\Pi y : A.B) \\
      c_e & = & \sref{coerce}~E~A.
    \end{eqnarray*}
    
    Clairement, $\pi_F$ et $c_e$ sont dérivables, puisqu'on part de
    jugements dérivables par la coercion algorithmique.
    
    Le lemme \ref{subst-coerce-algo} nous donne la dérivation suivante
    pour le jugement substitué:
    \typenva
    \begin{prooftree}
      \TAX{App}
      {$`G \seq f[u/x] : F' \quad \mualgo(F') = \Pi y : A'. B'$}
      {$`G \seq e[u/x] : E' \quad `G \seq E', A' : s$}
      {$E' \sub A' $}
      {$`G \seq (f~e)[u/x] : B' [ e[u/x] / y ]$}
      {}
    \end{prooftree}
    \typenvi
    
    Soit $e' = e[u/x]$ et $f' = f[u/x]$, on a donc d'autre part:
    \begin{eqnarray*}
      \ip{(f~e)[u/x]}{\Gr}
      & = & \ip{f'~e'}{\iG} \\
      & = & (\pi_{F'}~\ip{f'}{\Gr})~(c_{e'}~\ip{e'}{\Gr}) \\
      \text{ où } & & \\
      \pi_{F'} & = & \sref{coerce}~F'~(\Pi y : A'.B') \\
      c'' & = & \sref{coerce}~E'~A'
    \end{eqnarray*}
    
    On a donc les coercions suivantes:
    \[
    \xymatrix{
      F[u/x]\ar[rr]_{\pi_F\cux} & & 
      \Pi y : A[u/x].B[u/x] \\
      F'\ar[u]^{`a_f}\ar[rr]^{\pi_{F'}} & &
      {\Pi y : A'.B'}\ar@{-->}[u]^{c_f}}
    \]

    Par symmétrie et transitivité de la coercion, on en déduit qu'il
    existe $c_f : \Pi y : A'.B' \suba \Pi y : A[u/x].B[u/x]$.
    $c_f$ est nécessairement de la forme:
    $\lambda f.\lambda y.c_2~(f~(c_1~y))$ où $c_1 : A[u/x] \suba A'$ et
    $c_2 : B' \suba B[u/x]$.

    On a donc les coercions suivantes pour l'argument $e$:
    
    \[
    \xymatrix{
      E[u/x]\ar[rr]_{c_e\cux} & & 
      A[u/x]\ar[d]^{c_1} \\
      E'\ar[u]^{`a_e}
      \ar@{-->}[rr]^{c_{e'}} & & A'}
    \]

    \def\ipGr#1{\lbag #1 \rbag}
    \def\cuxobj#1{#1[`r]}
    Soit $\ipGr{t} = \ip{t}{\Gr}$ et $`r = \cux$.

    On en déduit que $c_{e'} = c_1 `o \cuxobj{c_e} `o `a_e$.    
    Soit $`a = c_2[\cuxobj{c_e}~(`a_e~\ipGr{e'})/y]$.
    On a $\Gr, y : A[u/x] \typec c_2 : B' \suba B[u/x]$, donc
    \[\Gr \typec c_2[\cuxobj{c_e}~(`a_e~\ipGr{e'})/y] : B'[e'/y] \suba
    B[u/x][e'/y]\]
    et $B[u/x][e'/y] = B[e/y][u/x]$ car $y$
    n'apparait pas dans $u$, et $e' = e[u/x]$. On peut appliquer
    l'hypothèse de récurrence car par inversion de $\Gr \typea f' : F'$
    avec $\mualgo(F') = \Pi y : A'.B'$ on a $\Gr, y : A' \typea B' : s$ et par restriction, $\Gr, y :
    A[u/x] \typea B' : s$ avec une dérivation de même taille.


    On peut vérifier:
    \[\begin{array}{ll}
      \firsteq{\afe~\ip{(f~e)[u/x]}{\Gr}}
      
      \step{Définition de l'interprétation}
      {=}{\afe~((\pi_{F'}~\ipGr{f'})~(c_{e'}~\ipGr{e'}))}
      
      \step{Définitions de $`a$ et $c_{e'}$}
      {=}{c_2[\cuxobj{c_e}(`a_e~\ipGr{e'})/y]~((\pi_{F'}~\ipGr{f'})~((c_1 `o \cuxobj{c_e} `o `a_e)~\ipGr{e'}))}
      
      \step{Définition de la composition}
      {=}{c_2[\cuxobj{c_e}(`a_e~\ipGr{e'})/y]~((\pi_{F'}~\ipGr{f'})~(c_1~(\cuxobj{c_e}~(`a_e~\ipGr{e'}))))}

      \step{Définition de $c_f$ ($= \lambda f.\lambda
        y.c_2~(f~(c_1~y))$)}
      {=}{c_f~(\pi_{F'}~\ipGr{f'})~(\cuxobj{c_e}~(`a_e~\ipGr{e'}))}
      
      \step{Commutation du diagramme 1}{=}
      {(\cuxobj{\pi_F}~(`a_f~\ipGr{f'}))~(\cuxobj{c_e}~(`a_e~\ipGr{e'}))}
      
      \step{Définitions de $`a_f$ et $`a_e$}{=}
      {(\cuxobj{\pi_F}~\ip{f}{\GD}\cux)~(\cuxobj{c_e}~\ip{e}{\GD}\cux)}

      \step{Définition de l'interprétation}{=}{\ip{f~e}{\GD}\cux}
      \vspace{0.2em}
    \end{array}\]
    
    \vspace{1cm}
    
    \TODO{Autres cas}

    \def\None{\sref{None}}
    \def\Some{\sref{Some}}
    \def\lift{\sref{lift}}
    \def\appopt{\sref{app\_opt}}
    \case{SubConv}
    Alors $T \eqbr T'$ donc $d = \None$. Comme $T[u/x] \eqbr T'[u/x]$, 
    $d\cux = d = \None = \coerce~(T[u/x])~(T'[u/x])$.
    
    \case{SubHnf}
    Par induction, $d\cux = \coerce~(\Gr)~(\hnf{T})[u/x]~(\hnf{T'})[u/x]$.
    On doit montrer que $\coerce~(\Gr)~T[u/x]~T'[u/x] = d\cux$.    
    Deux cas sont possibles:
    \begin{itemize}
    \item Si $(\hnf{T})[u/x] =
      \hnf{T[u/x]}$ alors on a la même dérivation que $d$. On vérifie
      juste $u \eqbr u$ à la place de $x \eqbr x$ aux axiomes, donc $d\cux
      = d$.
    \item Sinon,
      c'est que $\hnf{T} = x$, $\hnf{T[u/x]} = \hnf{u}$ et $(\hnf{T})[u/x]
      = u$. Dans ce cas, $\hnf{T'} = x$ sinon $T \suba T'$ ne serait pas
      dérivable. On doit dériver $T[u/x] \suba T'[u/x]$. 
      
      \begin{itemize}
      \item Si $T = x$ alors
        $T' = x$ et l'on doit dériver $u \suba u$ ce qui est direct par
        \irule{SubConv}: $d\cux = d = \sref{None}$.
        
      \item Sinon, on doit appliquer $\hnf{\_}$ et dériver:
        $\hnf{u} \suba \hnf{u}$. Encore une fois une application de
        \irule{SubConv} suffit et le terme engendré est $\None$.
      \end{itemize}
    \end{itemize}
    
    \case{SubProd}
    On a $T = \Pi y : U.V$ et $T' = \Pi y : U'.V'$ avec
    $c_1 = \coerce~(\GD)~U'~U$, $c_2 = \coerce~(\GD, y : U')~V~V'$ et
    \[d = \Some (\funml~f "=>"
    \sref{mkLambda}~(y, \ip{U'}{\GD},
    \appopt~c_2~((\lift~1~f)~(\appopt~c_1~y))))\].
    
    Par induction, 
    \[\coerce~(\Gr)~(U'[u/x])~(U[u/x]) = c_1\cux\]
    \[\coerce~(\Gr, y : U'[u/x])~(V[u/x])~(V'[u/x]) = c_2\cux\]

    Or par définition $\coerce~(\Gr)~T[u/x]~T'[u/x] =$
    \begin{eqnarray*}
       & &
      \coerce'~(\Gr)~(\Pi y : U[u/x].V[u/x])~(\Pi y : U'[u/x].V'[u/x])
      \\
      & = & (\funml~f "=>"
      \sref{mkLambda}~(y, \ip{U'[u/x]}{\Gr},
      \appopt~(c_2\cux)~((\lift~1~f)~(\appopt~(c_1\cux)~y)))).
    \end{eqnarray*}
    
    On a $\ip{U'[u/x]}{\Gr} = \ip{U'}{\GD}\cux$ par induction car $\Pi y :
    U'.V' : s$ implique $U' : s$ par inversion. On a donc bien $d\cux =
    \coerce~(\Gr)~(T[u/x])~(T'[u/x])$.
  \end{induction}
  

\end{proof}

On peut maintenant montrer notre théorème de correction.

\setboolean{displayLabels}{false}
\begin{theorem}[Correction de l'interprétation]
  \label{correct-interp}
  Si $`G \typea t : T$ alors on a $\iG \typec \ip{t}{\iG} :
  \ip{T}{\iG}$.
  Si $\wf `G$ alors $\wf \iG$.
\end{theorem}

\begin{proof}
  Par induction mutuelle sur la dérivation de typage ou de bonne
  formation.

  \begin{induction}
    \case{WfEmpty} Trivial.

    \case{WfVar}
    Par induction $\iG \typec \ip{A}{\iG} :
    \ip{s}{\iG}$. Par inversion du jugement de bonne formation
    dans \CCI{}, $\typewf \iG$.
    Or, $\ip{s}{\iG} = s$ ($s `: \{ \Prop, \Set, \Type \}$), donc 
    on peut appliquer \irule{WfVar} dans \CCI{} pour obtenir:
    $\wf \iG, x : \ip{A}{\iG}$, soit $\wf \ipG{`G, x : A}$.

    \case{PropSet}
    Direct par induction, $\iG \typec \ip{s}{\iG} = s :
    \ip{\Type}{\iG} = \Type$. La deuxième condition est directe
    par définition de l'interprétation.
    
    \case{Var} On a:
    \begin{prooftree}
      \Var
    \end{prooftree}
    Par induction, $\wf \iG$ et par simple inspection de la
    définition de l'interprétation des contextes, si $x : A `: `G$ alors
    $x : \ip{A}{`D} `: \iG$ pour $`D ``( `G$. Par affaiblissement
    dans \CCI{}, on obtient aisément $\iG \typec x : \ip{A}{`D}$ 
    à partir de  $\ipG{`D} \typec x : \ip{A}{\ipG{`D}}$. Or il est clair par
    la définition de l'interprétation que $\ip{A}{`D} = \ip{A}{`G}$  
    si $`G$ est une extension de $`D$ puisqu'on utilise l'environnement
    uniquement au moment de typer les variables et
    $\iG(x) = \ipG{`D}(x)$ pour tout $x$ utilisé dans $A$.
    On a donc $\iG \typec x : \ip{A}{\iG}$.
    La deuxième condition est montrée par $\ip{x}{`G} = \iG(x) = 
    \ip{A}{\ipG{`D}} = \ip{A}{\iG}$.
      
    \case{Prod} On a:
    \begin{prooftree}
      \Prod
    \end{prooftree}
    
    Par induction $\iG \typec \ip{T}{\iG} : \ip{s_1}{\iG}
    = s_1$ et $\ipG{`G, x : T} \typec \ip{U}{\ipG{`G, x : T}} :
    \ip{s_2}{\ipG{`G, x : T}} = s_2$.
    On déplie l'interprétation pour obtenir:
    $\iG, x : \ip{T}{\iG} \typec \ip{U}{\iG{}, x : \ip{T}{\iG}} :
    s_2$.
    
    Par \irule{Prod} dans \CCI, on obtient:
    $\iG{} \typec \Pi x : \ip{T}{\iG}. \ip{U}{\iG{}, x : \ip{T}{\iG}}
    : s_3$.
    Or $\ip{\Pi x : T.U}{\iG} = \Pi x : \ip{T}{\iG}. \ip{U}{\iG{}, x
      : \ip{T}{\iG}}$, donc on a bien:
    $\iG \typec \ip{\Pi x : T.U}{\iG} : s_3 = \ip{s_3}{\iG}$.
    La seconde condition est directe d'après la définition de $\mathcal{R}$ qui
    est une fonction.

    \case{Abs} On a:
    \begin{prooftree}
      \Abs
    \end{prooftree}
    
    Par induction $\iG \typec \ip{\Pi x : T.U}{\iG} : \ip{s}{\iG}$
    et $\ipG{`G, x : T} \typec \ip{M}{\ipG{`G, x : T}} :
    \ip{U}{\ipG{`G, x : T}}$.
    On déplie l'interprétation pour obtenir:
    $\iG, x : \ip{T}{\iG} \typec \ip{M}{\iG{}, x : \ip{T}{\iG}} :
    \ip{U}{\ipG{`G, x : T}}$.
    
    Par \irule{Abs} dans \CCI, on obtient:
    $\iG{} \typec \lambda x : \ip{T}{\iG}. \ip{M}{\iG{}, x : \ip{T}{\iG}}
    : \ip{\Pi x : T.U}{\iG}$.
    C'est équivalent à:
    $\iG \typec \ip{\lambda x : T.U}{\iG} : \ip{\Pi x : T.U}{\iG}$.
    La seconde condition se montre ainsi:
    $\ip{\lambda x : T.M}{\iG} = \Pi x : \ip{T}{\iG}. \ip{M}{\iG, x :
      \ip{T}{\iG}}$. Or par induction, 
    $\ip{M}{\iG, x : \ip{T}{\iG}} = \ip{U}{\iG, x : \ip{T}{\iG}}$, donc 
    $\ip{\lambda x : T.M}{\iG} = \Pi x : \ip{T}{\iG}. \ip{U}{\iG, x :
      \ip{T}{\iG}} = \ip{\Pi x : T.U}{\iG}$.
    
    \case{App} On a:
    \begin{prooftree}
      \AppA
    \end{prooftree}

    C'est le cas intéressant puisqu'on ajoute ici des coercions.
    Par induction, $\iG{} \typec \ip{f}{\iG} : \ip{T}{\iG}$,
    $\iG{} \typec \ip{u}{\iG} : \ip{U}{\iG}$,
    $\iG{} \typec \ip{U}{\iG}, \ip{V}{\iG} : \ip{s}{\iG}$.
    
    On procéde par étapes: d'abord la construction d'une fonction
    $\pi \ip{f}{\iG} : \ip{\Pi x : V.W}{\iG}$ puis la construction de
    l'argument $c \ip{u}{\iG} : \ip{V}{\iG}$. On n'a plus qu'à les
    appliquer pour obtenir le jugement:
    $\iG{} \typec \ip{f~u}{\iG} : \ip{W}{\iG, x :
      \ip{V}{\iG}}[c~u/x]$.
    Par substitutivité de l'interprétation, on a
    $\ip{W}{\iG, x : \ip{V}{\iG}}[c~u/x] \eqbr
    \ip{W[u/x]}{\iG}$ puisque les coercions de sorte à sorte ne peuvent
    être que l'identité. On peut donc déduire:
    $\iG{} \typec \ip{f~u}{\iG} : \ip{W[u/x]}{\iG}$ par \irule{Conv}.

    \item[- \irule{Sigma}, \irule{Sum}, \irule{LetSum}, \irule{Subset}:]
      Direct par induction ou un raisonnement similaire à \irule{App}.      
  \end{induction}
\end{proof}

%%% Local Variables: 
%%% mode: latex
%%% TeX-master: "subset-typing"
%%% End: 
