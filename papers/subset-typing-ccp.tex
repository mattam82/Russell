\chapter{Le calcul de coercion par pr�dicats}
Nous avons d�velopp� un langage supportant le \ps{} utilisable dans
\Coq. L'utilisateur peut d�finir des programmes dans un langage plus
souple puis prouver certains buts pour obtenir finalement un terme de
\CCI{} complet v�rifiable par le noyau. On peut finalement utiliser 
les types d�pendants comme des types
simples et s'occuper des d�pendances dans un deuxi�me temps (pour la preuve).
L'architecture de notre syst�me est la suivante:
on type le programme dans notre langage \lng{} o� l'on peut faire des
abus de notations avec les objets de type sous-ensemble, puis l'on r��crit le terme typ�
dans \CCI{} en laissant des ``trous'' dans les termes qui d�sambig�ent
les abus et enfin \Coq{} se charge de g�n�rer les obligations correspondant � ces trous.


On va donc tout d'abord pr�senter le langage \lng{}, puis un algorithme
de typage correct et complet pour les programmes �crits en \lng{}. Ensuite on montrera comment
plonger ce langage dans \CCI{} en ajoutant les coercions ad�quates et
enfin on expliquera comment se d�roule la g�n�ration des obligations de
preuves � partir des termes engendr�s par le plongement.

\setboolean{displayLabels}{true}

\section{Le langage \lng{}}
\label{section:russel}
Le langage que nous voulons est tr�s proche de \ML{}, plus les annotations
n�cessaires pour avoir un typage pr�cis et d�cidable. On �tudie ici une
restriction de \ML{}, purement fonctionnelle et sans filtrage, qu'on
�tendra dans la suite de notre travail. On n'a donc pas de types
inductifs mais on consid�re les types $\Sigma$, g�n�ralisation des
tuples de \ML{} form�s par l'op�rateur $*$.

\subsection{Syntaxe}
La syntaxe (figure \ref{fig:syntax}) est directement inspir�e des langages fonctionnels.
On part du \lc{} (variables, abstraction et application) puis l'on
ajoute des constantes (pour les entiers, bool�ens, etc...) ainsi que les
couples. La syntaxe $(x := `a, t : `t)$ permet de
cr�er des paires d�pendantes, de type $\Sigma x : `t. `t$. On peut aussi
appliquer un terme � un type pour instancier une fonction polymorphe par exemple.

Du c�t� des types, on a tout d'abord les types simples (constantes,
fl�che, produit cart�sien) qui sont des cas particuliers du produit ($\Pi$) et
de la somme ($\Sigma$) d�pendants. Les variables introduites par ces
types peuvent �tre utilis�es lors des applications de types. On
peut de plus abstraire sur les types avec le $\lambda$ (polymorphisme)
et les sortes.
Enfin on peut appliquer un type � un terme ($`t~`a$). 

%\vspace{-0.5cm}
\begin{figure}[ht]
  \begin{center}
    \subfigure[Termes]{\termgrammar}\quad
    \subfigure[Types]{\typegrammar}
  \end{center}
  \caption{Syntaxe}
  \label{fig:syntax}
\end{figure}
% \vspace{-1cm}

\subsection{S�mantique}
\typenvd

\typedFig
\subtdFig

La s�mantique du langage nous est donn�e par un syst�me de typage
(figure \ref{fig:typing-decl-rules} page \pageref{fig:typing-decl-rules}). Le
jugement de typage est d�fini inductivement par un ensemble de r�gles
d'inf�rence. 
Dans notre cas ce sont les r�gles du \CCfull{} (\CC{})
�tendu avec les $\Sigma$-types auxquelles on a ajout� une r�gle de
coercion (\irule{Coerce}, figure \ref{fig:typing-decl-rules}) que l'on trouve classiquement dans les syst�mes avec
sous-typage avec le nom de subsumption. 
Le jugement $\Gamma \typed t : T$ se lit: dans l'environnement $\Gamma$,
$t$ est de type $T$.

% \begin{remark}
%   En pratique, les types du Calcul des Constructions ne sont pas
%   toujours en forme normale et il peut donc �tre n�cessaire de les
%   r�duire (en t�te seulement) pour v�rifier des jugements du genre: 
%   $`G \seq t : \Pi x : T.V$.
% \end{remark}

La relation $\mathcal{R}$ d�finissant les produits formables
dans le syst�me est d�finie par les r�gles suivantes:
\[\begin{array}{cccll}
  s_1 & s_2 & s_3 & \text{Habitants} & \text{Exemple} \\
  \hline
  \Prop & \Prop & \Prop & \text{Implication logique} & x <= 0 "->" x = 0  \\
  \Set & \Set & \Set & \text{Fonctions} & \Pi x : \nat. \nat \\
  \Type & \Set & \Type & \text{Fonctions polymorphes} & \Pi A : \Set, A
  "->" A \\
  \Set & \Type & \Type & \text{Types d�pendants} & \sref{vector} : \nat "->" \Set : \Set "->" \Type \\
  \Set & \Prop & \Prop & \text{Termes dans les propositions} & 
  \Pi n : \nat. \Pi l : \text{list}~n. \text{length}~l = n \\
  \Type & \Prop & \Prop & \text{Impr�dicativit� de } \Prop & \Pi x : Prop. x `V `! x \\
%  \Type(i) & \Type(j) & \Type(max~i~j)  & \text{Connecteurs logiques,
%    \ldots} & \Pi A : \Prop. \Pi B : \Prop. A `^ B "->" B `^ A
\end{array}\]
On a un syst�me proche du \CCfull{} avec types $\Sigma$, mais
avec \Set{} pr�dicatif (comme dans \Coq{}).
On n'a pas $(\Prop,\Set,\Set)$ dans notre relation $\mathcal{R}$ pour
une bonne raison. Cela permet de cr�er des fonctions d�pendant de
propositions, par exemple $\Pi n : \nat, n > 0 "->" \Pi l :
\text{list}~A~n "->" A$. Or on veut � tout prix �viter d'introduire des
termes de preuve dans notre langage, et l'on voit que
cette fonction pourrait naturellement s'�crire $\Pi n : \mysubset{n}{\nat}{n > 0} "->" \Pi l :
\text{list}~A~n "->" A$. Encore une fois le type sous-ensemble nous
permet d'�viter d'avoir � passer des termes de preuve directement. 

Les sommes formables dans le syst�me sont r�duites au couples d'objets de
types de m�me sorte $s `: \{ \Prop, \Set \}$.
Dans le premier cas les habitants sont les couples de propositions
(codage du $`^$), dans le second ce sont les couples d'objets, soit les
paires de \ML.
Intuitivement, c'est le type sous-ensemble $\mysubset{x}{T}{P}$ qui permet
de faire des couples $\Set,\Prop$ habitant $\Set$. Les types $\Sigma x : U.V$ o� $U
: \Prop$ et $V : \Set$ n'ont pas d'int�r�t dans notre cas puisqu'ils
repr�sentent des objets de type $U `^ V$ mais on ne peut
pas utiliser $U$ dans notre syst�me. On pr�f�re coder ces objets par des
objets de type $\mysubset{x}{V}{U}$ (on n'est pas int�ress� par la preuve
de $U$ pour programmer).


La r�gle \irule{Coerce} formalise l'id�e que 
l'on peut utiliser un terme de type $T$ � la place d'un terme de type
$U$ si $T$ et $U$ sont dans une certaine relation. C'est l�
qu'interviendront les types sous-ensemble. \CC{} contient une r�gle de typage
similaire � \irule{Coerce}, la r�gle de conversion (\irule{Conv}), qui
dit essentiellement que deux types
$`b$-convertibles (on rappelle que l'on peut calculer dans les types
puisqu'on a l'abstraction, l'application, etc...) sont �quivalents.
On peut directement int�grer cette relation de $`b$-convertibilit� � notre
syst�me de coercion comme montr� figure \ref{fig:subtyping-decl-rules}
(\irule{SubConv}), � condition d'avoir l'inclusion
$`=_\beta~\subseteq~\subd + \text{\irule{SubConv}}$.
En fait notre notion de r�duction est un peu plus large que $\beta$
puisqu'on peut r�duire les $\sref{let}$:
$\letml~(x,y) = (u, v)~\inml~t$ se r�duit en $t[u/x][v/y]$. En
pratique cette constructions est du sucre syntaxique pratique au niveau
du typage (on peut inf�rer le type de $t$), mais elle est inessentielle au
niveau du calcul.
On peut ais�ment rajouter un $\letml~x=t~\inml~v$ � notre langage de
fa�on similaire: c'est �quivalent � $(\lambda x : T.v)~t$, mais $T$ peut
est inf�r� plut�t que donn� par l'utilisateur.

On consid�re les constantes comme des variables pr�d�finies 
dans nos contextes, par exemple on a la constante $\sref{list} : \Pi x :
nat. \Set$. 
L'ajout d'une constante � un contexte ne doit pas alt�rer sa
bonne formation comme pour le cas des variables, donc son type doit �tre
bien form� (en g�n�ral, toute d�finition de \Coq~donne lieu � une
constante dans notre syst�me si elle est bien typ�e).

\subsubsection{Jugement de coercion}
Notre syst�me de coercion par pr�dicats permet � l'utilisateur
d'utiliser une valeur de type $U$ l� o� l'on attend une valeur de type
$\mysubset{x}{V}{P}$ (\irule{SubProof}) si $U$ est lui-m�me coercible en $V$.
A l'inverse, on permet aussi d'utiliser une valeur de type
$\mysubset{x}{U}{P}$ (\irule{SubSub}) � la place d'une valeur de type
$V$ si $U$ est coercible vers $V$. Notre jugement de coercion est donc
sym�trique et laisse beaucoup de libert� � l'utilisateur au moment du
codage. Par exemple on peut d�river $u : \nat \type u : \mysubset{x}{\nat}{`_}$
Seulement, lors de la traduction de la d�rivation de coercion $\nat
\subd \mysubset{x}{\nat}{`_}$ (n�cessaire pour traduire l'abus de notation
$x : \mysubset{x}{\nat}{`_}$), l'utilisateur aura � r�soudre une obligation
de preuve de $`_$. On repose donc toujours sur la coh�rence du Calcul
des Constructions. 
Les r�gles \irule{SubProd} et \irule{SubSigma} permettent de faire des
coercions dans les types composites. Classiquement, la r�gle pour le 
produit fonctionnel est contravariante (une fonction sous-type d'une
autre accepte plus d'entr�es mais donne une sortie plus fine, voir
 \cite{journals/toplas/Castagna95}) et la r�gle pour le 
produit cart�sien covariante (une paire est coercible en une autre si 
leurs composantes sont coercibles deux-�-deux). Le sens des coercions
n'a pas d'importance dans le syst�me d�claratif puisqu'il est sym�trique
mais il est essentiel lors de la cr�ation des coercions que nous
d�crirons plus tard.

La r�gle \irule{SubTrans} assure que l'on a un syst�me compositionnel. Il y a ici une
analogie avec l'�limination des coupures dans les syst�mes logiques, o�
l'on montre que toute d�rivation utilisant la r�gle de \emph{modus ponens} ($A "=>" B$ et $B "=>" C$ implique
$A "=>" C$) peut se r��crire en une d�rivation ne l'utilisant
jamais. Dans les syst�mes � sous-typage, on montre de fa�on �quivalente
que l'on peut �liminer la r�gle de transitivit� ; premi�re �tape vers un
syst�me d�cidable.


Notre jugement de coercion identifie les types $U$ et $\mysubset{x}{U}{P}$
mais notre syst�me de typage ne permet pas d'�liminer (prendre la partie
preuve) ou d'introduire (cr�er un couple t�moin,preuve) des objets de
type sous-ensemble. Cela nous assure une certaine coh�rence, puisque
m�me si l'on ne v�rifie pas qu'un objet de type $U$ a bien la propri�t�
$P$, on ne peut pas raisonner sur le fait que $U$ a la propri�t� dans le
langage.


On ne fera pas la m�tath�orie du syst�me d�claratif ici, puisque
c'est une extension conservative du Calcul des Constructions et l'on
�tudiera en d�tail le syst�me algorithmique. Notre preuve de conservativit�
est simple: si l'on oublie les utilisations des types sous-ensemble de
notre syst�me de typage (\irule{Subset}) et de coercion
(\irule{SubProof} et \irule{SubSub}), alors le jugement de coercion est 
juste la $\beta$-convertibilit� et donc \irule{Coerce} et \irule{Conv} 
sont �quivalentes. Comme les autres r�gles de notre syst�me d�claratif
proviennent directement de \CC{}, on arrive � un syst�me strictement
�gal au syst�me du calcul des constructions. On peut donc s'appuyer sur
les r�sultats connus pour \CC{} pour une partie de notre syst�me.

Pour une �tude compl�te du \CCfull{}, se r�f�rer �
\cite{Barras99,Luo90}.
On va plut�t s'int�resser � la construction d'un algorithme de typage
correspondant � notre syst�me d�claratif.

% \section*{Propri�t�s �lementaires}

% \begin{fact}[Inversion du typage]
%   \label{inversion-typing-d}
%   On a les propri�t�s suivantes sur le jugement de typage:
%   \begin{enumerate}
%   \item Si $`G \type \lambda x : T.v : \Pi x : T.U : s$ alors $`G, x : T \type
%     v : U : s$.
%   \item Si $`G \type (t, v) : \Sigma x : T.U$ alors $`G, x : T \type U
%     : s1$ et $`G \type v : U[t/x]$.
%   \item Si $`G \type t : \mysubset{x}{U}{P}$ alors $`G \type t : U$ et
%     $`G, x : U \type P : \Prop$.
%   \end{enumerate}
% \end{fact}

% \begin{fact}[Convertibilit�]
%   \label{type-convertibility}
%   Si $`G \type T : s$ et $s `=_\beta s'$ alors $`G \type T : s'$.
% \end{fact}

% \begin{lemma}[Inversion pour le produit]
%   \label{inversion-prod-d}
%   Si $`G \type \Pi x : T.U : s$ alors $s `: \setproptype$, $`G \type T : t$ et 
%   $`G, x : T \type U : s$.
% \end{lemma}
% \begin{proof}
%   Par induction sur la taille de la d�rivation.
%   Les seules r�gles ayant pour conclusion possible un jugement de la
%   forme $\Pi x : T.U : s$ sont \irule{Prod} et \irule{Subsum}
%   (\irule{App} se termine par une application). Pour \irule{Prod} la
%   propri�t� est directe. Supposons que la derni�re r�gle appliqu�e fut
%   \irule{Subsum}. Alors il existe $s'$ tel que $`G \type \Pi x : T.U :
%   s'$ et par induction, $s' `: \setproptype$. Une
%   inspection des r�gles de sous-typage r�v�le que seules les
%   r�gles \irule{SubConv} et \irule{SubTrans} ont pu s'appliquer dans
%   la d�rivation $`G \type s' \sub s$. On en d�duit que $s `=_\beta s'$,
%   et il s'ensuit que $`G \type T : s'$ (par application de
%   \irule{Conv}). % \ref{type-convertibility}
%   On peut appliquer \irule{Subsum} � la fin de la d�rivation 
%   $`G, x : T \type U : s'$ pour obtenir le r�sultat $`G, x : T \type U :
%   s$.
  
  % La d�rivation est donc de la forme:
%   \begin{prooftree}
%     \AXC{$t `=_\beta s'$}
%     \UIC{$t \sub s'$}
%     \AXC{$\ldots$}
%     \AXC{$s `=_\beta t$}
%     \UIC{$s \sub t$}
%     \TIC{$s \sub s'$}
%   \end{prooftree}


%   Comme les classes de
%   $\beta$-�quivalence des �l�ments de $\setproptype$ sont r�duites � un
%   �l�ment,

% \end{proof}

% \TODO{Pas utilis�!}

% On montre tout d'abord qu'il existe des types principaux dans notre syst�me.

% \begin{lemma}[Principalit� du typage]
%   Pour tout $`G, t$, il existe $T$ tel que si $`G \typed t : T$ alors 
%   pour tout $S$ tel que $`G \typed t : S$ alors $T \subd S$.
% \end{lemma}

% \begin{proof}
%   Ce r�sultat d�coule du fait qu'il existe des sous-types principaux et
%   des sortes principales dans notre syst�me.
% \end{proof}

% \begin{lemma}[Unicit� du sortage]
%   Si $`G \typed T : s_1$ et $`G \typed T : s_2$ alors $s_1 = s_2$.
% \end{lemma}

% \begin{proof}
%   \begin{induction}{typing-decl}
%     \casetwo{PropSet}{Type}
%     Aucune autre r�gle ne peut typer $T$, donc $s_1 = s_2$
    
%     \case{Var} 
%     Les r�gles de conversion et de subsumption ne permettent pas de
%     typer $s1$...
    
%     \case{Prod}
%     \case{Abs}
%     \case{App}
%     \case{LetIn}
%     \case{Sigma}
%     \case{Sum}
%     \case{LetSum}
%     \case{Subset}
%     \case{Subsum}
%   \end{induction}
% \end{proof}


% \begin{lemma}[Sous-typage bien sort�]
%   \label{subtyping-sorts-d}
%   Si $S \subd T$, $`G \typed S : s_1$ et $`G \typed T : s_2$ alors $s_1 =
%   s_2$.
% \end{lemma}

% \begin{proof}
%   \begin{induction}[subtyping-decl]
%     \case{SubEq} Par unicit� du typage.

%     \case{SubTrans} Trivial.
    
%     \casetwo{SubProd}{SubSigma} Par induction les composantes
%     correspondantes sont dans la m�me sorte, donc les compos�s aussi.
    
%     \casetwo{SubProof}{SubSub} Comme le constructeur de types subset
%     est de type $\Set "->" \Prop "->" \Set$, on conserve bien les m�mes
%     sortes de part et d'autre.
%   \end{induction}
% \end{proof}

% \begin{lemma}[Bonne formation des contextes]
%   \label{wf-contexts-d}
%   Si $`G \type t : T$ alors $\typewf `G$.
% \end{lemma}
% \begin{proof}
%   \inductionon{typing-decl}
% \end{proof}

% \begin{fact}[Inversion du jugement de bonne formation]
%   \label{inversion-wf-d}
%   Si $\typewf `G, x : U$ alors $`G \type U : s$ et $s `: \{ \Set, \Prop, \Type(i) \}$.
% \end{fact}

% L'affaiblissement est n�cessaire pour montre le lemme de
% renforcement. Il �tablit que tout jugement peut �tre d�riv� dans un
% contexte �tendu par de nouvelles d�clarations.

% \begin{lemma}[Affaiblissement]
%   \label{weakening-d}
%   Si $`G, `D \type t : T$ alors pour tout $x : S `; `G, `D$ tel que
%   $\wf `G, x : S, `D$, $`G, x : S, `D \type t : T$
% \end{lemma}

% \begin{proof}
%   \begin{induction}[typing-decl]
%     \casetwo{PropSet}{Type} Trivial.

%     \case{Var}
%     On a $x : S `; `G, `D$, donc $`G, x : S, `D \type y : T$ est toujours d�rivable.
    
%     \case{Prod}
%     Par induction $`G, x : S, `D \type T : s1$ et $`G, x : S, `D,
%     y : T \seq U : s2$. On applique \irule{Prod} pour obtenir 
%     $`G, x : S, `D \type \Pi x : T.U : s2$. De m�me pour le reste des r�gles.

% %     \case{Abs}
% %     \case{App}
% %     \case{LetIn}
% %     \case{Sigma}
% %     \case{Sum}
% %     \case{LetSum}
% %     \case{Subset}
% %     \case{Subsum}
    

%   \end{induction}
% \end{proof}  

% Le renforcement montre que notre notion de sous-typage est correcte
% vis-�-vis du typage. On peut d�river les m�mes jugements dans des
% contextes o� les variables ont des types plus pr�cis.

% \begin{lemma}[Renforcement]
%   \label{narrowing-d}
%   \[ `G \seq S, S' : s, S' \sub S "=>" 
%   \left\{ \begin{array}{lcl}
%       \typewf `G, x : S, `D & "=>" & \typewf `G, x : S', `D \\
%       & `^{} & \\
%       `G, x : S, `D \seq t : T & "=>" & `G, x : S', `D \seq t : T
%     \end{array}
%   \right. \]
% \end{lemma}

% \begin{proof}
%   Par induction sur la taille de la d�rivation de typage ou de bonne formation.
    
%   \begin{induction}
%     \case{WfEmpty} Trivial.
    
%     \case{WfVar} 
%     La conclusion est $\typewf `G, x : S, `D$
    
%     \begin{induction}[text=Par induction sur la taille de $`D$]
%     \item[\protect{$`D = []$}]
%       La racine de la d�rivation est de la forme:
%       \begin{prooftree}
%         \UAX{WfVar}
%         {$`G \type S : s$}
%         {$\wf `G, x : S$}
%         {$s `: \{ \Set, \Prop, \Type(i) \}$}
%       \end{prooftree}
%       On a $`G \type S' : s$, donc par \irule{WfVar}, $\typewf `G, x : S'$.  

%     \item[\protect{$`D `= `D', y : U$}]
%       La racine de la d�rivation est de la forme:
%       \begin{prooftree}
%         \UAX{WfVar}
%         {$`G, x : S, `D' \type U : t$}
%         {$\wf `G, x : S, `D', y : U$}
%         {$s `: \{ \Set, \Prop, \Type(i) \}$}
%       \end{prooftree}
%       Par induction sur la d�rivation de typage $`G, x : S', `D' \seq U : t$,
%       on a donc bien $\typewf `G, x : S', `D', y : U$ par \irule{WfVar}.
%     \end{induction}
    
%     \casetwo{PropSet}{Type} 
%     Par induction, $\typewf `G, x : S', `D$, on applique simplement la r�gle.
    
%     \case{Var}
%     Par induction, $\typewf `G, x : S', `D$. La seule diff�rence avec le
%     contexte pr�cedent est le type associ� � $x$, donc si $t \not= x$, on
%     peut simplement r�appliquer \irule{Var}. Si $t `= x$ on construit la
%     d�rivation:

%     \begin{prooftree}
%       \BAX{Var}
%       {$\wf `G, x : S', `D$}
%       {$x : S' `: `G$}
%       {$`G, x : S', `D \seq x : S'$}
%       {}
%       \AXC{$`G, x : S', `D \type S : s$}
%       \AXC{$S' \sub S$} % `G \subt 
%       \TIC{$`G, x : S', `D \type x : S$}
%     \end{prooftree}
    
%     Par l'affaiblissement (lemme \ref{weakening-d}) et $`G \type S : s$,
%     on obtient la pr�misse $`G, x : S', `D \type S : s$.
    
%     \case{Prod} 
%     Par induction, $`G, x : S', `D \type T : s1$ et $`G, x : S', `D
%     y : T \seq U : s2$. On applique \irule{Prod} pour obtenir 
%     $`G, x : S' \type \Pi x : T.U : s2$. De m�me pour le reste des r�gles.
%   \end{induction}
% \end{proof}

% Maintenant que nous avons montr� que ce syst�me a bien les
% propri�t�s que l'ont veut pour la coercion par pr�dicats, on va le
% raffiner pour obtenir un algorithme de typage.


%$list nat \sub list {n:nat|n \neq 0}$ ?
%$list : Set -> Set$


%%% Local Variables: 
%%% mode: latex
%%% TeX-master: "subset-typing"
%%% LaTeX-command: "TEXINPUTS=\"style:$TEXINPUTS\" latex"
%%% End: 

\newpage
\section{�laboration du syst�me algorithmique et propri�t�s}
\typenva

Pour pouvoir implanter le typeur, il nous faut un syst�me dirig� par la
syntaxe. Ce n'est pas le cas du syst�me d�claratif, aussi bien pour le
typage que pour la coercion. On va donc raffiner ces syst�mes dans
l'optique d'en extraire un algorithme. On note $\typea$ le jugement de
typage algorithmique d�fini figure \ref{fig:typing-algo-rules} page
\pageref{fig:typing-algo-rules} et
$\suba$ le jugement de coercion algorithmique pr�sent� figure
\ref{fig:subtyping-algo-rules}. Ces deux syst�mes sont quelque peu
�loign�s des originaux, n�anmoins nous allons montrer qu'ils sont
corrects et complets vis-�-vis de leurs g�niteurs. La correction (si l'on
a une d�rivation d'un jugement dans le syst�me algorithmique, alors
c'est un jugement valide du syst�me d�claratif) est le
sens le plus facile � montrer, nous allons donc commencer par l�. On
d�crira ensuite la m�thode de construction des syst�mes algorithmiques
pour aboutir d'une part au th�or�me de compl�tude qui montre qu'on peut d�river les
m�mes jugements (� coercion pr�s) dans le syst�me algorithmique que dans
le syt�me d�claratif et d'autre part � la d�cidabilit� du jugement de
typage algorithmique.

Il nous a fallu changer quelque peu les r�gles pour obtenir le syst�me
algorithmique. En particulier, on a utilis� la fonction $\mu_0$ de \PVS{}
\cite{PVS-Semantics:TR} renomm�e $\mualgo$ (figure \ref{fig:mualgo-definition})
ici pour op�rer des
\emph{d�compr�hensions}. Cette fonction efface les constructeurs de type
sous-ensemble en t�te d'un type, par exemple: $\mualgo(\mysubset{f}{\nat
"->" \nat}{f~0 = 0}) = \nat "->" \nat$. On verra son utilit� dans la suite.

\subsection{Notations}
On note $\hnf{x}$ la mise en forme normale de t�te de $x$ selon la
r�duction d�finie pr�c�demment. On note $\hat{=}$ l'�galit�
d�finitionelle.


\setboolean{displayLabels}{true}
\typenva
\typeaFig
\typemuaFig
\subtaFig

\subsection{Correction}
On montre tout d'abord la correction de la coercion algorithmique qui
sera n�cessaire pour la correction du typage:

\begin{theorem}[Correction de la coercion]
  \label{correct-coercion}
  Si $U \suba V$ alors $U \subd V$.
\end{theorem}

\begin{proof}
  Les r�gles du syst�me algorithmique sont un sous-ensemble des r�gles
  du syst�me d�cla\-ratif, except� pour la r�gle \irule{SubHnf}.
  On utilise \irule{SubConv} et \irule{SubTrans} pour montrer son
  admissibilit� dans dans le syst�me d�claratif.
  \begin{prooftree}
    \AXC{$U \eqbr \hnf{U}$}  
    \UIC{$U~\subd \hnf{U}$}
    \AXC{$\hnf{U}~\subd \hnf{T}$}
    \AXC{$\hnf{T} \eqbr T$}
    \UIC{$\hnf{T}~\subd T$}
    \BIC{$\hnf{U}~\subd T$}
    \BIC{$U \subd T$}
  \end{prooftree}
\end{proof}

On a besoin d'un lemme sur l'op�ration $\mualgo$ d�finie figure
\ref{fig:mualgo-definition}.

\begin{lemma}[$\mualgo$ et coercion]
  \label{mu-coercion}
  $T \suba \mualgo(T)$.
\end{lemma}

\begin{proof}
  Il suffit de suivre la d�finition de $\mualgo$. La mise en forme
  normale de t�te est �quivalente � l'utilisation de \irule{SubHnf} dans notre jugement de
  coercion. $\mualgo$ est en fait l'application r�p�t�e de la r�gle \irule{SubSub}.
\end{proof}

\begin{lemma}[Conservation des sortes par $\mu$]
  \label{mu-sorts}
  Si $`G \type S : s$ alors $`G \type \mu~S : s$
\end{lemma}

\begin{proof}
  Par le simple fait que si $S = \mysubset{x}{U}{P}$ alors $S : \Set$ et
  $U : \Set$ (par \irule{Subset}), sinon $S = \mu~S$.
\end{proof}

\begin{theorem}[Correction du typage]
  \label{correct-typing}
  Si $`G \typea t : T$ alors $`G \typed t : T$
\end{theorem}

\setboolean{displayLabels}{false}
\begin{proof}
  \begin{induction}[typing-algo]
  \item[- \irule{WfEmpty},\irule{WfVar},\irule{PropSet},\irule{Var},\irule{Prod},\irule{Abs},
    \irule{Sigma}, \irule{Sum}:] r�gles inchang�es.
    
    \case{LetSum}
    On a 
    \begin{prooftree}
      \LetSumA
    \end{prooftree}
    
    Par induction, $`G \typed t : S$, et par le lemme \ref{mu-coercion}
    et la correction de la coercion $S \subd \Sigma x : T. U$.
    On peut donc d�river $`G \typed t : \Sigma x : T.U$ � l'aide de \irule{Coerce}.
    On peut directement appliquer \irule{LetSum} � cette pr�misse et �
    l'hypoth�se d'induction $`G, x : T, y : U \typed v : V$.
    
    \case{App} On a:
    \def\fCenter{\typea}
    \begin{prooftree}
      \AppA
    \end{prooftree}
    
    Par induction, $`G \typed f : T$, et $T \subd \Pi x : V. W$ par le
    lemme \ref{mu-coercion} et la correction de la coercion.
    On peut donc d�river $`G \typed f : \Pi x : V.W$ � l'aide de la r�gle
    \irule{Coerce}.
    Par le lemme \ref{correct-coercion}, et l'hypoth�se $`G \typed u :
    U$, on obtient $`G \typed u : V$ par \irule{Coerce}.
    Donc, par \irule{App}, on a bien $`G \typed f u : W[u/x]$.
  \end{induction}  
\end{proof}

On a prouv� que notre syst�me algorithmique �tait correct, c'est-�-dire
que ses jugements valides sont bien inclus dans ceux du syst�me
d�claratif, il faut maintenant montrer qu'il les inclut tous (ou presque!).

\begin{subsubsection}{Notations}
On introduit la notation $`G \typea T, U : s$ pour $`G \typea T : s `^
`G \typea U : s$.
\end{subsubsection}

\subsection{Compl�tude et d�cidabilit�}
On va maintenant repartir des syst�mes d�claratifs pour montrer comment
l'on a construit les syst�mes algorithmiques. 


On s'int�resse tout d'abord au jugement de coercion.
Pour rendre le jugement de coercion d�cidable, il faut traiter les r�gles
\irule{SubConv} et \irule{SubHnf} qu'on peut appliquer � n'importe quel 
moment et la r�gle \irule{SubTrans} qui n'est pas dirig�e par la syntaxe (il faut
``deviner'' un type $T$). Le syst�me de coercion algorithmique (figure
\ref{fig:subtyping-algo-rules}) est le m�me que le syst�me
d�claratif (figure \ref{fig:subtyping-decl-rules}) mais o� l'on n'applique 
\irule{SubConv} seulement si aucune autre r�gle ne
s'applique apr�s avoir appliqu� \irule{SubHnf} 
et sans la r�gle \irule{SubTrans}.

\subsubsection{D�cidabilit� et compl�tude de la coercion}
On va montrer que les deux syst�mes de coercion sont �quivalents vis-�-vis de la conversion. On montrera plus tard pourquoi on peut �liminer la
r�gle de transitivit�.

On rappelle qu'on consid�re qu'on peut appliquer la normalisation de
t�te \irule{ConvHnf} avant toute
application d'une r�gle de typage ou coercion. Ainsi on peut tr�s bien
d�river $((\lambda x : \Set.x)~\nat) \suba \mysubset{x}{\nat}{P}$ puisque 
$\hnf{((\lambda x : \Set.x)~\nat)} = \nat$ et $\nat \suba
\mysubset{x}{\nat}{P}$ par \irule{SubProof} et \irule{SubConv}.

Il nous faut tout d'abord des lemmes d'inversion sur la conversion:
\begin{lemma}
  \label{conversion-pi}
  Si $\Pi x : T. U \eqbr S$ alors $\hnf{S} = \Pi x : T'. U'$ avec $T \eqbr T'$ et $U
  \eqbr U'$.
\end{lemma}
\begin{lemma}
  \label{conversion-sigma}
  Si $\Sigma x : T. U \eqbr S$ alors $\hnf{S} = \Sigma x : T'. U'$ avec $T \eqbr T'$ et $U
  \eqbr U'$.
\end{lemma}

On peut maintenant montrer:
\begin{lemma}[Conservation de la conversion par coercion]
  \label{conversion-coercion}
  Si $`G \typea T, U : s$ et $T \eqbr U$ alors $T \suba U$.
\end{lemma}
\begin{proof}
  Par induction sur la forme de $\hnf{T}$.
  
  \def\seq{\suba}.
  
  \begin{itemize}
  \item[$\hnf{T} = \Pi x : X.Y$:]
    Alors $\hnf{U} = \Pi y : V.W$ et $X \eqbr V$, $Y \eqbr W$
    d'apr�s le lemme \ref{conversion-pi}.
    Par induction $X \sub Y$ et $V \sub W$. 
    On applique alors \irule{SubProd} � ces deux pr�misses.
    
  \item[$\hnf{T} = \Sigma x : X.Y$:]
    Alors $\hnf{U} = \Sigma y : V.W$, avec $X \eqbr V$ et $Y \eqbr
    W$. Par induction et application de \irule{SubSigma}.
    
  \item[$\hnf{T} `= \mysubset{x}{X}{P}$:] 
    On a alors $\hnf{U} = \mysubset{x}{X'}{P'}$ avec $X \eqbr X'$, $P \eqbr
    P'$, et la propri�t� est vraie par \irule{SubLeft} et \irule{SubRight}:
    
    \begin{prooftree}
      \AXC{$X \sub X'$}
      \LeftLabel{\SubLeftRule}
      \UIC{$\mysubset{x}{X}{P} \sub X'$}
      \LeftLabel{\SubRightRule}
      \UIC{$\mysubset{x}{X}{P} \sub \mysubset{x}{X'}{P'}$}
    \end{prooftree}

  \item[Sinon:]
    On applique obligatoirement \irule{SubConv} et l'on a la pr�misse $T
    \eqbr U$, c'est donc direct.
  \end{itemize}
\end{proof}

Il n'y a pas de probl�me d'identification de sortes dans ce syst�me,
contrairement au syst�me $\lambda~C_\leq$ de Gang Chen \cite{ChenPhD},
puisqu'on n'a pas de cumulativit�. Le seul fait que les arguments sont
tous les deux sort�s avec la m�me sorte
avant de d�river le jugement de coercion nous assure que l'on ne fera
pas d'identification erron�e.

On va maintenant montrer que la r�gle \irule{SubTrans} est admissible
dans notre syst�me algorithmique. On montre ceci en l'�liminant de toute
d�rivation possible la faisant intervenir.

\typenva
Tout d'abord quelques lemmes n�cessaires pour la preuve:
\begin{lemma}[Coercion et $\mualgo$]
  \label{coercion-mu}
  \quad
  \begin{itemize}
  \item Si $\Pi x : X.Y \sub U$ alors $\mualgo(U) = \Pi x : X'.Y'$ et $X' \sub
    X$, $Y \sub Y'$.
  \item Si $\Sigma x : X.Y \sub U$ alors $\mualgo(U) = \Sigma x : X'.Y'$ et $X \sub
    X'$, $Y \sub Y'$.
  \item Pour tout $S,U$, $S \sub \mualgo(U)$ \ssi~$S \sub U$.
  \end{itemize}
\end{lemma}
\begin{proof}
  Par induction sur les d�rivations de $\suba$ et la d�finition de $\mualgo$.

  Dans le dernier cas, de gauche � droite on construit la d�rivation en
  ajoutant des applications de \irule{SubProof} et dans l'autre sens on
  est assur� de trouver la preuve dans la d�rivation m�me de $S \sub U$:
  si $U$ n'est pas de la forme sous-ensemble c'est direct. Sinon, on
  peut trouver dans la preuve (en partant de la racine) la premi�re
  utilisation de la r�gle \irule{SubProof}. A partir de l�, on cherche
  la premi�re utilisation d'une r�gle autre que \irule{SubProof} ou 
  \irule{SubSub}. On a une d�rivation de $S' \suba \mualgo(U)$, on
  peut r�appliquer les r�gles \irule{SubSub} oubli�es pr�c�demment 
  pour obtenir la preuve de $S \suba \mualgo(U)$.
\end{proof}

\begin{lemma}[Coercion et conversion]
  \label{coercion-conversion}
  Si $`G \type S,T,U : s$, $S \eqbr T$ et $T \sub U$ alors $S \sub U$
\end{lemma}

\begin{proof}
  Par simple inspection des r�gles on voit que le jugement ne peut
  distinguer deux termes $\beta$-�quivalents (ils ont forc�ment les
  m�mes formes normales de t�te).
\end{proof}

\begin{lemma}[Coercion et formes normales de t�te]
  \label{coercion-hnf}
  Si $T \suba U$ alors $\hnf{T}~\suba \hnf{U}$ est d�rivable par une
  d�rivation plus petite ou �gale.
\end{lemma}

\begin{proof}
  \begin{induction}[subtyping-algo]

    \case{SubConv} Trivial.
    \case{SubHnf} On prend la d�rivation en pr�misse.
    \casetwo{SubProd}{SubSigma} $T$ et $U$ sont �gaux � leurs formes normales
    de t�te, direct.

    \case{SubProof}
    Par induction $\hnf{T}~\suba \hnf{V}$, on applique \irule{SubProof}
    \case{SubSub} idem.
  \end{induction}  
\end{proof}

\begin{lemma}[Transitivit� de la coercion]
  \label{transitive-coercion}
  Pour tout $S, T, U$ tel que $`G \type S,T,U : s$ si
  $S \sub T$ et $T \sub U$ alors $S \sub U$.
\end{lemma}

\begin{proof} 
  %\TODO{Dans \cite{Pierce:TypeSystems}, voir p. 420}
  On proc�de par �limination de la r�gle \irule{SubTrans} dans toute
  d�rivation de $S \sub U$.
  Par induction sur l'ordre lexicographique $< depth(S \sub T) +
  depth(T \sub U), depth(S \sub U) >$.
  On peut supposer sans perte de g�n�ralit� qu'il n'y a pas
  d'applications successive de la r�gle \irule{SubHnf} dans nos
  d�rivations, par idempotence de la mise en forme normale de t�te.
  
  \begin{induction}

    \casetwo{SubConv}{*}\quad
    \begin{prooftree}
      \AXC{$S \eqbr T$}
      \UIC{$S \sub T$}
      \AXC{$T \sub U$}
      \BIC{$S \sub U$}
    \end{prooftree}
    
    Par le lemme \ref{coercion-conversion}, on �limine trivialement \irule{SubTrans}.

    \casetwo{SubHnf}{*}\quad
    \begin{prooftree}
      \AXC{$\hnf{S}~\sub \hnf{T}$}
      \UIC{$S \sub T$}
      \AXC{$T \sub U$}
      \BIC{$S \sub U$}
    \end{prooftree}

    Par le lemme \ref{coercion-hnf}, il existe une d�rivation de 
    $\hnf{T} \sub \hnf{U}$ de taille plus petite ou �gale � la
    d�rivation de $T \sub U$ on peut donc se ramener au cas:

    \begin{prooftree}
      \AXC{$\hnf{S}~\sub \hnf{T}$}
      \UIC{$S \sub T$}
      \AXC{$\hnf{T} \sub \hnf{U}$}
      \UIC{$T \sub U$}
      \BIC{$S \sub U$}
    \end{prooftree}

    
    Par cas sur la d�rivation de $\hnf{S} \sub \hnf{T}$
    \begin{induction}
      \case{SubConv}\quad
      \begin{prooftree}
        \AXC{$\hnf{S} \eqbr \hnf{T}$}
        \UIC{$\hnf{S}~\sub \hnf{T}$}
        \UIC{$S \sub T$}
        \AXC{$\hnf{T} \sub \hnf{U}$}
        \UIC{$T \sub U$}
        \BIC{$S \sub U$}
      \end{prooftree}
      
      Par le lemme \ref{coercion-conversion}.
      
      \case{SubProd}\quad
      \begin{prooftree}
        \AXC{$C \sub A$}
        \AXC{$B \sub D$}
        \BIC{$\hnf{S}=\Pi x : A.B~\sub \Pi x : C.D = \hnf{T}$}
        \UIC{$S \sub T$}
        \AXC{$\hnf{T} = \Pi x : C.D \sub \hnf{U}$}
        \UIC{$T \sub U$}
        \BIC{$S \sub U$}
      \end{prooftree}
    
      Par cas sur la d�rivation de $\Pi x : C.D \sub \hnf{U}$.
      \begin{itemize}
        \case{SubConv} Trivial.
        \case{SubProd} Alors on a
        \begin{prooftree}
          \AXC{$E \sub C$}
          \AXC{$D \sub F$}
          \BIC{$\Pi x : C.D \sub \Pi x : E.F$}
        \end{prooftree}
        
        On a donc la d�rivation:
        \begin{prooftree}
          \AXC{$E \sub C$}\AXC{$C \sub A$}
          \BIC{$E \sub A$}
          
          \AXC{$B \sub D$}\AXC{$D \sub F$}
          \BIC{$B \sub F$}
          \BIC{$\hnf{S} = \Pi x : A.B \sub \Pi x : E.F = \hnf{U}$}
          \UIC{$S \sub U$}
        \end{prooftree}
        
        La taille des d�rivations de $E \sub C$, $C \sub A$ et $B \sub
        D$, $D \sub F$ �tant plus
        petites que $S \sub T$ et $T \sub U$, on �limine bien la transitivit� dans ce cas.

        \case{SubProof} On a:
        \begin{prooftree}
          \AXC{$\Pi x : C.D \sub E$}
          \UIC{$\Pi x : C.D \sub \mysubset{y}{E}{P}$}
        \end{prooftree}

        Par induction, on a:

        \begin{prooftree}
          \AXC{$\Pi x : A.B \sub \Pi x : C.D$}
          \AXC{$\Pi x : C.D \sub E$}
          \BIC{$\Pi x : A.B \sub E$}          
          \UIC{$\hnf{S} = \Pi x : A.B \sub \mysubset{y}{E}{P} = \hnf{U}$}
          \UIC{$S \sub U$}
        \end{prooftree}
        
        Car $\Pi x : C.D \sub E$ est une d�rivation plus petite que $T
        \sub U$.
      \end{itemize}
            
      \case{SubSigma} De fa�on �quivalente au produit.
    
      \case{SubProof}\quad
      \begin{prooftree}
        \AXC{$\hnf{S} \sub C$}
        \UIC{$\hnf{S}~\sub \mysubset{y}{C}{P} = \hnf{T}$}
        \UIC{$S \sub T$}
        \AXC{$\mysubset{y}{C}{P} = \hnf{T} \sub \hnf{U}$}
        \UIC{$T \sub U$}
        \BIC{$S \sub U$}
      \end{prooftree}
      Encore une fois, par cas sur la d�rivation de $\mysubset{y}{C}{P}
      = \hnf{T} \sub \hnf{U}$:
      \begin{itemize}
        \case{SubConv} Trivial.
        \case{SubSub} On a:
        \begin{prooftree}
          \AXC{$C \sub \hnf{U}$}
          \UIC{$\mysubset{y}{C}{P} = \hnf{T} \sub \hnf{U}$}
        \end{prooftree}
        
        Par induction, on peut donc d�river:
        \begin{prooftree}
          \AXC{$\hnf{S} \sub C$}
          \AXC{$C \sub \hnf{U}$}
          \BIC{$\hnf{S} \sub \hnf{U}$}
          \UIC{$S \sub U$}
        \end{prooftree}
        
        Les d�rivations $\hnf{S} \sub C$ et $C \sub \hnf{U}$ �tant bien
        plus petites que $S \sub T$ et $T \sub U$.     
      \end{itemize}
      
      \case{SubSub} De m�me:
      \begin{prooftree}
        \AXC{$A \sub \hnf{T}$}
        \UIC{$\hnf{S} = \mysubset{y}{A}{P}~\sub \hnf{T}$}
        \UIC{$S \sub T$}
        \AXC{$\hnf{T} \sub \hnf{U}$}
        \UIC{$T \sub U$}
        \BIC{$S \sub U$}
      \end{prooftree}

      Se r�ecrit en:
      \begin{prooftree}
        \AXC{$A \sub \hnf{T}$}
        \AXC{$\hnf{T} \sub \hnf{U}$}
        \BIC{$A \sub \hnf{U}$}
        \UIC{$\hnf{S} = \mysubset{y}{A}{P}~\sub \hnf{U}$}
        \UIC{$S \sub U$}
      \end{prooftree}
    \end{induction}
  
    On peut faire le m�me raisonnement par cas sur la d�rivation de
    $\hnf{T} \sub \hnf{U}$. On peut donc se restreindre aux cas ou l'on
    n'utilise plus la r�gle \irule{SubHnf} dans les pr�misses de \irule{SubConv}.
        
    Les cas restants se montrent de fa�on similaire � leurs �quivalents
    dans la preuve pour le cas \irule{SubHnf}. Par exemple pour le
    produit:

    
    \case{SubProd}\quad
    \begin{prooftree}
      \AXC{$C \sub A$}
      \AXC{$B \sub D$}
      \BIC{$\Pi x : A.B \sub \Pi x : C.D$}
      \AXC{$\Pi x : C.D \sub U$}
      \BIC{$\Pi x : A.B \sub U$}
    \end{prooftree}

    On n'a seulement � traiter le cas ou $\Pi x : C.D \sub U$ est d�riv� par
    \irule{SubProd} ou \irule{SubProof}.
    \begin{itemize}
      \case{SubProd} Alors on a
        \begin{prooftree}
          \AXC{$E \sub C$}
          \AXC{$D \sub F$}
          \BIC{$\Pi x : C.D \sub \Pi x : E.F$}
        \end{prooftree}
        
        On a donc la d�rivation:
        \begin{prooftree}
          \AXC{$E \sub C$}\AXC{$C \sub A$}
          \BIC{$E \sub A$}
          
          \AXC{$B \sub D$}\AXC{$D \sub F$}
          \BIC{$B \sub F$}
          \BIC{$S = \Pi x : A.B \sub \Pi x : E.F = U$}
        \end{prooftree}
        
        La taille des d�rivations de $E \sub C$, $C \sub A$ et $B \sub
        D$, $D \sub F$ �tant plus
        petites que $\Pi x : A.B \sub \Pi x : C.D$ et $\Pi x : C.D \sub
        \Pi x : E.F$, on �limine bien la transitivit� dans ce cas.

        \case{SubProof} On a:
        \begin{prooftree}
          \AXC{$\Pi x : C.D \sub E$}
          \UIC{$\Pi x : C.D \sub \mysubset{y}{E}{P}$}
        \end{prooftree}

        Par induction, on a:

        \begin{prooftree}
          \AXC{$\Pi x : A.B \sub \Pi x : C.D$}
          \AXC{$\Pi x : C.D \sub E$}
          \BIC{$\Pi x : A.B \sub E$}          
          \UIC{$S = \Pi x : A.B \sub \mysubset{y}{E}{P} = U$}
        \end{prooftree}
        
        Car $\Pi x : C.D \sub E$ est une d�rivation plus petite que $T
        \sub U$.
      \end{itemize}



    
  \end{induction}
\end{proof}

\begin{corrolary}[Compl�tude de la coercion]
  \label{complete-coercion}
  Si $U \subd V$ alors $U \suba V$.
\end{corrolary}

\begin{proof}
  Les r�gles des deux syst�mes sont les m�mes except� \irule{SubTrans}
  qui est admissible dans le syst�me algorithmique. De plus
  l'application restreinte de la conversion ne change pas les jugements
  d�rivables (lemme \ref{conversion-coercion}).
\end{proof}

En cons�quence $\subd$ et $\suba$ sont �quivalentes. Le syst�me
d'inf�rence de $\suba$ donne donc un algorithme pour d�cider de la relation
de coercion. L'ind�terminisme entre les r�gles \irule{SubProof} et
\irule{SubSub} ne pose pas de probl�me: on peut laisser le choix � 
l'implantation puisque le syst�me est confluent. \irule{SubHnf}
formalise le fait qu'on peut avoir � r�duire en t�te avant d'appliquer
les autres r�gles (pour obtenir un produit, une somme ou un sous-ensemble).

\subsubsection{D�cidabilit� et compl�tude du typage}
Le syst�me algorithmique correspond au syst�me d�claratif o� l'on a enlev� la r�gle
de coercion \irule{Coerce} et chang� certaines r�gles pour obtenir un syst�me d�cidable
(voir figure \ref{fig:typing-algo-rules}).
On va proc�der de fa�on similaire � l'�limination de la transitivit�
pour montrer que la r�gle \irule{Coerce} n'est plus n�cessaire dans le
syst�me algorithmique. On va montrer en fait que
toute d�rivation de typage utilisant \irule{Coerce} peut se r��crire en
une d�rivation n'utilisant cette r�gle qu'� sa racine.

% \paragraph{Sommes d�pendantes}
% On veut pouvoir faire le plus d'inf�rence possible dans notre syt�me
% algorithmique, on a donc introduit une r�gle \irule{SumInf} qui ne force
% pas � annoter les paires. Dans le cas ou le terme n'est pas annot�, on
% consid�re donc que la somme n'est pas d�pendante. En effet on remarque
%  qu'il n'est pas possible d'inf�rer le type $U$ � partir du seul terme $(t, u)$. Cela
% n�cessiterait de r�soudre un probl�me d'unification d'ordre sup�rieur
% auquel il n'y a pas de solution la plus g�n�rale. 
% On a donc dans le syst�me algorithmique deux r�gles pour les sommes, 
% dont une (\irule{SumDep}) permettant d'annoter le terme avec le type $U$ recherch�. 

\paragraph{\'Elimination de la coercion}
\typenva
On veut maintenant montrer la compl�tude de notre syst�me. Dans un
syst�me � sous-typage, le th�or�me correspondant est parfois nomm�
typage minimal ``\emph{minimal typing}'' puisque son �nonc� revient �
dire que tout terme a un type minimal dans les deux syst�mes. En effet
notre th�or�me est le suivant:
$`G \typed t : T "=>" `G \typea t : U \suba T$. Le typage algorithmique
assigne bien un seul type � un terme $t$ mais comme on a des coercions, le
type inf�r� $U$ peut �tre un peu diff�rent du type $T$. Dans notre cas
particulier $U$ sera certainement un type moins riche que $T$ (par
exemple $\nat$ par rapport � $\mysubset{x}{\nat}{P}$). Lorsque l'on
d�veloppera des programmes, on donnera une sp�cification forte et l'on
fera une coercion entre le type inf�r� et la sp�cification pour obtenir
au final (apr�s r��criture dans \Coq) un terme du type $T$ le plus riche.
On a besoin de quelques lemmes pour montrer que notre syst�me o�
\irule{Coerce} a �t� �limin� est complet:

\begin{lemma}[$\beta$-equivalence et $\mualgo$]
  \label{beta-mu}
  Si $X \sub Y$ et $\mualgo(Y) = \Sigma x : T.U$ alors $\mualgo(X) = \Sigma x : T'.U'$
  et $T' \sub T$, $U' \sub U$.
  Si $X \sub Y$ et $\mualgo(Y) = \Pi x : T.U$ alors $\mualgo(X) = \Pi x :
  T'.U'$ et $T \sub T'$, $U' \sub U$.
\end{lemma}
\begin{proof}
  Par induction sur la d�rivation de coercion, on fait le cas pour $\Sigma$.
  \begin{induction}  
    \case{SubConv} Trivial, puisqu'on aura $\mualgo(X) = \mualgo(Y)$.

    \case{SubHnf} Trivial puisque pour tout $x$, $\mualgo(x) =
    \mualgo(\hnf{x})$.
    
    \case{SubProd} Impossible, $\mualgo$ ne traversant pas les produits.

    \case{SubSigma} Direct, on a une d�rivation de $\Sigma x : T'. U'
    \sub \Sigma x : T.U$.
    
    \case{SubLeft} Ici, $Y `= \mysubset{x}{V}{P}$, on peut donc d�duire que
    $\mualgo(Y) = \mualgo(V) = \Sigma x : T.U$. On 
    applique l'hypoth�se de r�currence avec $X \sub V$ et on obtient:
    $\mualgo(X) = \Sigma x : T'.U' `^ T' \sub T `^ U' \sub U$.

    \case{SubRight} Ici, $X `= \mysubset{x}{V}{P}$. Par induction, 
    $\mualgo(V) = \mualgo(X) = \Sigma x : T'.U' `^ T' \sub T `^ U' \sub U$.
  \end{induction}
\end{proof}

\begin{lemma}[Bonne formation des contextes]
  \label{wf-contexts-a}
  Si $`G \type t : T$ alors $\typewf `G$.
\end{lemma}
\begin{proof}
  \inductionon{typing-decl}
\end{proof}

\begin{fact}[Inversion du jugement de bonne formation]
  \label{inversion-wf-a}
  Si $\typewf `G, x : U$ alors il existe $s$, $`G \type U : s$ et $s `: \setproptype$.
\end{fact}

\begin{lemma}[Affaiblissement]
  \label{weakening-a}
  Si $`G, `D \type t : T$ alors pour tout $x : S `; `G, `D$ tel que
  $\wf `G, x : S, `D$, $`G, x : S, `D \type t : T$
\end{lemma}

\begin{proof}
  \begin{induction}[typing-decl]
    \case{PropSet} Trivial.

    \case{Var}
    On a $x : S `; `G, `D$, donc $`G, x : S, `D \type y : T$ est toujours d�rivable.
    
    \case{Prod}
    Par induction $`G, x : S, `D \type T : s1$ et $`G, x : S, `D,
    y : T \type U : s2$. On applique \irule{Prod} pour obtenir 
    $`G, x : S, `D \type \Pi x : T.U : s2$. De m�me pour le reste des r�gles.
  \end{induction}
\end{proof}  

La restriction montre que notre notion de coercion est correcte
vis-�-vis du typage. On peut d�river les m�mes jugements dans des
contextes o� les variables ont des types �quivalents. Ici la taille des
d�rivations ne change pas.

\begin{lemma}[Restriction]
  \label{narrowing-a}
  \[ `G \seq S, S' : s, S' \sub S "=>" 
  \left\{ \begin{array}{lcl}
      \typewf `G, x : S, `D & "=>" & \typewf `G, x : S', `D \\
      & `^{} & \\
      `G, x : S, `D \seq t : T & "=>" & `G, x : S', `D \seq t : T' \suba T 
    \end{array}
  \right. \]
\end{lemma}

\begin{proof}
  On peut remarquer que si $T `: \setproptype$ alors $T'$ doit �tre �gal
  � $T$ par d�finition de la coercion.

  Par induction sur la taille de la d�rivation de typage ou de bonne formation.
    
  \begin{induction}
    \case{WfEmpty} Trivial.
    
    \case{WfVar} 
    La conclusion est $\typewf `G, x : S, `D$
    
    \begin{induction}[text=Par induction sur la taille de $`D$]
    \item[\protect{$`D = []$}]
      La racine de la d�rivation est de la forme:
      \begin{prooftree}
        \UAX{WfVar}
        {$`G \type S : s$}
        {$\wf `G, x : S$}
        {$s `: \setproptype$}
      \end{prooftree}
      On a donc $`G \type S' : s$, et par \irule{WfVar}, $\typewf `G, x : S'$.  
      
    \item[\protect{$`D `= `D', y : U$}]
      La racine de la d�rivation est de la forme:
      \begin{prooftree}
        \UAX{WfVar}
        {$`G, x : S, `D' \type U : s$}
        {$\wf `G, x : S, `D', y : U$}
        {$s `: \setproptype$}
      \end{prooftree}
      Par induction sur la d�rivation de typage $`G, x : S', `D' \seq U : s$,
      on a donc bien $\typewf `G, x : S', `D', y : U$ par \irule{WfVar}.
    \end{induction}
    
    \case{PropSet}
    Par induction, $\typewf `G, x : S', `D$, on applique simplement la r�gle.
    
    \case{Var}
    Par induction, $\typewf `G, x : S', `D$. La seule diff�rence avec le
    contexte pr�c�dent est le type associ� � $x$, donc si $t \not= x$, on
    peut simplement r�appliquer \irule{Var}. Si $t `= x$ on a la
    d�rivation:

    \begin{prooftree}
      \BAX{Var}
      {$\wf `G, x : S', `D$}
      {$x : S' `: `G$}
      {$`G, x : S', `D \seq x : S'$}
      {}
    \end{prooftree}
    
    On a bien $S' \suba S$, la propri�t� est donc bien v�rifi�e.
    
    \case{Prod} 
    Par induction, $`G, x : S', `D \type T : s1$ et $`G, x : S', `D
    y : T \seq U : s2$. On applique \irule{Prod} pour obtenir 
    $`G, x : S' \type \Pi x : T.U : s2$. De m�me pour le reste des r�gles.
  \end{induction}
\end{proof}

Il nous faut montrer des lemmes faisant intervenir la substitution pour
pouvoir prouver la compl�tude.
\begin{lemma}[Substitutivit� de $\mualgo$]
  \label{substitutive-mu}
  Si $\mualgo(T) = \Pi y : U.V$ alors $\mualgo(T[u/x]) = \Pi y :
  U[u/x].V[u/x]$.
  Si $\mualgo(T) = \Sigma y : U.V$ alors $\mualgo(T[u/x]) = \Sigma y : U[u/x].V[u/x]$.
\end{lemma}

\begin{proof}
  On montre la propri�t� pour les produits, la preuve est similaire pour
  les sommes. Par induction sur la taille de $T$.

  Il suffit de suivre la d�finition de $\mualgo$.
  Si $T$ est de la forme $\mysubset{y}{V}{P}$ alors
  on a $\mualgo(V) = \Pi y : U.V$ et par induction
  $\mualgo(V[u/x]) = \Pi y : U[u/x].V[u/x]$.
  Il s'ensuit directement que $\mualgo(T[u/x]) = \Pi y :
  U[u/x].V[u/x]$.

  Si $T$ est diff�rent d'un type sous-ensemble alors $\mualgo(T) = T$.
  Donc $T \eqbr \Pi y : U.V$. Il s'ensuit que $T$ est de la forme $\Pi y
  : U'.W'$ et donc $T[u/x] = \Pi y : U'[u/x].V'[u/x] =
  \mualgo(T[u/x])$. Par substitutivit� de la conversion, il s'ensuit que 
  $\Pi y : U'[u/x].V'[u/x] \eqbr \Pi y : U[u/x].V[u/x]$.
\end{proof}

\begin{lemma}[Substitutivit� de la coercion]
  \label{substitutive-coercion}
  Si $U \suba T$ alors pour tout $u$, $U[u/x] \suba T[u/x]$.
\end{lemma}

\begin{proof}
  \begin{induction}[subtyping-algo]
    \case{SubConv}
    Direct par pr�servation de l'�quivalence $\eqbr$ par substitution.
    
    \case{SubHnf}
    Par induction, $(\hnf{U})[u/x] \suba (\hnf{T})[u/x]$. 
    Par le lemme \ref{coercion-hnf}, $\hnf{((\hnf{U})[u/x])} \suba
    \hnf{((\hnf{T})[u/x])}$. 
    Donc $\hnf{U[u/x]}~\suba \hnf{T[u/x]}$ et par \irule{SubHnf}, 
    $U[u/x] \suba T[u/x]$.

    \case{SubProd}
    Par induction $U[u/x] \suba T[u/x]$ et $V[u/x] \suba W[u/x]$, donc
    $\Pi y : T[u/x].V[u/x] \suba \Pi y : U[u/x].W[u/x]$. La propri�t�
    est donc bien v�rifi�e.
    
    \case{SubSigma} Direct par induction.
    
    \case{SubSub} Par induction, $U'[u/x] \suba V[u/x]$. On applique
    \irule{SubLeft} pour obtenir $\mysubset{y}{U'[u/x]}{P} \suba V[u/x]$. 
    
    \case{SubRight} Direct par induction.
  \end{induction}
\end{proof}

\begin{lemma}[Substitutivit� du typage]
  \label{substitutive-typing}
  Si $`G \typea u : U$ alors
  \[ \left\{ \begin{array}{lcl}
      `G, x : U, `D \typea t : T & "=>" & `G, `D[u/x] \typea t[u/x] : T[u/x] \\
      \wf `G, x : U, `D & "=>" & \wf `G, `D[u/x]
    \end{array}\right. \]
\end{lemma}

\begin{proof}
  \typenva
  Par induction mutuelle sur la d�rivation de typage $`G, x : U, `D
  \typea t : T$ ou $\wf `G, x : U, `D$.
  
  \begin{induction}
    \case{WfEmpty} Trivial.

    \case{WfVar}
    Par induction sur $`D$.
    \begin{itemize}
    \item[\protect{$`D = []$}]
      On a alors $`G \typea U : s$ donc $\wf `G$ et trivialement, $\wf
      `G, `D[u/x]$.

    \item[\protect{$`D = `D', y : T$}]
      On a alors $`G, x : U, `D' \typea T : s$ et par induction
      $`G, `D'[u/x] \typea T[u/x] : s[u/x] = s$. Donc on peut appliquer
      \irule{WfVar} pour obtenir $\wf `G, `D'[u/x], y : T[u/x]$ soit
      $\wf `G, `D[u/x]$
    \end{itemize}
    
    \case{PropSet}
    La substitution n'a aucun effet et $`G, `D[u/x]$ est bien
    form� par induction.
    
    \case{Var}
    Par induction, $\wf `G, `D[u/x]$.
    Si $t `= x$ alors on a $T = U$ et $T[u/x] = U$ puisque $x$
    n'appara�t pas dans $U$. On a donc $`G, `D[u/x] \typea t[u/x] = u :
    T[u/x] = U$, qui peut s'obtenir par affaiblissement de $`G \typea u
    : U$. 
    Si $y : T `: `G$ alors on applique simplement \irule{Var}.
    Si $y : T `: `D$ alors $y : T[u/x] `: `D[u/x]$ et on obtient
    $`G, `D[u/x] \typea y[u/x] :  T[u/x]$ par \irule{Var}.
    
    \case{Prod}
    Par induction  $`G, `D[u/x] \typea T[u/x] : s_1[u/x]$ et
    $`G, `D[u/x], y : T[u/x] \typea M[u/x] : s_2[u/x]$. 
    On peut appliquer \irule{Prod} pour obtenir $`G, `D[u/x] \typea \Pi
    y : T[u/x].M[u/x] : s_2[u/x]$ soit $`G, `D[u/x] \typea (\Pi y :
    T.M)[u/x] : s_2[u/x]$.
    De fa�on similaire pour les autres cas.

    \case{App}
    On �tudie le cas de l'application qui requiert un lemme suppl�mentaire.
    Par induction, on a $`G, `D[u/x] \typea f[u/x] : T[u/x]$ et
    $`G, `D[u/x] \typea a[u/x] : A[u/x]$. Si $\mualgo(T) = \Pi y :
    V.W$ alors $\mualgo(T[u/x]) = \Pi y : V[u/x].W[u/x]$ (lemme
    \ref{substitutive-mu}). Par induction, on a aussi $`G, `D[u/x] \typea
    A[u/x],V[u/x] : s$. Enfin, par substitutivit� de la coercion on a $A[u/x]
    \suba V[u/x]$. On peut donc appliquer \irule{App} pour obtenir 
    $`G, `D[u/x] \typea (f[u/x]~a[u/x]) : W[u/x][a[u/x]/y]$. Or
    $W[u/x][a[u/x]/y] = W[a/y][u/x]$ ($y `; \freevars{u}$). On a donc bien $`G, `D[u/x] \typea (f~a)[u/x] :
    (W[a/y])[u/x]$.
    On a un raisonnement similaire pour \irule{LetSum}.
  \end{induction}
  
\end{proof}

\begin{lemma}[Substitutivit� du typage avec coercion]
  \label{substitutive-typing-coercion}
  Si $`G, x : V \typea t : T \sub U$ et $`G \typea u : V$
  alors $`G \typed t[u/x] : T[u/x] \sub U[u/x]$.
\end{lemma}

\begin{proof}
  Par substitutivit� du typage (\ref{substitutive-typing}) on a $`G \typed t[u/x] : T[u/x]$.
  Par le lemme pr�c�dent $T[u/x] \suba U[u/x]$.
\end{proof}

On a maintenant tout les ingr�dients pour montrer la compl�tude de notre
syst�me de typage vis-�-vis du syst�me d�claratif.

\setboolean{displayLabels}{true}
\begin{theorem}[Compl�tude du typage]
  \label{complete-typing}
  Si $`G \typed t : T$ alors $`E U, `G \typea t : U \sub T$.
  Si $\typewf `G$ dans le syst�me d�claratif alors $\typewf `G$ dans le
  syst�me algorithmique.
\end{theorem}

\begin{proof}
  \typenva
  Par induction mutuelle sur les d�rivations de typage et bonne formation.
  \begin{induction}
    \case{WfEmpty} Trivial.
    
    \case{WfVar}\quad
    \typenvd
    \begin{prooftree}
      \WfVar
    \end{prooftree}
    Par induction $`E s', `G \typea A : s' \sub s$. On a forc�ment $s' =
    s$ puisque les sortes ne sont en relation qu'avec elles-m�mes.
    On applique \irule{WfVar} pour obtenir $\typewf `G, x : A$.
    
    \case{PropSet} Trivial.

    \case{Var}\quad
    \typenvd
    \begin{prooftree}
      \Var
    \end{prooftree}
    Par induction $\typewf `G$ et $x : A `: `G$ , direct par
    \irule{Var}.
    
    \case{Prod}\quad
    \typenvd
    \begin{prooftree}
      \Prod
    \end{prooftree}
    
    Direct par induction et le fait qu'une sorte ne peut �tre en relation
    qu'avec elle m�me.
        
    \case{Abs} \quad
    \typenvd
    \begin{prooftree}
      \Abs
    \end{prooftree}
    Par induction $`E U', `G, x : T \typea M : U' \suba U$.
    Or cela implique $`G, x : T \typea U' : s$, donc on peut d�river
    $`G \typea \Pi x : T.U' : s$. On a donc bien $`G \typea \lambda x :
    T.M : \Pi x : T.U' \suba \Pi x : T.U$.
    
    \case{App} On a 
    \typenvd
    \begin{prooftree}
      \App
    \end{prooftree}
    
    \typenva
    Par induction, $`E T, `G \typea f : T \suba \Pi x : V. W$ et
    $`E U, `G \subta u : U \sub V$.
    
    Si $T \suba \Pi x : V.W$ alors $\mualgo(T) = \Pi x : V'.W'$ avec
    $V \suba V'$ et $W' \suba W$ (lemme \ref{beta-mu}).

    Par transitivit� de la coercion: $U \suba V'$, on peut donc d�river 
    \begin{prooftree}
      \TAX{App}
      {$`G \seq f : T \quad \mualgo(T) = \Pi x : V'. W'$}
      {$`G \seq u : U \quad `G \seq U, V' : s$}
      {$U \suba V'$}
      {$`G \seq (f u) : W' [ u / x ]$}
      {}
    \end{prooftree}
    
    Par substitutivit� de la coercion (lemme
    \ref{substitutive-coercion}), $W'[u/x] \suba W[u/x]$, la propri�t�
    est donc bien v�rifi�e.
    
    \case{Sigma}\quad
    \typenvd
    \begin{prooftree}
      \SigmaR
    \end{prooftree}

    Par induction $`E s', `G \typea T : s' \sub s$ et $`E s'', `G, x : T 
    \typea U : s' \suba s$ o� $s `: \{ \Prop, \Set \}$. Encore une fois  
    les sortes $s$, $s'$ et $s''$ doivent �tre �gales. C'est direct
    par \irule{Sigma}.

    \case{Sum}\quad
    \typenvd
    \begin{prooftree}
      \Sum
    \end{prooftree}
    \typenva
    
    Ici, l'annotation nous force � utiliser le jugement de coercion.
    Par induction, $`E s', \Sigma x : T.U : s' \suba s$, $`E T', `G
    \type t : T' \suba T$ et $`E U', `G \typea u : U' \suba U[t/x]$.
    On peut montrer $`G \type \Sigma x : T'.U : s$.
    En effet, par inversion de $`G \seq \Sigma x : T.U : s$ on a
    $`G, x : T \seq U : s$ et par restriction ($T' \sub T$), $`G, x : T' \seq
    U : s$.
    Comme $T' \suba T$ on obtient $\Sigma x : T'.U \suba \Sigma x : T.U$.
    On peut donc d�river:
    \begin{prooftree}
      \QAX{SumDep}
      {$`G \seq t : T'$}
      {$`G \seq u : U'$}
      {$`G \seq U[t/x], U' : s \quad U' \suba U[t/x]$}
      {$`G \seq \Sigma x : T'.U : s$}
      {$`G \seq (x \coloneqq~t, u : U) : \Sigma x : T'.U$}
      {}
    \end{prooftree} 
        
    \case{LetSum} On a
    \typenvd
    \begin{prooftree}
      \LetSum
    \end{prooftree}
    
    \typenva
    Par induction, $`E S, `G \typea t : S \suba \Sigma x : T.U$ et 
    $`E V', `G, x: T, y : U \typea v : V' \suba V$.
    On a $\mualgo(S) = \Sigma x : T'.U'$ avec $T' \suba T$ et $U'
    \suba U$. Par restriction on peut donc d�river $`G, x : T', y : U'
    \seq v : V'' \suba V'$.
    
    On a donc la d�rivation suivante dans le syst�me algorithmique:
    \begin{prooftree}
      \TAX{LetSum}
      {$`G \seq t : S$}
      {$\mualgo(S) = `S x : T'. U'$}
      {$`G, x : T', y : U' \seq v : V''$}
      {$`G \seq \letml~(x, u) = t~\inml~v : V''$}
      {}
    \end{prooftree}
    
    Comme $V'' \suba V' \suba V$, la propri�t� est vraie par
    transitivit� de la coercion.

    \casetwo{Conv}{Coerce}
    Dans les deux cas on a inductivement $`E T', `G \typea t : T'
    \suba T$. Avec \irule{Conv} on a $T \eqbr S$, donc $T' \suba S$ par
    le lemme \ref{coercion-conversion}. Pour \irule{Coerce} on a $T \subd S$.
    Par compl�tude de la coercion, $T \suba S$ et par transitivit� de la
    coercion, $T' \suba S$. La propri�t� est donc bien v�rifi�e dans les
    deux cas.
  \end{induction}
\end{proof}

On combine les th�or�mes de correction et de compl�tude pour obtenir
 la propri�t� suivante entre les deux syst�mes:
\begin{corrolary}[�quivalence des syst�mes d�claratif et algorithmique]
  $`G \typed t : T$ \ssi{} il existe $U$ tel que $`G \typea t : U$ et $U \suba T$.
\end{corrolary}

On a maintenant un syst�me raffin� d�rivant les m�mes jugements (�
coercion pr�s) que le syst�me d�claratif. On veut en extraire un
algorithme de typage. Pour cela on doit pouvoir r�soudre deux probl�mes:
\begin{itemize}
\item\textbf{V�rification de type.} On donne $`G$,$t$ et $T$ et l'on doit
  d�cider si $`G \typea t : T$ ;
\item\textbf{Inf�rence de type.} On donne $`G$,$t$ et l'on doit trouver $T$ tel
  que $`G \typea t : T$ si c'est d�rivable, sinon on �choue.
\end{itemize}
En pratique, la v�rification a besoin de l'inf�rence puisque lorsqu'on
v�rifie une application $f u : T$ on doit inf�rer le type de $f$.
On montre donc les th�or�mes suivants:

\begin{theorem}[D�cidabilit� de l'inf�rence dans le syst�me algorithmique]
  Le probl�me d'inf�rence $`G \typea t :~?$ est d�cidable.
\end{theorem}

\begin{proof}
  Il suffit d'observer que les r�gles de typage sont dirig�es par la
  syntaxe du deuxi�me argument et permettent donc d'inf�rer un type pour
  tout terme. En lisant les pr�misses de chaque r�gle de gauche �
  droite, on voit que l'inf�rence est d�cidable.
\end{proof}

\begin{theorem}[D�cidabilit� de $\typea$]
  La relation de typage $`G \typea t : T$ est d�cidable.
\end{theorem}
\begin{proof}
  Direct. On utilise le th�or�me pr�c�dent pour le cas de l'application.
\end{proof}

On a d�sormais un algorithme de typage pour notre syst�me avec
coercions. Ce syst�me est tr�s lib�ral puisqu'il permet de consid�rer
des objets comme v�rifiant des propri�t�s arbitraires sans les
montrer. Il nous faut maintenant remettre de la logique dans nos termes
pour s'assurer qu'ils sont corrects.

%%% Local Variables: 
%%% mode: latex
%%% TeX-master: "subset-typing"
%%% LaTeX-command: "TEXINPUTS=\"style:$TEXINPUTS\" latex"
%%% End: 

\section{G�n�ration des obligations de preuve}
On a maintenant un syst�me de typage d�cidable et l'on veut d�sormais
traduire ses d�rivations dans \CCI{} dont le jugement de typage est $\typec$. 

\subsection{D�finition de la r��criture vers \Coq}
\typenvi
La traduction transforme une d�rivation dans notre syst�me algorithmique
vers une d�rivation de \CCI{} valide. Le jugement 
$\timpl{`G}{t}{T}{`G'}{t'}{T'}$ se lit: on transforme le s�quent
$`G \typea t : T$ (syst�me algorithmique) en $`G' \typec t' : T'$
(\Coq). Le jugement $\subimpl{`G}{c}{T}{U}$ se lit: la coercion de $T$ �
$U$ est $c$ et on construit le s�quent $`G \typec c : T "->" U$.
La traduction est un homomorphisme (elle conserve la structure de la
d�rivation et se rappelle r�cursivement) except� pour l'application, ce qui
est normal puisque nous avons un syst�me tr�s proche de \CCI{}. Le fait
de traduire aussi les environnements $\Gamma$ est d� au fait que nous
faisons la coercion dans les types,  donc les environnements (listes
de couples $(\text{nom}, \text{type})$) doivent aussi �tre r��crits. Cela assure aussi la
coh�rence avec l'environnement g�n�ral de \Coq, c'est-�-dire
l'int�gration transparente de notre tactique dans les d�veloppements
\Coq~et la r�utilisabilit� des programmes g�n�r�s. En cons�quence, les
types sp�cifi�s ne sont donc pas toujours pr�serv�s (on veut pouvoir y
introduire des coercions).

\typeiFig
\typemuiFig

Ici la fonction $\muimpl$ renvoie � la fois un type (qu'on demande
�quivalent � un produit) et une fonction de coercion qui va faire les
projections n�cessaires sur l'objet \Coq~$f'$. En effet dans \Coq~les
objets de type sous-ensemble $\subset{x}{T}{P}$ sont cod�s par un terme 
de la forme $\sref{elt}~t~p$ dont on peut extraire les parties objet 
(un certain $t$ de type $T$, par la projection $\Pi_1$) et preuve
 (de type $P[t/x]$). Il faut donc faire exactement
une projection pour atteindre par exemple la fonction d'un objet de type
 $\subset{f}{\nat "->" \nat}{f~0 \neq 0}$.

Le jugement de coercion $U \suba V$ nous assure qu'il est possible de
d�river le jugement $`G' \typec U' \subi V'$ et donc de cr�er une coercion de $U'$
� $V'$ soit une fonction de type $U' "->" V'$ dans \CCI.
On trouve ici l'essence du m�canisme de coercion par pr�dicats. 

\subtiFig

\vspace{1.4em}
\begin{itemize}
\item[ \SubConvI\DP:] \quad\\

  Cr�e une coercion identit� puisque \CCI{}~a la r�gle de conversion. 
  \vspace{1em}

\item[\SubLeftI\DP:] \quad\\
  
  Engendre une projection,
  c'est le cas o� l'on ne s'int�resse pas � la preuve accompagnant
  l'objet. 
  \vspace{1em}

\item[\SubRightI\DP:] \quad\\

  Correspond � la g�n�ration d'une
  obligation de preuve dans \PVS. On utilise le m�canisme des variables
  existentielles (not�es $?:\text{type}$) d�crit plus loin pour donner 
  l'information au syst�me qu'il faut compl�ter le terme � un endroit
  donn� avec un nouveau terme de type appropri�. On peut ais�ment cr�er
  des obligations qui ne seront pas prouvables mais cela rel�ve de la
  responsabilit� de l'utilisateur.
  \vspace{1em}

\item[\SubProdI\DP,] \quad\\
  

\item[\SubSigmaI\DP:] \quad\\

  R�alisent respectivement
  les coercions pour les produits fonctionnels et cart�siens.

\end{itemize}
  
\subsection{Propri�t�s}
On veut montrer que si l'on a une d�rivation dans notre syst�me
algorithmique, alors son image par la r��criture est une d�rivation
valide de \CCI{} (par induction sur la d�rivation dans le syst�me
algorithmique). 


% Ce travail est en cours � ce jour, nous nous sommes
% plut�t pench�s sur l'impl�mentation du typeur et de la fonction de
% r��criture avant de commencer cette derni�re preuve.

%%% Local Variables: 
%%% mode: latex
%%% TeX-master: "subset-typing"
%%% LaTeX-command: "TEXINPUTS=\"style:$TEXINPUTS\" latex"
%%% End: 


\newpage
\section{La tactique \Subtac}
Nous avons d\'evelopp\'e la tactique \Subtac{} disponible dans la version
\CVS{}~courante de \Coq{} (\url{http://coq.inria.fr}). Elle permet de
cr\'eer un programme, le typer et g\'en\'erer un terme incomplet
correspondant (voir annexe \ref{fig:euclid-subtac}). 

\subsection{Existentielles}
La g\'en\'eration des buts correspondant aux variables existentielles et la
formation du terme final devaient originellement \^etre laiss\'ees \`a la
tactique \Refine~et au syst\`eme de gestion des existentielles de \Coq. Certaines limitations 
dans l'impl\'ementation du raffinement (le m\'ecanisme permettant de manipuler
des termes ``\`a trous'') nous ont emp\^ech\'e d'utiliser \Refine. En
particulier, la gestion des d\'efinitions r\'ecursives et la pr\'esence de
variables existentielles dans les types d'autres existentielles
 n'\'etaient pas support\'ees. En
cons\'equence, nous avons d\'evelopp\'e une nouvelle tactique \Coq{}
permettant de g\'erer les termes avec existentielles de fa\c con plus
g\'en\'erale. 

\subsubsection{La tactique \eterm}
L'id\'ee de d\'epart de la tactique \Refine{} est de prendre un terme \`a
trous et d'en faire une traduction en une s\'equence de tactiques. Par
exemple, lorsque \Refine{} rencontre une abstraction, il fait une
introduction, lorsque c'est un cast, on applique l'identit\'e et ainsi de
suite. Intuitivement, la s\'equence de tactiques engendr\'ee va construire
le terme de d\'epart implicitement. 

La tactique \eterm{} fonctionne diff\'erement. \`A partir d'un terme $t$
contenant des existentielles, \eterm{} va g\'en\'eraliser le terme par
rapport \`a celles-ci, et g�n�raliser chaque existentielle par rapport �
son contexte, cr\'eant ainsi un objet $(\lambda ex_1 : T_1, \ldots, ex_n :
T_n, t[?_1 := ex_1, \ldots, ?_n := ex_n])$, ou chaque $ex_i$ est appliqu�
aux variables introduites dans son contexte par les abstractions.
Habituellement, on propose $t$ comme habitant d'un type $T$ donn\'e (le
but), par exemple on peut proposer $\lambda x : \nat.x$ comme preuve du
but $\nat "->" \nat$.
Plut\^ot que de donner directement $t$, on applique le nouveau
terme, et \Coq{} va automatiquement nous demander d'instancier les
arguments $ex_1 \ldots ex_n$ correspondants aux existentielles du terme 
$t$. Cette technique permet d'avoir des d\'ependances entre existentielles
(par exemple, $ex_1$ peut apparaitre dans tout les types $T_2 \ldots
T_n$) et de ne pas reposer sur la gestion des existentielles de \Coq{}
qui n'est pas tr\`es flexible \`a l'heure actuelle.

Il nous faut nous pencher un peu plus avant sur la g\'en\'eralisation des
existentielles pour comprendre le m\'ecanisme d'\eterm.
Puisqu'on veut pouvoir avoir des d\'ependances entres les $n$
existentielles d'un termes et qu'on s\'erialise celles-ci en un produit
$n$-aire, il nous faut \^etre tr\`es attentifs \`a l'ordre dans lequel on
g\'en\'eralise les variables existentielles. Si $?_3 : T_3$ o\`u $T_3$
r\'ef\'erence $?_4$, il faut que l'existentielle $ex_4$ apparaisse
\emph{avant} $ex_3$ dans notre produit. Il est toujours possible de
trouver un ordre compatible avec ces d\'ependances puisqu'il est
impossible de cr\'eer un cycle o\`u $?_i$ r\'ef\'erencerait $?_j$ et vice-versa
(ceci est assur\'e par le caract\`ere fonctionnel des objets impl\'ementants
les existentielles dans \Coq). La tactique actuelle est cod\'ee avec
l'hypoth\`ese que toute existentielle $?_i$ ne d\'epend pas des
existentielles d'indice sup\'erieur \`a $i$. Il est cependant envisagable de
r\'eecrire tout terme contenant des existentielles comme un terme
\'equivalent avec des indices respectant cet ordre.

\subsection{Traitement de la r\'ecursion}
Lorsque l'on d\'eveloppe un programme r\'ecursif dans un syst\`eme tel que
\Coq, on est forc\'e de fournir une preuve de sa terminaison. Pour cela,
on montre g\'en\'eralement qu'on a un ordre bien
fond\'e sur le type de l'argument de r\'ecursion et que chaque appel respecte
cet ordre. Nous avons ajout\'e des facilit\'es d'\'ecriture de fonctions
r\'ecursives \`a notre langage ; on ajoute les existentielles
correspondant aux preuves que l'ordre est bien fond\'e ou qu'il est bien
respect\'e par les termes. Ainsi lors du raffinement on obtient naturellement
les buts correspondants \`a prouver.

\subsection{Traitement des inductifs}
Notre langage ne prend pas encore en compte les d\'efinitions inductives g\'en\'erales.
Au-del\`a du traitement des types sous-ensemble, on a un support minimal
pour les inductifs \`a deux constructeurs qui correspondent \`a des bool\'eens
annot\'es par des propri\'et\'es logiques (voir traitement de la
conditionnelle figure \ref{fig:euclid-subtac}). A long terme on devrait
pouvoir traiter les inductifs dans $\Set$ pr\'edicatif, qui ne peuvent 
embarquer des propositions qu'en utilisant des types sous-ensemble avec
le m\^eme m\'ecanisme de coercion et conserver l'inf\'erence.

\TODO{langage de Subtac}

%%% Local Variables: 
%%% mode: latex
%%% TeX-master: "subset-typing"
%%% LaTeX-command: "TEXINPUTS=\"style:$TEXINPUTS\" latex"
%%% End: 


%%% Local Variables: 
%%% mode: latex
%%% TeX-master: "subset-typing"
%%% LaTeX-command: "TEXINPUTS=\"style:$TEXINPUTS\" latex"
%%% End: 
