\chapter{Le calcul de coercion par pr�dicats}
Nous avons d�velopp� un langage supportant le \ps{} utilisable dans
\Coq. L'utilisateur peut d�finir des programmes dans un langage 
souple puis prouver certains buts pour obtenir finalement un terme de
\CCI{} complet v�rifiable par le noyau. On peut finalement utiliser 
les types d�pendants comme des types
simples et s'occuper des d�pendances dans un deuxi�me temps (pour la preuve).
L'architecture de notre syst�me est la suivante:
on type le programme dans notre langage \lng{} o� l'on peut faire des
abus de notations avec les objets de type sous-ensemble, puis l'on r��crit le terme typ�
dans \CCI{} en laissant des ``trous'' dans les termes qui d�sambig�ent
les abus et enfin \Coq{} se charge de g�n�rer les obligations correspondant � ces trous.


On va donc tout d'abord pr�senter le langage \lng{} (section \ref{section:russel}), puis un algorithme
de typage correct et complet pour les programmes �crits en \lng{}. Ensuite on montrera comment
plonger ce langage dans \CCI{} en ajoutant les coercions ad�quates et
enfin on expliquera comment se d�roule la g�n�ration des obligations de
preuves � partir des termes engendr�s par le plongement.

\setboolean{displayLabels}{true}

\section{Le langage \lng{}}
\label{section:russel}
Le langage que nous voulons est tr�s proche de \ML{}, plus les annotations
n�cessaires pour avoir un typage pr�cis et d�cidable. On �tudie ici une
restriction de \ML{}, purement fonctionnelle et sans filtrage, qu'on
�tendra dans la suite de notre travail. On n'a donc pas de types
inductifs mais on consid�re les types $\Sigma$, g�n�ralisation des
tuples de \ML{} form�s par l'op�rateur $*$.

\subsection{Syntaxe}
La syntaxe (figure \ref{fig:syntax}) est directement inspir�e des langages fonctionnels.
On part du \lc{} (variables, abstraction et application) puis l'on
ajoute des constantes (pour les entiers, bool�ens, etc...) ainsi que les
couples. La syntaxe $(x := `a, t : `t)$ permet de
cr�er des paires d�pendantes, de type $\Sigma x : `t. `t$. On peut aussi
appliquer un terme � un type pour instancier une fonction polymorphe par exemple.

Du c�t� des types, on a tout d'abord les types simples (constantes,
fl�che, produit cart�sien) qui sont des cas particuliers du produit ($\Pi$) et
de la somme ($\Sigma$) d�pendants. Les variables introduites par ces
types peuvent �tre utilis�es lors des applications de types. On
peut de plus abstraire sur les types avec le $\lambda$ (polymorphisme)
et les sortes.
Enfin on peut appliquer un type � un terme ($`t~`a$). 

%\vspace{-0.5cm}
\begin{figure}[ht]
  \begin{center}
    \subfigure[Termes]{\termgrammar}\quad
    \subfigure[Types]{\typegrammar}
  \end{center}
  \caption{Syntaxe}
  \label{fig:syntax}
\end{figure}
% \vspace{-1cm}

\subsection{S�mantique}
\typenvd

\typedFig
\subtdFig

La s�mantique du langage nous est donn�e par un syst�me de typage
(figure \ref{fig:typing-decl-rules} page \pageref{fig:typing-decl-rules}). Le
jugement de typage est d�fini inductivement par un ensemble de r�gles
d'inf�rence. 
Dans notre cas ce sont les r�gles du \CCfull{} (\CC{})
�tendu avec les $\Sigma$-types auxquelles on a ajout� une r�gle de
coercion (\irule{Coerce}, figure \ref{fig:typing-decl-rules}) que l'on trouve classiquement dans les syst�mes avec
sous-typage avec le nom de subsumption. 
Le jugement $\Gamma \typed t : T$ se lit: dans l'environnement $\Gamma$,
$t$ est de type $T$.

L'�quivalence utilis�e est $\eqbr$, soit la $\beta$-r�duction
classique ainsi que les projections $\pi_i$ pour les paires.

% \begin{remark}
%   En pratique, les types du Calcul des Constructions ne sont pas
%   toujours en forme normale et il peut donc �tre n�cessaire de les
%   r�duire (en t�te seulement) pour v�rifier des jugements du genre: 
%   $`G \seq t : \Pi x : T.V$.
% \end{remark}

La relation $\mathcal{R}$ d�finissant les produits formables
dans le syst�me est d�finie par les r�gles suivantes:
\begin{figure}
  \[\begin{array}{cccll}
    s_1 & s_2 & s_3 & \text{Habitants} & \text{Exemple} \\
    \hline
    \Prop & \Prop & \Prop & \text{Implication logique} & x ``<= 0 "->" x = 0  \\
    \Set & \Set & \Set & \text{Fonctions} & \Pi x : \nat. \nat \\
    \Type & \Set & \Type & \text{Fonctions polymorphes} & \Pi A : \Set, A
    "->" A \\
    \Set & \Type & \Type & \text{Types d�pendants} & \sref{vector} : \nat "->" \Set : \Set "->" \Type \\
    \Set & \Prop & \Prop & \text{Termes dans les propositions} & 
    \Pi n : \nat. \Pi l : \text{list}~n. \text{length}~l = n \\
    \Type & \Prop & \Prop & \text{Impr�dicativit� de } \Prop & \Pi x : Prop. x `V `! x \\
    % \Type(i) & \Type(j) & \Type(max~i~j)  & \text{Connecteurs logiques,
    %   \ldots} & \Pi A : \Prop. \Pi B : \Prop. A `^ B "->" B `^ A
  \end{array}\]
  \caption{D�finition de $\mathcal{R}$}
  \label{R-definition}
\end{figure}
On a un syst�me proche du \CCfull{} avec types $\Sigma$, mais
avec \Set{} pr�dicatif (comme dans \Coq{}).
On n'a pas $(\Prop,\Set,\Set)$ dans notre relation $\mathcal{R}$ pour
une bonne raison. Cela permet de cr�er des fonctions d�pendant de
propositions, par exemple $\Pi n : \nat, n > 0 "->" \Pi l :
\text{list}~A~n "->" A$. Or on veut � tout prix �viter d'introduire des
termes de preuve dans notre langage, et l'on voit que
cette fonction pourrait naturellement s'�crire $\Pi n : \mysubset{n}{\nat}{n > 0} "->" \Pi l :
\text{list}~A~n "->" A$. Encore une fois le type sous-ensemble nous
permet d'�viter d'avoir � passer des termes de preuve directement. 

Les sommes formables dans le syst�me sont r�duites au couples d'objets de
types de m�me sorte $s `: \{ \Prop, \Set \}$.
Dans le premier cas les habitants sont les couples de propositions
(codage du $`^$), dans le second ce sont les couples d'objets, soit les
paires de \ML.
Intuitivement, c'est le type sous-ensemble $\mysubset{x}{T}{P}$ qui permet
de faire des couples $\Set,\Prop$ habitant $\Set$. Les types $\Sigma x : U.V$ o� $U
: \Prop$ et $V : \Set$ n'ont pas d'int�r�t dans notre cas puisqu'ils
repr�sentent des objets de type $U `^ V$ mais on ne peut
pas utiliser $U$ dans notre syst�me. On pr�f�re coder ces objets par des
objets de type $\mysubset{x}{V}{U}$ (on n'est pas int�ress� par la preuve
de $U$ pour programmer).


La r�gle \irule{Coerce} formalise l'id�e que 
l'on peut utiliser un terme de type $T$ � la place d'un terme de type
$U$ si $T$ et $U$ sont dans une certaine relation. C'est l�
qu'interviendront les types sous-ensemble. \CC{} contient une r�gle de typage
similaire � \irule{Coerce}, la r�gle de conversion (\irule{Conv}), qui
dit essentiellement que deux types
$`b$-convertibles (on rappelle que l'on peut calculer dans les types
puisqu'on a l'abstraction, l'application, etc...) sont �quivalents.
On peut directement int�grer cette relation de $`b$-convertibilit� � notre
syst�me de coercion comme montr� figure \ref{fig:subtyping-decl-rules}
(\irule{SubConv}), � condition d'avoir l'inclusion
$\eqbr~\subseteq~\subd + \text{\irule{SubConv}}$.
En fait notre notion de r�duction est un peu plus large que $\beta$
puisqu'on peut r�duire les $\sref{let}$:
$\letml~(x,y) = (u, v)~\inml~t$ se r�duit en $t[u/x][v/y]$. En
pratique cette constructions est du sucre syntaxique pratique au niveau
du typage (on peut inf�rer le type de $t$), mais elle est inessentielle au
niveau du calcul.
On peut ais�ment rajouter un $\letml~x=t~\inml~v$ � notre langage de
fa�on similaire: c'est �quivalent � $(\lambda x : T.v)~t$, mais $T$ peut
est inf�r� plut�t que donn� par l'utilisateur.

On consid�re les constantes comme des variables pr�d�finies 
dans nos contextes, par exemple on a la constante $\sref{list} : \Pi x :
nat. \Set$. 
L'ajout d'une constante � un contexte ne doit pas alt�rer sa
bonne formation comme pour le cas des variables, donc son type doit �tre
bien form� (en g�n�ral, toute d�finition de \Coq~donne lieu � une
constante dans notre syst�me si elle est bien typ�e).

\subsubsection{Jugement de coercion}
Notre syst�me de coercion par pr�dicats permet � l'utilisateur
d'utiliser une valeur de type $U$ l� o� l'on attend une valeur de type
$\mysubset{x}{V}{P}$ (\irule{SubProof}) si $U$ est lui-m�me coercible en $V$.
A l'inverse, on permet aussi d'utiliser une valeur de type
$\mysubset{x}{U}{P}$ (\irule{SubSub}) � la place d'une valeur de type
$V$ si $U$ est coercible vers $V$. Notre jugement de coercion est donc
sym�trique et laisse beaucoup de libert� � l'utilisateur au moment du
codage. Par exemple on peut d�river $u : \nat \type u : \mysubset{x}{\nat}{`_}$
Seulement, lors de la traduction de la d�rivation de coercion $\nat
\subd \mysubset{x}{\nat}{`_}$ (n�cessaire pour traduire l'abus de notation
$x : \mysubset{x}{\nat}{`_}$), l'utilisateur aura � r�soudre une obligation
de preuve de $`_$. On repose donc toujours sur la coh�rence du Calcul
des Constructions. 
Les r�gles \irule{SubProd} et \irule{SubSigma} permettent de faire des
coercions dans les types composites. Classiquement, la r�gle pour le 
produit fonctionnel est contravariante (une fonction sous-type d'une
autre accepte plus d'entr�es mais donne une sortie plus fine, voir
 \cite{journals/toplas/Castagna95}) et la r�gle pour le 
produit cart�sien covariante (une paire est coercible en une autre si 
leurs composantes sont coercibles deux-�-deux). Le sens des coercions
n'a pas d'importance dans le syst�me d�claratif puisqu'il est sym�trique
mais il est essentiel lors de la cr�ation des coercions que nous
d�crirons plus tard.

La r�gle \irule{SubTrans} assure que l'on a un syst�me compositionnel. Il y a ici une
analogie avec l'�limination des coupures dans les syst�mes logiques, o�
l'on montre que toute d�rivation utilisant la r�gle de \emph{modus ponens} ($A "=>" B$ et $B "=>" C$ implique
$A "=>" C$) peut se r��crire en une d�rivation ne l'utilisant
jamais. Dans les syst�mes � sous-typage, on montre de fa�on �quivalente
que l'on peut �liminer la r�gle de transitivit� ; premi�re �tape vers un
syst�me d�cidable.


Notre jugement de coercion identifie les types $U$ et $\mysubset{x}{U}{P}$
mais notre syst�me de typage ne permet pas d'�liminer (prendre la partie
preuve) ou d'introduire (cr�er un couple t�moin,preuve) des objets de
type sous-ensemble. Cela nous assure une certaine coh�rence, puisque
m�me si l'on ne v�rifie pas qu'un objet de type $U$ a bien la propri�t�
$P$, on ne peut pas raisonner sur le fait que $U$ a la propri�t� dans le
langage.


On ne fera pas la m�tath�orie du syst�me d�claratif ici, puisque
c'est une extension conservative du Calcul des Constructions et l'on
�tudiera en d�tail le syst�me algorithmique. Notre preuve de conservativit�
est simple: si l'on oublie les utilisations des types sous-ensemble de
notre syst�me de typage (\irule{Subset}) et de coercion
(\irule{SubProof} et \irule{SubSub}), alors le jugement de coercion est 
juste la $\beta$-convertibilit� et donc \irule{Coerce} et \irule{Conv} 
sont �quivalentes. Comme les autres r�gles de notre syst�me d�claratif
proviennent directement de \CC{}, on arrive � un syst�me strictement
�gal au syst�me du calcul des constructions. On peut donc s'appuyer sur
les r�sultats connus pour \CC{} pour une partie de notre syst�me.

Pour une �tude compl�te du \CCfull{}, se r�f�rer �
\cite{Barras99,Luo90}.
On va plut�t s'int�resser � la construction d'un algorithme de typage
correspondant � notre syst�me d�claratif.

%%% Local Variables: 
%%% mode: latex
%%% TeX-master: "subset-typing"
%%% LaTeX-command: "TEXINPUTS=\"style:$TEXINPUTS\" latex"
%%% End: 

\newpage
\subsection{�laboration du syst�me algorithmique \& propri�t�s}
\typenva

Pour pouvoir impl�menter le typeur, il nous faut un syst�me dirig� par la
syntaxe. 
C'est presque le cas pour la coercion, il y a juste la r�gle
\irule{SubConv} qu'on peut appliquer � n'importe quel moment. 


\subsubsection*{Conversion}
On note $\suba$ le m�me syst�me que figure \ref{subtyping-decl-rules} mais
o� l'on n'applique \irule{SubConv} seulement si aucune autre r�gle ne
s'applique.

On montre que les deux syst�mes de coercion sont �quivalents: 

Il nous faut tout d'abord un lemme sur la conversion:
\begin{lemma}
  \label{conversion-pi}
  Si $\Pi T U \eqbi S$ alors $S \eqbi \Pi T' U'$ avec $T \eqbi T'$ et $U
  \eqbi U'$.
\end{lemma}

Le seul cas int�ressant est si les deux termes sont dans la relation de $\beta$-�quivalence:
\begin{lemma}[Conservation de la conversion par sous-typage]
  \label{conversion-coercion}
  Si $`G \typea T, U : s$ et $T \eqbi U$ alors $T \suba U$.
\end{lemma}

\begin{proof}
  Par induction sur la forme de $T$.
  
  \def\seq{\suba}.
  
  \begin{itemize}
  \item[$T$ atomique:]
    On a alors $U = T$, trivial.
    
  \item[$T `= \Pi X.Y$:]
    Alors $U `= \Pi V.W$ et $X \eqbi V$, $Y \eqbi W$
    d'apr�s le lemme \ref{conversion-pi}.
    Par induction $X \sub Y$ et $V \sub W$. 
    On applique alors \irule{SubProd} � ces deux pr�misses.
    
  \item[$T `= \Sigma X.Y$:]
    $U$ est de la forme $\Sigma V.W$, avec $X \eqbi V$ et $Y \eqbi
    W$. Par induction et application de \irule{SubSigma}.
    
  \item[$T `= \subset{x}{X}{P}$:] 
    On a alors $U `= \subset{x}{X'}{P'}$ avec $X \eqbi X'$, $P \eqbi
    P'$, et la propri�t� est vraie par \irule{SubLeft} et \irule{SubRight}:
    
    \begin{prooftree}
      \AXC{$X \sub X'$}
      \LeftLabel{\SubLeftRule}
      \UIC{$\subset{x}{X}{P} \sub X'$}
      \LeftLabel{\SubRightRule}
      \UIC{$\subset{x}{X}{P} \sub \subset{x}{X'}{P'}$}
    \end{prooftree}
  \end{itemize}
\end{proof}


Il faut cependant s'assurer que deux types de sortes
diff�rentes ne peuvent �tre identifi�s. En effet dans notre syst�me
la $\beta$-convertibilit� n'assure pas que deux termes ont la m�me
sorte, par exemple:
$\typed (\lambda x : \Type. x)~(\nat:\Set) : \Type "->"_\beta \nat : \Set$.
\TODO{Dans mon syst�me, on ne peut pas faire la coercion de nat � Set}

Le fait que les arguments sont tout deux sort�s avec la m�me sorte
avant de d�river le jugement de coercion nous assure que l'on ne fera
pas d'identification erron�e dans les autres r�gles que \irule{SubConv}.

En cons�quence $\subd$ et $\suba$ sont �quivalentes. Le syst�me
d'inf�rence de $\suba$ donne donc un algorithme pour d�cider de la relation
de coercion. L'ind�terminisme entre les r�gles \irule{SubProof} et
\irule{SubSub} ne pose pas de probl�me: on peut laisser le choix � 
l'impl�mentation puisque le syst�me est confluent.

Cependant, il reste une source importante d'ind�cidabilit� dans le
syst�me de typage, c'est la r�gle de coercion. On va l'�liminer du
syst�me algorithmique apr�s avoir montr� que
toute d�rivation de typage utilisant \irule{Coerce} peut se r��crire en
une d�rivation n'utilisant cette r�gle qu'� sa racine.
\def\subs{\subset{x}{U}{P}}
Il nous faut changer quelque peu les r�gles pour obtenir le syst�me
algorithmique. En particulier, on va utiliser la fonction $\mu0$ de \PVS{}
\cite{PVS-Semantics:TR} renom�e $\mu$ ici pour op�rer des
\emph{d�compr�hension}. Cette fonction efface les constructeurs de type
sous-ensemble en t�te d'un type, par exemple: $\mualgo(\subset{f}{\nat
  "->" \nat}{f~0 = 0}) = \nat "->" \nat$. Cela va nous permettre de
restreindre l'utilisation du jugement du sous-typage � l'application.

\TODO{bof}
\begin{remark}
  En pratique, les types du Calcul des Constructions ne sont pas
  toujours en forme normale et il peut donc �tre n�cessaire de les
  r�duire pour v�rifier des jugements du genre: 
  $`G \seq t : \Pi x : T.V$. On pourrait voir $\mu$ comme une
  extension de la relation de $\beta$-�quivalence avec la r�duction
  $\subset{x}{U}{P} "->"_\mu U$. 
\end{remark}

\subsection*{Sommes d�pendantes}
En inspectant la r�gle \irule{Sum}, on remarque qu'il n'est pas possible
d'inf�rer le type $U$ � partir du seul terme $(t, u)$. Cela
n�cessiterait de r�soudre un probl�me d'unification d'ordre sup�rieur
auquel il n'y a pas de solution la plus g�n�rale. On introduit donc dans
le syst�me algorithmique deux nouvelles r�gles, dont une (\irule{SumDep})
permettant d'annoter le terme avec le type $U$ recherch�. On consid�re
que c'est � l'utilisateur d'annoter suffisament les termes. Par d�faut,
si le terme n'est pas annot� on consid�re l'objet $(u, v)$ comme une
paire non-d�pendante (\irule{SumInf}).

\typenva
\typeaFig
\typemuaFig
\subtaFig

\subsection*{Subsumption}
\typenva

% On enl�ve \irule{Coerce} du syst�me et on change la r�gle 
% d'application \irule{App} de la figure \ref{typing-decl-rules}.
% On note $\typea$ le syst�me de typage obtenu. 

\begin{proposition}[Passage de la subsumption � l'application]
  \label{subsum-elim}
  On peut r�ecrire toute d�rivation $`G \typea t : T$ utilisant la
  subsumption ailleurs qu'� sa racine vers une d�rivation $`G \type t :
  U$ avec $U \suba T$.
\end{proposition}

On a besoin de quelques lemmes auparavant:

\begin{lemma}[$\beta$-equivalence et $\mu$]
  \label{beta-mu}
  Si $X \sub Y$ et $\mu~Y \eqbi \Sigma x : T.U$ alors $\mu X \eqbi \Sigma x : T'.U'$
  et $T' \sub T$, $U' \sub U$.
  Si $X \sub Y$ et $\mu~Y \eqbi \Pi x : T.U$ alors $\mu X \eqbi \Pi x :
  T'.U'$ et $T \sub T'$, $U' \sub U$.
\end{lemma}
\begin{proof}
  \begin{induction}
    Par induction sur la d�rivation de coercion, on fait le cas pour $\Sigma$.
    
    \case{SubConv} Trivial, puisqu'on aura $\mu~X = \mu~Y$.

    \case{SubProd} Impossible, $\mu$ ne traversant pas les produits.

    \case{SubSigma} Direct, on a une d�rivation de $\Sigma x : T'. U'
    \sub \Sigma x : T.U$.
    
    \case{SubLeft} Ici, $Y `= \subset{x}{V}{P}$, on peut donc d�duire que
    $\mu~Y = \mu~V \eqbi \Sigma x : T.U$. On 
    applique l'hypoth�se de r�curence avec $X \sub V$ et on obtient:
    $\mu~X \eqbi \Sigma x : T'.U' `^ T' \sub T `^ U' \sub U$.

    \case{SubRight} Ici, $X `= \subset{x}{V}{P}$. Par induction, 
    $\mu~V = \mu~X \eqbi \Sigma x : T'.U' `^ T' \sub T `^ U' \sub U$.
  \end{induction}
\end{proof}

\begin{lemma}[Bonne formation des contextes]
  \label{wf-contexts-a}
  Si $`G \type t : T$ alors $\typewf `G$.
\end{lemma}
\begin{proof}
  \inductionon{typing-decl}
\end{proof}

\begin{fact}[Inversion du jugement de bonne formation]
  \label{inversion-wf-a}
  Si $\typewf `G, x : U$ alors $`G \type U : s$ et $s `: \{ \Set, \Prop, \Type(i) \}$.
\end{fact}

\begin{lemma}[Affaiblissement]
  \label{weakening-a}
  Si $`G, `D \type t : T$ alors pour tout $x : S `; `G, `D$ tel que
  $\wf `G, x : S, `D$, $`G, x : S, `D \type t : T$
\end{lemma}

\begin{proof}
  \begin{induction}[typing-decl]
    \casetwo{PropSet}{Type} Trivial.

    \case{Var}
    On a $x : S `; `G, `D$, donc $`G, x : S, `D \type y : T$ est toujours d�rivable.
    
    \case{Prod}
    Par induction $`G, x : S, `D \type T : s1$ et $`G, x : S, `D,
    y : T \seq U : s2$. On applique \irule{Prod} pour obtenir 
    $`G, x : S, `D \type \Pi x : T.U : s2$. De m�me pour le reste des r�gles.
  \end{induction}
\end{proof}  

Le renforcement montre que notre notion de sous-typage est correcte
vis-�-vis du typage. On peut d�river les m�mes jugements dans des
contextes o� les variables ont des types plus pr�cis.

\begin{lemma}[Renforcement]
  \label{narrowing-a}
  \[ `G \seq S, S' : s, S' \sub S "=>" 
  \left\{ \begin{array}{lcl}
      \typewf `G, x : S, `D & "=>" & \typewf `G, x : S', `D \\
      & `^{} & \\
      `G, x : S, `D \seq t : T & "=>" & `G, x : S', `D \seq t : T
    \end{array}
  \right. \]
\end{lemma}

\begin{proof}
  Par induction sur la taille de la d�rivation de typage ou de bonne formation.
    
  \begin{induction}
    \case{WfEmpty} Trivial.
    
    \case{WfVar} 
    La conclusion est $\typewf `G, x : S, `D$
    
    \begin{induction}[text=Par induction sur la taille de $`D$]
    \item[\protect{$`D = []$}]
      La racine de la d�rivation est de la forme:
      \begin{prooftree}
        \UAX{WfVar}
        {$`G \type S : s$}
        {$\wf `G, x : S$}
        {$s `: \{ \Set, \Prop, \Type(i) \}$}
      \end{prooftree}
      On a $`G \type S' : s$, donc par \irule{WfVar}, $\typewf `G, x : S'$.  
      
    \item[\protect{$`D `= `D', y : U$}]
      La racine de la d�rivation est de la forme:
      \begin{prooftree}
        \UAX{WfVar}
        {$`G, x : S, `D' \type U : t$}
        {$\wf `G, x : S, `D', y : U$}
        {$s `: \{ \Set, \Prop, \Type(i) \}$}
      \end{prooftree}
      Par induction sur la d�rivation de typage $`G, x : S', `D' \seq U : t$,
      on a donc bien $\typewf `G, x : S', `D', y : U$ par \irule{WfVar}.
    \end{induction}
    
    \casetwo{PropSet}{Type} 
    Par induction, $\typewf `G, x : S', `D$, on applique simplement la r�gle.
    
    \case{Var}
    Par induction, $\typewf `G, x : S', `D$. La seule diff�rence avec le
    contexte pr�cedent est le type associ� � $x$, donc si $t \not= x$, on
    peut simplement r�appliquer \irule{Var}. Si $t `= x$ on construit la
    d�rivation:

    \begin{prooftree}
      \BAX{Var}
      {$\wf `G, x : S', `D$}
      {$x : S' `: `G$}
      {$`G, x : S', `D \seq x : S'$}
      {}
      \AXC{$`G, x : S', `D \type S,S' : s$}
      \AXC{$S' \sub S$} % `G \subt 
      \TIC{$`G, x : S', `D \type x : S$}
    \end{prooftree}
    
    Par l'affaiblissement (lemme \ref{weakening-a}) et $`G \type S,S' : s$,
    on obtient la pr�misse $`G, x : S', `D \type S,S' : s$.
    
    \case{Prod} 
    Par induction, $`G, x : S', `D \type T : s1$ et $`G, x : S', `D
    y : T \seq U : s2$. On applique \irule{Prod} pour obtenir 
    $`G, x : S' \type \Pi x : T.U : s2$. De m�me pour le reste des r�gles.
  \end{induction}
\end{proof}

On peut maintenant montrer notre proposition:
\begin{proof}
  On inspecte les d�rivations possibles utilisant \irule{Coerce} juste avant
  une autre r�gle.
  
  \typenva
  \begin{induction}
    \case{Var} Par de pr�misse de la forme $`G \type t : T$, donc
    pas d'application de \irule{Subsum} possible.
    
    \case{Abs} \quad
    
    \begin{prooftree}
      \AXC{\vdots}
      \UIC{$`G \seq \Pi x : T. U : s $}
      \AXC{\vdots}
      \UIC{$`G, x : T \seq M : U'$}
      \AXC{\vdots}
      \UIC{$`G, x : T \seq U', U : s$}
      \AXC{\vdots}
      \UIC{$U' \sub U$}
      \TIC{$`G, x : T \seq M : U $}
      \BIC{$`G \seq \lambda x : T. M : \Pi x : T.U$}
    \end{prooftree}
    
    Comme $`G, x : T \typed U' : s$ on peut former le produit 
    $`G \typea \Pi x : T. U' : s$.
    On r�ecrit donc la d�rivation en:     
    
    \begin{prooftree}
      \AXC{\vdots}
      \UIC{$`G \seq \Pi x : T. U' : s $}
      \AXC{\vdots}
      \UIC{$`G, x : T \seq M : U'$}
      \BIC{$`G \seq \lambda x : T. M : \Pi x : T.U'$}
      \AXC{$T \eqbi T$}
      \UIC{$T \sub T$}
      \AXC{\vdots}
      \UIC{$U' \sub U$}
      \BIC{$\Pi x : T.U' \sub \Pi X : T.U$}        
      \BIC{$`G \seq \lambda x : T. M : \Pi x : T.U$}
    \end{prooftree}
    
    \case{Sum}\quad
    
    \begin{prooftree}
      \AXC{\vdots}
      \UIC{$`G \seq \Sigma x : T.U : s $}
      \AXC{\vdots}
      \UIC{$`G \seq t : S$}
      \AXC{\vdots}
      \UIC{$S \sub T$}
      \BIC{$`G \seq t : T $}
      \AXC{\vdots}
      \UIC{$`G \seq u : U[t/x]$}
      \TIC{$`G \seq (t,u) : \Sigma x : T.U$}
    \end{prooftree}
    
    Comme $S \sub T$, on a bien $\Sigma x : S.U \sub \Sigma x : T.U$
    par \irule{SubSigma}. On sait de plus que $\Sigma x : S.U$ est bien
    sort� de sorte $s$. En effet, par inversion de $`G \seq \Sigma x :
    T.U : s$ on a $`G, x : T \seq U : s$ et par renforcement ($S \sub T$), $`G, x : S \seq
    U : s$. On peut donc d�river:

    \begin{prooftree}
      \AXC{\vdots}
      \UIC{$`G \seq \Sigma x : S.U : s $}
      \AXC{\vdots}
      \UIC{$`G \seq t : S$}
      \AXC{\vdots}
      \UIC{$`G \seq u : U[t/x]$}
      \TIC{$`G \seq (t,u) : \Sigma x : S.U$}
      \AXC{\vdots}
      \UIC{$\Sigma x : S.U \sub \Sigma x : T.U$}
      \BIC{$`G \seq (t,u) : \Sigma x : T.U$}
    \end{prooftree}
    
    \case{LetSum}\quad
    \begin{prooftree}
      \AXC{\vdots}        
      \UIC{$`G \seq t : S $}
      \AXC{\vdots}
      \UIC{$S \sub V$}
      \BIC{$`G \seq t : V$}
      \AXC{$\mu~V \eqbi \Sigma x : T.U$}
      \AXC{\vdots}
      \UIC{$`G, x : T, u : U \seq v : V $}
      \TIC{$`G \seq \letml~(x, u) = t~\inml~v : V$}
    \end{prooftree}
    
    Par le lemme \ref{beta-mu}, on a $\mu~S \eqbi \Sigma x : T'.U'$,
    $T' \suba T$ et $U' \suba U$. Par renforcement (lemme
    \ref{narrowing-a}) on a:

    \begin{prooftree}
      \AXC{\vdots}        
      \UIC{$`G \seq t : S $}
      \AXC{$\mu~S \eqbi \Sigma x : T'.U'$}
      \AXC{\vdots}
      \UIC{$`G, x : T', u : U' \seq v : V $}
      \TIC{$`G \seq \letml~(x, u) = t~\inml~v : V$}
    \end{prooftree}

    Si l'on applique la subsumption � droite c'est direct ($V' \sub V$).
    
    \case{LetIn}
    De fa�on similaire, par renforcement on obtient la d�rivation sans \irule{Coerce}.
    
    \case{App}\quad    
    \begin{prooftree}
      \AXC{\vdots}
      \UIC{$`G \seq f : T $}
      \AXC{\vdots}
      \UIC{$ T \sub \Pi x : V. W$}
      \BIC{$`G \seq f : \Pi x : V. W $}
      \noLine
      \UIC{$\mualgo(\Pi x : V.W) \eqbi \Pi x : V.W$}
      \AXC{\vdots}
      \UIC{$`G \seq u : U $}
      \AXC{\vdots}
      \UIC{$U \sub V$}
      \TIC{$`G \seq f u : W [ u / x ]$}
    \end{prooftree}
    
    Par le lemme \ref{beta-mu} on a $\mu(T) \eqbi \Pi x : V'.W'$ avec
    $V \suba V'$ et $W' \suba W$. Par la transitivit� de la coercion on
    a: $U \suba V `^ V \suba V' "=>" U \suba V'$.
    On peut donc d�river:
    
    \begin{prooftree}
      \AXC{\vdots}
      \UIC{$`G \seq f : T $}
      \noLine
      \UIC{$\mualgo(T) \eqbi \Pi x : V'.W'$}
      \AXC{\vdots}
      \UIC{$`G \seq u : U $}
      \AXC{\vdots}
      \UIC{$U \sub V'$}
      \TIC{$`G \seq f u : W [ u / x ]$}
    \end{prooftree}
    
    \case{Subsum}\quad

    Dans le cas o� l'on a un enchainement de r�gles \irule{Coerce}:
    \begin{prooftree}
      \AXC{\vdots}
      \UIC{$`G \seq t : S$}
      \AXC{\vdots}
      \UIC{$S \sub T$}
      \BIC{$`G \seq t : T$}
      \AXC{\vdots}
      \UIC{$T \sub U$}
      \BIC{$`G \seq t : U$}
    \end{prooftree}
    
    En utilisant la transitivit� de la coercion on obtient:
    \begin{prooftree}
      \AXC{\vdots}
      \UIC{$`G \seq t : S$}
      \AXC{\vdots}
      \UIC{$S \sub U$}
      \BIC{$`G \seq t : U$}
    \end{prooftree}

  \end{induction}
  
\end{proof}

Par induction, on a donc montr� que l'on peut r�duire l'utilisation de la r�gle de
subsumption � la racine d'une d�rivation. On peut ignorer sans perte de g�n�ralit� 
l'utilisation de la coercion � la racine de la d�rivation, 
on fera de toute fa�on un test de coercion entre le type inf�r� et le
type sp�cifi� juste avant la r��criture. On consid�re maintenant le
syst�me algorithmique comme pr�sent� figure \ref{fig:typing-algo-rules}.

\begin{lemma}[Substitutivit� de $\mualgo$]
  \label{substitutive-mu}
  Si $\mualgo(T) = U$ alors $\mualgo(T[u/x]) = U[u/x]$.
\end{lemma}

\begin{proof}
  Il suffit de suivre la d�finition de $\mualgo$.
  Si $T$ est de la forme $\sub{y}{V}{P}$ alors $\mualgo(T) =
  \mualgo(V)$ et $T[u/x]$ est de la forme
  $\sub{y}{V[u/x]}{P[u/x]}$. Par induction, $\mualgo(T[u/x]) =
  \mualgo(V[u/x]) = \mualgo(V)[u/x] = \mualgo(T)[u/x]$.
  Sinon c'est trivial.
\end{proof}

\begin{lemma}[Substitutivit� de la coercion]
  \label{substitutive-coercion}
  Si $U \suba T$ alors pour tout $u$, $U[u/x] \suba T[u/x]$.
\end{lemma}

\begin{proof}
  \begin{induction}[subtyping-algo]
    \case{SubConv}
    Direct par pr�servation de l'�quivalence $\eqbi$ par substitution.
    
    \case{SubProd}
    Par induction $U[u/x] \suba T[u/x]$ et $V[u/x] \suba W[u/x]$, donc
    $\Pi y : T[u/x].V[u/x] \suba \Pi y : U[u/x].W[u/x]$. La propri�t�
    est donc bien v�rifi�e.
    
    \case{SubSigma} Direct par induction.
    
    \case{SubSub} Par induction, $U'[u/x] \suba V[u/x]$. On applique
    \irule{SubLeft} pour obtenir $\sub{y}{U'[u/x]}{P} \suba V[u/x]$. 
    
    \case{SubRight} Direct par induction.
  \end{induction}
\end{proof}

\begin{lemma}[Substitutivit� du typage]
  \label{substitutive-typing}
  Si $`G \typea u : U$ alors
  \[ \left\{ \begin{array}{lcl}
      `G, x : U, `D \typea t : T & "=>" & `G, `D[u/x] \typea t[u/x] :
      T[u/x] \\
      \wf `G, x : U, `D & "=>" & \wf `G, `D[u/x]
    \end{array}\right. \]
\end{lemma}

\begin{proof}
  \typenva
  Par induction mutuelle sur la d�rivation de typage $`G, x : U, `D
  \typea t : T$ ou $\wf `G, x : U, `D$.
  
  \begin{induction}
    \case{WfEmpty} Trivial.

    \case{WfVar}
    Par induction sur $`D$.
    \begin{itemize}
    \item[\protect{$`D = []$}]
      On a alors $`G \typea U : s$ donc $\wf `G$ et triviallement, $\wf
      `G, `D[u/x]$.

    \item[\protect{$`D = `D', y : T$}]
      On a alors $`G, x : U, `D' \typea T : s$ et par induction
      $`G, `D'[u/x] \typea T[u/x] : s[u/x]$. Donc on peut appliquer
      \irule{WfVar} pour obtenir $\wf `G, `D'[u/x], T[u/x]$ soit
      $\wf `G, `D[u/x]$
    \end{itemize}
    
    \casetwo{PropSet}{Type} 
    La substitution n'a aucun effet et $`G, `D[u/x]$ est bien
    form� par induction.
    
    \case{Var}
    Par induction, $\wf `G, `D[u/x]$.
    Si $t `= x$ alors on a $T = U$ et $T[u/x] = U$ puisque $x$
    n'apparait pas dans $U$. On a donc bien $`G, `D[u/x] \typea t[u/x] = u :
    T[u/x] = U$. 
    Si $y : T `: `G$ alors on applique simplement \irule{Var}.
    Si $y : T `: `G$ alors $y : T[u/x] `: `D[u/x]$ et on obtient
    $`G, `D[u/x] \typea y[u/x] :  T[u/x]$ par \irule{Var}.
    
    \case{Prod}
    Par induction  $`G, `D[u/x] \typea T[u/x] : s_1[u/x]$ et
    $`G, `D[u/x], y : T[u/x] \typea M[u/x] : s_2[u/x]$. 
    On peut appliquer \irule{Prod} pour obtenir $`G, `D[u/x] \typea \Pi
    y : T[u/x].M[u/x] : s_2[u/x]$ soit $`G, `D[u/x] \typea (\Pi y :
    T.M)[u/x] : s_2[u/x]$.
    De fa�on similaire pour les autres cas.

    \case{App}
    On �tudie le cas de l'application qui requiert un lemme suppl�mentaire.
    Par induction, $`G, `D[u/x] \typea f[u/x] : T[u/x]$ et
    $`G, `D[u/x] \typea a[u/x] : A[u/x]$. Si $\mualgo(T) \eqbi \Pi y :
    V.W$ alors $\mualgo(T[u/x]) \eqbi \Pi y : V[u/x].W[u/x]$ (lemme
    \ref{substitutive-mu}). Par induction, on a aussi $`G, `D[u/x] \typea
    A[u/x],V[u/x] : s$. Enfin, par substitutivit� de la coercion on a $A[u/x]
    \suba V[u/x]$. On peut donc appliquer \irule{App} pour obtenir 
    $`G, `D[u/x] \typea (f[u/x] a[u/x]) : W[u/x][a[u/x]/y]$. Or $W[u/x][a[u/x]/y] =
    W[a/y][u/x]$. On a donc bien $`G, `D[u/x] \typea (f a)[u/x] :
    (W[a/y])[u/x]$.
    On a un raisonnement similaire pour \irule{LetSum}.


  \end{induction}
  
\end{proof}

\begin{lemma}[Substitutivit� du typage avec coercion]
  \label{substitutive-typing-coercion}
  Si $`G, x : V \typea t : T \sub U$, $G \typea u : V$
  alors $`G \typed t[u/x] : T[u/x] \sub U[u/x]$.
\end{lemma}

\begin{proof}
  Par substitutivit� du typage (\ref{substitutive-typing}) on a $`G \typed t[u/x] : T[u/x]$.
  Par le lemme pr�c�dent $T[u/x] \suba U[u/x]$.
\end{proof}

\begin{lemma}[Inversion de la coercion]
  \label{inversion-coercion}
  Si $`G \subta \lambda x : T. M : \Pi x : T.U \sub \Pi x : V.W$ 
  alors $`G, x : T \typea M : U$.
\end{lemma}
\begin{proof}
  \TODO{pas utilis� ?}
\end{proof}

\begin{fact}[Sym�trie de la coercion]
  La relation $\sub$ est sym�trique.
\end{fact}

\begin{lemma}[Coercion et $\mu$]
  \label{coercion-mu}
  Si $\Pi x : X.Y \sub U$ alors $\mu(U) \eqbi \Pi x : X'.Y'$ et $X' \sub
  X$, $Y \sub Y'$.
  Si $\Sigma x : X.Y \sub U$ alors $\mu(U) \eqbi \Sigma x : X'.Y'$ et $X \sub
  X'$, $Y \sub Y'$.
  Pour tout $U$, $U \sub \mu(U)$.
\end{lemma}
\begin{proof}
  Par induction sur les d�rivations de $\suba$ et la d�finition de $\mu$.
\end{proof}

\begin{lemma}[Coercion et conversion]
  \label{coercion-conversion}
  Si $S \eqbi T$ et $T \sub U$ alors $S \sub U$
\end{lemma}

\begin{proof}
  Par simple inspection des r�gles on voit que le jugement ne peut
  distinguer deux termes $\beta$-�quivalents.
\end{proof}

\begin{lemma}[Transitivit� de la coercion]
  \label{transitive-coercion}
  Pour tout $S, T, U$,  si $S \sub T$ et $T \sub U$ alors $S \sub U$.
\end{lemma}

\begin{proof}  
  \TODO{Dans \cite{Pierce:TypeSystems}, voir p. 420}
  On proc�de par �limination de la r�gle \irule{SubTrans} dans toute
  d�rivation de $S \sub U$.
  
  \begin{induction}[subtyping-decl]

    \case{SubConv}\quad
    \begin{prooftree}
      \AXC{$S \eqbi T$}
      \UIC{$S \sub T$}
      \AXC{$T \sub U$}
      \BIC{$S \sub U$}
    \end{prooftree}
    
    Par le lemme pr�c�dent, on �limine trivialement \irule{SubTrans}.

    \case{SubProd}\quad
    \begin{prooftree}
      \AXC{$X' \sub X$}
      \AXC{$Y \sub Y'$}
      \BIC{$\Pi~X~Y \sub \Pi~X'~Y'$}
      \AXC{$\Pi~X'~Y' \sub U$}
      \BIC{$\Pi~X~Y \sub U$}
    \end{prooftree}
    
    Si $\Pi~X'~Y' \sub U$, alors il existe $S$, $T$, tel que $\mu(U)
    \eqbi \Pi~S~T$ et $S \sub X'$, $Y' \sub T$.
    Par induction, $S \sub X$ et $Y \sub T$ donc $\Pi~X~Y \sub \Pi~S~T$ 
    et enfin $\Pi~X~Y \sub U$.

    De facon �quivalente pour le second cas.

    \case{SubSigma}\quad
    \begin{prooftree}
      \AXC{$X \sub X'$}
      \AXC{$Y \sub Y'$}
      \BIC{$\Sigma~X~Y \sub \Sigma~ X'~Y'$}
      \AXC{$\Sigma~X'~Y' \sub U$}
      \BIC{$\Sigma~X~Y \sub U$}
    \end{prooftree}
    
    Si $\Sigma~X'~Y' \sub U$, alors il existe $S$, $T$, tel que $\mu(U)
    \eqbi \Sigma~S~T$ et $X' \sub S$, $Y' \sub T$.
    Par induction, $X \sub S$ et $Y \sub T$ donc $\Sigma~X~Y \sub
    \Sigma~S~T$ et enfin $\Sigma~X~Y \sub U$ (lemme \ref{coercion-mu}).
    
    De fa�on �quivalente pour le second cas.

    \case{SubProof}\quad
    \begin{prooftree}
      \AXC{$S \seq V$}
      \UIC{$S \seq \subset{x}{V}{P}$}
      \AXC{$\subset{x}{V}{P} \sub U$}
      \BIC{$S \sub U$}
    \end{prooftree}
    
    Si $\subset{x}{V}{P} \sub U$ alors $\mu(U) \eqbi V$.
    Comme $S \seq V$, par induction, $S \seq \mu(U)$ donc
    $S \sub U$.
    
    De fa�on �quivalente pour le second cas.

    \case{SubSub}
    Cas identique � \irule{SubProof}
    
  \end{induction}
\end{proof}

\begin{corrolary}[Compl�tude de la coercion]
  \label{complete-coercion}
  Si $U \subd V$ alors $U \suba V$.
\end{corrolary}

\begin{proof}
  Les r�gles des deux syst�mes sont les m�mes except� \irule{SubTrans}
  qu'on peut �liminer dans le syst�me algorithmique. De plus
  l'application restreinte de la conversion ne change pas les jugements
  d�rivables (lemme \label{conversion-sub}).
\end{proof}

\begin{theorem}[Correction de la coercion]
  \label{correct-coercion}
  Si $U \suba V$ alors $U \subd V$.
\end{theorem}

\begin{proof}
  Les r�gles du syst�me algorithmique sont un sous-ensemble des r�gles
  du syst�me d�claratif.
\end{proof}

\begin{theorem}[Correction du typage]
  \label{correct-typing}
  Si $`G \typea t : T$ alors $`G \typed t : T$
\end{theorem}

\setboolean{displayLabels}{false}
\begin{proof}
  \begin{induction}[typing-algo]
  \item[\irule{WfEmpty},\irule{WfVar},\irule{PropSet},\irule{Var},\irule{Prod},\irule{Abs},
    \irule{LetIn}, \irule{Sigma}, \irule{Sum}:] r�gles inchang�es.
    
    \case{LetSum}
    On a 
    \begin{prooftree}
      \LetSumA
    \end{prooftree}
    
    Par induction, $`G \typed t : S$, et par correction de la coercion $S \subd \Sigma x : T. U$.
    On peut donc d�river $`G \typed t : \Sigma x : T.U$ � l'aide de \irule{Coerce}.
    On peut directement appliquer \irule{LetSum} � cette pr�misse et �
    l'hypoth�se d'induction $`G, x : T, y : U \typed v : V$.
    
    \case{App} On a:
    \def\fCenter{\typea}
    \begin{prooftree}
      \AppA
    \end{prooftree}
    
    Par induction, $`G \typed f : T$, et $T \subd \Pi x : V. W$.
    On peut donc d�river $`G \typed f : \Pi x : V.W$ � l'aide de la
    subsumption.
    Par le lemme \ref{correct-coercion}, et l'hypoth�se $`G \typed u :
    U$, on obtient $`G \typed u : V$ par \irule{Subsum}.
    Donc, par \irule{App}, on a bien $`G \typed f u : W[u/x]$.
  \end{induction}  
\end{proof}

\setboolean{displayLabels}{true}
\begin{lemma}[Compl�tude du typage]
  \label{complete-typing}
  $`G \typed t : T "=>" `E U, `G \subta t : U \sub T$
\end{lemma}

\begin{proof}
  \begin{induction}[typing-decl]
  \item[\irule{WfEmpty},\irule{WfVar},\irule{PropSet},\irule{Var},\irule{Prod},\irule{Abs},
    \irule{LetIn}, \irule{Sigma}, \irule{Sum}, \irule{Subset}:] r�gles inchang�es.
    
    \case{App} On a 
    \typenvd
    \begin{prooftree}
      \App
    \end{prooftree}
    
    \typenva
    Par induction, $`E T, `G \typea f : T \suba \Pi x : V. W$ et
    $`E U, `G \subta u : U \sub V$.
    
    Si $T \suba \Pi x : V.W$ alors $\mualgo(T) \eqbi \Pi x : V'.W'$ avec
    $V \suba V'$ et $W' \suba W$.

    Par transivit� de la coercion: $U \suba V'$, on peut donc d�river 
    \begin{prooftree}
      \TAX{App}
      {$`G \seq f : T \quad \mualgo(T) \eqbi \Pi x : V'. W'$}
      {$`G \seq u : U \quad `G \seq U, V' : s$}
      {$U \suba V'$}
      {$`G \seq (f u) : W' [ u / x ]$}
      {}
    \end{prooftree}
    
    Par substitutivit� de la coercion (lemme
    \ref{substitutive-coercion}), $W'[u/x] \suba W[u/x]$, la propri�t�
    est donc bien v�rifi�e.

    \case{LetSum} On a
    \typenvd
    \begin{prooftree}
      \LetSum
    \end{prooftree}
    
    \typenva
    Par induction, $`E S, `G \typea t : S \suba \Sigma x : T.U$ et 
    $`E V', `G, x: T, y : U \typea v : V' \suba V$.
    On a $\mualgo(S) \eqbi \Sigma x : T'.U'$ avec $T' \suba T$ et $U'
    \suba U$. Par renforcement on peut donc d�river $`G, x : T', y : U'
    \seq v : V'$.
    
    On a donc la d�rivation suivante dans le syst�me algorithmique:
    \begin{prooftree}
      \TAX{LetSum}
      {$`G \seq t : S$}
      {$\mualgo(S) \eqbi `S x : T'. U'$}
      {$`G, x : T', y : U' \seq v : V'$}
      {$`G \seq \letml~(x, u) = t~\inml~v : V'$}
      {}
    \end{prooftree}
    
    Comme $V' \suba V$, la propri�t� est vraie.

    \casetwo{Conv}{Subsum}
    Dans les deux cas on a inductivement $`E T', `G \typea t : T'
    \suba T$. Avec \irule{Conv} on a $T \eqbi S$, donc $T' \suba S$ par
    le lemme \ref{coercion-conversion}. Pour \irule{Subsum} on a $T \subd S$.
    Par compl�tude de la coercion, $T \suba S$ et par transitivit� de la
    coercion, $T' \suba S$. La propri�t� est donc bien v�rifi�e dans les
    deux cas.
    
  \end{induction}
  
\end{proof}

On combine les th�or�mes de correction et compl�tude pour obtenir la propri�t� suivante entre les deux syst�mes:
\begin{corrolary}[�quivalence des syst�mes d�claratifs et algorithmiques]
  $`G \typed t : T$ \ssi{} il existe $U$ tel que $`G \typea t : U$ et $U \suba T$.
\end{corrolary}

On a maintenant un syst�me raffin� d�rivant les m�me jugements (�
coercion pr�s) que le syst�me d�claratif. On veut en extraire un
algorithme de typage. Pour cela on doit pouvoir r�soudre deux probl�mes:
\begin{itemize}
\item\textbf{V�rification de type.} On donne $`G$,$t$ et $T$ et l'on doit
  d�cider si $`G \typea t : T$ ;
\item\textbf{Inf�rence de type.} On donne $`G$,$t$ et l'on doit trouver $T$ tel
  que $`G \typea t : T$ si c'est d�rivable, sinon on �choue.
\end{itemize}
En pratique, la v�rification a besoin de l'inf�rence puisque lorsqu'on
v�rifie une application $f u : T$ on doit inf�rer le type de $f$.
On montre donc les th�or�mes suivants:

\begin{theorem}[D�cidabilit� de l'inf�rence dans le syst�me algorithmique]
  Le probl�me d'inf�rence $`G \typea t : ?$ est d�cidable.
\end{theorem}

\begin{proof}
  Il suffit d'observer que les r�gles de typage sont dirig�es par la
  syntaxe du deuxi�me argument et permettent donc d'inf�rer un type pour
  tout terme. En lisant les pr�misses de chaque r�gle de gauche �
  droite, on voit que l'inf�rence est d�cidable.
\end{proof}

\begin{theorem}[D�cidabilit� de $\typea$]
  La relation de typage $`G \typea t : T$ est d�cidable.
\end{theorem}
\begin{proof}
  Direct. On utilise le th�or�me pr�c�dent pour le cas de l'application.
\end{proof}

%%% Local Variables: 
%%% mode: latex
%%% TeX-master: "subset-typing"
%%% LaTeX-command: "TEXINPUTS=\"style:$TEXINPUTS\" latex"
%%% End: 

\section{G�n�ration des obligations de preuve}
On veut d�sormais traduire les d�rivations du syst�me algorithmique
dans \CCI{} dont le jugement de typage est $\typec$. 
Les termes de \lng{} ne sont pas directement typables dans \CCI{}
puisque nous avons permis d'utiliser des objets comme s'ils avaient des
types diff�rents de leurs types originaux avec la r�gle de coercion. Il
va donc falloir maintenant expliciter ces coercions pour obtenir des
termes typables dans \CCI{}. Cependant, on ne peut pas cr�er un terme
complet � partir de notre d�rivation, puisqu'on ne peut pas inf�rer des
preuves arbitraires. On utilise donc des existentielles (intuitivement
des trous dont on ne connait que le type des habitants) pour traduire le
fait qu'il est de la responsabilit� de l'utilisateur de prouver que son
utilisation de la coercion n'�tait pas incorrecte.

\subsection{Interpr�tation}
On d�finit l'interpr�tation $\ip{t}{`G}$ par r�currence sur la forme des
termes (figure \ref{fig:interp}). Cette interpr�tation renvoie
un terme $t'$ r�ecrit que l'on montrera bien typ� dans l'environnement \CCI{} $\ipG{`G}$.

\begin{definition}[Interpr�tation des contextes]
  \label{ctx-interp}
  On fait l'extension aux contextes de la fa�on suivante:
  \begin{itemize}
  \item $\ipG{[]} = []$
  \item $\ipG{`G, x : T} =  \ipG{`G}, x : \ip{T}{`G}$
  \end{itemize}
\end{definition}

\def\typeafn#1#2{\typeml_{#1}(#2)}
\begin{figure}
  \[\begin{array}{lcll}
    \ip{x}{`G} & = & x & \\
    \\
    \ip{s}{`G} & = & s & s `: \setproptype\\
    \\
    \ip{\Pi x : T.U}{`G} 
    & = & \Pi x : \ip{T}{`G}.\ip{U}{`G, x : T} & \\
    & \\
    \ip{\lambda x : `t.v}{`G} 
    & = & \letml~`t' = \ip{`t}{`G}~\inml & \\
    & & \letml~v' = \ip{v}{`G, x : `t}~\inml & \\
    & & (\lambda x : `t'. v') & \\
    & \\
    \ip{f~u}{`G} 
    & = & \letml~F~=\typeafn{`G}{f}~\andml~U = \typeafn{`G}{u}~\inml & \\
    & & \letml~(\Pi x : V.W) = \mualgo{F}~\inml & \\
    & & \letml~\pi = \coerce{`G}{F}{(\Pi x : V.W)} & \\
    & & \letml~c = \coerce{`G}{U}{V}\inml & \\
    & & (\pi~\ip{f}{`G})~(c~\ip{u}{`G}) & \\
    & \\
    \ip{\Sigma x : T.U}{`G} 
    & = & \Sigma x : \ip{T}{`G}.\ip{U}{`G, x : T} & \\
    & \\
    \ip{\pair{\Sigma x : T.U}{t}{u}}{`G}
    & = & \letml~t' = \ip{t}{`G}~\inml & \\
    & & \letml~T' = \typeafn{`G}{t}~\inml& \\
    & & \letml~ct = \coerce{`G}{T'}{T}~\inml& \\
    & & \letml~U' = \typeafn{`G}{u}~\inml& \\
    & & \letml~u' = \ip{u}{`G}~\inml & \\
    & & \letml~cu = \coerce{`G}{U'}{U[t/x]}~\inml & \\
    & & \pair{\ip{\Sigma x : T.U}{`G}}{ct[t']}{cu[u']} & \\
    & \\
    \ip{\pi_i~t}{`G} 
    & = & \letml~t' = \ip{t}{`G}~\inml & i `: \{ 1, 2 \} \\
    & & \letml~T = \typeafn{`G}{t}~\inml & \\
    & & \letml~\Sigma x : V.W = \mualgo{T}~\inml & \\
    & & \letml~c = \coerce{`G}{T}{(\Sigma x : V.W)}~\inml & \\
    & & \pi_i~c[t'] & \\
    & \\
    \ip{\mysubset{x}{U}{P}}{`G}
    & = & \mysubset{x}{\ip{U}{`G}}{\ip{P}{`G, x : U}}
  \end{array}\]
  \caption{Interpr�tation dans \CCI{}}
  \label{fig:interp}
\end{figure}

Chaque jugement de coercion du syst�me algorithmique permet de d�river
une coercion explicite qui sera directement appliqu�e � un objet.

On formalise donc les coercions par des contextes d'�valuation classiques.
\begin{definition}[Contextes d'�valuation]
  \label{eval-ctx}
  Un contexte d'�valuation est un terme form� � partir de la grammaire
  originale des termes � laquelle on ajoute des terminaux $\ctxdot$ dans
  chacune de r�gles.
\end{definition}

\begin{definition}[Substitution et composition de coercions]
  La substitution (l'application) dans un contexte d'�valuation est not�e $c[d]$, elle
  remplace toutes les occurrences de $\ctxdot$ dans $c$ par $d$.

  Le composition de deux coercions not�e $c `o d$ est �gale � $c[d]$,
  son �l�ment neutre est $\ctxdot$.
  
  La substitution d'un terme pour une variable dans un contexte
  d'�valuation est not�e $c[t/x]$ comme pour les termes.
\end{definition}

\subsubsection{Coercions explicites}
On d�finit le syst�me $\suba$ (figure \ref{fig:coerce-impl-rules})
qui d�rive une coercion � partir de deux types $S$ et $T$ dans un environnement $`G$.
On a introduit du d�terminisme par rapport au jugement de
coercion algorithmique puisqu'on donne priorit� � la r�gle
\irule{SubSub} par rapport � la r�gle \irule{SubProof} (ces r�gles sont
confluentes comme nous le monterons lemme \ref{coerce-unicity}). On explicite aussi la priorit� donn�e � la mise en forme
normale de t�te (figure \ref{fig:hnfdef}) puis � la d�rivation
par rapport au test de conversion dans la pr�misse de \irule{SubConv}.

Notre op�ration de mise en forme normale de t�te est d�finie de la fa�on suivante:
\begin{figure}[ht]
  \[\begin{array}{lcll}
    \hnf{((\lambda x : T.e)~v)} & = & \hnf{e[v/x]} & \\
    \hnf{\pi_1~(x, y)} & = & \hnf{x} & \\
    \hnf{\pi_2~(x, y)} & = & \hnf{y} & \\
    \hnf{e} & = & e & \{\text{si $e$ est d'une autre forme}\}
  \end{array}\]
  \caption{D�finition de la r�duction de t�te}
  \label{fig:hnfdef}
\end{figure}

On note donc $\hnf{T}$ la forme normale de t�te $T$ et $\nf{T}$ la forme normale
de $T$.

\begin{figure}[ht]
  \[\begin{array}{llcll}
    (\beta) & (\lambda x : X.e)~v & = & e[v/x] & \\
    (\pi_i) & \pi_i~\pair{T}{e_1}{e_2} & = & e_i & \\
    (\sigma_i) & \sigma_i~(\elt{E}{P}{e_1}{e_2}) & = & e_i & \\
    (\eta) & (\lambda x : X.e~x) & = & e & \text{si $x `; FV(e)$} \\ % et $e : \Pi x : X.Y$} \\
    (\rho) & \pair{\Sigma x : X.Y}{\pi_1~e}{\pi_2~e} & = & e & \text{si $e : \Sigma x : X. Y$} \\
    & \elt{E}{P}{(\eltpit~e)}{(\eltpip~e)} & = & e & \text{si $e : \mysubset{x}{E}{P}$} \\
    (\sigma) & \elt{E}{P}{t}{p} & = & \elt{E}{P}{t'}{p'} & \text{si $t
      `= t'$}
  \end{array}\]
  \caption{Th�orie �quationnelle de \CCI{}}
  \label{fig:eqcci}
\end{figure}

On utilise l'�quivalence $\eqbpers$ d�finie comme la cl�ture r�flexive,
sym�trique et transitive de la relation d�finie figure
\ref{fig:eqcci}. Cette relation sera d�not�e par $`=$ pour plus de
clart�. Cette relation contient la $\beta$-r�duction et les projections
pour les sommes d�pendantes, mais aussi des relations n�cessaires pour
supporter l'interpr�tation de termes de \lng{} dans le langage. On a
donc la r�gle $\eta$ pour l'abstraction et $\rho$ pour le
\emph{surjective pairing} qui s'applique aux sommes d�pendantes et aux
objets de type sous-ensemble. Enfin on a une forme limit�e
d'indiff�rence aux preuves pour les objets de type sous-ensemble.
On ajoute une r�gle de typage au syst�me de \CCI{} pour typer les
existentielles:
\begin{prooftree}
  \AXC{$\tcoq{`G}{P}{\Prop}$}
  \UIC{$\tcoq{`G}{\ex{P}}{P}$}
\end{prooftree}

% enrichie avec les
%existentielles, on �tend donc la relation aux existentielles de la fa�on
%suivante: \[\ex{`G}{P} \eqbres \ex{`G'}{P'} `= `G \eqbres `G' `^ P \eqbres P'\]

\subtiFig

Le syst�me figure \ref{fig:coerce-impl-rules} d�rive les termes de
coercion. Il a de bonnes propri�t�s pour la preuve et l'impl�mentation
telles que l'unicit� et l'admissibilit� de la transitivit� que nous
montrerons plus tard. 

\subsection{Propri�t�s}
On veut montrer que si l'on a un jugement valide dans notre syst�me
algorithmique, alors son image par l'interpr�tation est un jugement
valide de \CCI{}. On rappelle que \CCI{} est �quivalent au premier
calcul pr�sent� o� la r�gle de coercion est remplac�e par la r�gle 
de conversion.

\typenvi

\subsubsection{Correction}
Notre probl�me se ram�ne � montrer le th�or�me suivant: 
\[`G \typea t : T "=>" \iG \typec \ip{t}{`G} : \ip{T}{`G}\]
Ce r�sultat ne se montre pas ais�ment.
En effet le jugement de coercion rend la preuve
tr�s difficile � cause de son caract�re
non local. Pour mieux comprendre ce probl�me, consid�rons l'exemple
suivant:

\paragraph{Exemple}
Dans le syst�me algorithmique, on peut tr�s bien d�river
$\Pi n : \nat. \sref{list}~n \suba \Pi n :
\mysubset{x}{\nat}{P}.\listml~n$ puisque $\mysubset{x}{\nat}{P} \suba
\nat$ et $\listml~n \eqbr \listml~n$.
Si l'on interpr�te ces deux types, une coercion va �tre ins�r�e dans le
second type: $\ip{\Pi n : \mysubset{x}{\nat}{P}.\listml~n}{`G} = \Pi n :
\mysubset{x}{\nat}{P}.\listml~(\pi_1~n)$. La coercion g�n�r�e doit donc
avoir pour type: $\Pi n : \nat. \listml~n "->" \Pi n :
\mysubset{x}{\nat}{P}.\listml~(\pi_1~n)$, mais elle est d�riv�e en se
basant seulement sur les types algorithmiques. On peut v�rifier ici
que l'intuition de la coercion par pr�dicats est bonne, puisqu'on peut
d�river ce jugement:

\begin{prooftree}
  \AXC{$\nat \eqbr \nat$}
  \UIC{$\subimpl{`G'}{\ctxdot}{\nat}{\nat}$}
  \UIC{$\subimpl{`G'}{\pi_1~\ctxdot}{\mysubset{x}{\nat}{P}}{\nat}$}

  \AXC{$\listml~n \eqbr \listml~n$}
  \UIC{$\subimpl{`G, n : \mysubset{x}{\nat}{P}}{\ctxdot}
    {\listml~n}{\listml~n}$}

  \BIC{$\subimpl{`G'}{\lambda x : \ip{\mysubset{x}{\nat}{P}}{`G}.
      \ctxdot~[\ctxdot~(\pi_1~x)] = \ctxdot~(\pi_1 x)}{\Pi n : \nat. \listml~n}
    {\Pi n : \mysubset{x}{\nat}{P}.\listml~n}$}
\end{prooftree}

Supposons $`G \typec t : \Pi n : \nat. \listml~n$ alors on a la d�rivation
de typage suivante:
\begin{prooftree}
  
  \AXC{$\timpl{`G, x : \mysubset{x}{\nat}{\ip{P}{`G}}}{t}
    {\Pi n : \nat. \listml~n}$}

  \AXC{$\timpl{`G, x : \mysubset{x}{\nat}{\ip{P}{`G}}}{\pi_1~x}{\nat}$}
  
  \BIC{$\timpl{`G, x : \mysubset{x}{\nat}{\ip{P}{`G}}}
    {t~(\pi_1~x)}
    {\listml~(\pi_1~x)}$}

  \UIC{$\timpl{`G}
    {\lambda x : \ip{\mysubset{x}{\nat}{P}}{`G}. t~(\pi_1~x)}
    {\Pi n : \ip{\mysubset{x}{\nat}{P}}{`G}.\listml~(\pi_1~x)}$}
\end{prooftree}

On cr�e donc bien un terme de type $\ip{\Pi n :
  \mysubset{x}{\nat}{P}.\listml~n}{`G}$ en appliquant la coercion a un
terme de type $\ip{\Pi n : \nat. \listml~n}{`G}$, c'est l'effet recherch�.

\subsection{Coercions explicites}
On d�finit le syst�me $\subi$ (figure \ref{fig:subtyping-impl-rules})
qui d�rive une coercion � partir de deux types $S$ et $T$ dans un environnement $`G$.
On a introduit du d�terminisme par rapport au jugement de
coercion algorithmique puisqu'on donne priorit� � la r�gle
\irule{SubSub} par rapport � la r�gle \irule{SubProof} (ces r�gles sont
confluentes). On explicite aussi la priorit�e donn�e � la mise en forme
normale de t�te (figure \ref{fig:hnfdef}) puis � la d�rivation
par rapport au test de
conversion dans la pr�misse de \irule{SubConv}.

Notre op�ration de mise en forme normale de t�te est d�finie de la fa�on suivante:
\begin{figure}[h]
  \[\begin{array}{lcl}
    \hnf{((\lambda x : T.e)~v)} & = & \hnf{e[v/x]} \\
    \hnf{x} & = & x
  \end{array}\]
  \caption{D�finition de la forme normale de t�te}
  \label{fig:hnfdef}
\end{figure}

On utilise la $\beta$-�quivalence de $CCI$ enrichi avec les
existentielles, on �tend donc la relation aux existentielles de la fa�on
suivante: $\ex{`G}{P} \eqbi \ex{`G'}{P'} `= `G \eqbi `G' `^ P \eqbi P'$.

\coerceFig
\subticFig


\begin{lemma}[Coercion et conversion dans le contexte]
  \label{coercion-conversion-ctx-impl}
  Si $`G, x : V, `D \typec c : S \subi T$ et $V \eqbi V'$ alors
  $`G, x : V', `D \typec c' : S \subi T$ et $c \eqbi c'$.
\end{lemma}
\begin{proof}
  Par induction sur la d�rivation de coercion. Tout les cas passent par
  induction sauf \irule{SubConv} qui n'utilise pas $`G$ en pr�misse donc
  ne pose pas de probl�me. Le seul cas int�ressant est \irule{SubProof}
  qui utilise la $\beta$-�quivalence sur les existentielles.
\end{proof}

On conserve les propri�t�s suivantes sur la coercion:
\def\subhnf{\subihnf}
\begin{lemma}[Coercion et conversion]
  \label{coercion-conversion-impl}
  Si $`G \typea S,T,U,V : s$, $S \eqbi T$, $U \eqbi V$ et
  $`G \typec c : T \sub U$ alors $`G \typec c' : S \sub V$ avec 
  $c \eqbi c'$. 
\end{lemma}

\begin{proof}
  Par simple inspection des r�gles on voit que le jugement ne peut
  distinguer deux termes $\beta$-�quivalents (ils ont forc�ment des
  formes normales de t�te �quivalentes). On en d�duit que la derni�re
  r�gle appliqu�e dans le jugement $T \sub U$ est la seule r�gle
  applicable � $S \sub V$.

  Par induction sur la d�rivation dans $\sub$.
   
  \begin{induction}
    \case{SubHnf}\quad
    Comme $S \eqbi T$, $S \eqbi \hnf{T}$. 
    De m�me comme $U \eqbi V$, $\hnf{U} \eqbi V$.
    On a l'hypoth�se $\hnf{T} \sub \hnf{U}$ donc
    par induction on a $c' : \hnf{S} \sub \hnf{V}$ avec $c' \eqbi c$. 
    On d�rive bien $`G \typec c' : S \sub V$ avec $c' \eqbi c$ par \irule{SubHnf}. 
    
    \case{SubConv}\quad
    On a  $S \eqbi T$, $T \eqbi U$ et $U \eqbi V$ donc $S \eqbi V$ par transitivit�
    de la $\beta$-�quivalence. On a donc $c' = \lambda x : S.x
    \eqbi \lambda x : T.x$ d�rivable par une application optionnelle de
    \irule{SubHnf} puis \irule{SubConv}. 
    C'est la seule r�gle applicable pour d�river
    le jugement $S \sub V$, sinon on aurait des symboles de t�te
    diff�rents pour deux termes $\beta$-�quivalents. 

    \case{SubProd}\quad
    Soit $T = \Pi x : A.B$ et $U = \Pi x : C.D$.
    On a $\subimpl{`G}{c_1}{A}{C}$ et $\subimpl{`G, x : C}{c_2}{B[c_1~x/x]}{D}$
    Or si $S \eqbi \Pi x : A.B$ et $\Pi x : C.D \eqbi V$ alors $\hnf{S}
    = \Pi x : A'.B'$ avec $A \eqbi A'$ et $B \eqbi B'$ et $\hnf{V} =
    \Pi x : C'.D'$ avec $C \eqbi C'$ et $D \eqbi D'$.
      
    Par induction on a $`G \typec c_1' : C' \sub A'$ avec $c_1'
    \eqbi c_1$. Par hypoth�se on a $`G, x : C \typec c_2 : B[c_1~x/x]
    \sub D$. Par induction: $`G, x : C \typec c_2' : B'[c_1'~x/x] \sub
    D'$ avec $c_2' \eqbi c_2$ car $B[c_1~x/x] \eqbi B'[c_1~x/x]$.
    On peut donc d�duire $`G, x : C' \typec c_2' : B'[c_1'~x/x] \sub D'$
    par le lemme \ref{coercion-conversion-ctx-impl}. 
    On d�duit par
    \irule{SubProd} que $c' : \hnf{S} \sub \hnf{V}$ est
    d�rivable. On a bien $c \eqbi c'$ car $c_1' \eqbi c_1$ et $c_2'
    \eqbi c_2$, donc $\lambda f.\lambda x.c_2~(f~(c_1~x)) \eqbi
    \lambda f.\lambda x.c_2'~(f~(c_1'~x))$.

    \case{SubSigma}\quad
    De m�me par \irule{SubSigma}:
    
    \begin{prooftree}      
      \AXC{$\subimpl{`G}{c_1'}{A'}{C'}$}
      \AXC{$\subimpl{`G, x : A'}{c_2'}{B'}{D'[c_1'~x/x]}$}
      \BIC{$\subimplhnf{`G}{\lambda t : \Sigma x : A'. B'.~\letml~(x, y) = t~\inml~(c_1'~x, c_2'~y)}
        {\Sigma x : A'. B'}{\Sigma x : C'. D'}$}
    \end{prooftree}
 
    \case{SubSub}\quad  
    On a $\hnf{T} = \mysubset{x}{T'}{P}$ donc $\hnf{S} = \mysubset{x}{S'}{P'}$
    avec $T' \eqbi S'$ et $P \eqbi P'$.
    Par hypoth�se, $\subimpl{`G}{c}{T'}{U}$, donc par induction,
    $\subimpl{`G}{c'}{S'}{\hnf{V}}$ ($U \eqbi \hnf{V}$) et par \irule{SubSub}, 
    $\subimpl{`G}{c' `o \pi_1}{\mysubset{x}{S'}{P'} = \hnf{S}}{\hnf{V}}$.
    Clairement, $c' `o \pi_1 \eqbi c `o \pi_1$, la propri�t� est donc
    bien v�rifi�e.
    
    \case{SubProof}\quad
    On a $U = \mysubset{x}{U'}{P}$.
    De m�me on obtient:
    \begin{prooftree}
      \UAX{SubProof}
      {$\subimpl{`G}{c'}{\hnf{S}}{V' \eqbi U'}$}
      {$\subimplhnf{`G}
        {\lambda t : \hnf{S}.~\elt{V'}{(\lambda x : V'.P')}{(c'~t)}{?_{P'[c'~t/x]}}}
        {\hnf{S}}{\hnf{V} = \mysubset{x}{V'}{P'}}$}
      {}
    \end{prooftree}

    Encore une fois $c \eqbi c'$ est v�rifi� car $\ex{`G, t :
      \hnf{S}}{P'[c'~t/x]} \eqbi \ex{`G, x : T}{P[c~t/x]}$.
  \end{induction}
\end{proof}

\begin{lemma}[Coercion et formes normales de t�te]
  \label{substi-coercion-hnf}
  Si $`G \typec c : T \suba U$ alors $`G \typec c' : \hnf{T}~\suba
  \hnf{U}$ avec $c \eqbi c'$ est d�rivable par une
  d�rivation plus petite ou �gale.
\end{lemma}

\begin{proof}
  Par idempotence de la mise en forme normale de t�te, on a la m�me
  d�rivation dans le cas ou la derni�re r�gle appliqu�e �tait
  \irule{SubHnf}, sinon c'est trivial.
\end{proof}

\begin{lemma}[Coercion de termes convertibles]
  \label{subti-eqb-coercion-eqbe-id}
  Si $`G \typea T, U : s$ et $T \eqb U$ alors il existe $c$,
  $`G \typec c : T \suba U$ avec $c \eqbe \lambda x : T.x$.
\end{lemma}

\begin{proof}
  Par induction sur la forme normale de $T$ not�e $\nf{T}$.

  \begin{itemize}
  \item $\nf{T} = \Pi y : A.B$ Alors, $\nf{U} = \Pi y : A'.B'$ avec 
    $A \eqb A'$, $B \eqb B'$. Par induction, $c_1 \eqbe \lambda x : \ip{A'}{`G}.x
    : A' \sub A$ et $c_2 \eqbe \lambda y : \ip{B'}{`G, x : A'}.y : B
    \sub B'$. On a donc:
    \begin{eqnarray*}
      c & = & \lambda f : \ip{\Pi x : A.B}{`G}.\lambda x :
      \ip{A'}{`G}. c_2~f~(c_1~x) \\
      & \eqbe & \lambda f : \ip{\Pi x : A.B}{`G}.\lambda x :
      \ip{A'}{`G}. f~x \\
      & \eqbe & \lambda f : \ip{\Pi x : A.B}{`G}. f
    \end{eqnarray*}
    De fa�on �quivalente pour $\Sigma, \{|\}$.
    
  \item Si $\nf{T}$ n'a pas pour symbole de t�te, $\Pi, \Sigma$ ou
    $\{|\}$, alors il est possible que $U$ est pour symbole de t�te un
    type sous-ensemble auquel cas le m�canisme est similaire. 
    Dans tout les autres cas, \irule{SubConv} est la seule r�gle
    applicable et on a $c = \lambda x : T.x$.
  \end{itemize}
  
\end{proof}


\def\GD{`G, x : U, `D}
\def\Gr{`G, `D[u/x]}
\def\iGD{\ipG{\GD}}
\def\iGr{\ipG{\Gr}}


\begin{lemma}[Coercion de sortes]
  \label{subti-coercion-sorts}
  Si $`G \typec e : s \subi T$ ou $`G \typec e : T \subi s$ alors $T
  \eqbi s$ et $e \eqbi \lambda x : s.x$.
\end{lemma}
\begin{proof}  
  Clairement on ne peut d�river $s \suba T$ que par \irule{SubConv}
  (�ventuellement pr�c�d� de \irule{SubHnf}). En effet seule la r�gle
  \irule{SubProof} pourrait s'appliquer, mais cela impliquerait que
  $T \eqbi \mysubset{x}{U}{P}$ avec $s \suba U$ et ainsi de suite. La
  seule possibilit� est de d�river $s \eqbi T$ ou $s \eqbi U$, auquel cas $U$ est une
  sorte ce qui contredit le fait que $\mysubset{x}{U}{P} : s$ dans le
  cas pr�c�dent. On a d�rive donc $s \subi T$ si et seulement si $s
  \eqbi T$.
\end{proof}

\begin{lemma}[Stabilit� de la coercion par substitution]
  \label{subti-coercion-subst}
  Si $\GD \typea T, T' : s$, $`G \typec u :
  U$, alors
  $\begin{array}{lcl}
    \GD \typea t : T & "=>" & \ip{t[u/x]}{\Gr} \eqbe
    \ip{t}{\GD}[\ip{u}{`G}/x] \\
    \subimpl{\GD}{c}{T}{T'} & "=>" & \subimpl{\Gr}{c'}{T[u/x]}{T'[u/x]}
    `^ c' \eqbei c[\ip{u}{`G}/x]
  \end{array}$
  
\end{lemma}

\begin{proof}
  Par induction mutuelle sur les d�rivation de $c$ et $t$.

  \begin{induction}
    \case{SubHnf} On a: 
    \begin{prooftree}
      \AXC{$\subimpl{\GD}{c}{\hnf{T}}{\hnf{T'}}$}
      \UIC{$\subimpl{\GD}{c}{T}{T'}$}
    \end{prooftree}
    
    \def\as{\overrightarrow{a}}    
    \def\bs{\overrightarrow{b}}    
    \def\asux{\overrightarrow{a[u/x]}}
    \def\bsux{\overrightarrow{b[u/x]}}
    \def\ipux{[\ip{u}{`G}/x]}

    Par induction on a
    $\subimpl{\Gr}{c'}{\hnf{T}[u/x]}{\hnf{T'}[u/x]}$ avec $c' \eqbe
    c\ipux$.
    Si $\hnf{T[u/x]} = \hnf{T}[u/x]$ et $\hnf{T'[u/x]} = \hnf{T'}[u/x]$
    c'est direct par induction. Sinon, on a $\hnf{T} = x~\as$ ou $\hnf{T'} =
    x~\as$. Les deux cas sont similaires, on traite le cas ou $\hnf{T} =
    x~\as$. Le jugement $x~\as \suba \hnf{T'}$
    ne peut �tre d�riv� que par \irule{SubProof} ou
    \irule{SubConv}. Dans le premier cas, cela implique qu'on a une
    d�rivation de $d : x~\as \suba U'$ o� $\hnf{T'} = \mysubset{y}{U'}{P}$. Par
    induction on a donc une d�rivation de
    $\subimpl{\Gr}{d'}{u~\asux}{U'[u/x]}$ avec $d' \eqbei
    d\ipux$. 
    Par le lemme \ref{substi-coercion-hnf} on a donc une d�rivation de 
    $\subimpl{\Gr}{d'}{\hnf{(u~\asux)}}{\hnf{(U'[u/x])}}$ avec $d' \eqbei
    d\ipux$. 
    
    \begin{prooftree}
      \AXC{$\subimpl{\Gr}{d'}{\hnf{(u~\asux)}}{\hnf{(U'[u/x])}}$}
      \UIC{$\subimpl{\Gr}{d'}{\hnf{(u~\asux)}}{U'[u/x]}$}
      \UIC{$\subimpl{\Gr}{c'}{\hnf{T[u/x]}}{\hnf{(T'[u/x])} = \mysubset{y}{U'[u/x]}{P[u/x]}}$}
      \UIC{$\subimpl{\Gr}{c'}{T[u/x]}{T'[u/x]}$}
    \end{prooftree}
      
      On a $\ip{\hnf{T}}{\GD}\ipux \eqbi
      \ip{\hnf{T}[u/x]}{\Gr}$ par induction. Or $\hnf{(\hnf{T}[u/x])} =
      \hnf{(u~\asux)} = \hnf{(T[u/x])}$, donc, du fait qu'on met en forme
      normale de t�te avant interpr�tation \TODO{Pas explicite dans la
        def de l'interpr�tation}:
      \begin{eqnarray}
        \ip{\hnf{(T[u/x])}}{\Gr} & = & \ip{\hnf{T}[u/x]}{\Gr} \\
        & \eqbi & \ip{\hnf{T}}{\GD}\ipux \label{eq:hnftux}
      \end{eqnarray}
      
      Par induction, on a $\ip{U'[u/x]}{\Gr} \eqbi \ip{U'}{\GD}\ipux$ et
      $\ip{P[u/x]}{\Gr} = \ip{P}{\GD}\ipux$. 
      On en d�duit:
      \begin{eqnarray*}
        \ip{P[u/x]}{\Gr}[d'~t/y] & \eqbi & \ip{P}{\GD}\ipux[d'~t/y] \\
        & \eqbi & \ip{P}{\GD}\ipux[d\ipux~t/y] \\
        & = & \ip{P}{\GD}[d~t/y]\ipux
      \end{eqnarray*}
      
      Soit $A = \ipG{\Gr, t : \hnf{T[u/x]}} \typec \ip{P[u/x]}{}[d'~t/y]$ et
      $B = \ipG{\GD, t : \hnf{T}} \typec \ip{P}{}[d~t/y]$. On a
      $A \eqbi B\ipux$ car $\ipG{\Gr, t : \hnf{T[u/x]}} \eqbi \ipG{\GD, t :
        \hnf{T}}\ipux$ par \ref{eq:hnftux} et les conclusion des
      existentielles sont convertibles par le r�sultat pr�c�dent.
      
      On a donc:
      \begin{eqnarray*}
        c' & = & \lambda t : \ip{\hnf{(T[u/x])}}{\Gr}. \\
        & &
        \elt{\ip{U'[u/x]}{\Gr}}
        {\ip{(\lambda y : (U'[u/x]).P[u/x])}{\Gr}}
        {(d'~t)}
        {?_A} \\
        & \eqbe & \lambda t : \ip{\hnf{(T[u/x])}}{\Gr}. \\
        & &
        \elt{\ip{U'[u/x]}{\Gr}}
        {\ip{(\lambda y : (U'[u/x]).P[u/x])}{\Gr}}
        {(d\ipux~t)}
        {?_A} \\
        & \eqbe & 
        \lambda t : \ip{\hnf{T}}{\GD}\ipux. \\
        & &
        \elt{\ip{U'}{\GD}\ipux}
        {\ip{(\lambda y : U'.P)}{\GD}\ipux}
        {(d\ipux~t)}
        {?_A} \\
        & \eqbe &
        (\lambda t : \ip{\hnf{T}}{\GD}.
        \elt{\ip{U'}{\GD}}{\ip{\lambda y : U'.P}{\GD}}
        {(d~t)}
        {?_B})\ipux \\
        & = & c\ipux
      \end{eqnarray*}
          
      Dans le cas ou l'on applique directement \irule{SubConv} cela
      implique que $\hnf{T} = x = \hnf{T'}$ donc par reflexivit� de la
      coercion \ref{subti-reflexive}, $\subimpl{\Gr}{c'}{u}{u}$ est d�rivable et $c' \eqbei
      \lambda y : \ip{u}{`G}.y \eqbei (\lambda y : x.y)\ipux$.
      
    \case{SubConv}
    On a $T \eqbi T'$, donc $T[u/x] \eqbi T'[u/x]$ par substitutivit� de
    la $\beta$-�quivalence. Par le lemme \ref{subti-eqb-coercion-eqbe-id}, on
    sait qu'il existe $c'$, $\Gr \typec c' : T[u/x] \sub T'[u/x]$ telle
    que $c' \eqbe \lambda x : \ip{T[u/x]}{\Gr}.x$. On a bien $c' \eqbe
    c\ipux$ car $\ip{T[u/x]}{\Gr} \eqb \ip{T}{\GD}\ipux$ par induction.
    
    \case{SubProd}
    On a:
    \begin{prooftree}
      \AXC{$\subimplhyp{\GD}{c_1}{A'}{A}$}
      \AXC{$\subimplhyp{\GD, y : A'}{c_2}{B}{B'}$}
      \BIC{$\subimplconcl{\GD}{f}
        {\lambda x : \ip{A'}{\GD}.~c_2~(f~(c_1~x))}
        {\Pi y : A.B}{\Pi y : A'.B'}$}
    \end{prooftree}
    
    Par induction et application de \irule{SubProd}:
    \begin{prooftree}
      \AXC{$\subimplhyp{\Gr}{c_1'}{A'[u/x]}{A[u/x]}$}
      \AXC{$\subimplhyp{\Gr, y : A'[u/x]}{c_2'}{B[u/x]}{B'[u/x]}$}
      \BIC{$\subimpl{\Gr}{c'}
        {\Pi y : A[u/x].B[u/x]}{\Pi y : A'[u/x].B'[u/x]}$}
    \end{prooftree}
    
    o� \[c' = \lambda f : \ip{\Pi y : A[u/x].B[u/x]}{\Gr}.\lambda x :
    \ip{A'[u/x]}{\Gr}.~c_2'~(f~(c_1'~x))\]
    Par induction, $\ip{A'[u/x]}{\Gr} \eqbe \ip{A'}{\GD}\ipux$ et 
    $\ip{\Pi y : A[u/x].B[u/x]}{\Gr} \eqbe \ip{\Pi y : A.B}{\GD}\ipux$.
    On a donc bien $c' \eqbe c\ipux$.
   
    \casethree{SubSigma}{SubProof}{SubSub} Idem, direct par induction.
  
    \case{PropSet} Trivial.
    
    \case{Var} 
    On a:
    \typenva
    \begin{prooftree}
      \BAX{Var}
      {$\wf \GD$}
      {$y : T `: \GD$}
      {$\GD \seq y : T$}
      {}
    \end{prooftree}    
    \typenvi

    \begin{itemize}
    \item Si $x "/=" y$, alors par d�finition de l'interpr�tation,
      on doit montrer
      $\ip{y[u/x]}{\Gr} = y = \ip{y}{\GD}\ipux$.
      
    \item Sinon, $t[u/x] = u$ et $\ip{u}{\Gr} = \ip{x}{\GD}\ipux =
      \ip{u}{\GD}$. On utilise ici le fait que $u$ est bien typ� dans
      l'environnement $`G$, donc dans toute extension bien form�e de cet
      environnement par affaiblissement.      
    \end{itemize}
    
    \case{App}\quad
    \typenva
    \begin{prooftree}
      \TAX{App}
      {$\GD \seq f : F \quad \mualgo(F) = \Pi y : A. B : s$}
      {$\GD \seq e : E \quad `G \seq E, A : s$}
      {$E \sub A$}
      {$\GD \seq (f~e) : B [ e / y ]$}
      {}
    \end{prooftree}
    \typenvi

    Par induction:
    \[\ip{f[u/x]}{\Gr} \eqbi \ip{f}{\GD}\ipux\]
    \[\ip{e[u/x]}{\Gr} \eqbi \ip{e}{\GD}\ipux\]
    
    \def\afe{`a}    
    Par d�finition de la substitution et de l'interpr�tation, 
    \begin{eqnarray*}
      \ip{(f~e)}{\GD}
      & = & (((\pi_F~\ip{f}{\GD})~(c_e~\ip{e}{\GD})) \\
      \text{ o� } & & \\
      \pi_F & = & \sref{coerce}~F~(\Pi y : A.B) \\
      c_e & = & \sref{coerce}~E~A.
    \end{eqnarray*}
    
    Clairement, $\pi_F$ et $c_e$ sont d�rivables, puisqu'on part de
    jugements d�rivables par la coercion algorithmique.

    On a la substitutivit� du typage \ref{substitutive-typing} donc le
    jugement substitu� est:
    \typenva
    \begin{prooftree}
      \TAX{App}
      {$\Gr \seq f[u/x] : F[u/x] \quad \mualgo(F[u/x]) = \Pi y : A[u/x]. B[u/x] : s$}
      {$\Gr \seq e[u/x] : E[u/x] \quad `G \seq E[u/x], A[u/x] : s$}
      {$E[u/x] \sub A[u/x]$}
      {$\Gr \seq (f~e)[u/x] : B' [ e[u/x] / y ]$}
      {}
    \end{prooftree}
       
    Soit $e' = e[u/x]$ et $f' = f[u/x]$, on a donc d'autre part:
    \begin{eqnarray*}
      \ip{(f~e)[u/x]}{\Gr}
      & = & \ip{f'~e'}{\Gr} \\
      & = & (\pi_{F[u/x]}~\ip{f'}{\Gr})~(c_{e'}~\ip{e'}{\Gr}) \\
      \text{ o� } & & \\
      \pi_{F[u/x]} & = & \sref{coerce}~F[u/x]~(\Pi y : A[u/x].B[u/x]) \\
      c_{e'} & = & \sref{coerce}~E[u/x]~A[u/x]
    \end{eqnarray*}
    
    Par induction, il existe $d$, $e$:
    $\Gr \typec d \eqbe \pi_F\ipux : F[u/x] \suba (\Pi y : A.B)[u/x]$
    et 
    $\Gr \typec e \eqbe c_e\ipux : E[u/x] \suba A[u/x]$.
    
    \[\begin{array}{ll}
      \firsteq{\ip{f~e}{\GD}\ipux}
      
      \step{D�finition de l'interpr�tation}
      {=}(\pi_F~\ip{f}{\GD})~(c_e~\ip{e}{\GD})\ipux 

      \step{D�finition de la substitution}
      {=}{(\pi_F\ipux~\ip{f}{\GD}\ipux)~(c_e\ipux~\ip{e}{\GD}\ipux)}

      \step{Application de l'hypoth�se d'induction pour les termes}
      {\eqbe}{(\pi_F\ipux~\ip{f'}{\Gr})~(c_e\ipux~\ip{e'}{\Gr})}

      \step{Application de l'hypoth�se d'induction pour les coercions}
      {\eqbe}{(d~\ip{f'}{\Gr})~(e~\ip{e'}{\Gr})}

      \step{Unicit� des coercions: $d \eqbe \pi_{F'}$ et $e \eqb
        c_{e'}$}
      {\eqbe}{(\pi_{F'}~\ip{f'}{\Gr})~(c_{e'}~\ip{e'}{\Gr})}
      
      \step{D�finition de l'interpr�tation}
      {\eqbe}{\ip{(f~e)[u/x]}{\Gr}}
    \end{array}\]
    
    \casethree{Prod}{Sigma}{Subset} Par induction.

    \case{Abs}
    \typenva
    On a:
    \begin{prooftree}
      \AXC{$\GD \seq \Pi y : T. U : s $}
      \AXC{$\GD, y : T \seq M : U $}
      \BIC{$\GD \seq \lambda y : T. M : \Pi y : T.U$}
    \end{prooftree}
    
    On a bien:
    \[\begin{array}{ll}
      \firsteq{\ip{\lambda y : T.M}{\GD}\ipux}
      
      \step{D�finition de l'interpr�tation}
      {=}{\lambda y : \ip{T}{\GD}\ipux.\ip{M}{\GD, y : T}\ipux}

      \step{Application de l'hypoth�se de r�currence}
      {\eqbe}{\lambda y : \ip{T[u/x]}{\Gr}.\ip{M[u/x]}{\Gr, y : T[u/x]}}

      \step{D�finition de l'interpr�tation}
      {=}{\ip{\lambda y : T[u/x].M[u/x]}}
    \end{array}\]
      
    \casetwo{LetSum}{SumDep} \TODO{Par induction}


  \end{induction}
\end{proof}

On va maintenant �tendre la relation de coercion aux contextes de
mani�re canonique.
\typenva
\begin{definition}[Coercion de contextes]
  \label{coercion-ctx}
  On d�finit inductivement la coercion de deux contextes de coercions
  algorithmiques par les r�gles suivantes:
  \begin{itemize}
  \item $[] \sub []$
  \item $`G, x : T \sub `G', x : T'$ si $`G \sub `G'$ et $T \sub T'$.
  \end{itemize}
\end{definition}

De m�me pour les coercions explicites d�riv�es par le jugement $`G
\typec c : T \suba S$.
\begin{definition}[Coercion explicites de contextes]
  \label{coercion-ctx-i}
  On d�finit inductivement la coercion de deux contextes de coercions
  explicites par les r�gles suivantes:
  \begin{itemize}
  \item $[] \sub []$
  \item $`r, c : `G, x : T \sub `G', x : T'$ si $`r : `G \sub `G'$ et 
    $`G \typec c : T \sub T'$.
  \end{itemize}
\end{definition}

Clairement toute coercion de contexte algorithmique correspond � une
coercion de contexte explicite et vice-versa.

\begin{definition}[Extension de la substitution aux coercions de
  contextes]
  \label{subst-coercion-ctx}
  On d�finit la substitution d'une coercion de contexte inductivement:
  \begin{itemize}
  \item $t[[]] = t$
  \item $t[`r, c : `G, x : T \sub T'] = t[`r : `G][c~x/x]$
  \end{itemize}
\end{definition}
\begin{lemma}[Stabilit� par affaiblissement]
  Si $`r : `D \sub `G$, $`G \typec c : T \subi T'$ et $\ip{T'}{`D} \eqbe
  \ip{T}{`G}[`r]$, alors $`D \typec c' : T \sub T'$ et $c' \eqb c[`r]$.
\end{lemma}

\begin{proof}
  Par induction sur la d�rivation de coercion:

  \begin{induction}
    \case{SubHnf} 
    On a $\subimpl{`G}{c}{\hnf{T}}{\hnf{T'}}$. Par induction, 
    $`D \typec c[`r] : \hnf{T} \sub \hnf{T'}$. 
    On peut donc d�river $`D \typec c[`r] : T \sub T'$ par \irule{SubHnf}.

    \case{SubConv}
    On a $T \eqb T'$ et $c = \lambda x : \ip{T}{`G}.x$. 
    On peut donc d�river $\subimpl{`D}{c' = \lambda x : \ip{T}{`D}.x}{T}{T'}$.
    Or $\ip{T}{`D} \eqb \ip{T'}{`G}[`r]$, on a donc bien $c' = c[`r]$.

    \case{SubProd}
    On a:
    \begin{prooftree}
      \BAX{}
      {$\subimpl{`G}{c_1}{C}{A}$}
      {$\subimpl{`G, y : C}{c_2}{B}{D}$}
      {$\subimplhnf{`G}{\lambda f : \ip{\Pi x : A.B}{`G}.~\lambda x :
          \ip{C}{`G}.~c_2~(f~(c_1~x))}
        {\Pi x : A.B}{\Pi x : C.D}$}
      {}
    \end{prooftree}
    Par induction on a {$\subimpl{`D}{c_1[`r]}{C}{A}$}. On peut d�finir
    la coercion $`s = `r, (\lambda x : \ip{C}{`D}.x) : `D, x : C \sub `G, x : C$ et obtenir par
    induction:
    $\subimpl{`G, y : C}{c_2[`s]}{B}{D}$
    
    On peut alors appliquer \irule{SubProd} pour obtenir:
    \[\subimplhnf{`G}{c' = \lambda f : \ip{\Pi x : A.B}{`G}.~\lambda x :
      \ip{C}{`G}.~c_2[`s]~(f~(c_1[`r]~x))}{\Pi x : A.B}{\Pi x : C.D}\]
    
    Comme $\ip{\Pi x : A.B}{`G} = \Pi x : \ip{A}{`G}.\ip{B}{`G, x : A}$
    on a:
    \[\begin{array}{ll}
      \firsteq{(\lambda f : \ip{\Pi x : A.B}{`G}.~\lambda x :
          \ip{C}{`G}.~c_2~(f~(c_1~x)))[`r]}

        \step{D�finition de la substitution}
        {=}{\lambda f : \ip{\Pi x : A.B}{`G}[`r].~\lambda x :
          \ip{C}{`G}[`r].c_2[`r]~(f~(c_1[`r]~x))}

        \step{Application de l'hypoth�se de r�currence}
        {\eqbe}{\lambda f : \ip{\Pi x : A.B}{`D}.~\lambda x :
          \ip{C}{`D}.c_2[`r]~(f~(c_1[`r]~x))}
        
        \step{Coercion identit� dans $`s$}
        {=}{\lambda f : \ip{\Pi x : A.B}{`D}.~\lambda x :
          \ip{C}{`D}.c_2[`s]~(f~(c_1[`r]~x))}      
      \end{array}\]
      
    \case{SubSigma}
    \TODO{todo!}
    On a:
    \begin{prooftree}
      \BAX{}
      {$\subimpl{`G}{c_1}{C}{A}$}
      {$\subimpl{`G, y : C}{c_2}{B}{D}$}
      {$\subimplhnf{`G}{\lambda f : \ip{\Pi x : A.B}{`G}.~\lambda x :
          \ip{C}{`G}.~c_2~(f~(c_1~x))}
        {\Pi x : A.B}{\Pi x : C.D}$}
      {}
    \end{prooftree}
    Par induction on a {$\subimpl{`D}{c_1[`r]}{C}{A}$}. On peut d�finir
    la coercion $`s = `r, (\lambda x : \ip{C}{`D}.x) : `D, x : C \sub `G, x : C$ et obtenir par
    induction:
    $\subimpl{`G, y : C}{c_2[`s]}{B}{D}$
    
    On peut alors appliquer \irule{SubProd} pour obtenir:
    \[\subimplhnf{`G}{c' = \lambda f : \ip{\Pi x : A.B}{`G}.~\lambda x :
      \ip{C}{`G}.~c_2[`s]~(f~(c_1[`r]~x))}{\Pi x : A.B}{\Pi x : C.D}\]
    
    Comme $\ip{\Pi x : A.B}{`G} = \Pi x : \ip{A}{`G}.\ip{B}{`G, x : A}$
    on a:
    \[\begin{array}{ll}
      \firsteq{(\lambda f : \ip{\Pi x : A.B}{`G}.~\lambda x :
          \ip{C}{`G}.~c_2~(f~(c_1~x)))[`r]}

        \step{D�finition de la substitution}
        {=}{\lambda f : \ip{\Pi x : A.B}{`G}[`r].~\lambda x :
          \ip{C}{`G}[`r].c_2[`r]~(f~(c_1[`r]~x))}

        \step{Application de l'hypoth�se de r�currence}
        {\eqbe}{\lambda f : \ip{\Pi x : A.B}{`D}.~\lambda x :
          \ip{C}{`D}.c_2[`r]~(f~(c_1[`r]~x))}
        
        \step{Coercion identit� dans $`s$}
        {=}{\lambda f : \ip{\Pi x : A.B}{`D}.~\lambda x :
          \ip{C}{`D}.c_2[`s]~(f~(c_1[`r]~x))}      
      \end{array}\]
      

  \end{induction}
\end{proof}


\begin{lemma}[Substitutivit� de la coercion]
  \label{subti-coercion-subst}
  Si $`G \typec X, V, T, U : s$, $`G \typec e : X \subi V$ et
  $`G, x : V, `D \typec c : T \subi U$ alors
  $`G, x : X, `D[e~x/x] \typec c[e~x/x] : T[e~x/x] \subi U[e~x/x]$.   
\end{lemma}

\begin{proof}
  Soit $T' = T[e~x/x]$, $U' = U[e~x/x]$.

  \begin{induction}[subtyping-algo]
  \case{SubHnf} 
  On a: 
  \begin{prooftree}
    \AXC{$\subimplhnf{\GD}{c}{\hnf{T}}{\hnf{U}}$}
    \UIC{$\subimpl{\GD}{c}{T}{U}$}
  \end{prooftree}
  
  On a plusieurs possibilit�s:
  \begin{itemize}
  \item Si $\hnf{T} = x$. Alors la seule r�gle appliquable est
    \irule{SubProof}, on a donc:
    \begin{prooftree}
      \AXC{$\subimpl{\GD}{c'}{x}{W}$}
      \UIC{$\subimplhnf{\GD}{c = (\lambda t : x.\ldots(c'~t))}{\hnf{T} = x}{\hnf{U} = \mysubset{y}{W}{P}}$}
    \end{prooftree}
    
    Par induction: \[\subimpl{\Gr}{c'[e~x/x]}{e~x}{W[e~x/x]}\]
    Or on peut appliquer \irule{SubHnf} et \irule{SubProof} 
    pour d�river le jugement $\subimpl{\GD}{?}{T'}{U'}$:

    \begin{prooftree}
      \AXC{$\subimpl{\Gr}{c'[e~x/x]}{e~x}{W[e~x/x]}$}
      \UIC{$\subimplhnf{\GD}{\lambda t :
          x[e~x/x].\ldots(c'~t))}{\hnf{T'} = e~x}{\hnf{U'} = \mysubset{y}{W[e~x/x]}{P[e~x/x]}}$}
      \UIC{$\subimpl{\Gr}{?}{T'}{U'}$}
    \end{prooftree}
    
    La coercion inf�r�e est bien $c[e~x/x]$.
    On ne peut appliquer d'autre r�gle que \irule{SubProof}, en effet,
    comme $T : s$ on a $\hnf{T} = x : s$ et donc $X \eqbi s$. Par le
    lemme \ref{subti-coercion-sorts} on en d�duit que $\hnf{T'} = x$
    donc aucune autre r�gle ne peut s'appliquer.
    
  \item Si $\hnf{U} = x$ alors la seule r�gle applicable est
    \irule{SubSub} et l'on a un r�sultat similaire au cas pr�c�dent:
    
    \begin{prooftree}
      \AXC{$\subimpl{\Gr}{c'[e~x/x]}{W[e~x/x]}{x}$}
      \UIC{$\subimplhnf{\GD}{c'[e~x/x] `o \pi_1}{\hnf{T'} =
          \hnf{T}[e~x/x] = \mysubset{y}{W[e~x/x]}{P[e~x/x]}}{\hnf{U'}}$}
      \UIC{$\subimpl{\Gr}{?}{T'}{U'}$}
    \end{prooftree}
   
  \item Sinon, $\hnf{T} "/=" x$ et $\hnf{U} "/=" x$ donc $\hnf{T'} =
    \hnf{T}[e~x/x]$ et $\hnf{U'} = \hnf{U}[e~x/x]$.

    Dans ce cas on va <<rejouer>> la d�rivation $\subimplhnf{\GD}{c}{\hnf{T}}{\hnf{U}}$.
    
    Par cas sur cette d�rivation:
    \begin{induction}
      \case{SubProd}\quad
      On a:
      \begin{prooftree}
        \BAX{}
        {$\subimpl{\GD}{c_1}{C}{A}$}
        {$\subimpl{\GD, y : C}{c_2}{B[c_1~y/y]}{D}$}
        {$\subimplhnf{\GD}{\lambda f : \Pi x : A.B.~\lambda x :
            C.~c_2~(f~(c_1~x))}
          {\Pi x : A.B}{\Pi x : C.D}$}
        {}
      \end{prooftree}

      Par induction: 
      \[\subimpl{\Gr}{c_1[e~x/x]}{C[e~x/x]}{A[e~x/x]}\]
      \[\subimpl{\Gr, y : C[e~x/x]}{c_2[e~x/x]}{B[c_1~y/y][e~x/x] = B[e~x/x][c_1[e~x/x]/y]}{D[e~x/x]}\]
      
      On en d�duit par \irule{SubProd}:
      \[\subimplhnf{\Gr}{c[e~x/x]}{(\Pi y : A.B)[e~x/x]}{(\Pi y : C.D)[e~x/x]}\]
      On peut v�rifier:
      $c[e~x/x] : \hnf{T'} \subihnf \hnf{U'} = \lambda f : (\Pi x : A.B)[e~x/x].~\lambda x :
      C[e~x/x].~c_2[e~x/x]~(f~(c_1[e~x/x]~x))$.
      
      \case{SubSigma} idem.
      
      \case{SubSub}\quad
      \begin{prooftree}
        \AXC{$\subimpl{\GD}{c}{T}{\hnf{U}}$}
        \UIC{$\subimplhnf{\GD}{\lambda t : \mysubset{y}{T}{P}.~c~(\pi_1~t)}
          {\mysubset{y}{T}{P}}{\hnf{U}}$}       
      \end{prooftree}
      
      donne:
      \begin{prooftree}
        \AXC{$\subimpl{\Gr}{c[e~x/x]}{T[e~x/x]}{\hnf{U}[e~x/x]}$}
        \UIC{$\subimplhnf{\Gr}{\lambda t : (\mysubset{y}{T}{P})[e~x/x].~c[e~x/x]~(\pi_1~t)}
          {\mysubset{y}{T[e~x/x]}{P[e~x/x]}}{\hnf{U}[e~x/x]}$}
      \end{prooftree}
      
      \case{SubProof} idem.
    \end{induction}
  \end{itemize}

  \case{SubConv}
  Dans ce cas, c'est direct par pr�servation de la $\beta$-�quivalence par
  substitution. Pour montrer que la nouvelle coercion est �gale �
  $c[e~x/x]$ il faut raisonner par cas sur la forme de $T$ et $U$. 
  \begin{itemize} 
  \item Si la forme de $T$ (respectivement $U$) est �gale � $x$ 
    alors $\hnf{U} = x$ (reps. $\hnf{T} = x$). Or $\GD \typec T,
    U : s$, donc $\GD \typec x : s$, soit $X \eqbi s$. Par le lemme
    \ref{subti-coercion-sorts}, on d�duit que $e = \lambda x : s.x$.
    On a donc: $\hnf{T'} = \hnf{U'} = \hnf{e~x} = x$. Or $x \subhnf x$
    n'est pas d�rivable donc on a la d�rivation:
    \begin{prooftree}
      \AXC{$T' \eqbi U'$}
      \UIC{$\subimpl{\Gr}{\lambda y : T'.y}{T'}{U'}$}
    \end{prooftree}
    
    On a bien $c[e~x/x] = \lambda y : T[e~x/x].y$.
  \item Sinon, on a $\hnf{T'} = \hnf{T}[e~x/x]$ et $\hnf{U'} =
    \hnf{U}[e~x/x]$. Or on sait que $\hnf{T} \subhnf \hnf{U}$ n'est pas
    d�rivable, donc la subsitution ne peut pas l'�tre non plus (on
    substitue sous les symboles de t�te dans ce cas). Cependant, on a toujours $T'
    \eqbi U'$ donc on va d�river la coercion attendue.
    
  \end{itemize}
  


\end{induction}
\end{proof}


%%% Local Variables: 
%%% mode: latex
%%% TeX-master: "subset-typing"
%%% LaTeX-command: "TEXINPUTS=\"style:$TEXINPUTS\" latex"
%%% End: 


\begin{lemma}[Transitivit� de la coercion]
  \label{subi-trans}
  S'il existe $c_1, c_2$ tels que $`G \typec c_1 : S \sub T$ et $`D
  \typec c_2 : T \sub U$ avec $`r : `D \sub `G$ et $\ip{T}{`G}[`r] \eqbr
  \ip{T}{`D}$,
  alors $`E!c, `D \typec c : S \sub U$ et $c \eqbr c_2 `o c_1[`r]$.
\end{lemma}
\begin{proof}
  Par induction lexicographique sur la paire de d�rivations de $c_1$ et $c_2$.

  \begin{induction}
       
    \case{SubConv} On va traiter les cas o� cette r�gle est utilis�e en
    racine d'une des deux d�rivations, d'abord � gauche puis � droite.

    \begin{prooftree}
      \AXC{$S \eqbr T$}
      \UIC{$\subimpl{`G}{c_1 = \ctxdot}{S}{T}$}
      \AXC{$\subimpl{`D}{c_2}{T}{U}$}
      \noLine\BIC{}
    \end{prooftree}
    
    Les conditions de bord de \irule{SubConv} nous donnent comme
    hypoth�ses que $S$ et $T$ sont en forme normale de t�te et que 
    $S "/=" \Pi, \Sigma, \{|\}$ et $T "/=" \{|\}$.
    Par inversion de la $\beta\rho$-�quivalence $S \eqbr T$, on a aussi,
    $T "/=" \Pi, Sigma$.
    Les seules r�gles pouvant s'appliquer a la fin de la d�rivation de
    $c_2$ sont donc \irule{SubConv}, \irule{SubHnf} et \irule{SubProof}.

    \begin{itemize}
      \case{SubConv} Alors on a $U = \hnf{U} "/=" \Pi, \Sigma, \{|\}$.
      La coercion compos�e est alors $\ctxdot `o \ctxdot$. C'est bien la coercion
      d�riv�e pour le jugement $\subimpl{`D}{\ctxdot}{S}{U}$ par \irule{SubConv}.
      
      \case{SubHnf} On a alors $\subimpl{`D}{c_2}{\hnf{T}}{\hnf{U}}$.
      Comme $T = \hnf{T}$, on peut appliquer l'hypoth�se d'induction pour
      obtenir une coercion $c$ telle que $\subimpl{`D}{c}{S}{\hnf{U}}$ et
      $c \eqbr c_2 `o c_1[`r]$.
      Une application de \irule{SubHnf} suffit pour obtenir une
      d�rivation de $\subimpl{`D}{c}{S}{U}$ avec $c \eqbr c_2 `o c_1[`r]$.
      
      \case{SubProof} Ici on a:
      \begin{prooftree}
        \AXC{$\subimpl{`D}{d}{T}{U'}$}
        \UIC{$\subimpl{`D}{c_2 = {\elt{\ip{U'}{`D}}{\ip{\lambda x : U'.P}{`D}}{d}
              {\ex{\ipG{`G}}{\ip{P}{`D}[d/x]}}}}{T}{U = \mysubset{x}{U'}{P}}$}
      \end{prooftree}

      Par induction, il existe une coercion $d' \eqbr d `o c_1[`r] = d$ telle que
      $\subimpl{`D}{d'}{S}{U'}$. On applique \irule{SubProof} pour obtenir
      la coercion $c$ de $S$ � $U$. Clairement $c \eqbr c_2 `o c_1[`r] =
      c_2 `o [] = c_2$.
      
    \end{itemize}
    
    Supposons maintenant que la d�rivation de $c_2$ termine par une
    application de \irule{SubConv}. Alors $T$ et $U$ sont en forme
    normale de t�te et $T, U "/=" \Pi, \Sigma, \{|\}$.
    Les seules r�gles pouvant appara�tre en racine de la d�rivation de
    $c_1$ sont donc \irule{SubConv}, \irule{SubHnf} et \irule{SubSub}.

    \begin{itemize}
      \case{SubConv} Alors on a $S = \hnf{S} "/=" \Pi, \Sigma, \{|\}$.
      La coercion compos�e est alors $\ctxdot `o \ctxdot$. C'est bien la coercion
      d�riv�e pour le jugement $\subimpl{`D}{\ctxdot}{S}{U}$ par \irule{SubConv}.
      
      \case{SubHnf} On a alors $\subimpl{`G}{c_1}{\hnf{S}}{\hnf{T}}$.
      Comme $T = \hnf{T}$, on peut appliquer l'hypoth�se d'induction pour
      obtenir une coercion $c$ telle que $\subimpl{`D}{c}{\hnf{S}}{U}$ et
      $c \eqbr c_2 `o c_1[`r]$.
      Une application de \irule{SubHnf} suffit pour obtenir une
      d�rivation de $\subimpl{`D}{c}{S}{U}$ avec $c = c_2 `o c_1[`r]$.
      
      \case{SubSub} Ici on a:
      \begin{prooftree}
        \AXC{$\subimpl{`G}{d}{S'}{T}$}
        \UIC{$\subimpl{`G}{c_1 = d[\pi_1~\ctxdot]}{S = \mysubset{x}{S'}{P}}{T}$}
      \end{prooftree}

      Par induction, il existe une coercion $d' \eqbr c_2 `o d[`r] = d[`r]$ telle que
      $\subimpl{`D}{d'}{S'}{U}$. On applique \irule{SubSub} pour obtenir
      la coercion $c = d[`r][\pi_1~\ctxdot]$ de $S$ � $U$. On a \[c =
      d[`r][\pi_1~\ctxdot] = d[\pi_1~\ctxdot][`r] \eqbr c_2 `o c_1[`r] \]

    \end{itemize}

    \case{SubHnf}\quad
    \begin{prooftree}
      \AXC{$\subimpl{`G}{c}{\hnf{S}}{\hnf{T}}$}
      \UIC{$\subimpl{`G}{c}{S}{T}$}
      \AXC{$\subimpl{`D}{d}{T}{U}$}
      \noLine\BIC{} % $\subimpl{`G}{d `o c}{S}{U}$
    \end{prooftree}
    
    Si $T = \hnf{T}$ alors c'est trivial par induction et application de
    \irule{SubHnf}.
    Sinon, la seule r�gles permettant de d�river $\subimpl{`D}{d}{T}{U}$
    est \irule{SubHnf}. Il suffit alors d'appliquer l'hypoth�se
    d'induction pour obtenir une d�rivation de
    $\subimpl{`D}{c}{\hnf{S}}{\hnf{U}}$ avec $c \eqbr c_2 `o c_1[`r]$ puis
    \irule{SubHnf} nous permet de construire le jugement
    $\subimpl{`G}{c}{S}{U}$. 

    De m�me si l'on a une application de \irule{SubHnf} � la racine de
    la d�rivation de droite.
    
    On peut donc se ramener au cas o� \irule{SubConv} et
    \irule{SubHnf} ne sont appliqu�es � la racine d'aucune des 
    deux d�rivations.

    
    \case{SubProd}\quad
    \begin{prooftree}
      \AXC{$\subimpl{`G}{c_1}{X'}{X}$}
      \AXC{$\subimpl{`G, x : X'}{c_2}{Y}{Y'}$}
      \BIC{$\subimpl{`G}{c = \lambda x : \ip{X'}{`G}.c_2[\ctxdot~c_1[x]]}{\Pi x : X.Y}{\Pi x : X'.Y'}$}
      \AXC{$\subimpl{`D}{d}{\Pi x : X'.Y'}{U}$}
      \noLine\BIC{}
    \end{prooftree}
    
    Par induction sur la d�rivation $\subimpl{`D}{d}{\Pi x :
      X'.Y'}{U}$. Seules deux r�gles peuvent s'appliquer � la racine:
    \begin{induction}
      \case{SubProd}\quad
      On a $U = \Pi x : S.T$ et la d�rivation a la forme:
      \begin{prooftree}
        \AXC{$\subimpl{`D}{d_1}{S}{X'}$}
        \AXC{$\subimpl{`D, x : S}{d_2}{Y'}{T}$}
        \BIC{$\subimpl{`D}{d = (\lambda x : \ip{S}{`D}.d_2[\ctxdot~d_1[x]])}
          {\Pi x : X'.Y'}{\Pi x : S.T}$}
      \end{prooftree}

      On peut construire une coercion de contextes de $`q = `r, d_1 : (`D, x : S) \sub `G, x : X'$. 

      Par le lemme \ref{narrowing-i-coercion} on obtient:
      \[\subimpl{`D, x : S}{c_2'}{Y}{Y'}\] de m�me
      taille que la d�rivation de $c_2$ tel que $c_2' \eqbr c_2[`q]$
      
      On obtient de m�me $\subimpl{`D}{c_1'}{X'}{X}$ avec $c_1' \eqbr c_1[`r]$.

      En utilisant une coercion de contextes partout l'identit�, on
      obtient par induction avec les d�rivations de $c_1[`r]$ et $d_1$
      d'une part et $c_2'$ et $d_2$ d'autre part, deux coercions
      $e_1, e_2$ telles que:
      \[\subimpl{`D}{e_1 \eqbr c_1' `o d_1}{S}{X}\] et 
      \[\subimpl{`D, x : S}{e_2 \eqbr d_2 `o c_2'}{Y}{T}\]
      
      On en d�duit:
      \begin{prooftree}
        \AXC{$\subimpl{`D}{e_1}{S}{X}$}
        \AXC{$\subimpl{`D, x : S}{e_2}{Y}{T}$}
        \BIC{$\subimpl{`D}{e = \lambda x : \ip{S}{`D}.e_2[\ctxdot~e_1[x]]}
          {\Pi x : X.Y}{\Pi x : S.T}$}
      \end{prooftree}

      On a bien $e \eqbr d `o c[`r]$:
      \[\begin{array}{ll}
        \firsteq{d `o c[`r]}

        \step{D�finition de $c$ et $d$}
        {=}{(\lambda x : \ip{S}{`D}.d_2[\ctxdot~d_1[x]]) `o (\lambda
          y : \ip{X'}{`G}.c_2[\ctxdot~c_1[y]])[`r]}
        
        \step{Composition des contextes}
        {=}{\lambda x : \ip{S}{`D}.d_2[(\lambda y :
          \ip{X'}{`G}.c_2[\ctxdot~c_1[y]])[`r]~d_1[x]]}

        \step{Substitution dans l'abstraction}
        {=}{\lambda x : \ip{S}{`D}.d_2[(\lambda y :
          \ip{X'}{`G}[`r].c_2[\ctxdot~c_1[y]][`r])~d_1[x]]}
        
        \step{R�duction}
        {"->"_{\beta}}
        {\lambda x :
          \ip{S}{`D}.d_2[(c_2[\ctxdot~c_1[y]][`r])[d_1[x]/y]]}
        
        \step{$y `; `r$}
        {=}
        {\lambda x :
          \ip{S}{`D}.d_2[(c_2[`r][\ctxdot~c_1[`r][y]])[d_1[x]/y]]}
        
        \step{D�finition de $`q$}
        {=}{\lambda x : \ip{S}{`D}.d_2[c_2[`q][\ctxdot~c_1[`q][d_1[x]]]]}

        \step{$d_2 `o c_2[`q] \eqbr e_2$}
        {\eqbr}{\lambda x : \ip{S}{`D}.e_2[\ctxdot~c_1[`q][d_1[x]]]}

        \step{$c_1[`q] = c_1[`r]$}
        {\eqbr}{\lambda x : \ip{S}{`D}.e_2[\ctxdot~e_1[x]]}
        
        \step{D�finition}
        {=}{e}
      \end{array}\]
      
      \case{SubProof}
      Ici $U = \mysubset{y}{U'}{P}$ et la d�rivation commence
      par:
      \begin{prooftree}
        \AXC{$\subimpl{`D}{e}{\Pi x : X'.Y'}{U'}$}
        \UIC{$\subimpl{`D}{d = \elt{\ip{U'}{`D}}
            {\ip{\lambda x : U'.P}{`D}}{e}
            {\ex{\ipG{`D}}{\ip{P}{`D}[e/x]}}}
          {\Pi x : X'.Y'}{\mysubset{y}{U'}{P}}$}
      \end{prooftree}

      Par induction on a $\subimpl{`D}{f \eqbr e `o c[`r]}{\Pi x : X.Y}{U'}$.
      On peut donc d�river:
      \begin{prooftree}
        \AXC{$\subimpl{`G}{f}{\Pi x : X.Y}{U'}$}
        \UIC{$\subimpl{`G}{d' = \elt{\ip{U'}{`G}}
            {\ip{\lambda x : U'.P}{`G}}{f}
            {\ex{\iG}{\ip{P}{`G}[f/x]}}}
          {\Pi x : X.Y}{U}$}
      \end{prooftree}
      
      On peut v�rifier que $d' \eqbr d `o c[`r]$:

      \begin{eqnarray*}
        d `o c[`r] & = & (\elt{\ip{U'}{`G}}
        {\ip{\lambda x : U'.P}{`G}}{e}
        {\ex{\iG}{\ip{P}{`G}[e/x]}})[c[`r]] \\
        & = & \elt{\ip{U'}{`G}}
        {\ip{\lambda x : U'.P}{`G}}{e[c[`r]]}
        {\ex{\iG}{\ip{P}{`G}[e[c[`r]]/x]}} \\
        & \eqbr & \elt{\ip{U'}{`G}}
        {\ip{\lambda x : U'.P}{`G}}{f}
        {\ex{\iG}{\ip{P}{`G}[f/x]}} \\
        & = & d'
      \end{eqnarray*}
      
    \end{induction}
    
    \case{SubSigma}\quad
    De fa�on �quivalente � \irule{SubProd}, on fait le cas si
    \irule{SubSigma} est utilis�e � la pr�misse droite.

    \begin{prooftree}
      \AXC{$\subimpl{`G}{c}{T}{\Sigma x : X'.Y'}$}
      \AXC{$\subimpl{`D}{d_1}{X'}{X}$}
      \AXC{$\subimpl{`D, x : X'}{d_2}{Y'}{Y}$}
      \BIC{$\subimpl{`D}{d = (d_1[\pi_1~\ctxdot], d_2[\pi_1~\ctxdot/x][\pi_2~\ctxdot])}{\Sigma x : X'.Y'}{\Sigma x : X.Y}$}
      \noLine\BIC{}
    \end{prooftree}

    Par induction sur la d�rivation de $\subimpl{`G}{c}{T}{\Sigma x
      : X'.Y'}$:
    \begin{induction}
      \case{SubSigma}
      On a:
      \begin{prooftree}
        \AXC{$\subimpl{`G}{c_1}{S}{X'}$}
        \AXC{$\subimpl{`G, x : S}{c_2}{T}{Y'}$}
        \BIC{$\subimpl{`G}{c = (c_1[\pi_1~\ctxdot], c_2[\pi_1~\ctxdot/x][\pi_2~\ctxdot])}{\Sigma x
            : S.T}{\Sigma x : X'.Y'}$}
      \end{prooftree}
      
      Par affaiblissement pour les coercions (lemme
      \ref{narrowing-i-coercion}) on a $\subimpl{`D}{c_1'}{S}{X'}$ avec $c_1' \eqbr c_1[`r]$.
      On peut aussi construire la coercion de contexte
      $`q = `r, \ctxdot : (`D, x : S) \sub (`G, x : S)$. Par le m�me lemme on obtient
      $\subimpl{`D, x : S}{c_2'}{T}{Y'}$ avec $c_2' \eqbr c_2[`q] =
      c_2[`r]$. On peut enfin construire le jugement
      $\subimpl{`D, x : S}{d_2'}{Y'}{Y}$ avec $d_2' \eqbr d_2[c_1'[x]/x]$.
      
      Par induction en utilisant une coercion de contexte partout
      l'identit� on a donc:
      \begin{prooftree}
        \AXC{$\subimpl{`D}{e_1 \eqbr d_1 `o c_1[`r]}{S}{X}$}
        \AXC{$\subimpl{`D, x : S}{e_2 \eqbr d_2' `o c_2'}{T}{Y}$}
        \BIC{$\subimpl{`D}{e = (e_1[\pi_1~\ctxdot],
            e_2[\pi_1~\ctxdot/x][\pi_2~\ctxdot])}
          {\Sigma x : S.T}{\Sigma x : X.Y}$}
      \end{prooftree}
      

      On peut v�rifier:
      \[\begin{array}{ll}
        \firsteq{d `o c[`r]}

        \step{D�finition de $c$}
        {=}{(d_1[\pi_1~c[`r]], d_2[\pi_1~c[`r]/x][\pi_2~c[`r]])}
        
        \step{R�duction}
        {"->"_\rho}{(d_1[c_1[`r][\pi_1~\ctxdot]], d_2[(c_1[\pi_1~\ctxdot])[`r]/x]
          [c_2[`r][\pi_1~\ctxdot/x][\pi_2~\ctxdot]])}

        \step{D�finition de $e_1$}
        {\eqbr}
        {(e_1[\pi_1~\ctxdot], d_2[c_1[`r][\pi_1~\ctxdot]/x][c_2[`r][\pi_1~\ctxdot/x][\pi_2~\ctxdot]])}
        
        \step{D�finition de $c_1'$}
        {\eqbr}
        {(e_1[\pi_1~\ctxdot], d_2[c_1'[\pi_1~\ctxdot]/x][c_2[`r][\pi_1~\ctxdot/x][\pi_2~\ctxdot]])}
        

        \step{D�finition de $d_2'$}
        {\eqbr}
        {(e_1[\pi_1~\ctxdot],
          d_2'[\pi_1~\ctxdot/x][c_2[`r][\pi_1~\ctxdot/x][\pi_2~\ctxdot]])}

        \step{$x `; c_2[\pi_1~\ctxdot/x]$}
        {\eqbr}
        {(e_1[\pi_1~\ctxdot], d_2'[c_2[`r][\pi_1~\ctxdot/x][\pi_2~\ctxdot]][\pi_1~\ctxdot/x])}
        
        \step{D�finition de $c_2'$}
        {\eqbr}
        {(e_1[\pi_1~\ctxdot], d_2'[c_2'[\pi_1~\ctxdot/x][\pi_2~\ctxdot]][\pi_1~\ctxdot/x])}
        
        \step{D�finition de $e_2$}
        {\eqbr}
        {(e_1[\pi_1~\ctxdot],
          e_2'[\pi_1~\ctxdot/x][\pi_2~\ctxdot][\pi_1~\ctxdot/x])}

        \step{Deuxi�me substitution inutile}
        {=}{(e_1[\pi_1~\ctxdot], e_2[\pi_1~\ctxdot/x][\pi_2~\ctxdot])}

        \step{D�finition de $e$}
        {=}{e}
      \end{eqnarray*}
      
      \case{SubSub}
      On a: 
      \begin{prooftree}
        \AXC{$\subimpl{`G}{c'}{T}{\Sigma x : X'.Y'}$}
        \UIC{$\subimpl{`G}{c = c' `o \pi_1}{\mysubset{y}{T}{P}}{\Sigma x : X'.Y'}$}
      \end{prooftree}
      
      Par induction, $\subimpl{`G}{f \eqbr d `o c'}{T}{\Sigma x :
        X.Y}$, on a donc:
      \begin{prooftree}
        \AXC{$\subimpl{`G}{f \eqbr d `o c'}{T}{\Sigma x : X.Y}$}
        \UIC{$\subimpl{`G}{g = f `o \pi_1}{\mysubset{y}{T}{P}}{\Sigma x : X.Y}$}
      \end{prooftree}
      
      On a bien: $g = f `o \pi_1 \eqbr d `o c' `o \pi_1 \eqbr d `o c$.

    \end{induction}
    
    
    \case{SubProof}\quad
    \begin{prooftree}
      \AXC{$\subimpl{`G}{e}{S}{T}$}
      \UIC{$\subimpl{`G}{c = (\lambda t : S.~\elt{T}{(\lambda x :
            T.P)}{(e~t)}{?_{P[e~t/x]}})}
        {\hnf{S}}{\mysubset{x}{T}{P}}$}
      \AXC{$\subimpl{`G}{d}{\mysubset{x}{T}{P}}{\hnf{U}}$}
      \BIC{$$} 
    \end{prooftree}
    
    Si $\subimpl{`G}{d}{\mysubset{x}{T}{P}}{\hnf{U}}$ alors
    $\subimpl{`G}{e'}{T}{\hnf{U}}$ est
    d�rivable par une d�rivation plus petite et $d = e' `o \pi_1$.
    Par induction, avec l'hypoth\`ese $\subimpl{`G}{e}{\hnf{S}}{T}$, il existe $f$ tel que
    $\subimpl{`G}{f}{\hnf{S}}{\hnf{U}}$ est d�rivable par une d�rivation
    n'utilisant pas \irule{SubTrans} et $f \eqbr e' `o e$. 
    On a bien $f \eqbr e' `o e \eqbr d `o c \eqbr e' `o \pi_1 `o c$ car
    \begin{eqnarray*}
      e' `o \pi_1 `o c & = & e' `o \pi_1 `o (\lambda t : S.~\elt{T}{(\lambda x :
        T.P)}{(e~t)}{?_{P[e~t/x]}}) \\
      & = &
      \lambda x.e'~(\pi_1~((\lambda t : S.~\elt{T}{(\lambda x :
        T.P)}{(e~t)}{?_{P[e~t/x]}})~x)) \\
      & "->"_{\beta} &
      \lambda x.e'~(\pi_1~(\elt{T}{(\lambda x :
        T.P)}{(e~x)}{?_{P[e~x/x]}})) \\
      & "->"_{\beta} & \lambda x.e'~(e~x) \\
      & = & e' `o e
    \end{eqnarray*}      
    
    \case{SubSub}\quad
    \begin{prooftree}
      \AXC{$\subimpl{`G}{c}{S}{\hnf{T}}$}
      \UIC{$\subimpl{`G}{d = c `o \pi_1}
        {\mysubset{x}{S}{P}}{\hnf{T}}$}
      \AXC{$\subimpl{`G}{e}{\hnf{T}}{\hnf{U}}$}
      \BIC{$$}
    \end{prooftree}
    
    Par induction il existe une d�rivation de $\subimpl{`G}{f \eqbr e `o c}
    {S}{U}$ n'utilisant pas \irule{SubTrans}. On peut donc d�river:

    \begin{prooftree}
      \AXC{$\subimpl{`G}{f}{S}{U}$}
      \UIC{$\subimpl{`G}{f `o \pi_1}
        {\mysubset{x}{S}{P}}{U}$}
    \end{prooftree}
    
    On a bien $f `o \pi_1 \eqbr e `o c `o \pi_1$.
    
  \end{induction}
\end{proof}


On peut maintenant �noncer le lemme de symm�trie:

\begin{lemma}[Sym�trie de la coercion]
  \label{subi-sym}
  S'il existe $c$ tel que $`G \typec c : A \subi B$ alors $`E!c^{-1}, `G
  \typec c^{-1} : B \subi A$ et $c^{-1} `o c \eqbre \sref{id}~A$ et $c
  `o c^{-1} \eqbre \sref{id}~B$, o� $\sref{id} \coloneqq \lambda X :
  Set.\lambda x : X. x$.
\end{lemma}
\begin{proof}
  On sait que le jugement $\subi$ est symm�trique, c'est � dire qu'on a
  l'existence des inverses. On utilise la transitivit� pour montrer le
  reste du lemme. Si $\subimpl{`G}{c}{A}{B}$ et
  $\subimpl{`G}{c^{-1}}{B}{A}$ alors il existe $f$ et $f^{-1}$ tel que
  $\subimpl{`G}{f \eqbr c^{-1} `o c}{A}{A}$ et $\subimpl{`G}{f^{-1}
    \eqbr c `o c^{-1}}{B}{B}$. Par unicit� des coercions, $f \eqbre
  \lambda x : A.x$ et $f^{-1} \eqbre \lambda x : B.x$. En g�n�ral $f$ et
  $f'$ sont en forme $\eta$-longue et pas l'identit�.
\end{proof}

%%% Local Variables: 
%%% mode: latex
%%% TeX-master: "subset-typing"
%%% LaTeX-command: "TEXINPUTS=\"style:$TEXINPUTS\" latex"
%%% End: 


\begin{lemma}[Coercion identit�]
  Si $`G \typec T : s$ alors il existe $cid_T$,
  $\subimpl{`G}{cid_T}{T}{T}$ et pour tout $t$, $`G \typec t : T$,
  $cid_T t \eqbi t$.
\end{lemma}

\begin{proof}
  Par induction sur la forme de $T$.

  

\end{proof}


On peut maintenant �noncer le lemme de symm�trie:

\begin{lemma}[Sym�trie de la coercion]
  \label{subi-sym}
  S'il existe $c$ tel que $`G \typec c : A \subi B$
  alors $`E!c^{-1}, `G \typec c^{-1} : B \subi A$ et $c^{-1} `o c \eqbei
  \sref{id}~A$ et $c `o c^{-1} \eqbei \sref{id}~B$, o�
  $\sref{id} \coloneqq \lambda X : Set.\lambda x : X. x$.
\end{lemma}
\begin{proof}
  On sait que le jugement $\subi$ est symm�trique, c'est � dire qu'on a
  l'existence des inverses. On utilise la transitivit� pour montrer le
  reste du lemme. Si $\subimpl{`G}{c}{A}{B}$ et
  $\subimpl{`G}{c^{-1}}{B}{A}$ alors il existe $f$ et $f^{-1}$ tel que
  $\subimpl{`G}{f \eqbi c^{-1} `o c}{A}{A}$ et $\subimpl{`G}{f^{-1}
    \eqbi c `o c^{-1}}{B}{B}$. Par unicit� des coercions, $f \eqbei \lambda
  x : A.x$ et $f^{-1} \eqbei \lambda x : B.x$. En g�n�ral $f$ et $f'$
  sont en forme $\eta$-longue et pas l'identit�.
\end{proof}

%%% Local Variables: 
%%% mode: latex
%%% TeX-master: "subset-typing"
%%% LaTeX-command: "TEXINPUTS=\"style:$TEXINPUTS\" latex"
%%% End: 


%%% Local Variables: 
%%% mode: latex
%%% TeX-master: "subset-typing"
%%% LaTeX-command: "TEXINPUTS=\"style:$TEXINPUTS\" latex"
%%% End: 


\chapter{\Subtac}
Nous avons d\'evelopp\'e la contribution \Subtac{} (pour
``subset-tactics'') disponible dans la version
\CVS{}~courante de \Coq{} (\url{http://coq.inria.fr}). Elle permet de
typer un programme en \lng{} et g\'en\'erer un terme incomplet
correspondant (voir annexe \ref{fig:euclid-subtac}). 

\section{Existentielles}
La g\'en\'eration des buts correspondant aux variables existentielles et la
formation du terme final devaient originellement \^etre laiss\'ees \`a la
tactique \Refine~et au syst\`eme de gestion des existentielles de \Coq. Certaines limitations 
dans l'implantation du raffinement (le m\'ecanisme permettant de manipuler
des termes ``\`a trous'') nous ont emp\^ech\'e d'utiliser \Refine. En
particulier, la gestion des d\'efinitions r\'ecursives et la pr\'esence de
variables existentielles dans les types d'autres existentielles
 n'\'etaient pas support\'ees. En
cons\'equence, nous avons d\'evelopp\'e une nouvelle tactique \Coq{}
permettant de g\'erer les termes avec existentielles de fa\c con plus
g\'en\'erale. 

\subsection{La tactique \eterm}
L'id\'ee de d\'epart de la tactique \Refine{} est de prendre un terme \`a
trous et d'en faire une traduction en une s\'equence de tactiques. Par
exemple, lorsque \Refine{} rencontre une abstraction, il fait une
introduction, lorsque c'est un cast, on applique l'identit\'e et ainsi de
suite. Intuitivement, la s\'equence de tactiques engendr\'ee va construire
le terme de d\'epart implicitement. 

La tactique \eterm{} fonctionne diff\'eremment. \`A partir d'un terme $t$
contenant des existentielles, \eterm{} va g\'en\'eraliser le terme par
rapport \`a celles-ci, et g�n�raliser chaque existentielle par rapport �
son contexte, cr\'eant ainsi un objet $(\lambda ex_1 : T_1, \ldots, ex_n :
T_n, t[?_1 := ex_1, \ldots, ?_n := ex_n])$, o� chaque $ex_i$ est appliqu�
aux variables introduites dans son contexte par les abstractions.
Habituellement, on propose $t$ comme habitant d'un type $T$ donn\'e (le
but), par exemple on peut proposer $\lambda x : \nat.x$ comme preuve du
but $\nat "->" \nat$.
Plut\^ot que de donner directement $t$, on applique le nouveau
terme, et \Coq{} va automatiquement nous demander d'instancier les
arguments $ex_1 \ldots ex_n$ correspondant aux existentielles du terme 
$t$. Cette technique permet d'avoir des d\'ependances entre existentielles
(par exemple, $ex_1$ peut appara�tre dans tous les types $T_2 \ldots
T_n$) et de ne pas reposer enti�rement sur la gestion des existentielles
de \Coq{} qui n'est pas tr\`es flexible \`a l'heure actuelle. En
particulier, si l'on applique une produits o� il y a des d�pendances
entre arguments, \Coq{} va recr�er des existentielles.

Il nous faut nous pencher un peu plus avant sur la g\'en\'eralisation des
existentielles pour comprendre le m\'ecanisme d'\eterm.
Puisqu'on veut pouvoir avoir des d\'ependances entres les $n$
existentielles d'un terme et qu'on s\'erialise celles-ci en un produit
$n$-aire, il nous faut \^etre tr\`es attentifs \`a l'ordre dans lequel on
g\'en\'eralise les variables existentielles. Si $?_3 : T_3$ o\`u $T_3$
r\'ef\'erence $?_4$, il faut que l'existentielle $ex_4$ apparaisse
\emph{avant} $ex_3$ dans notre produit. Il est toujours possible de
trouver un ordre compatible avec ces d\'ependances puisqu'il est
impossible de cr\'eer un cycle o\`u $?_i$ r\'ef\'erencerait $?_j$ et vice-versa
(ceci est assur\'e par le caract\`ere fonctionnel des objets implantant
les existentielles dans \Coq). La tactique actuelle est cod\'ee avec
l'hypoth\`ese que toute existentielle $?_i$ ne d\'epend pas des
existentielles d'indice sup\'erieur \`a $i$. Il est cependant envisageable de
r��crire tout terme contenant des existentielles comme un terme
\'equivalent avec des indices respectant cet ordre.

\section{Traitement de la r\'ecursion}
Lorsque l'on d\'eveloppe un programme r\'ecursif dans un syst\`eme tel que
\Coq, on est forc\'e de fournir une preuve de sa terminaison. Pour cela,
on montre g\'en\'eralement qu'on a un ordre bien
fond\'e sur le type de l'argument de r\'ecursion et que chaque appel respecte
cet ordre. Nous avons ajout\'e des facilit\'es d'\'ecriture de fonctions
r\'ecursives \`a notre langage ; on ajoute les existentielles
correspondant aux preuves que l'ordre est bien fond\'e ou qu'il est bien
respect\'e par les termes. Ainsi lors du raffinement on obtient naturellement
les buts correspondants \`a prouver.


La possibilit� de faire des fonctions r�cursives �
l'int�rieur des termes devraient �tre facile � introduire, il suffit
d'avoir un combinateur $\sref{fix}$ comme constante avec une forme
pratique pour le langage: par exemple la fonction utilisable pour
l'appel r�cursif devrait avoir un type de la forme:
$\mysubset{x}{T}{R~x~a} "->" B$ o� $R$ est la relation bien fond�e et $T$ le type de
l'argument de r�cursion ainsi on g�n�rera les obligations automatiquement
lors des appels r�cursifs.

\section{Traitement des inductifs}
Notre langage ne prend pas encore en compte les d\'efinitions inductives g\'en\'erales.
Au-del\`a du traitement des types sous-ensemble, on a un support minimal
pour les inductifs \`a deux constructeurs qui correspondent \`a des bool\'eens
annot\'es par des propri\'et\'es logiques (voir traitement de la
conditionnelle figure \ref{fig:euclid-subtac}). A long terme on devrait
pouvoir traiter les inductifs dans $\Set$ pr\'edicatif, qui ne peuvent 
embarquer des propositions qu'en utilisant des types sous-ensemble avec
le m\^eme m\'ecanisme de coercion et conserver l'inf\'erence.

\section{\texttt{Program} et \texttt{Recursive program}}
On peut utiliser notre contribution � l'aide des deux tactiques
\texttt{Program} et \texttt{Recursive program}. 

\paragraph{\texttt{Program}}
La syntaxe pour l'appel de cette tactique est la suivante:
$\texttt{Program}~\sref{name}~:~`t := `a.$ o� $`t, `a$ d�nottent les
cat�gories syntaxiques des types et des termes respectivement.
La tactique fonctionne ainsi:
On inf�re le type du terme, on applique la coercion du terme vers le
type sp�cifi� puis on r��crit le terme obtenu dans \CCI{}. On r��crit
ensuite le type sp�cifi� dans \CCI{} et on le pose comme but. On utilise
\eterm{} qui va s'occuper du terme � trous et l'appliquer au but. On
obtient alors automatiquement les obligations de preuves. 

\paragraph{\texttt{Recursive program}}
Cette deuxi�me tactique permet d'�crire des d�finitions r�cursives, on va donc demander
plus d'informations � l'utilisateur lors de la d�finition:
\[\texttt{Recursive program}~\sref{name}~(a : `t_a) \{ \sref{wf}~R~(\sref{auto} `|
  \sref{proof}~p)? \} : ~`t := `a.\]
L'argument $a$ de type $`t_a$ est l'argument de r�cursion, $R$ est la
relation d'ordre qu'on doit montrer bien fond�e. Si l'on utilise
$\sref{auto}$ la tactique va chercher dans une base une preuve de bonne
fondation. Si l'on utilise $\sref{proof}$ alors $p$ doit r�f�rencer une
preuve de la bonne fondation de $R$ dans l'environnement courant. Sinon
on g�n�re une obligation de preuve que $R$ est bien fond�.
La tactique fait ensuite � peu pr�s la m�me chose que \texttt{Program}.
Le typage et la r��criture se font dans un contexte o� la
fonction $name : \Pi a : `t_a.`t$ appara�t et
l'on v�rifie � l'application si l'on fait un appel r�cursif auquel cas on
ins�re une existentielle correspondant � la preuve que l'ordre est bien
respect� (dans \Coq, la fonction permettant de faire la r�cursion aura
un type de la forme $\Pi x : `t_a. R~x~a "->" `t$). Un exemple de
l'utilisation de \texttt{Recursive program} est donn� figures
\ref{fig:euclid-subtac} et \ref{fig:euclid-subtac-script}.


%%% Local Variables: 
%%% mode: latex
%%% TeX-master: "subset-typing"
%%% LaTeX-command: "TEXINPUTS=\"style:$TEXINPUTS\" latex"
%%% End: 


%%% Local Variables: 
%%% mode: latex
%%% TeX-master: "subset-typing"
%%% LaTeX-command: "TEXINPUTS=\"style:$TEXINPUTS\" latex"
%%% End: 
