\documentclass{article}
\usepackage[latin1]{inputenc}
\usepackage{xspace} % To get the right spacings in front of : and so on
\usepackage[english]{babel}
\usepackage{abbrevs}
\usepackage{subfigure}
\usepackage{ifthen}
\usepackage{coqdoc}
\usepackage{bussproofs}
\usepackage{pgf,pgfarrows,pgfnodes}
\usepackage{amsmath}
\usepackage{concmath}
\usepackage{url}
\usepackage{fullpage}

\def\tactic#1{\texttt{#1}}

\title{\Russell's Metatheoretic Study - Notes}
\date{\today}
\author{Matthieu Sozeau \\
  LRI - Paris Sud - XI University \\
  sozeau@lri.fr}
\begin{document}
\maketitle
\begin{abstract}
We are working on the formalization of \Russell{}'s type theory in the
\Coq{} proof assistant \cite{sozeau:coq/Russell/meta}.  The type system
of \Russell{} is based on the Calculus of Constructions with
$\Sigma$-types (dependent sums), extended by an equivalence on types
which subsumes $\beta$-conversion. The extension permits to identify
types and subsets based on them in a manner similar to the
\emph{Predicate Subtyping} feature of \PVS{}.

We are aiming at a complete proof of \Russell{} metatheoretic properties
(structural properties, Subject Reduction, maybe Strong Normalization),
the refining steps which led us to the algorithmic system and the
corresponding typing algorithm and also the correctness of a traduction
from \Russell{} to the Calculus of Inductive Constructions with
metavariables.

We started the development using the formalization of the Calculus of
Constructions by Bruno Barras \cite{Barras96a}.  We kept the standard de
Bruijn encoding for variable bindings and defined our judgements using
\emph{dependent} inductive predicates.  This alone causes some problems
for the faithful formalization of the paper results.  The proofs offer
several other technical difficulties including:
\begin{itemize}
\item Elimination of transitivity in a system with an \emph{untyped} type
  conversion relation.

\item Subject Reduction, which is \emph{not} directly provable for the
  declarative system. Here we adapted the technique developped by
  Robin Adams \cite{adams:PTSEQ}. It includes a new term algebra, with
  associated reduction operations and a new type system
  for which we have to prove metatheoretic properties.

\item Correction of the traduction: the target system include
  metavariables, introducing a second, \emph{unusual} kind of variable binding.
\end{itemize}
\end{abstract}

- Works only with $``>=$ coq-v8.2.

\begin{itemize}
\item Work on a dependency analyser for terms.
\item Problem of mutual induction, conjunction of goals, generating the
  principle and applying it.
\item Problem of induction on partially instanciated goals, see McBride.
\item Induction on depths:
  \begin{itemize}
  \item Adding a parameter on judgements.
  \item Adding a parameter and going in \Set.
  \end{itemize}
\item Useful tactics: \tactic{subst}, \tactic{eauto}, \tactic{generalize dependent}.
\item Reusability of proofs (multiple systems).
\end{itemize}

\bibliography{bib-joehurd,barras,pvs-bib,bcp,Luo,biblio-subset,%
cparent,demons,demons2,demons3}
\bibliographystyle{acm}

%% \renewcommand{\thefootnote}{}
%% \footnotetext{This article was typeset in \LaTeX~using the
%%   \texttt{Concrete Math} font}

\end{document}

%%% Local Variables: 
%%% mode: latex
%%% LaTeX-command: "x=pdf; TEXINPUTS=\"..:../style:../figures:\" ${pdfx}latex"
%%% TeX-master: t
%%% End: 
