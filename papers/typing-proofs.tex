\setboolean{displayLabels}{false}

\begin{lemma}[Conservation de la conversion par sous-typage]
  Si $`G \typed t : T$, $T \eqbi U$ alors $`G \judgesub t : T \impsub U$.
\end{lemma}

\begin{proof}
  Par induction sur la forme de $T$.
  
  \def\seq{\judgesub}.
  
  \begin{itemize}
  \item[$T$ atomique:]
    On a alors $U = T$, trivial.
    
  \item[$T `= \Pi X.Y$:]
    Alors $U `= \Pi V.W$ et $X \eqbi V$, $Y \eqbi W$, 
    donc par induction $`G \judgesub x : U \impsub T$ et 
    $`G, x : U \judgesub v : V \impsub W$. On applique alors
    \SubProdRule{} � ces deux pr�misses.
    
  \item[$T `= \Sigma x : X.Y$:]
    $U$ est de la forme $\Sigma x : V.W$, avec $X \eqbi V$ et $Y \eqbi
    W$. Par induction et application de \SubSigmaRule.
    
  \item[$T `= \subset{x}{X}{P}$:] 
    On a alors $U `= \subset{x}{X'}{P'}$ avec $X \eqbi X'$, $P \eqbi
    P'$, et la propri�t� est vraie par \SubLeftRule{} et \SubRightRule{}:
    
    \begin{prooftree}
      \AXC{$`G \seq t : X \impsub X'$}
      \AXC{$`G \typei P : \Pi x : X.\Prop$}
      \LeftLabel{\SubLeftRule}
      \BIC{$`G \seq t : \subset{x}{X}{P} \impsub X'$}
      \LeftLabel{\SubRightRule}
      \UIC{$`G \seq t : \subset{x}{X}{P} \impsub \subset{x}{X'}{P'}$}
    \end{prooftree}
    
  \end{itemize}
\end{proof}

\begin{lemma}[Substitutivit� des termes du sous-typage]
  \label{substitutive-term-subtyping}
  Si $`G \judgesub x : U \impsub T$ alors pour tout $u$ tel que $`G
  \typed u : U$, $`G \judgesub u : U \impsub T$.
\end{lemma}
\begin{proof}
  Par induction sur la d�rivation de sous-typage 
  $`G \judgesub x : U \impsub T$:
  \def\seq{\judgesub}.
  
  \begin{itemize}
  \item[\SubConvRule:]
    Direct par pr�servation de l'�quivalence $\eqbi$ par le sous-typage.
    
  \item[\SubProdRule,\SubSigmaRule:] $x$ est une variable, on ne peut
    donc pas utiliser ces r�gles.
    
  \item[\SubLeftRule:] On a
    \begin{prooftree}
      \BAX{Sub-Left}
      {$`G \seq x : U' \impsub V$}
      {$`G \typei P : \Pi x : U'. \Prop$}
      {$`G \seq x : \subset{x}{U'}{P} \impsub V$}
      {}
    \end{prooftree}
    
    On a $U `= \subset{x}{U'}{P}$ et $`G \judgesub x : U' \impsub V$.
    Par induction, si $`G \typed u : U'$ alors $`G \judgesub u : U' \impsub T$.
    Si $`G \typed u : \subset{x}{U'}{P}$ alors
    $`G \typed u : U'$ par application de \LetSubRule et donc par
    application de \SubLeftRule, $`G \judgesub u : \subset{x}{U'}{P}
    \impsub V$.
    
  \item[\SubRightRule:] On a
    \begin{prooftree}
      \UAX{}
      {$`G \seq x : U \impsub V'$}
      {$`G \seq x : U \impsub \subset{x}{V'}{P}$}
      {}
    \end{prooftree}
    
    On a $V `= \subset{x}{V'}{P}$ et $`G \judgesub x : U \impsub V'$.
    Par induction, si $`G \typed u : U$ alors $`G \judgesub u : U
    \impsub V'$. Donc par application de \SubRightRule, $`G \judgesub u : U \impsub V$.
    
  \end{itemize}
\end{proof}

\begin{lemma}[Substitutivit� du sous-typage]
  \label{substitutive-subtyping}

  Si $`G, x : V \judgesub t : T \impsub U$, $G \typei u : V$ et 
  $`G \typei t[u/x] : T[u/x]$, alors $`G \typed t[u/x] : T[u/x] \impsub U[u/x]$.
\end{lemma}
%% \begin{proof}
%%   Par induction sur la d�rivation de sous-typage:
%%   \def\seq{\judgesub}.
  
%%   \begin{itemize}
%%   \item[\SubConvRule:]
%%     On a $T \eqbi U$, $`G, x : V \judgesub y : T \impsub U$ et
%%     $`G \typei y[u/x] : T[u/x]$.    
%%     Si $y = x$, alors on doit montrer $`G \judgesub u : T[u/x] \impsub
%%     U[u/x]$ qui est vrai car $\eqbi$ est substitutive et $T[u/x], U[u/x]
%%     $ sont bien typ�s.
%%     Sinon, $`G \typei y : T[u/x] \impsub U[u/x]$, vrai par
%%     substitutivit� de $\eqbi$.
    
%%   \item[\SubProdRule:] On a
%%     \begin{prooftree}
%%       \SubProd
%%     \end{prooftree}
    
    
%%   \item[\SubSigmaRule:] On a 
%%     \begin{prooftree}
%%       \SubSigma
%%     \end{prooftree}
    
%%     Par hypoth�se d'induction, $`G \typed t : T, U$, $`G \typed v : V[t/x], W[t/y]$.
%%     Par \SumRule, $(t, v) : \Sigma y : U, W$.

%%   \item[\SubLeftRule:] On a
%%     \begin{prooftree}
%%       \SubLeft
%%     \end{prooftree}
    
%%     Par induction, $`G \typed p : V$.

%%   \item[\SubRightRule:] On a
%%     \begin{prooftree}
%%       \SubRight
%%     \end{prooftree}
    
%%     Par induction, $`G \typed p : U$. Par \SubsetRule, $`G \typed p :
%%     \subset{x}{U}{P}$.
    
%%   \end{itemize}
%% \end{proof}

\begin{lemma}[Inversion du typage]
\label{inversion-typing}

On a les propri�t�s suivantes sur le jugement de typage:
\begin{enumerate}
\item Si $`G \type t : \Pi x : T.U$ alors $t = \lambda x : T.v$ et $`G, x : T
  \type v : U$.
\item Si $`G \type t : \Sigma x : T.U$ alors $t = (t', v)$, $`G, x : T \type U
  : s1$ et $`G \type v : U[t/x]$.
\item Si $`G \type t : \subset{x}{U}{P}$ alors $`G \type t : U$.
\end{enumerate}
\end{lemma}

\begin{lemma}[Inversion du sous-typage]
  \label{inversion-subtyping}
  Si $`G \judgesub t : \Pi x : T.U \impsub \Pi x : V.W$ 
  alors $t = \lambda x : T. M$ et $`G, x : T \typei M : U$.
  
\end{lemma}

\begin{lemma}[Admissibilit� de la r�flexivit�, transitivit� du sous-typage]
  \quad
  \label{refl-trans-subtyping}
  \begin{enumerate}
  \item Pour tout $t, S$, si $`G \typed t : S$ alors $`G \judgesub t : S \impsub S$.
  \item Pour tout $t, S, T, U$,  si $`G \judgesub t : S \impsub T$ et $`G \judgesub t
    : T \impsub U$ alors $`G \judgesub t : S \impsub U$.
  \end{enumerate}
\end{lemma}

\begin{proof}  
  \begin{enumerate}
  \item $S \eqbi S$, donc pour tout $x : S$, $`G \judgesub x : S \impsub
    S$. Par substivit� du sous-typage, la propri�t� est vraie pour tout
    $t$ tel que $`G \typed t : S$.
  \item Dans \cite{Pierce:TypeSystems}, voir p. 420.
  \end{enumerate}
\end{proof}

\begin{lemma}[Correction du sous-typage]
  \label{correct-subtyping}
  Si $`G \judgesub t : U \impsub V$ alors $`G \typed t : U "=>" `G
  \typed t : V$.
\end{lemma}

\begin{proof}
  Par induction sur la d�rivation de sous-typage:
  
  \begin{itemize}
  \item[\SubConvRule:] On a $T \eqbi U$ et l'on suppose 
    $x \typed T$. On a donc bien $`G \typed x : U$.
    
  \item[\SubProdRule:] On a
    \begin{prooftree}
      \SubProd
    \end{prooftree}
    
    Par induction, $`G \typed x : U "=>" `G \typed x : T$, 
    $`G, x : U \typed v : V "=>" `G, x : U \typed v : W$.
    On suppose $`G \typed \lambda x : T.v : \Pi x : T.V$.
    Par inversion du typage, on a $`G, x : T \typed v : V$.
    
    Par \AbsRule, $`G \typed \lambda x : U.v : \Pi x : U.W$.
    
  \item[\SubSigmaRule:] On a 
    \begin{prooftree}
      \SubSigma
    \end{prooftree}
    
    Par hypoth�se d'induction, $`G \typed t : T, U$, $`G \typed v : V[t/x], W[t/y]$.
    Par \SumRule, $(t, v) : \Sigma y : U, W$.

  \item[\SubLeftRule:] On a
    \begin{prooftree}
      \SubLeft
    \end{prooftree}
    
    Par induction, $`G \typed p : V$.

  \item[\SubRightRule:] On a
    \begin{prooftree}
      \SubRight
    \end{prooftree}
    
    Par induction, $`G \typed p : U$. Par \SubsetRule, $`G \typed p :
    \subset{x}{U}{P}$.
    
  \end{itemize}
  
\end{proof}


\begin{lemma}[Correction du typage]
  \label{correct-typing}
  $`G \typei t : T "=>" `G \typed t : T$
\end{lemma}

\begin{proof}
  Par induction sur la d�rivation dans le syst�me algorithmique:

  \begin{description}
  \item[\WfAtomRule,\WfVarRule,\PropSetRule,\VarRule,\ProdRule,\AbsRule,
    \LetInRule, \SigmaRule, \SumRule, \LetSumRule:] r�gles inchang�es.

  \item[\AppRule:] On a
    \def\fCenter{\typei}
    \begin{prooftree}
      \AppI
    \end{prooftree}
    
    Par induction, $`G \typed f : \Pi x : V. W $.
    Par le lemme \ref{correct-subtyping}, et l'hypoth�se $`G \typed u : U$, 
    $`G \typed u : V$. Donc, par \AppRule, on a bien $`G \typed f u :
    W[u/x]$.
  \end{description}
  
\end{proof}

\begin{lemma}[Compl�tude du typage]
  \label{complete-typing}
  $`G \typed t : T "=>" `E U, `G \typei t : U `^ `G \judgesub t : U \impsub T$
\end{lemma}

\begin{proof}
  Par induction sur la d�rivation dans le syst�me d�claratif:

  \begin{description}
  \item[\WfAtomRule,\WfVarRule,\PropSetRule,\VarRule,\ProdRule,\AbsRule,
    \LetInRule, \SigmaRule, \SumRule, \LetSumRule:] r�gles inchang�es.
    
  \item[\AppRule:] On a 
    \begin{prooftree}
      \App
    \end{prooftree}
    
    Par induction, $`E X, Y, `G \typei f : \Pi x : X. Y `^ 
    `G \judgesub f : \Pi x : X. Y \impsub \Pi x : V. W$ et
    $`E U, `G \typei u : U `^ `G \judgesub u : U \impsub V$.
    
    Si $`G \judgesub f : \Pi x : X.Y \impsub \Pi x : V.W$, alors 
    $f$ est de la forme $\lambda x : X.v$ et par inversion (lemme
    \ref{inversion-subtyping}) $`G \judgesub x : V \impsub X$.
    Par substitutivit� du sous-typage (lemme
    \ref{substitutive-term-subtyping}), on a donc $`G \judgesub u : V
    \impsub X$. Par transitivit� du sous-typage \ref{refl-trans-subtyping}, 
    $`G \judgesub u : U \impsub X$. On peut donc appliquer \AppRule{} pour
    obtenir $`G \typei f u : Y[u/x]$. 
    
    Par la covariance du produit en son codomaine, 
    $`G, x : V \judgesub v : Y \impsub W$. Par substitivit� du
    sous-typage (lemme \ref{substitutive-subtyping}), on a donc
    $`G \judgesub v[u/x] : Y[u/x] \impsub W[u/x]$, la propri�t� est
    donc bien v�rifi�e.
    
  \item[\SubsetRule:] On a
    \begin{prooftree}
      \Subset
    \end{prooftree}
    
    Par induction, $`E T, `G \typei x : T `^ `G \judgesub x : T \impsub U$.
    

  \item[\LetSubRule:]
    
  \item[\ConvRule:]

  \end{description}
  
\end{proof}


%%% Local Variables: 
%%% mode: latex
%%% TeX-master: "subset-typing"
%%% End: 
