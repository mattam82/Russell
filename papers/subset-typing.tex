\documentclass[a4paper,11pt]{article}
\usepackage[francais]{babel} 
\usepackage[latin1]{inputenc}  %% les accents dans le fichier.tex
\usepackage[T1]{fontenc}       %% Pour la c\'{e}sure des mots accentu\'{e}s
\usepackage{indentfirst}
\usepackage{a4}
\usepackage[dvips]{graphicx}
\usepackage{coqdoc}
\usepackage{amssymb}
\usepackage{amsmath}
\usepackage{amsthm}
\usepackage{amsfonts}
\usepackage{array}
\usepackage{myabbrevs}
\usepackage{bnf}
\usepackage{bussproofs}
\usepackage{hyperref}
\usepackage{fullpage}
%\usepackage{concmath}
\usepackage{cmbright}
\usepackage{fancyhdr}
\usepackage{ifthen}

\def\infvspace{2em}
% This is the "centered" symbol
\def\fCenter{\vdash}
\def\seq{\fCenter}
% Optional to turn on the short abbreviations
\EnableBpAbbreviations

\newtheorem{lemma}{Lemme}[section]
\newtheorem{theo}[lemma]{Th�or�me}
\newtheorem{prop}[lemma]{Proposition}

\newcommand{\src}[1]{\texttt{#1}}
\newcommand{\srcm}[1]{\text{\texttt{#1}}}
%\newcommand{\Set}{\ensuremath{\text{\texttt{Set}}}}
\newcommand{\Prop}{\ensuremath{\text{\texttt{Prop}}}}

\newcommand{\rname}[1]{{\bf #1}}
\newcommand{\rulelabel}[1]{{\bf (#1)}}
\newcommand{\UR}[2]{\RightLabel{\rulelabel{#1}}\UIC{#2}}
\newcommand{\URL}[2]{\LeftLabel{\rulelabel{#1}}\UIC{#2}}

\def\impsub{\rightslice}

\newboolean{displayLabels}
\setboolean{displayLabels}{true}

\def\HOL{{\tt HOL}}
\def\Coq{{\tt Coq}}
\def\PVS{{\tt PVS}}

\makeatletter

\newcommand{\LeftRuleLabel}[1]{
  \@ifnotmtarg{#1}{\ifthenelse{\boolean{displayLabels}}{\LeftLabel{\rulelabel{#1}}}{}}
}

\newcommand{\UAX}[4]{\AXC{#2}
  \LeftRuleLabel{#1}
  \@ifnotmtarg{#4}{\RightLabel{#4}}
  \UIC{#3}}

\newcommand{\BAX}[5]{\AXC{#2}\AXC{#3}
  \LeftRuleLabel{#1}
  \@ifnotmtarg{#5}{\RightLabel{#5}}
  \BIC{#4}}

\newcommand{\TAX}[6]{\AXC{#2}\AXC{#3}\AXC{#4}
  \LeftRuleLabel{#1}
  \@ifnotmtarg{#6}{\RightLabel{#6}}
  \TIC{#5}}
\makeatother

\newcommand{\BR}[2]{\RightLabel{\rulelabel{#1}}\BIC{#2}}
\newcommand{\BRL}[2]{\LeftLabel{\rulelabel{#1}}\BIC{#2}}

\newcommand{\letml}{\textbf{let}~}
\newcommand{\inml}{~\textbf{in}~}
\newcommand{\ifml}{~\textbf{if}~}
\newcommand{\thenml}{~\textbf{then}~}
\newcommand{\elseml}{~\textbf{else}~}
\newcommand{\funml}{~\textbf{fun}~}

\newcommand{\eqbi}{`=_{\beta\iota}}

\def\judgewf{\vdash_{wf}}
\def\judgetyped{\vdash}
\def\judgetypei{\vdash_{\bullet}}
\def\judgesub{\vdash_{\impsub}}
\def\judgerw{~{\mbox{$"~>"$}}~}
\def\typed{\judgetyped}
\def\type{\typed}
\def\typei{\judgetypei}
\def\wf{\judgewf}

\newcommand{\elt}[4]{\text{elt}~#1~#2~#3~#4}
\renewcommand{\subset}[3]{\{ #1 : #2 `| #3~#1 \}}

\def\thetitle{Sous-typage par pr�dicats en Coq}

\pagestyle{fancy}
\fancyhead[RO,LE]{\thetitle}
\fancyfoot[C]{\thepage}
%\renewcommand{\headrulewidth}{0pt}

\newcommand{\matht}[1]{\text{{\tt #1}}}

\def\even{\matht{even}}
\def\odd{\matht{odd}}
\def\Set{\matht{Set}}
\def\Prop{\matht{Prop}}
\def\Type{\matht{Type}}
\def\ps{\emph{predicate subtyping}}

\title{\thetitle}

\author{Matthieu Sozeau}

\date{\today}

\begin{document}

\maketitle

\begin{abstract}
  blabla 
\end{abstract}

\section*{Introduction}
Le \ps{} impl�ment� dans PVS (\cite{PVS-Semantics:TR,Shankar&Owre:WADT99,Rushby98:TSE}).


\def\WfAtomRule{\rname{Wf-Atom}}
\def\WfVarRule{\rname{Wf-Var}}
\def\PropSetRule{\rname{PropSet}}
\def\VarRule{\rname{Var}}
\def\ProdRule{\rname{Prod}}
\def\AbsRule{\rname{Abs}}
\def\AppRule{\rname{App}}
\def\LetInRule{\rname{LetIn}}
\def\SigmaRule{\rname{Sigma}}
\def\SumRule{\rname{Sum}}
\def\LetSumRule{\rname{LetSum}}
\def\SubsetRule{\rname{Subset}}
\def\LetSubRule{\rname{LetSub}}
\def\ConvRule{\rname{Conv}}

\newcommand{\WfEmptyFull}[1]{
  \UAX{WfEmpty}
  {}
  {$\seq []$}
  {}
}  
\newcommand{\WfEmpty}[1][\Gamma]{\WfEmptyFull{#1}}
  
\newcommand{\WfVarFull}[4]{
  \UAX{WfVar}
  {$\tgen{#1}{#2}{#3}$}
  {$\wf #1, #4 : #2$}
  {$#3 `: \setproptype{} `^{} #4 `; #1$}
}
\newcommand{\WfVar}[1][\Gamma]{\WfVarFull{#1}{A}{s}{x}}

\newcommand{\PropSetFull}[2]{
  \UAX{PropSet}
  {$\wf #1$}
  {$\tgen{#1}{#2}{\Type}$}
  {$#2 `: \setprop$} 
}
\newcommand{\PropSet}[1][\Gamma]{\PropSetFull{#1}{s}}


\newcommand{\TypeTypeFull}[1]{
  \UAX{Type}
  {$\wf #1$}
  {$\tgen{#1}{\Type(i)}{\Type(i + 1)}$}
  {$i `: `N$}
}
\newcommand{\TypeType}[1][\Gamma]{\TypeTypeFull{#1}}

\newcommand{\VarFull}[3]{
  \BAX{Var}
  {$\wf #1$}
  {$#2 : #3 `: #1$}
  {$\tgen{#1}{#2}{#3}$}
  {}
}
\newcommand{\Var}[1][\Gamma]{\VarFull{#1}{x}{A}} 

\newcommand{\ProdFull}[7]{
  \BAX{Prod}
  {$\tgen{#1}{#2}{#3}$}
  {$\tgen{#1, #4 : #2}{#5}{#6}$}
  {$\tgen{#1}{\Pi #4 : #2.#5}{#7}$}
  {$(#3, #6, #7) `: \mathcal{R} `^{} #4 `; #1$}
}
\newcommand{\Prod}[1][\Gamma]{\ProdFull{#1}{T}{s_1}{x}{U}{s_2}{s_3}}

\newcommand{\AbsFull}[6]{
  \BAX{Abs}
  {$\tgen{#1}{\Pi #2 : #3.#4}{#5}$}
  {$\tgen{#1, #2 : #3}{#6}{#4}$}
  {$\tgen{#1}{\lambda #2 : #3. #6}{\Pi #2 : #3.#4}$}
  {$#2 `; #1$}
}

\newcommand{\Abs}[1][\Gamma]{\AbsFull{#1}{x}{T}{U}{s}{M}}

 \newcommand{\AppFull}[6]{
  \BAX{App}
  {$\tgen{#1}{#2}{\Pi #3 : #4. #5}$}
  {$\tgen{#1}{#6}{#4}$}
  {$\tgen{#1}{(#2 #6)}{#5 [ #6 / #3 ]}$}
  {$$}
}

\newcommand{\App}[1][\Gamma]{\AppFull{#1}{f}{x}{V}{W}{u}}

\newcommand{\SigmaRFull}[5]{
  \BAX{Sigma}
  {$\tgen{#1}{#2}{#3}$}
  {$\tgen{#1, #4 : #2}{#5}{#3}$}
  {$\tgen{#1}{\Sigma #4 : #2.#5}{#3}$}
  {$#3 `: \{ \Prop, \Set \} `^{} #4 `; #1$} 
}
\newcommand{\SigmaR}[1][\Gamma]{\SigmaRFull{#1}{T}{s}{x}{U}}

\newcommand{\SumDepFull}[7][\Gamma]{
  \TAX{SumDep}
  {$\tgen{#1}{\Sigma #2 : #5. #6}{#7}$}
  {$\tgen{#1}{#3}{#5}$}
  {$\tgen{#1}{#4}{#6[#3 / #2]}$}
  {$\tgen{#1}{\pair{\Sigma #2 : #5.#6}{#3}{#4}}{\Sigma #2 : #5.#6}$}
  {}
}

\newcommand{\SumDep}[1][\Gamma]{\SumDepFull[#1]{x}{t}{u}{T}{U}{s}}
 
\newcommand{\PiLeftFull}[5][\Gamma]{
  \UAX{PiLeft}
  {$\tgen{#1}{#2}{`S #3 : #4.#5}$}
  {$\tgen{#1}{\pi_1~#2}{#4}$}
  {}
}
\newcommand{\PiLeft}[1][\Gamma]{\PiLeftFull[#1]{t}{x}{T}{U}}

\newcommand{\PiRightFull}[5][\Gamma]{
  \UAX{PiRight}
  {$\tgen{#1}{#2}{`S #3 : #4.#5}$}
  {$\tgen{#1}{\pi_2~#2}{#5[\pi_1~#2/#3]}$}
  {}
}
\newcommand{\PiRight}[1][\Gamma]{\PiRightFull[#1]{t}{x}{T}{U}}


\newcommand{\SubsetFull}[4]{
  \BAX{Subset}
  {$\tgen{#1}{#3}{\Set}$}
  {$\tgen{#1, #2 : #3}{#4}{\Prop}$}
  {$\tgen{#1}{\mysubset{#2}{#3}{#4}}{\Set}$}
  {$#2 `; #1$}
}
\newcommand{\SubsetR}[1][\Gamma]{\SubsetFull{#1}{x}{U}{P}}


\newcommand{\SubsumFull}[5]{
  \TAX{Subsum}
  {$\tgen{#1}{#4}{#5}$}
  {$\tgen{#1}{#2}{#3}$}
  {$#5 \sub #2$} % #1 \subt 
  {$\tgen{#1}{#4}{#2}$}
  {}
}
\newcommand{\Subsum}[1][\Gamma]{\SubsumFull{#1}{T}{s}{t}{U}} 

\def\Coerce{\Subsum}

\newcommand{\ConvFull}[5]{
  \TAX{Conv}
  {$\tgen{#1}{#2}{#3}$}
  {$\tgen{#1}{#4}{#5}$}
  {$#5 \eqbr #2$}
  {$\tgen{#1}{#4}{#2}$}
  {}
}
\newcommand{\Conv}[1][\Gamma]{\ConvFull{#1}{T}{s}{t}{U}}

\newcommand{\typedRules}{
  \begin{center}
    \def\seq{\typed}
    \def\fCenter{\wf}
    \WfEmpty \DP\quad
    \WfVar \DP
    
    \def\fCenter{\typed}
    \vspace{\infvspace}
    \PropSet\DP

    \vspace{\infvspace}
    \Var\DP
    
    \vspace{\infvspace}
    \Prod\DP
    
    \vspace{\infvspace}
    \Abs\DP

    \vspace{\infvspace}
    \App\DP

    \vspace{\infvspace}
    \SigmaR\DP

    \vspace{\infvspace}
    \SumDep\DP
    
    \vspace{\infvspace}
    \PiLeft\DP
    \quad
    \PiRight\DP

    \vspace{\infvspace}
    \SubsetR\DP

    \vspace{\infvspace}
    \Subsum\DP
      
  \end{center}
}

\def\typedFig
{
\begin{figure}[tb]
  \typedRules
  \caption{Calcul de coercion par pr�dicats - version d�clarative}
  \label{fig:typing-decl-rules}
\end{figure}
}

%%% Local Variables: 
%%% mode: latex
%%% TeX-master: "subset-typing"
%%% LaTeX-command: "TEXINPUTS=\"style:$TEXINPUTS\" latex"
%%% End: 


\def\AppA{
\TAX{App}
{$\talgo{`G}{f}{T} \quad \mualgo(T) = \Pi x : V. W : s$}
{$\talgo{`G}{u}{U} \quad \talgo{`G}{U, V}{s'}$}
{$\subalgo{`G}{U}{V} $}
{$\talgo{`G}{(f u)}{W [ u / x ]}$}
{}
}

\def\LetSumA{
  \TAX{LetSum}
  {$`G \seq t : S$}
  {$\mualgo(S) = `S x : T. U $}
  {$`G, x : T, u : U \seq v : V $}
  {$`G \seq \letml~(x, u) = t~\inml~v : V$}
  {$x,y `; `G$}
}

\def\PiLeftA{
  \BAX{PiLeft}
  {$\tgen{`G}{t}{S}$}
  {$\mualgo(S) = `S x : T.U$}
  {$\tgen{`G}{\pi_1~t}{T}$}
  {}
}

\def\PiRightA{
  \BAX{PiRight}
  {$\tgen{`G}{t}{S}$}
  {$\mualgo(S) = `S x : T.U$}
  {$\tgen{`G}{\pi_2~t}{U[\pi_1~t/x]}$}
  {}
}


\def\SumInfA{
  \TAX{SumInf}
  {$`G \seq t : T $}
  {$`G \seq u : U $}
  {$`G \seq \Sigma \_ : T.U : s$}
  {$`G \seq (t, u) : \Sigma \_ : T.U$}
  {}
}

\def\SumDepAold{
  \QAX{SumDep}
  {$`G \seq t : T$}
  {$`G \seq u : U'$}
  {$`G \seq \Sigma x : T.U : s$}
  {$`G \seq U' : s \quad U' \suba U[t/x]$}
  {$`G \seq (x \coloneqq~t, u : U) : \Sigma x : T.U$}
  {}
}

\def\SumDepA{
  \QAX{SumDep}
  {$`G \seq t : T$}
  {$`G \seq u : U'$}
  {$`G \seq \Sigma x : T.U : s$}
  {$`G \seq U' : s \quad U' \suba U[t/x]$}
  {$`G \seq \pair{\Sigma x : T.U}{t}{u} : \Sigma x : T.U$}
  {}
}

\def\typeaFig{
\begin{figure}[t]
  \begin{center}
    \def\fCenter{\wf}
    \def\type{\typea}
    \def\subt{\subta}
    \def\sub{\suba}
    
    \WfEmpty\DP
    \quad
    \WfVar\DP
    
    \def\fCenter{\typea}
    \vspace{\infvspace}
    \PropSet\DP
%    \quad
%    \TypeType\DP
    
    \vspace{\infvspace}
    \Var\DP
    
    \vspace{\infvspace}
    \Prod\DP
    
    \vspace{\infvspace}
    \Abs\DP

    \vspace{\infvspace}
    \AppA\DP

%    \vspace{\infvspace}
%    \LetIn\DP
    
    \vspace{\infvspace}
    \SigmaR\DP

%    \vspace{\infvspace}
%    \SumInfA\DP

    \vspace{\infvspace}
    \SumDepA\DP
    
    \vspace{\infvspace}
    \PiLeftA\DP
    \quad
    \PiRightA\DP
    %\LetSumA\DP
    
    \vspace{\infvspace}
    \Subset\DP
  \end{center}
  \caption{Calcul de coercion par pr�dicats - version algorithmique}
  \label{fig:typing-algo-rules}
\end{figure}
}

\def\typemuaFig{
  \begin{figure}[ht]
    \begin{eqnarray*}
      \mualgo'(\mysubset{x}{U}{P}) & "=>" & \mualgo'(\downarrow{U}) \\
      \mualgo'(x)                & "=>" & x \\
      \\
      \mualgo(x) & "=>" & \mualgo' (\downarrow{x})
    \end{eqnarray*}
    \caption{D�finition de $\mualgo$}
    \label{fig:mualgo-definition}
\end{figure}
}

%%% Local Variables: 
%%% mode: latex
%%% TeX-master: "subset-typing"
%%% LaTeX-command: "TEXINPUTS=\"style:$TEXINPUTS\" latex"
%%% End: 

%% \def\SubProd{
%%   \BAX{Sub-Prod}
%%   {$`G \seq U \sub T$} %"<|-|>"
%%   {$`G, x : U \seq V \sub W$}
%%   {$`G \seq \Pi x : T.V \sub \Pi x : U.W$}
%%   {}
%% }

%% \def\SubSigma{
%%   \BAX{Sub-Sigma}
%%   {$`G \seq T \sub U$}
%%   {$`G \seq V \sub W$}
%%   {$`G \seq \Sigma x : T. V \sub \Sigma x : U. W$}
%%   {}
%% }

\def\SubHnf{
  \UAX{SubHnf}
  {$\hnf{T}~\sub \hnf{U}$}
  {$T \sub U$}
  {} 
}

\def\SubLeft{
  \BAX{Sub-Left}
  {$`G \seq U \sub V$}
  {$`G \type P : \Pi x : U. \Prop$}
  {$`G \seq \subset{x}{U}{P} \sub V$}
  {}
}
    
\def\SubRight{
  \UAX{Sub-Right}
  {$`G \seq T \sub U$}
                                %{$`G \judgetype h : P~p$}
  {$`G \seq T \sub \subset{x}{U}{P}$}
  {}
}

\def\subtaFig{
\begin{figure}[ht]
  \begin{center}
    \def\fCenter{\suba}
    \def\type{\typea}
    \def\sub{\suba}

    \SubConv\DP
    \quad
    \SubHnf\DP
    \vspace{\infvspace}

    \SubProd\DP

    \vspace{\infvspace}
    \SubSigma\DP
    
    \vspace{\infvspace}
    \SubProof\DP

    \vspace{\infvspace}    
    \SubSub\DP
  \end{center}
  \caption{Coercion par pr�dicats - version algorithmique}
  \label{fig:subtyping-algo-rules}
\end{figure}
}

%%% Local Variables: 
%%% mode: latex
%%% TeX-master: "subset-typing"
%%% LaTeX-command: "TEXINPUTS=\"style:$TEXINPUTS\" latex"
%%% End: 


\clearpage
 
\def\SubConvRule{\rname{Sub-Conv}}
\def\SubProdRule{\rname{Sub-Prod}}
\def\SubSigmaRule{\rname{Sub-Sigma}}
\def\SubLeftRule{\rname{Sub-Left}}
\def\SubRightRule{\rname{Sub-Right}}

\section*{Journal}

\subsection*{8 mars}
Nouvelle r�gle de produit fonctionel avec contravariance bien typ�e,
produit d�pendant (\SubSigmaRule) covariant.
Un exemple de produit avec contravariance se trouve dans \cite{cal00coherence}, p. 6.
Exemple d'utilisation int�ressante:

\BAX{}
{$`G \judgetype f : \{ \phi : \even "->" `N `| `A x : `N, \phi~x ``<= x \}$}
{$`G \judgetype g : `N "->" `N := \matht{pred}$}
{$`G \judgesub g ``<= f$}
{}
\DP

Le sous-typage avec coercions: Luo, Callaghan, Sa�bi \cite{saibi97inheritance}... 
\begin{itemize}
\item Uniformit� du sous-typage: ne d�pend pas du contexte.
\item Coercions d�clar�es dans l'environement (ex: Coq).
\end{itemize}

Dans HOL, Joe Hurd simule le \emph{predicate subtyping} � la PVS avec
des \emph{predicate sets} \cite{hurd2001a}. Technique adaptable � Coq ?

%%% Local Variables: 
%%% mode: latex
%%% TeX-master: "~/research/coq/papers/subset-typing"
%%% End: 

\subsection*{9 mars}
Le \ps{} dans HOL n'est pas correct, il peut �tre subverti ais�ment �
cause d'une sorte de covariance des domaines de fonctions et de la
fonction de suppression des predicats (similaire � $\mu$):
$inv : `R^{\neq 0} "->" `R `: `R "->" `R$. Le d�veloppement dans \HOL{}
est fait directement dans le language, et il est argu� qu'il n'est pas
possible de le faire correctement � cause de l'exemple
pr�c�dent. L'algorithme de sous-typage est simplement une g�n�ration de
tout les sous-types possibles � partir d'un ensemble de r�gles de
sous-typage pour les constantes et constructions logiques ou fonctionelles.

Trouv� un article \cite{stumpsubset} sur les types sous-ensembles dans PF, logique d'ordre sup�rieur
avec fonctions partielles... Permet de traiter le cas
$\ifml 1 / i > 0 \thenml i \neq 0 \elseml `_$, qui g�n�re une obligation de
preuve $i \neq 0$ dans \PVS{}. Evidemment on ne risque pas de pouvoir
typer ce code en \Coq{} lorsque $`/ : `Z "->" \{ x : `Z `| x \neq 0 \} "->" `Z$ mais
certaines id�es peuvent �tre int�ressantes. 

%%% Local Variables: 
%%% mode: latex
%%% TeX-master: "~/research/coq/papers/subset-typing"
%%% End: 

\subsection*{11 mars}
Typage de la division euclidienne:
$\matht{div} : `A a : `N, `A b : \{ x : `N `| x \neq 0 \}, `S q : `N, `S r : \{ n : `N
`| n < b \}, a = bq + r := \funml a~b "=>" \ifml a < b \thenml (0, a) \elseml \letml (q, r) =
\matht{div}~(a - b)~b \inml (q + 1, r)$.

Soit $`t_{div} = `A a : `N, `A b : \{ x : `N `| x \neq 0 \}, `E q : `N, `E r : \{ n : `N
`| n < b \}, a = bq + r$ et $`G = \matht{div} : `t_{div}$:



\AXC{$1$}
\AXC{$2$}
\BIC{$`G, a : `N, b : `N^{*} \seq \ifml a < b \thenml (0, a) \elseml \letml (q, r) =
  \matht{div}~(a - b)~b \inml (q + 1, r) : `E q \dots$}
\doubleLine
\UIC{$`G \seq \funml a~b "=>" \ifml a < b \thenml (0, a) \elseml \letml (q, r) =
  \matht{div}~(a - b)~b \inml (q + 1, r) : `t_{div}$}
\DisplayProof

Soit $`G_{if} = `G, a : `N, b : `N^{*}, a ``/< b$.
\begin{prooftree}
  \AXC{$`G_{if} \seq b : `N^{*}$}
  \RightLabel{$`b = `N^{*}$}
  \UIC{$`G_{if} \seq b : `b$}

  \AXC{$`G_{if} \seq b : `g$}
  \AXC{$`G_{if} \seq (- a) : `g "->" `b''$}
  \AXC{$`G_{if} \judgetypea $}
  \TIC{$`G_{if} \seq (- a) b : `b''$}
  \AXC{$`G_{if} \seq \matht{div} : `b''' "->" `b' "->" `a$}
  \AXC{$`G_{if} \judgesubd a - b : `b'' ``<= `b''' "~>" t$}
  \UIC{$`G_{if} \seq \matht{div}~(a - b) "~>" \matht{div}~t : `b' "->" `a$}

  \TIC{$`G_{if} \seq \matht{div}~(a - b) : `b' "->" `a$}

  \AXC{$`G_{if} \judgesubd b : `b ``<= `b' "~>" p $}
  \TIC{$`G_{if} \seq \matht{div}~(a - b)~b "~>" \matht{div}~(a - b)~p : `a$}
  
  
  \AXC{$`G_{if}, (q, r) : `a \seq (q + 1, r) : `E q \dots$}
  \RightLabel{1}
  \BIC{$`G_{if} \seq \letml (q, r) = \matht{div}~(a - b)~b \inml (q + 1, r) : `E q \dots$}
\end{prooftree}  

\RightLabel{2}
\AXC{$`G, a : `N, b : `N^{*}, a < b \seq (0, a) : `E q \dots$}
\DisplayProof


%%% Local Variables: 
%%% mode: latex
%%% TeX-master: "~/research/coq/papers/subset-typing"
%%% End: 

\subsection*{14 mars}
R�gles d'introduction, d'�limination et de formation pour $\Pi$, $\Sigma$, sous-types
pr�dicats. Nouveau jugement de typage par r�ecriture. Nombreuses
questions � discuter avec Christine. Beaucoup de bruit dans le buro !


%%% Local Variables: 
%%% mode: latex
%%% TeX-master: "~/research/coq/papers/subset-typing"
%%% End: 

\subsection*{15 mars}
Distinction inf\'erence et typage. 
\begin{description}
\item[Inf\'erence ($"~>"$)] \`a la ML, on v\'erifie: $`G \typei p "~>" T "=>"
  `G \typed p ":" T$. 
\item[Typage ($:$)]. On a $`G \typed p : T$, on veut $`E U, `G \typei p
  "~>" U `^ `G \typei U "~>" T$. On utilise le sous-typage g\'en\`erant les
  obligations de preuve.
\end{description}

Eliminer \rname{LetSub}, inutilisable en pratique.

Quelques points \`a m\'editer:
\begin{itemize}
\item $`O \type 3 : `N "~>" `O \type 3 : `N^*$ ? D\'ependance envers le
  terme pour le sous-typage. De m\^eme, $2 : `N ``<= `N^*$, on devrait
  parler de renforcement.
\item On peut restreindre le sous-typage aux projections de types
  subsets avec l'\'egalit\'e syntaxique.
\item Je peux garder mes r\^egles de sous-typages, si elles sont syntax-directed!
\end{itemize}

Sous-typage \`a l'application et variable suffisante pour l'ad\'equation ?
On distingue les deux phases, pas de sous-typage \`a l'application.

V\'erifier Sub-{Left, Right}, l'application du sous-typage.

%%% Local Variables: 
%%% mode: latex
%%% TeX-master: "~/research/coq/papers/subset-typing"
%%% End: 

\subsection*{16 mars}
Il faut faire du sous-typage dans la sp�cification aussi:
$f : x : \subset{n}{`N}{0 \neq} "->" \subset{n}{`N}{x <}$.

Les deux phases:
\begin{description}
\item[Inf�rence] on donne les types impr�cis, ie: dans $x > n$, 
  $n "~>" `N$.
\item[Typage] on traverse la premi�re d�rivation de typage en ajoutant
  les coercions appropri�es, par exemple:
  $`G \type n : \subset{n}{`N}{0 \neq}$,  $\type_{inf} n "~>" `N$ est
  r�ecrit en: 
  $`G \type \pi_{1}~n : `N$.
\end{description}

\subsubsection*{Soir!}

Formalisation des trois jugements:
\begin{description}
\item[$\typed$] Typage d�claratif, syst�me ind�cidable, repr�sentant
  exactement ce qu'on veut ajouter comme fonctionnalit�.
\item[$\typei$] Version algorithmique, utilisant le dernier jugement
  pour r�aliser l'ad�quation avec la pr�sentation d�clarative.
\item[$\judgesubi$] ``Sous-typage'', sans obligations de preuves,
  d�cidable et � peu pr�s d�terministe.
\end{description}

Il faudra ensuite faire la traduction dans \Coq, avec de nouveaux
jugements r�ecrivant les d�rivations.

Propri�t�s � montrer:
\begin{itemize}
\item $`G \typed t : T "=>" `E U, `G \typei t : U `^ `G \judgesubi t : U \sub T$
\item $`G \typei t : T "=>" `G \typed t : T$
\end{itemize}

%%% Local Variables: 
%%% mode: latex
%%% TeX-master: "~/research/coq/papers/subset-typing"
%%% End: 

\subsection*{17 mars}
Preuves de substitutivit� \ref{substitutive-subtyping}, 
inversion \ref{inversion-subtyping}, admissibilit� de refl, trans dans
le sous-typage \ref{refl-trans-subtyping}, correction et compl�tude du typage.


%%% Local Variables: 
%%% mode: latex
%%% TeX-master: "~/research/coq/papers/subset-typing"
%%% End: 


\bibliography{subset-typing,../bib/bib-joehurd,../bib/pvs-bib}
\bibliographystyle{plain}

\renewcommand{\thefootnote}{}
\footnotetext{Ce rapport a �t� pr�par� sous \LaTeX~avec la fonte 
  \texttt{Computer Modern Bright}}

\end{document}
