\documentclass[a4paper,11pt]{article}
\usepackage[francais]{babel} 
\usepackage[latin1]{inputenc}  %% les accents dans le fichier.tex
\usepackage[T1]{fontenc}       %% Pour la c\'{e}sure des mots accentu\'{e}s
\usepackage{indentfirst}
\usepackage{a4}
\usepackage[dvips]{graphicx}
\usepackage{coqdoc}
\usepackage{amssymb}
\usepackage{amsmath}
\usepackage{amsthm}
\usepackage{amsfonts}
\usepackage{array}
\usepackage{myabbrevs}
\usepackage{bnf}
\usepackage{bussproofs}
\usepackage{hyperref}
\usepackage{fullpage}
%\usepackage{concmath}
\usepackage{cmbright}
\usepackage{fancyhdr}

\def\infvspace{2em}
% This is the "centered" symbol
\def\fCenter{~{\mbox{$\vdash$}}}
\def\seq{\fCenter}
% Optional to turn on the short abbreviations
\EnableBpAbbreviations

\newtheorem{lemma}{Lemme}[section]
\newtheorem{theo}[lemma]{Th�or�me}
\newtheorem{prop}[lemma]{Proposition}

\newcommand{\src}[1]{\texttt{#1}}
\newcommand{\srcm}[1]{\text{\texttt{#1}}}
%\newcommand{\Set}{\ensuremath{\text{\texttt{Set}}}}
\newcommand{\Prop}{\ensuremath{\text{\texttt{Prop}}}}

\newcommand{\rname}[1]{{\sc #1}}
\newcommand{\rulelabel}[1]{{\bf {\sc (#1)}}}
\newcommand{\UR}[2]{\RightLabel{\rulelabel{#1}}\UIC{#2}}
\newcommand{\URL}[2]{\LeftLabel{\rulelabel{#1}}\UIC{#2}}

\def\impsub{\rightslice}

\def\HOL{{\tt HOL}}
\def\Coq{{\tt Coq}}
\def\PVS{{\tt PVS}}

\makeatletter

\newcommand{\UAX}[4]{\AXC{#2}
  \@ifnotmtarg{#1}{\LeftLabel{\rulelabel{#1}}}
  \@ifnotmtarg{#4}{\RightLabel{#4}}
  \UIC{#3}}

\newcommand{\BAX}[5]{\AXC{#2}\AXC{#3}
  \@ifnotmtarg{#1}{\LeftLabel{\rulelabel{#1}}}
  \@ifnotmtarg{#5}{\RightLabel{#5}}
  \BIC{#4}}

\newcommand{\TAX}[6]{\AXC{#2}\AXC{#3}\AXC{#4}
  \@ifnotmtarg{#1}{\LeftLabel{\rulelabel{#1}}}
  \@ifnotmtarg{#6}{\RightLabel{#6}}
  \TIC{#5}}
\makeatother

\newcommand{\BR}[2]{\RightLabel{\rulelabel{#1}}\BIC{#2}}
\newcommand{\BRL}[2]{\LeftLabel{\rulelabel{#1}}\BIC{#2}}

\newcommand{\letml}{\textbf{let}~}
\newcommand{\inml}{~\textbf{in}~}
\newcommand{\ifml}{~\textbf{if}~}
\newcommand{\thenml}{~\textbf{then}~}
\newcommand{\elseml}{~\textbf{else}~}
\newcommand{\funml}{~\textbf{fun}~}

\newcommand{\eqbi}{`=_{\beta\iota}}

\def\judgewf{~{\mbox{$\vdash_{wf}$}}~}
\def\judgetype{~{\mbox{$\vdash$}}~}
\def\judgesub{~{\mbox{$\vdash_{\impsub}$}}~}
\def\judgerw{~{\mbox{$"~>"$}}~}

\newcommand{\elt}[4]{\text{elt}~#1~#2~#3~#4}
\renewcommand{\subset}[3]{\{ #1 : #2 `| #3~#1 \}}

\def\thetitle{Sous-typage par pr�dicats en Coq}

\pagestyle{fancy}
\fancyhead[RO,LE]{\thetitle}
\fancyfoot[C]{\thepage}
%\renewcommand{\headrulewidth}{0pt}

\newcommand{\matht}[1]{\text{{\tt #1}}}

\def\even{\matht{even}}
\def\odd{\matht{odd}}

\def\ps{\emph{predicate subtyping}}


\title{\thetitle}

\author{Matthieu Sozeau}

\date{\today}

\begin{document}

\maketitle

\begin{abstract}
  blabla 
\end{abstract}

\def\WfAtomRule{\rname{Wf-Atom}}
\def\WfVarRule{\rname{Wf-Var}}
\def\PropSetRule{\rname{PropSet}}
\def\AbsRule{\rname{Abs}}
\def\VarRule{\rname{Var}}
\def\ProdRule{\rname{Prod}}
\def\SubsetRule{\rname{Subset}}
\def\AppRule{\rname{App}}
\def\type{\judgetype}
\def\wf{\judgewf}

\begin{figure}[h]
  \begin{center}
    \def\fCenter{\judgerw}
    \UAX{Wf-Atom}{}{$\wf [] "~>" \wf []$}{} \DisplayProof
    
    \vspace{\infvspace}
    \UAX{Wf-Var}
    {$`G \type A : s \seq `G' \type A' : s$}
    {$\wf `G, x : A \seq \wf `G', x : A'$}
    {$s `: \{ \matht{Set}, \matht{Prop}, \matht{Type(i)} \}$} \DisplayProof

    \vspace{\infvspace}
    \UAX{PropSet}
    {$\wf `G \seq \wf `G'$}
    {$`G \type s : Type \seq `G' \type s : \matht{Type}$}
    {$s `: \{ \matht{Prop}, \matht{Set} \}$} 
    \DisplayProof
    
    \vspace{\infvspace}
    \BAX{Var}{$\wf `G \seq \wf `G'$}
    {$x : A `: `G \seq x : A' `: `G'$}
    {$`G \type x : A "~>" `G' \type x : A'$}{} \DisplayProof
    
    \vspace{\infvspace}
    \BAX{Prod}
    {$`G \type T : s1 \seq `G' \type T' : s1$}
    {$`G, x : T \type U : s2 \seq `G, x' : T' \type U' : s2$}
    {$`G \type \Pi x : T.U : s2 \seq `G' \type \Pi x' : T'.U' : s2$}
    {$s1, s2 `: \text{{\sc Sort}}$}
    \DisplayProof
    
    \vspace{\infvspace}
    \TAX{Abs}
    {$`G \type T : s1 \seq `G' \type T' : s1$}
    {$`G, x : T \type U : s2 \seq `G', x : T' \type U' : s2$}
    {$`G, x : T \type M : U \seq `G', x : T' \type M' : U'$}
    {$`G \type \lambda x : T. M : \Pi x : T.U
      \seq `G' \type \lambda x : T'. M' : \Pi x : T'.U'
      $}
    {$s1, s2 `: \matht{Sort}$} \DisplayProof

    \vspace{\infvspace}
    \BAX{App}
    {$`G \type f : \Pi x : V. W \seq `G' \type f' : \Pi x : V'. W'$}
    {$`G \type u : V \seq `G' \type u' : V'$}
    {$`G \type (f u) : W [ u / x ] \seq
      `G' \type (f u') : W' [ u' / x ]$}
    {$$} \DisplayProof
    
    \vspace{\infvspace}
    \BAX{Sigma}
    {$`G \type T : s1 \seq `G' \type T' : s1$}
    {$`G, x : T \type U : s2 \seq `G', x : T' \type U' : s2$}
    {$`G \type \Sigma x : T.U : s2 \seq `G' \type \Sigma x : T'.U' : s2$}
    {$s1, s2 `: \matht{Sort}$} \DisplayProof

    \vspace{\infvspace}
    \BAX{Sum}
    {$`G \type t : T \seq `G' \type t' : T'$}
    {$`G \type u : U[t/x] \seq `G' \type u' : U'[t'/x]$}
    {$`G \type (t, u) : \Sigma x : T.U \seq 
      `G' \type (t', u') : \Sigma x : T'.U'$}
    {} \DisplayProof
    
    \vspace{\infvspace}
    \BAX{LetSum}
    {$`G \type t : `S x : T. U \seq `G' \type t' : `S x : T'. U'$}
    {$`G, x : T, u : U \type v : V \seq `G', x : T', u : U' \type v' : V'$}
    {$`G \type \letml (x, u) = t \inml v : V \seq 
      `G' \type \letml (x, u) = t' \inml v' : V'$}
    {} \DP

    \vspace{\infvspace}
    \TAX{Subset}
    {$`G \type P : \Pi x : U. Prop \seq `G' \type P' : \Pi x : U'. Prop$}
    {$`G \type x : U \seq `G' \type x : U'$}
    {$`G \type p : P~x \seq `G' \type p' : P'~x$}
    {$`G \type x : \{ x : U `| P~x \} \seq 
      `G' \type \elt{U'}{P'}{x}{p'} : \subset{x}{U'}{P'}$}
    {$$} \DisplayProof

%%     \vspace{\infvspace}
%%     \BAX{LetSub}
%%     {$`G \type t : \subset{x}{T}{P} \seq `G' \type t' : \subset{x}{T'}{P'}$}
%%     {$`G, x : T, p : P~x \type v : V \seq `G', x : T', p : P'~x \type v' : V'$}
%%     {$`G \type \letml x = t \inml v : V \seq 
%%       `G' \type \letml (x, p) = t' \inml v' : V'$}
%%     {} \DP

    %\elt{U}{f}{x}{p}
%%     \vspace{\infvspace}
%%     \TAX{App}
%%     {$`G \type u : U$}
%%     {$`G \type f : \Pi x : V. W$}
%%     {$`G \judgesub u : U \impsub V "~>" p$}
%%     {$`G \type (f u) "~>" (f p) : W [ p / x ]$}
%%     {$$} \DisplayProof

  \end{center}
  \label{typing-rules}
  \caption{Typage du Calcul des Constructions avec sous-types pr�dicats}
\end{figure}

\UAX{[]}
{$`G \type t : T \seq `G' \type t' : T'$}
{$`G'$}
{}
\DP
\begin{lemma}[Substitutivit� de la traduction]
  Pour tout $U : s$, $t : T$,
  $U[t/x] "~>" Q$ ssi $U "~>" P$ et $P[[t]_{T'}/x] = Q$.
\end{lemma}

\begin{proof}
  Par induction sur $U$:

  \begin{description}
  \item[\VarRule:]  $U = x : T$, donc $Q = t'$ et $P = x : T'$ donc 
    $P[[t]_{T'}/x] = [t]_{T'}$.
    
  \item[\:]
    
  \end{description}
  
\end{proof}

\def\SubConvRule{\rname{Sub-Conv}}
\def\SubProdRule{\rname{Sub-Prod}}
\def\SubSigmaRule{\rname{Sub-Sigma}}
\def\SubLeftRule{\rname{Sub-Left}}
\def\SubRightRule{\rname{Sub-Right}}

\begin{figure}[h]
  \begin{center}
    \def\fCenter{\judgesub}
    
    \BAX{Sub-Conv}
    {$`G \judgetype x : T$}
    {$T \eqbi U$}
    {$`G \seq x : T \impsub U "~>" x$}
    {} 
    \DisplayProof
    \vspace{\infvspace}

    \BAX{Sub-Prod}
    {$`G \seq x : U \impsub T "~>" p$} %"<|-|>"
    {$`G, x : U \seq v[p/x] : V[p/x] \impsub W "~>" q$}
    {$`G \seq \lambda x : T. v : \Pi x : T.V \impsub \Pi x : U.W "~>" 
      \lambda x : U. q$}
    {}
    \DisplayProof
    
    \vspace{\infvspace}
    \BAX{Sub-Sigma}
    {$`G \seq x : T \impsub U "~>" p$}
    {$`G \seq v : V \impsub W[p/y] "~>" q$}
    {$`G \seq (x, v) : \Sigma x : T. V \impsub \Sigma y : U. W "~>"
      (p, q)$}
    {}
    \DisplayProof

    \vspace{\infvspace}
    \BAX{Sub-Left}
    {$`G \judgetype q : P~p$}
    {$`G, q : P~p \seq p : U \impsub V "~>" t$}
    {$`G \seq p : \subset{x}{U}{P} \impsub V "~>" t$}
    {}
    \DisplayProof

    \vspace{\infvspace}
    \BAX{Sub-Right}
    {$`G \seq p : A \impsub U "~>" p'$}
    {$`G \judgetype h : P~p'$}
    {$`G \seq p : A \impsub \subset{x}{U}{P} "~>" \elt{U}{P}{p'}{h}$}
    {$A$ atomique}
    \DisplayProof
    
  \end{center}
  \label{subtyping-rules}
  \caption{Sur-typage}
\end{figure}

\begin{figure}[h]
  \begin{center}
    \begin{tabular}{ccc}
      atom & $"=>"$ & atom \\
      $\subset{x}{`t}{P}$ & $"=>"$ & $\mu~`t$ \\
      $\Pi x : `t. `t'$ & $"=>"$ & $\Pi x : `t. \mu~`t'$ \\
      $\Sigma x : `t. `t'$ & $"=>"$ & $\Sigma x : \mu~`t. (\mu~`t') `/
      ((\pi~`t) x)$ 
    \end{tabular}    
  \end{center}
  \label{mu-def}
  \caption{$\mu$: Supertypes maximaux}
\end{figure}

\begin{figure}[h]
  \begin{center}
    \begin{tabular}{ccc}
      $s `: \text{atom}$ & $"=>"$ & $\lambda x : s. True$ \\
      $\subset{x}{`t}{P}$ & $"=>"$ & $\lambda x : \mu~`t. ((\pi~`t) x)
      `^ P[x/y]$ \\
      $\Pi x : `t. `t'$ & $"=>"$ & $\lambda x : (\Pi x : `t. \mu~`t'). 
      `A y : A, ((\pi~`t') (x~y))$ \\
      $\Sigma x : `t, `t'$ & $"=>"$ & $\lambda x : 
      (\Sigma x : \mu~`t, (\mu~`t') `/ ((\pi~`t) x)). 
      \letml (a, b) = x \inml ((\pi~`t) a) `^ ((\pi~`t') b)[a/y]$ 
    \end{tabular}
  \end{center}
  \label{pi-def}
  \caption{$\pi$: Collection des contraintes}
\end{figure}

\clearpage

\begin{lemma}[Pr�servation de l'�quivalence par $\mu$]
  \label{mu-equiv-preserve}
  Si $U \eqbi V$ alors $\mu~U \eqbi \mu~V$.
\end{lemma}

\begin{proof}
  Par induction sur $(U, V)$ : si $U `= \subset{x}{A}{P}$, 
  par hypoth�se, $V \eqbi \subset{x}{A'}{P'}$ avec $A \eqbi A'$ et $P
  \eqbi P'$. Par induction $A \eqbi A' "=>" \mu~A \eqbi \mu~A'$. Sinon,
  $U = \mu~U$ et $V = \mu~V$.
\end{proof}

\begin{prop}[Equivalence des supertypes maximaux]
  Si $`G \judgesub U \impsub V$ alors 
  $`G \judgetype \mu~U \eqbi \mu~V$ et 
  $`G \judgetype (\pi~U)~(\mu~U) "->" (\pi~V)~(\mu~V)$.
\end{prop}
\begin{proof}
  Par induction sur la d�rivation de typage:
  
  \begin{description}
  \item[\SubConvRule:] $U \eqbi V$, donc par le lemme
    \ref{mu-equiv-preserve}, $\mu~U \eqbi \mu~V$.
    
  \item[\SubProdRule:]
    
  \item[\SubSigmaRule:] Par induction, $\mu~T \eqbi \mu~U$ et     
    $\mu~V \eqbi \mu~W[p/y]$. Donc 
    \begin{eqnarray*}
      \mu~\Sigma x : T.V & = & \Sigma x : \mu~T. \mu~V `/ ((\pi T) x) \\
      & \eqbi & \Sigma x : \mu~U. \mu~W[p/y] `/ ((\pi T) x) \\
      & & \{ p : T "->" U, \pi~T "->" \pi~U \} \\
      & = & \mu~\Sigma x : U. W[p/y]
    \end{eqnarray*}
    
  \item[\SubLeftRule:]
    
  \item[\SubRightRule:]    
  \end{description}
  
\end{proof}


\section*{Journal}

\subsection*{8 mars}
Nouvelle r�gle de produit fonctionel avec contravariance bien typ�e,
produit d�pendant (\SubSigmaRule) covariant.
Un exemple de produit avec contravariance se trouve dans \cite{cal00coherence}, p. 6.
Exemple d'utilisation int�ressante:

\BAX{}
{$`G \judgetypei f : \{ \phi : \even "->" `N `| `A x : `N, \phi~x ``<= x \}$}
{$`G \judgetypei g : `N "->" `N := \matht{pred}$}
{$`G \judgesubi g ``<= f$}
{}
\DP

Le sous-typage avec coercions: Luo, Callaghan, Sa�bi \cite{saibi97inheritance}... 
\begin{itemize}
\item Uniformit� du sous-typage: ne d�pend pas du contexte.
\item Coercions d�clar�es dans l'environement (ex: Coq).
\end{itemize}

Dans HOL, Joe Hurd simule le \emph{predicate subtyping} � la PVS avec
des \emph{predicate sets} \cite{hurd2001a}. Technique adaptable � Coq ?

%%% Local Variables: 
%%% mode: latex
%%% TeX-master: "~/research/coq/papers/subset-typing"
%%% End: 

\subsection*{9 mars}
Le \ps{} dans HOL n'est pas correct, il peut �tre subverti ais�ment �
cause d'une sorte de covariance des domaines de fonctions et de la
fonction de suppression des predicats (similaire � $\mu$):
$inv : `R^{\neq 0} "->" `R `: `R "->" `R$. Le d�veloppement dans \HOL{}
est fait directement dans le language, et il est argu� qu'il n'est pas
possible de le faire correctement � cause de l'exemple
pr�c�dent. L'algorithme de sous-typage est simplement une g�n�ration de
tout les sous-types possibles � partir d'un ensemble de r�gles de
sous-typage pour les constantes et constructions logiques ou fonctionelles.

Trouv� un article \cite{stumpsubset} sur les types sous-ensembles dans PF, logique d'ordre sup�rieur
avec fonctions partielles... Permet de traiter le cas
$\ifml 1 / i > 0 \thenml i \neq 0 \elseml `_$, qui g�n�re une obligation de
preuve $i \neq 0$ dans \PVS{}. Evidemment on ne risque pas de pouvoir
typer ce code en \Coq{} lorsque $`/ : `Z "->" \{ x : `Z `| x \neq 0 \} "->" `Z$ mais
certaines id�es peuvent �tre int�ressantes. 

%%% Local Variables: 
%%% mode: latex
%%% TeX-master: "~/research/coq/papers/subset-typing"
%%% End: 

\subsection*{11 mars}
Typage de la division euclidienne:
$\matht{div} : `A a : `N, `A b : \{ x : `N `| x \neq 0 \}, `S q : `N, `S r : \{ n : `N
`| n < b \}, a = bq + r := \funml a~b "=>" \ifml a < b \thenml (0, a) \elseml \letml (q, r) =
\matht{div}~(a - b)~b \inml (q + 1, r)$.

Soit $`t_{div} = `A a : `N, `A b : \{ x : `N `| x \neq 0 \}, `E q : `N, `E r : \{ n : `N
`| n < b \}, a = bq + r$ et $`G = \matht{div} : `t_{div}$:



\AXC{$1$}
\AXC{$2$}
\BIC{$`G, a : `N, b : `N^{*} \seq \ifml a < b \thenml (0, a) \elseml \letml (q, r) =
  \matht{div}~(a - b)~b \inml (q + 1, r) : `E q \dots$}
\doubleLine
\UIC{$`G \seq \funml a~b "=>" \ifml a < b \thenml (0, a) \elseml \letml (q, r) =
  \matht{div}~(a - b)~b \inml (q + 1, r) : `t_{div}$}
\DisplayProof

Soit $`G_{if} = `G, a : `N, b : `N^{*}, a ``/< b$.
\begin{prooftree}
  \AXC{$`G_{if} \seq b : `N^{*}$}
  \RightLabel{$`b = `N^{*}$}
  \UIC{$`G_{if} \seq b : `b$}

  \AXC{$`G_{if} \seq b : `g$}
  \AXC{$`G_{if} \seq (- a) : `g "->" `b''$}
  \AXC{$`G_{if} \judgetypea $}
  \TIC{$`G_{if} \seq (- a) b : `b''$}
  \AXC{$`G_{if} \seq \matht{div} : `b''' "->" `b' "->" `a$}
  \AXC{$`G_{if} \judgesubd a - b : `b'' ``<= `b''' "~>" t$}
  \UIC{$`G_{if} \seq \matht{div}~(a - b) "~>" \matht{div}~t : `b' "->" `a$}

  \TIC{$`G_{if} \seq \matht{div}~(a - b) : `b' "->" `a$}

  \AXC{$`G_{if} \judgesubd b : `b ``<= `b' "~>" p $}
  \TIC{$`G_{if} \seq \matht{div}~(a - b)~b "~>" \matht{div}~(a - b)~p : `a$}
  
  
  \AXC{$`G_{if}, (q, r) : `a \seq (q + 1, r) : `E q \dots$}
  \RightLabel{1}
  \BIC{$`G_{if} \seq \letml (q, r) = \matht{div}~(a - b)~b \inml (q + 1, r) : `E q \dots$}
\end{prooftree}  

\RightLabel{2}
\AXC{$`G, a : `N, b : `N^{*}, a < b \seq (0, a) : `E q \dots$}
\DisplayProof


%%% Local Variables: 
%%% mode: latex
%%% TeX-master: "~/research/coq/papers/subset-typing"
%%% End: 

\subsection*{14 mars}
R�gles d'introduction, d'�limination et de formation pour $\Pi$, $\Sigma$, sous-types
pr�dicats. Nouveau jugement de typage par r�ecriture. Nombreuses
questions � discuter avec Christine. Beaucoup de bruit dans le buro !


%%% Local Variables: 
%%% mode: latex
%%% TeX-master: "~/research/coq/papers/subset-typing"
%%% End: 

\subsection*{15 mars}
Distinction inf\'erence et typage. 
\begin{description}
\item[Inf\'erence ($"~>"$)] \`a la ML, on v\'erifie: $`G \typea p "~>" T "=>"
  `G \typed p : T$. 
\item[Typage ($:$)]. On a $`G \typed p : T$, on veut $`E U, `G \typea p
  "~>" U `^ `G \typea U "~>" T$. On utilise le sous-typage g\'en\`erant les
  obligations de preuve.
\end{description}

Eliminer \rname{LetSub}, inutilisable en pratique.

Quelques points \`a m\'editer:
\begin{itemize}
\item $`O \type 3 : `N "~>" `O \type 3 : `N^*$ ? D\'ependance envers le
  terme pour le sous-typage. De m\^eme, $2 : `N ``<= `N^*$, on devrait
  parler de renforcement.
\item On peut restreindre le sous-typage aux projections de types
  subsets avec l'\'egalit\'e syntaxique.
\item Je peux garder mes r\^egles de sous-typages, si elles sont syntax-directed!
\end{itemize}

Sous-typage \`a l'application et variable suffisante pour l'ad\'equation ?
On distingue les deux phases, pas de sous-typage \`a l'application.

V\'erifier Sub-{Left, Right}, l'application du sous-typage.

%%% Local Variables: 
%%% mode: latex
%%% TeX-master: "~/research/coq/papers/subset-typing"
%%% LaTeX-command: "TEXINPUTS=\"style:$TEXINPUTS\" latex"
%%% End: 

\subsection*{16 mars}
Il faut faire du sous-typage dans la sp�cification aussi:
$f : x : \subset{n}{`N}{0 \neq} "->" \subset{n}{`N}{x <}$.

Les deux phases:
\begin{description}
\item[Inf�rence] on donne les types impr�cis, ie: dans $x > n$, 
  $n "~>" `N$.
\item[Typage] on traverse la premi�re d�rivation de typage en ajoutant
  les coercions appropri�es, par exemple:
  $`G \type n : \subset{n}{`N}{0 \neq}$,  $\type_{inf} n "~>" `N$ est
  r�ecrit en: 
  $`G \type \pi_{1}~n : `N$.
\end{description}

\subsubsection*{Soir!}

Formalisation des trois jugements:
\begin{description}
\item[$\typed$] Typage d�claratif, syst�me ind�cidable, repr�sentant
  exactement ce qu'on veut ajouter comme fonctionnalit�.
\item[$\typei$] Version algorithmique, utilisant le dernier jugement
  pour r�aliser l'ad�quation avec la pr�sentation d�clarative.
\item[$\judgesubi$] ``Sous-typage'', sans obligations de preuves,
  d�cidable et � peu pr�s d�terministe.
\end{description}

Il faudra ensuite faire la traduction dans \Coq, avec de nouveaux
jugements r�ecrivant les d�rivations.

Propri�t�s � montrer:
\begin{itemize}
\item $`G \typed t : T "=>" `E U, `G \typei t : U `^ `G \judgesubi t : U \sub T$
\item $`G \typei t : T "=>" `G \typed t : T$
\end{itemize}

%%% Local Variables: 
%%% mode: latex
%%% TeX-master: "~/research/coq/papers/subset-typing"
%%% End: 


\bibliography{subset-typing,../bib/bib-joehurd}
\bibliographystyle{plain}

\renewcommand{\thefootnote}{}
\footnotetext{Ce rapport a �t� pr�par� sous \LaTeX~avec la fonte 
  \texttt{Computer Modern Bright}}

\end{document}
