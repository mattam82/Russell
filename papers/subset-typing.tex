\documentclass[a4paper,11pt]{article}
\usepackage[francais]{babel} 
\usepackage[latin1]{inputenc}  %% les accents dans le fichier.tex
\usepackage[T1]{fontenc}       %% Pour la c\'{e}sure des mots accentu\'{e}s
\usepackage{indentfirst}
\usepackage{a4}
\usepackage[dvips]{graphicx}
\usepackage{coqdoc}
\usepackage{amssymb}
\usepackage{amsmath}
\usepackage{amsthm}
\usepackage{amsfonts}
\usepackage{array}
\usepackage{myabbrevs}
\usepackage{bnf}
\usepackage{bussproofs}
\usepackage{hyperref}
%\usepackage{concmath}
\usepackage{cmbright}
\usepackage{fancyhdr}

\def\infvspace{3em}
% This is the "centered" symbol
\def\fCenter{~{\mbox{$\vdash$}}}
\def\seq{\fCenter}
% Optional to turn on the short abbreviations
\EnableBpAbbreviations

\newtheorem{lemma}{Lemme}[section]
\newtheorem{theo}[lemma]{Th�or�me}
\newtheorem{prop}[lemma]{Proposition}

\newcommand{\src}[1]{\texttt{#1}}
\newcommand{\srcm}[1]{\text{\texttt{#1}}}
%\newcommand{\Set}{\ensuremath{\text{\texttt{Set}}}}
\newcommand{\Prop}{\ensuremath{\text{\texttt{Prop}}}}

\newcommand{\rname}[1]{{\sc #1}}
\newcommand{\rulelabel}[1]{{\bf {\sc (#1)}}}
\newcommand{\UR}[2]{\RightLabel{\rulelabel{#1}}\UIC{#2}}
\newcommand{\URL}[2]{\LeftLabel{\rulelabel{#1}}\UIC{#2}}

\def\HOL{{\tt HOL}}
\def\Coq{{\tt Coq}}
\def\PVS{{\tt PVS}}

\makeatletter

\newcommand{\UAX}[4]{\AXC{#2}
  \@ifnotmtarg{#1}{\LeftLabel{\rulelabel{#1}}}
  \@ifnotmtarg{#4}{\RightLabel{#4}}
  \UIC{#3}}

\newcommand{\BAX}[5]{\AXC{#2}\AXC{#3}
  \@ifnotmtarg{#1}{\LeftLabel{\rulelabel{#1}}}
  \@ifnotmtarg{#5}{\RightLabel{#5}}
  \BIC{#4}}

\newcommand{\TAX}[6]{\AXC{#2}\AXC{#3}\AXC{#4}
  \@ifnotmtarg{#1}{\LeftLabel{\rulelabel{#1}}}
  \@ifnotmtarg{#6}{\RightLabel{#6}}
  \TIC{#5}}
\makeatother

\newcommand{\BR}[2]{\RightLabel{\rulelabel{#1}}\BIC{#2}}
\newcommand{\BRL}[2]{\LeftLabel{\rulelabel{#1}}\BIC{#2}}

\newcommand{\letml}{\textbf{let}~}
\newcommand{\inml}{~\textbf{in}~}
\newcommand{\ifml}{~\textbf{if}~}
\newcommand{\thenml}{~\textbf{then}~}
\newcommand{\elseml}{~\textbf{else}~}

\newcommand{\eqbi}{`=_{\beta\iota}}

\def\judgewf{~{\mbox{$\vdash_{wf}$}}~}
\def\judgetype{~{\mbox{$\vdash$}}~}
\def\judgesub{~{\mbox{$\vdash_{``<=}$}}~}

\newcommand{\elt}[4]{\text{elt}~#1~#2~#3~#4}
\renewcommand{\subset}[3]{\{ #1 : #2 `| #3~#1 \}}

\def\thetitle{Sous-typage par pr�dicats en Coq}

\pagestyle{fancy}
\fancyhead[RO,LE]{\thetitle}
\fancyfoot[C]{\thepage}
%\renewcommand{\headrulewidth}{0pt}

\title{\thetitle}

\author{Matthieu Sozeau}

\date{\today}

\begin{document}

\maketitle

\begin{abstract}
  blabla 
\end{abstract}

\def\WfAtomRule{\rname{Wf-Atom}}
\def\WfVarRule{\rname{Wf-Var}}
\def\PropSetRule{\rname{PropSet}}
\def\AbsRule{\rname{Abs}}
\def\VarRule{\rname{Var}}
\def\ProdRule{\rname{Prod}}
\def\SubsetRule{\rname{Subset}}
\def\AppRule{\rname{App}}
  
\begin{figure}[h]
  \begin{center}
    \def\fCenter{\judgewf}
    \UAX{Wf-Atom}{}{$\seq []$}{} \DisplayProof
    \UAX{Wf-Var}{$`G \judgetype A : s$}{$\seq `G, x : A$}
    {$s `: \{ Set, Prop, Type \}$} \DisplayProof
    \vspace{\infvspace}
    \def\fCenter{\judgetype}
    \UAX{PropSet}{$\judgewf `G$}{$`G \seq s : Type$}
    {$s `: \{ Prop, Set \}$} \DisplayProof
    \BAX{Var}{$\judgewf `G$}{$x : A `: `G$}{$`G \seq x : A$}{} \DisplayProof
    \vspace{\infvspace}
    \TAX{Abs}{$`G \seq T : s1$}{$`G, x : T \seq M : U$}
    {$`G, x : T \seq U : s2$}
    {$`G \seq \lambda x : T. M : \Pi x : T.U$}
    {$s1, s2 `: \text{{\sc Sort}}$} \DisplayProof
    \vspace{\infvspace}
    \BAX{Prod}{$`G \seq T : s1$}{$`G, x : T \seq U : s2$}
    {$`G \seq \Pi x : T.U : s2$}
    {$s1, s2 `: \text{{\sc Sort}}$} \DisplayProof
    
    \vspace{\infvspace}
    \TAX{Subset}{$`G \seq f : \Pi x : U. Prop$}
    {$`G \seq x : U$}{$`G \seq p : P~x$}
    {$`G \seq \elt{U}{f}{x}{p} : \{ x : U `| P~x \}$}{$$} \DisplayProof
    
    \vspace{\infvspace}
    \TAX{App}
    {$`G \seq u : U$}
    {$`G \seq f : \Pi x : V. W$}
    {$`G \judgesub u : U ``<= V "~>" p$}
    {$`G \seq (f u) "~>" (f p) : W [ p / x ]$}
    {$$} \DisplayProof

  \end{center}
  \label{typing-rules}
  \caption{Typage du Calcul des Constructions avec sous-types pr�dicats}
\end{figure}

\def\SubConvRule{\rname{Sub-Conv}}
\def\SubProdRule{\rname{Sub-Prod}}
\def\SubSigmaRule{\rname{Sub-Sigma}}
\def\SubLeftRule{\rname{Sub-Left}}
\def\SubRightRule{\rname{Sub-Right}}

\begin{figure}[h]
  \begin{center}
    \def\fCenter{\judgesub}
    
    \BAX{Sub-Conv}
    {$`G \judgetype x : T$}
    {$T \eqbi U$}
    {$`G \seq x : T ``<= U "~>" x$}
    {} 
    \DisplayProof
    \vspace{\infvspace}

    \BAX{Sub-Prod}
    {$`G \seq x : U ``<= T "~>" p$} %"<|-|>"
    {$`G, x : U \seq v[p/x] : V[p/x] ``<= W "~>" q$}
    {$`G \seq \lambda x : T. v : \Pi x : T.V ``<= \Pi x : U.W "~>" 
      \lambda x : U. q$}
    {}
    \DisplayProof
    
    \vspace{\infvspace}
    \BAX{Sub-Sigma}
    {$`G \seq x : T ``<= U "~>" p$}
    {$`G \seq v : V ``<= W[p/y] "~>" q$}
    {$`G \seq (x, v) : \Sigma x : T. V ``<= \Sigma y : U. W "~>"
      (p, q)$}
    {}
    \DisplayProof

    \vspace{\infvspace}
    \BAX{Sub-Left}
    {$`G \judgetype q : P~p$}
    {$`G, q : P~p \seq p : U ``<= V "~>" t$}
    {$`G \seq p : \subset{x}{U}{P} ``<= V "~>" t$}
    {}
    \DisplayProof

    \vspace{\infvspace}
    \BAX{Sub-Right}
    {$`G \seq p : A ``<= U "~>" p'$}
    {$`G \judgetype h : P~p'$}
    {$`G \seq p : A ``<= \subset{x}{U}{P} "~>" \elt{U}{P}{p'}{h}$}
    {$A$ atomique}
    \DisplayProof
    
  \end{center}
  \label{subtyping-rules}
  \caption{Sous-typage}
\end{figure}

\begin{figure}[h]
  \begin{center}
    \begin{tabular}{ccc}
      atom & $"=>"$ & atom \\
      $\subset{x}{`t}{P}$ & $"=>"$ & $\mu~`t$ \\
      $\Pi x : `t. `t'$ & $"=>"$ & $\Pi x : `t. \mu~`t'$ \\
      $\Sigma x : `t. `t'$ & $"=>"$ & $\Sigma x : \mu~`t. (\mu~`t') `/
      ((\pi~`t) x)$ 
    \end{tabular}    
  \end{center}
  \label{mu-def}
  \caption{$\mu$: Supertypes maximaux}
\end{figure}

\begin{figure}[h]
  \begin{center}
    \begin{tabular}{ccc}
      $s `: \text{atom}$ & $"=>"$ & $\lambda x : s. True$ \\
      $\subset{x}{`t}{P}$ & $"=>"$ & $\lambda x : \mu~`t. ((\pi~`t) x)
      `^ P[x/y]$ \\
      $\Pi x : `t. `t'$ & $"=>"$ & $\lambda x : (\Pi x : `t. \mu~`t'). 
      `A y : A, ((\pi~`t') (x~y))$ \\
      $\Sigma x : `t, `t'$ & $"=>"$ & $\lambda x : 
      (\Sigma x : \mu~`t, (\mu~`t') `/ ((\pi~`t) x)). 
      \letml (a, b) = x \inml ((\pi~`t) a) `^ ((\pi~`t') b)[a/y]$ 
    \end{tabular}
  \end{center}
  \label{pi-def}
  \caption{$\pi$: Collection des contraintes}
\end{figure}

\clearpage

\begin{lemma}[Pr�servation de l'�quivalence par $\mu$]
  \label{mu-equiv-preserve}
  Si $U \eqbi V$ alors $\mu~U \eqbi \mu~V$.
\end{lemma}

\begin{proof}
  Par induction sur $(U, V)$ : si $U `= \subset{x}{A}{P}$, 
  par hypoth�se, $V \eqbi \subset{x}{A'}{P'}$ avec $A \eqbi A'$ et $P
  \eqbi P'$. Par induction $A \eqbi A' "=>" \mu~A \eqbi \mu~A'$. Sinon,
  $U = \mu~U$ et $V = \mu~V$.
\end{proof}

\begin{prop}[Equivalence des supertypes maximaux]
  Si $`G \judgesub U ``<= V$ alors 
  $`G \judgetype \mu~U \eqbi \mu~V$ et 
  $`G \judgetype (\pi~U)~(\mu~U) "->" (\pi~V)~(\mu~V)$.
\end{prop}
\begin{proof}
  Par induction sur la d�rivation de typage:
  
  \begin{description}
  \item[\SubConvRule:] $U \eqbi V$, donc par le lemme
    \ref{mu-equiv-preserve}, $\mu~U \eqbi \mu~V$.
    
  \item[\SubProdRule:]
    
  \item[\SubSigmaRule:] Par induction, $\mu~T \eqbi \mu~U$ et     
    $\mu~V \eqbi \mu~W[p/y]$. Donc 
    \begin{eqnarray*}
      \mu~\Sigma x : T.V & = & \Sigma x : \mu~T. \mu~V `/ ((\pi T) x) \\
      & \eqbi & \Sigma x : \mu~U. \mu~W[p/y] `/ ((\pi T) x) \\
      & & \{ p : T "->" U, \pi~T "->" \pi~U \} \\
      & = & \mu~\Sigma x : U. W[p/y]
    \end{eqnarray*}
    
  \item[\SubLeftRule:]
    
  \item[\SubRightRule:]    
  \end{description}
  
\end{proof}


\section*{Journal}

\subsection*{8 mars}
Nouvelle r�gle de produit fonctionel avec contravariance bien typ�e,
produit d�pendant (\SubSigmaRule) covariant.
Un exemple de produit avec contravariance se trouve dans \cite{cal00coherence}, p. 6.
Exemple d'utilisation int�ressante:

\newcommand{\matht}[1]{\text{{\tt #1}}}

\def\even{\matht{even}}
\def\odd{\matht{odd}}

\BAX{}
{$`G \judgetype f : \{ \phi : \even "->" `N `| `A x : `N, \phi~x ``<= x \}$}
{$`G \judgetype g : `N "->" `N := \matht{pred}$}
{$`G \judgesub g ``<= f$}
{}
\DP

Le sous-typage avec coercions: Luo, Callaghan, Sa�bi \cite{saibi97inheritance}... 
\begin{itemize}
\item Uniformit� du sous-typage: ne d�pend pas du contexte.
\item Coercions d�clar�es dans l'environement (ex: Coq).
\end{itemize}

Dans HOL, Joe Hurd simule le \emph{predicate subtyping} � la PVS avec
des \emph{predicate sets} \cite{hurd2001a}. Technique adaptable � Coq ?

\def\ps{\emph{predicate subtyping}}

\subsection*{9 mars}
Le \ps{} dans HOL n'est pas correct, il peut �tre subverti ais�ment �
cause d'une sorte de covariance des domaines de fonctions et de la
fonction de suppression des predicats (similaire � $\mu$):
$inv : `R^{\neq 0} "->" `R `: `R "->" `R$. Le d�vellopement dans \HOL{}
est fait directement dans le language, et il est argu� qu'il n'est pas
possible de le faire correctement � cause de l'exemple
pr�c�dent. L'algorithme de sous-typage est simplement une g�n�ration de
tout les sous-types possibles � partir d'un ensemble de r�gles de
sous-typage pour les constantes et constructions logiques ou fonctionelles.

Trouv� un article \cite{stumpsubset} sur les types sous-ensembles dans PF, logique d'ordre sup�rieur
avec fonctions partielles... Permet de traiter le cas
$\ifml 1 / i > 0 \thenml i \neq 0 \elseml `_$, qui g�n�re une obligation de
preuve $i \neq 0$ dans \PVS{}. Evidemment on ne risque pas de pouvoir
typer ce code en \Coq{} lorsque $`/ : `Z "->" \{ x : `Z `| x \neq 0 \} "->" `Z$ mais
certaines id�es peuvent �tre int�ressantes. 


\bibliography{subset-typing,../bib/bib-joehurd}
\bibliographystyle{plain}

\renewcommand{\thefootnote}{}
\footnotetext{Ce rapport a �t� pr�par� sous \LaTeX~avec la fonte 
  \texttt{Computer Modern Bright}}

\end{document}
