\documentclass[a4paper,11pt]{article}
\usepackage[francais]{babel} 
\usepackage[latin1]{inputenc}  %% les accents dans le fichier.tex
\usepackage[T1]{fontenc}       %% Pour la c\'{e}sure des mots accentu\'{e}s
\usepackage{indentfirst}
\usepackage{a4}
\usepackage[dvips]{graphicx}
\usepackage{coqdoc}
\usepackage{amssymb}
\usepackage{amsmath}
\usepackage{amsthm}
\usepackage{amsfonts}
\usepackage{array}
\usepackage{myabbrevs}
\usepackage{bnf}
\usepackage{bussproofs}
\usepackage{hyperref}
\usepackage{fullpage}
%\usepackage{concmath}
\usepackage{cmbright}
\usepackage{fancyhdr}
\usepackage{ifthen}

\def\infvspace{2em}
% This is the "centered" symbol
\def\fCenter{\vdash}
\def\seq{\fCenter}
% Optional to turn on the short abbreviations
\EnableBpAbbreviations

\newtheorem{lemma}{Lemme}[section]
\newtheorem{theorem}[lemma]{Th�or�me}
\newtheorem{proposition}[lemma]{Proposition}
\newtheorem{definition}[lemma]{D�finition}

\newtheorem{techremark}[]{Remarque technique}

\newcommand{\src}[1]{\texttt{#1}}
\newcommand{\srcm}[1]{\text{\texttt{#1}}}
%\newcommand{\Set}{\ensuremath{\text{\texttt{Set}}}}
\newcommand{\Prop}{\ensuremath{\text{\texttt{Prop}}}}

\newcommand{\rname}[1]{{\bf #1}}
\newcommand{\rulelabel}[1]{{\bf (#1)}}
\newcommand{\UR}[2]{\RightLabel{\rulelabel{#1}}\UIC{#2}}
\newcommand{\URL}[2]{\LeftLabel{\rulelabel{#1}}\UIC{#2}}

\def\impsub{\rightslice}

\newboolean{displayLabels}
\setboolean{displayLabels}{true}

\def\HOL{{\tt HOL}}
\def\Coq{{\tt Coq}}
\def\PVS{{\tt PVS}}

\makeatletter

\newcommand{\LeftRuleLabel}[1]{
  \@ifnotmtarg{#1}{\ifthenelse{\boolean{displayLabels}}{\LeftLabel{\rulelabel{#1}}}{}}
}

\newcommand{\UAX}[4]{\AXC{#2}
  \LeftRuleLabel{#1}
  \@ifnotmtarg{#4}{\RightLabel{#4}}
  \UIC{#3}}

\newcommand{\BAX}[5]{\AXC{#2}\AXC{#3}
  \LeftRuleLabel{#1}
  \@ifnotmtarg{#5}{\RightLabel{#5}}
  \BIC{#4}}

\newcommand{\TAX}[6]{\AXC{#2}\AXC{#3}\AXC{#4}
  \LeftRuleLabel{#1}
  \@ifnotmtarg{#6}{\RightLabel{#6}}
  \TIC{#5}}

\newcommand{\QAX}[7]{\AXC{#2}\noLine\UIC{#3}\AXC{#4}\AXC{#5}
  \LeftRuleLabel{#1}
  \@ifnotmtarg{#7}{\RightLabel{#7}}
  \TIC{#6}}

\makeatother

\newcommand{\BR}[2]{\RightLabel{\rulelabel{#1}}\BIC{#2}}
\newcommand{\BRL}[2]{\LeftLabel{\rulelabel{#1}}\BIC{#2}}

\newcommand{\letml}{\textbf{let}~}
\newcommand{\inml}{~\textbf{in}~}
\newcommand{\ifml}{~\textbf{if}~}
\newcommand{\thenml}{~\textbf{then}~}
\newcommand{\elseml}{~\textbf{else}~}
\newcommand{\funml}{~\textbf{fun}~}

\newcommand{\eqbi}{`=_{\beta}}

\def\judgewf{\vdash_{wf}}
\def\judgetyped{\vdash}
\def\judgetypea{\vdash_{\bullet}}
\def\judgetypei{\vdash_{\Box}}
\def\typed{\judgetyped}
\def\typea{\judgetypea}
\def\typei{\judgetypei}
\def\type{\typed}
\def\judgesubd{\vdash}
\def\judgesuba{\vdash_{\bullet}}
\def\judgesubi{\vdash_{\Box}}
\def\subtd{\judgesubd}
\def\subta{\judgesuba}
\def\subti{\judgesubi}
\def\subt{\subtd}
\def\subd{\rightslice}
\def\suba{\rightslice_{\bullet}}
\def\subi{\rightslice_{\Box}}
\def\sub{\subd}

\def\judgerw{~{\mbox{$"~>"$}}~}
\def\wf{\judgewf}

\newcommand{\elt}[4]{\text{elt}~#1~#2~#3~#4}
\renewcommand{\subset}[3]{\{ #1 : #2 `| #3 \}}

\def\thetitle{Sous-typage par pr�dicats en Coq}

\pagestyle{plain}
\fancyhead[RO,LE]{\thetitle}
\fancyfoot[C]{\thepage}
%\renewcommand{\headrulewidth}{0pt}

\newcommand{\matht}[1]{\text{{\tt #1}}}

\def\even{\matht{even}}
\def\odd{\matht{odd}}
\def\Set{\matht{Set}}
\def\Prop{\matht{Prop}}
\def\Type{\matht{Type}}
\def\ps{\emph{predicate subtyping}}

\title{\thetitle}

\author{Matthieu Sozeau}

\date{\today}

\begin{document}

\maketitle

\begin{abstract}
  blabla 
\end{abstract}

\section*{Introduction}
Le \ps{} impl�ment� dans PVS (\cite{PVS-Semantics:TR,Shankar&Owre:WADT99,Rushby98:TSE}).


\def\WfAtomRule{\rname{Wf-Atom}}
\def\WfVarRule{\rname{Wf-Var}}
\def\PropSetRule{\rname{PropSet}}
\def\VarRule{\rname{Var}}
\def\ProdRule{\rname{Prod}}
\def\AbsRule{\rname{Abs}}
\def\AppRule{\rname{App}}
\def\LetInRule{\rname{LetIn}}
\def\SigmaRule{\rname{Sigma}}
\def\SumRule{\rname{Sum}}
\def\SumInfRule{\rname{SumInf}}
\def\SumDepRule{\rname{SumDep}}
\def\LetSumRule{\rname{LetSum}}
\def\SubsetRule{\rname{Subset}}
\def\LetSubRule{\rname{LetSub}}
\def\SubsumRule{\rname{Subsumption}}
\def\ConvRule{\rname{Conv}}


\def\SubReflRule{\rname{Sub-Refl}}
\def\SubTransRule{\rname{Sub-Trans}} 
\def\SubConvRule{\rname{Sub-Conv}}
\def\SubProdRule{\rname{Sub-Prod}}
\def\SubSigmaRule{\rname{Sub-Sigma}}
\def\SubLeftRule{\rname{Sub-Left}}
\def\SubRightRule{\rname{Sub-Right}}
\def\SubProofRule{\rname{Sub-Proof}}
\def\SubSubRule{\rname{Sub-Subset}}
\def\SubTransRule{\rname{Sub-Trans}}

\newcommand{\WfEmptyFull}[1]{
  \UAX{WfEmpty}
  {}
  {$\seq []$}
  {}
}  
\newcommand{\WfEmpty}[1][\Gamma]{\WfEmptyFull{#1}}
  
\newcommand{\WfVarFull}[4]{
  \UAX{WfVar}
  {$\tgen{#1}{#2}{#3}$}
  {$\wf #1, #4 : #2$}
  {$#3 `: \setproptype{} `^{} #4 `; #1$}
}
\newcommand{\WfVar}[1][\Gamma]{\WfVarFull{#1}{A}{s}{x}}

\newcommand{\PropSetFull}[2]{
  \UAX{PropSet}
  {$\wf #1$}
  {$\tgen{#1}{#2}{\Type}$}
  {$#2 `: \setprop$} 
}
\newcommand{\PropSet}[1][\Gamma]{\PropSetFull{#1}{s}}


\newcommand{\TypeTypeFull}[1]{
  \UAX{Type}
  {$\wf #1$}
  {$\tgen{#1}{\Type(i)}{\Type(i + 1)}$}
  {$i `: `N$}
}
\newcommand{\TypeType}[1][\Gamma]{\TypeTypeFull{#1}}

\newcommand{\VarFull}[3]{
  \BAX{Var}
  {$\wf #1$}
  {$#2 : #3 `: #1$}
  {$\tgen{#1}{#2}{#3}$}
  {}
}
\newcommand{\Var}[1][\Gamma]{\VarFull{#1}{x}{A}} 

\newcommand{\ProdFull}[7]{
  \BAX{Prod}
  {$\tgen{#1}{#2}{#3}$}
  {$\tgen{#1, #4 : #2}{#5}{#6}$}
  {$\tgen{#1}{\Pi #4 : #2.#5}{#7}$}
  {$(#3, #6, #7) `: \mathcal{R} `^{} #4 `; #1$}
}
\newcommand{\Prod}[1][\Gamma]{\ProdFull{#1}{T}{s_1}{x}{U}{s_2}{s_3}}

\newcommand{\AbsFull}[6]{
  \BAX{Abs}
  {$\tgen{#1}{\Pi #2 : #3.#4}{#5}$}
  {$\tgen{#1, #2 : #3}{#6}{#4}$}
  {$\tgen{#1}{\lambda #2 : #3. #6}{\Pi #2 : #3.#4}$}
  {$#2 `; #1$}
}

\newcommand{\Abs}[1][\Gamma]{\AbsFull{#1}{x}{T}{U}{s}{M}}

 \newcommand{\AppFull}[6]{
  \BAX{App}
  {$\tgen{#1}{#2}{\Pi #3 : #4. #5}$}
  {$\tgen{#1}{#6}{#4}$}
  {$\tgen{#1}{(#2 #6)}{#5 [ #6 / #3 ]}$}
  {$$}
}

\newcommand{\App}[1][\Gamma]{\AppFull{#1}{f}{x}{V}{W}{u}}

\newcommand{\SigmaRFull}[5]{
  \BAX{Sigma}
  {$\tgen{#1}{#2}{#3}$}
  {$\tgen{#1, #4 : #2}{#5}{#3}$}
  {$\tgen{#1}{\Sigma #4 : #2.#5}{#3}$}
  {$#3 `: \{ \Prop, \Set \} `^{} #4 `; #1$} 
}
\newcommand{\SigmaR}[1][\Gamma]{\SigmaRFull{#1}{T}{s}{x}{U}}

\newcommand{\SumDepFull}[7][\Gamma]{
  \TAX{SumDep}
  {$\tgen{#1}{\Sigma #2 : #5. #6}{#7}$}
  {$\tgen{#1}{#3}{#5}$}
  {$\tgen{#1}{#4}{#6[#3 / #2]}$}
  {$\tgen{#1}{\pair{\Sigma #2 : #5.#6}{#3}{#4}}{\Sigma #2 : #5.#6}$}
  {}
}

\newcommand{\SumDep}[1][\Gamma]{\SumDepFull[#1]{x}{t}{u}{T}{U}{s}}
 
\newcommand{\PiLeftFull}[5][\Gamma]{
  \UAX{PiLeft}
  {$\tgen{#1}{#2}{`S #3 : #4.#5}$}
  {$\tgen{#1}{\pi_1~#2}{#4}$}
  {}
}
\newcommand{\PiLeft}[1][\Gamma]{\PiLeftFull[#1]{t}{x}{T}{U}}

\newcommand{\PiRightFull}[5][\Gamma]{
  \UAX{PiRight}
  {$\tgen{#1}{#2}{`S #3 : #4.#5}$}
  {$\tgen{#1}{\pi_2~#2}{#5[\pi_1~#2/#3]}$}
  {}
}
\newcommand{\PiRight}[1][\Gamma]{\PiRightFull[#1]{t}{x}{T}{U}}


\newcommand{\SubsetFull}[4]{
  \BAX{Subset}
  {$\tgen{#1}{#3}{\Set}$}
  {$\tgen{#1, #2 : #3}{#4}{\Prop}$}
  {$\tgen{#1}{\mysubset{#2}{#3}{#4}}{\Set}$}
  {$#2 `; #1$}
}
\newcommand{\SubsetR}[1][\Gamma]{\SubsetFull{#1}{x}{U}{P}}


\newcommand{\SubsumFull}[5]{
  \TAX{Subsum}
  {$\tgen{#1}{#4}{#5}$}
  {$\tgen{#1}{#2}{#3}$}
  {$#5 \sub #2$} % #1 \subt 
  {$\tgen{#1}{#4}{#2}$}
  {}
}
\newcommand{\Subsum}[1][\Gamma]{\SubsumFull{#1}{T}{s}{t}{U}} 

\def\Coerce{\Subsum}

\newcommand{\ConvFull}[5]{
  \TAX{Conv}
  {$\tgen{#1}{#2}{#3}$}
  {$\tgen{#1}{#4}{#5}$}
  {$#5 \eqbr #2$}
  {$\tgen{#1}{#4}{#2}$}
  {}
}
\newcommand{\Conv}[1][\Gamma]{\ConvFull{#1}{T}{s}{t}{U}}

\def\typedFig
{
\begin{figure}[tb]
  \begin{center}
    \def\fCenter{\wf}
    \WfEmpty \DP\quad
    \WfVar \DP
    
    \def\fCenter{\typed}
    \vspace{\infvspace}
    \PropSet\DP

%    \vspace{\infvspace}
%    \TypeType\DP

    \vspace{\infvspace}
    \Var\DP
    
    \vspace{\infvspace}
    \Prod\DP
    
    \vspace{\infvspace}
    \Abs\DP

    \vspace{\infvspace}
    \App\DP

    \vspace{\infvspace}
    \SigmaR\DP

    \vspace{\infvspace}
    \SumDep\DP
    
    \vspace{\infvspace}
    \PiLeft\DP
    \quad
    \PiRight\DP

    \vspace{\infvspace}
    \SubsetR\DP

    \vspace{\infvspace}
    \Subsum\DP
      
  \end{center}
  \caption{Calcul de coercion par pr�dicats - version d�clarative}
  \label{fig:typing-decl-rules}
\end{figure}
}

%%% Local Variables: 
%%% mode: latex
%%% TeX-master: "subset-typing"
%%% LaTeX-command: "TEXINPUTS=\"style:$TEXINPUTS\" latex"
%%% End: 

\def\SubConv{
\UAX{Sub-Conv}
{$T \eqbi U$}
{$T \sub U$}
{} 
}

\def\SubRefl{
  \UAX{Sub-Refl}
  {}
  {$S \seq S$}
  {}
}

\def\SubTrans{
\BAX{Sub-Trans}
{$S \seq T$}
{$T \seq U$}
{$S \seq U$}
{}
} 

\def\SubProd{
\BAX{Sub-Prod}
{$U \seq T$} %"<|-|>"
{$V \seq W$}
{$\Pi x : T.V \seq \Pi x : U.W$}
{}
}

\def\SubSigma{
  \BAX{Sub-Sigma}
  {$T \seq U$}
  {$V \seq W$}
  {$\Sigma x : T. V \seq \Sigma y : U. W$}
  {}
}

\def\SubProof{
  \UAX{Sub-Proof}
  {$U \seq V$}
  {$U \seq \subset{x}{V}{P}$}
  {}
}

\def\SubSub{
  \UAX{Sub-Subset}
  {$U \seq V$}
  {$\subset{x}{U}{P} \seq V$}
  {}
}

\begin{figure}[h]
  \begin{center}
    \def\fCenter{\subd}
    
    \vspace{\infvspace}
    \SubConv\DP

    \vspace{\infvspace}
    \SubTrans\DP

    \vspace{\infvspace}
    \SubProd\DP

    \vspace{\infvspace}
    \SubSigma\DP
    
    \vspace{\infvspace}
    \SubProof\DP
    
    \vspace{\infvspace}
    \SubSub\DP
    
  \end{center}
  \label{subtyping-algo-rules}
  \caption{Sous-typage d�claratif}
\end{figure}

%%% Local Variables: 
%%% mode: latex
%%% TeX-master: "subset-typing"
%%% End: 

\def\AppA{
\TAX{App}
{$`G \seq f : T \quad \mu~T `= \Pi x : V. W$}
{$`G \seq u : U$}
{$U \sub V $}
{$`G \seq (f u) : W [ u / x ]$}
{}
}

\def\SumInfA{
  \TAX{SumInf}
  {$`G \seq t : T $}
  {$`G \seq u : U $}
  {$`G \seq \Sigma \_ : T.U : s$}
  {$`G \seq (t, u) : \Sigma \_ : T.U$}
  {}
}

\def\SumDepA{
  \TAX{SumDep}
  {$`G \seq t : T $}
  {$`G \seq u : U[t/x] $}
  {$`G \seq \Sigma x : T.U : s$}
  {$`G \seq (t, u : U) : \Sigma x : T.U$}
  {}
}

\def\typeaFig{
\begin{figure}[h]
  \begin{center}
    \def\fCenter{\wf}
    \def\type{\typea}
    \def\subt{\subta}
    \def\sub{\suba}
    
    \WfAtom\DP    
    \WfVar\DP
    
    \def\fCenter{\typea}
    \vspace{\infvspace}
    \PropSet\DP
    
    \vspace{\infvspace}
    \Var\DP
    
    \vspace{\infvspace}
    \Prod\DP
    
    \vspace{\infvspace}
    \Abs\DP

    \vspace{\infvspace}
    \AppA\DP

    \vspace{\infvspace}
    \LetIn\DP
    
    \vspace{\infvspace}
    \SigmaR\DP

    \vspace{\infvspace}
    \SumInfA\DP

    \vspace{\infvspace}
    \SumDepA\DP
    
    \vspace{\infvspace}
    \LetSum\DP
    
    \vspace{\infvspace}
    \Subset\DP
  \end{center}
  \label{typing-algo-rules}
  \caption{Calcul des constructions avec sous-typage (version algorithmique)}
\end{figure}
}

\def\typemuaFig{
\begin{figure}[h]
  \begin{eqnarray*}
    \subset{x}{U}{P} & "=>" & \mu~U \\
    x                & "=>" & x
  \end{eqnarray*}
  \label{mu-definition}
  \caption{D�finition de $\mu$}
\end{figure}
}

%%% Local Variables: 
%%% mode: latex
%%% TeX-master: "subset-typing"
%%% LaTeX-command: "TEXINPUTS=\"style:$TEXINPUTS\" latex"
%%% End: 

\def\SubConv{
\UAX{Sub-Conv}
                                %{$`G \judgetypei x : T$}
{$T \eqbi U$}
{$`G \seq x : T \impsub U$}
{} 
}

\def\SubProd{
  \BAX{SubProd}
  {$`G \seq x : U \impsub T$} %"<|-|>"
  {$`G, x : U \seq v : V \impsub W$}
  {$`G \seq \lambda x : T. v : \Pi x : T.V 
    \impsub \Pi x : U.W$}
  {}
}
\def\SubSigma{
  \BAX{Sub-Sigma}
  {$`G \seq t : T \impsub U$}
  {$`G \seq v : V[t/x] \impsub W[t/y]$}
  {$`G \seq (t, v) : \Sigma x : T. V
    \impsub \Sigma y : U. W$}
  {}
}

\def\SubLeft{
  \BAX{Sub-Left}
  {$`G \seq p : U \impsub V$}
  {$`G \typei P : \Pi x : U. \Prop$}
  {$`G \seq p : \subset{x}{U}{P} \impsub V$}
  {}
}
    
\def\SubRight{
  \UAX{Sub-Right}
  {$`G \seq p : T \impsub U$}
                                %{$`G \judgetype h : P~p$}
  {$`G \seq p : T \impsub \subset{x}{U}{P}$}
  {}
}

\begin{figure}[h]
  \begin{center}
    \def\fCenter{\judgesub}
    
    \SubConv\DP
    \vspace{\infvspace}

    \SubProd\DP

    \vspace{\infvspace}
    \SubSigma\DP
    
    \vspace{\infvspace}
    \SubLeft\DP
    
    \vspace{\infvspace}
    \SubRight\DP
    
  \end{center}
  \label{subtyping-algo-rules}
  \caption{Sous-typage}
\end{figure}

%%% Local Variables: 
%%% mode: latex
%%% TeX-master: "subset-typing"
%%% End: 

\newcommand{\timpl}[6]{#1 \typea #2 : #3 "~>" #4 \typec #5 : #6}
\newcommand{\subimpl}[4]{#1 \typec #2 : #3 \sub #4}

\def\WfAtomI{
  \UAX{Wf-Atom}
  {}
  {$\wf [] "~>"~\wf []$}
  {}
}  
  
\def\WfVarI{
  \UAX{Wf-Var}
  {$\timpl{`G}{A}{s}{`G'}{A'}{s}$}
  {$\wf `G, x : A "~>"~\wf `G', x : A'$}
  {$s `: \{ \Set, \Prop, \Type(i) \}$}
}

\def\PropSetI{
  \UAX{PropSet}
  {$\wf `G "~>" ~\wf `G'$}
  {$\timpl{`G}{s}{\Type(0)}{`G'}{s}{\Type(0)}$}
  {$s `: \{ \Prop, \Set \}$} 
}

\def\TypeTypeI{
  \UAX{Type}
  {$\wf `G "~>" \wf `G$}
  {$\timpl{`G}{\Type(i)}{\Type(i + 1)}{`G'}{\Type(i)}{\Type(i + 1)}$}
  {}
}

\def\VarI{
  \BAX{Var}
  {$\wf `G "~>"~\wf `G'$}
  {$x : A `: `G "~>" x : A' `: `G'$}
  {$\timpl{`G}{x}{A}{`G'}{x}{A'}$}
  {}
}

\def\ProdI{
  \BAX{Prod}
  {$\timpl{`G}{T}{s1}{`G'}{T'}{s1}$}
  {$\timpl{`G, x : T}{U}{s2}{`G', x : T'}{U'}{s2}$}
  {$\timpl{`G}{\Pi x : T.U}{s2}{`G'}{\Pi x : T'.U'}{s2}$}
  {$(s1, s2) `: \text{{\sc Sort}}$}
}

\def\AbsI{
  \BAX{Abs}
  {$\timpl{`G}{\Pi x : T. U}{s}{`G'}{\Pi x : T'. U'}{s}$}
  {$\timpl{`G, x : T}{M}{U}{`G', x : T'}{M'}{U'}$}
  {$\timpl{`G}{\lambda x : T. M}{\Pi x : T.U}
    {`G'}{\lambda x : T'. M'}{\Pi x : T'.U'}$}
  {}
}

\def\AppI{
  \QAX{App}
  {$\timpl{`G}{f}{T}{`G'}{f'}{T'}$}
  {$\mu~T' `= (\pi, \Pi x : V'. W')$}
  {$\timpl{`G}{u}{U}{`G'}{u'}{U'}$}
  {$\subimpl{`G'}{c}{U'}{V'}$}
  {$\timpl{`G}{f u}{W[u/x]}{`G'}{(\pi~f')~(c~u')}{W'[ c~u' / x ]}$}
  {}
}

\def\LetInI{
  \BAX{LetIn}
  {$\timpl{`G}{t}{T}{`G'}{t'}{T'}$}
  {$\timpl{`G, x : T}{v}{V}{`G', x : T'}{v'}{V'}$}
  {$\timpl{`G}{\letml x = t \inml v}{V[t / x]}
    {`G'}{\letml x = t' \inml v'}{V'[t' / x]}$}
  {}
}

\def\SigmaRI{
  \BAX{Sigma}
  {$\timpl{`G}{T}{s1}{`G'}{T'}{s1}$}
  {$\timpl{`G, x : T}{U}{s2}{`G', x : T'}{U'}{s2}$}
  {$\timpl{`G}{\Sigma x : T.U}{s2}{`G'}{\Sigma x : T'.U'}{s2} $}
  {$(s1, s2) `: \matht{Sort}$}
}


\def\SumInfI{
  \TAXWide{SumInf}
  {$\timpl{`G}{t}{T}{`G'}{t'}{T'}$}
  {$\timpl{`G}{u}{U}{`G'}{u'}{U'}$}
  {$\timpl{`G}{\Sigma \_ : T.U}{s}{`G'}{\Sigma \_ : T'.U'}{s}$}
  {$\timpl{`G}{(t, u)}{\Sigma \_ : T.U}{`G'}{(t', u')}{\Sigma \_ : T'.U'}$}
  {}
}

\def\SumDepI{
  \TAXWide{SumDep}
  {$\timpl{`G}{t}{T}{`G'}{t'}{T'}$}
  {$\timpl{`G}{u}{U[t/x]}{`G'}{u'}{U'[t'/x]}$}
  {$\timpl{`G}{\Sigma x : T.U}{s}{`G'}{\Sigma x : T'.U'}{s}$}
  {$\timpl{`G}{(t, u : U)}{\Sigma x : T.U}{`G'}{(t', u')}{\Sigma x : T'.U'}$}
  {}
}
 
\def\LetSumI{
  \BAX{LetSum}
  {$\timpl{`G}{t}{\Sigma x : T. U}{`G'}{t'}{\Sigma x : T'.U'}$}
  {$\timpl{`G, x : T, u : U}{v}{V}{`G, x : T', u : U'}{v'}{V'}$}
  {$\timpl{`G}{\letml (x, u) = t \inml v}{V}
    {`G'}{\letml (x, u) = t' \inml v'}{V'}$}
  {}
}

\def\SubsetI{
  \BAX{Subset}
  {$\timpl{`G}{U}{\Type}{`G'}{U'}{\Type}$}
  {$\timpl{`G, x : U}{P}{\Prop}{`G', x : U'}{P'}{\Prop} $}
  {$\timpl{`G}{\subset{x}{U}{P}}{\Type}{`G'}{\subset{x}{U'}{P'}}{\Type}$}
  {$$}
}

\def\marginleft{0em}

\def\typeiFig{
\begin{figure}[h]
  \begin{center}
    \def\fCenter{\wf}
    \def\type{\typec}
    
    \WfAtomI\DP

    \vspace{\infvspace}
    \WfVarI\DP
    
    \vspace{\infvspace}
    \PropSetI\DP
    
    \vspace{\infvspace}
    \VarI\DP
    
    \vspace{\infvspace}
    \ProdI\DP
    
    \vspace{\infvspace}
    \AbsI\DP
    
    \vspace{\infvspace}
    \AppI\DP

    \vspace{\infvspace}
    \LetInI\DP
    
    \vspace{\infvspace}
    \SigmaRI\DP

    \vspace{\infvspace}
    \SumInfI\DP

    \vspace{\infvspace}
    \SumDepI\DP
    
    \vspace{\infvspace}
    \LetSumI\DP
    
    \vspace{\infvspace}
    \SubsetI\DP
  \end{center}
  \label{fig:typing-impl-rules}
  \caption{R�ecriture du typage vers \CCI}
\end{figure}
}


\def\typemuiFig{
\begin{figure}[h]
  \begin{eqnarray*}
    (f, \subset{x}{U}{P}))   & "=>" & \letml (f, t) = \muimpl~(f, U) \inml (f `o
    \pi_1, t) \\
    x                        & "=>"  & x
  \end{eqnarray*}
  \label{fig:muimpl-definition}
  \caption{D�finition de $\muimpl$}
\end{figure}
}

%%% Local Variables: 
%%% mode: latex
%%% TeX-master: "subset-typing"
%%% LaTeX-command: "TEXINPUTS=\"style:$TEXINPUTS\" latex"
%%% End: 

\def\SubConvI{
\UAX{Sub-Conv}
{$T \eqbi U$}
{$\subimpl{`G}{T}{U}{`O}{(\lambda x. x)}{T'}{U'}$}
{} 
}

\def\SubProdI{
  \BAX{Sub-Prod}
  {$\subimpl{`G}{U}{T}{`G'}{c_1}{U'}{T'}$} %"<|-|>"
  {$\subimpl{`G}{V}{W}{`G'}{c_2}{V'}{W'}$}
  {$\subimpl{`G}{\Pi x : T.V}{\Pi x : U.W}{`G'}
    {\lambda f. c_2 `o f `o c_1}
    {\Pi x : T'.V'}{\Pi x : U'.W'}$}
  {}
}

\def\SubSigmaI{
  \BAX{Sub-Sigma}
  {$\subimpl{`G}{T}{U}{`G'}{c_1}{T'}{U'}$}
  {$\subimpl{`G}{V}{W}{`G'}{c_2}{V'}{W'}$}
  {$\subimpl{`G}{\Sigma x : T. V}{\Sigma x : U. W}{`G'}
    {\lambda (x, y). (c_1~x, c_2~y)}
    {\Sigma x : T'. V'}{\Sigma x : U'. W'}
    $}
  {}
}

\def\SubLeftI{
  \BAX{Sub-Left}
  {$\subimpl{`G}{U}{V}{`G'}{c}{U'}{V'}$}
  {$\timpl{`G, x : U}{P}{\Prop}{`G', x : U'}{P'}{\Prop}$}
  {$\subimpl{`G}{\subset{x}{U}{P}}{V}{`G'}
    {c `o \pi_1}
    {\subset{x}{U'}{P'}}{V'}$}
  {}
}
    
\def\SubRightI{
  \BAX{Sub-Right}
  {$\timpl{`G, x : U}{P}{\Prop}{`G', x : U'}{P'}{\Prop}$}
  {$\subimpl{`G}{T}{U}{`G'}{c}{T'}{U'}$}
  {$\subimpl{`G}{T}{\subset{x}{U}{P}}
    {`G'}{\lambda t : T'. \elt{U'}{P'}{(c~t)}{?(P'[c~t/x])}}
    {T'}{\subset{x}{U'}{P'}}$}
  {}
}

\begin{figure}[h]
  \begin{center}
    \def\fCenter{\subtd}
    
    \vspace{\infvspace}
    \SubConvI\DP

    \vspace{\infvspace}
    \SubProdI\DP

    \vspace{\infvspace}
    \SubSigmaI\DP
    
    \vspace{\infvspace}
    \SubLeftI\DP
    
    \vspace{\infvspace}
    \SubRightI\DP
    
  \end{center}
  \label{subtyping-impl-rules}
  \caption{R�ecriture du sous-typage vers \Coq}
\end{figure}

%%% Local Variables: 
%%% mode: latex
%%% TeX-master: "subset-typing"
%%% End: 

\setboolean{displayLabels}{false}

\begin{lemma}[Conservation de la conversion par sous-typage]
  Si $`G \typed t : T$, $T \eqbi U$ alors $`G \subti t : T \sub U$.
\end{lemma}

\begin{proof}
  Par induction sur la forme de $T$.
  
  \def\seq{\subti}.
  
  \begin{itemize}
  \item[$T$ atomique:]
    On a alors $U = T$, trivial.
    
  \item[$T `= \Pi X.Y$:]
    Alors $U `= \Pi V.W$ et $X \eqbi V$, $Y \eqbi W$, 
    donc par induction $`G \subti x : U \sub T$ et 
    $`G, x : U \subti v : V \sub W$. On applique alors
    \SubProdRule{} � ces deux pr�misses.
    
  \item[$T `= \Sigma x : X.Y$:]
    $U$ est de la forme $\Sigma x : V.W$, avec $X \eqbi V$ et $Y \eqbi
    W$. Par induction et application de \SubSigmaRule.
    
  \item[$T `= \subset{x}{X}{P}$:] 
    On a alors $U `= \subset{x}{X'}{P'}$ avec $X \eqbi X'$, $P \eqbi
    P'$, et la propri�t� est vraie par \SubLeftRule{} et \SubRightRule{}:
    
    \begin{prooftree}
      \AXC{$`G \seq t : X \sub X'$}
      \AXC{$`G \typei P : \Pi x : X.\Prop$}
      \LeftLabel{\SubLeftRule}
      \BIC{$`G \seq t : \subset{x}{X}{P} \sub X'$}
      \LeftLabel{\SubRightRule}
      \UIC{$`G \seq t : \subset{x}{X}{P} \sub \subset{x}{X'}{P'}$}
    \end{prooftree}
    
  \end{itemize}
\end{proof}

\begin{lemma}[Substitutivit� des termes du sous-typage]
  \label{substitutive-term-subtyping}
  Si $`G \subti x : U \sub T$ alors pour tout $u$ tel que $`G
  \typed u : U$, $`G \subti u : U \sub T$.
\end{lemma}
\begin{proof}
  Par induction sur la d�rivation de sous-typage 
  $`G \subti x : U \sub T$:
  \def\seq{\subti}.
  
  \begin{itemize}
  \item[\SubConvRule:]
    Direct par pr�servation de l'�quivalence $\eqbi$ par le sous-typage.
    
  \item[\SubProdRule,\SubSigmaRule:] $x$ est une variable, on ne peut
    donc pas utiliser ces r�gles.
    
  \item[\SubLeftRule:] On a
    \begin{prooftree}
      \BAX{Sub-Left}
      {$`G \seq x : U' \sub V$}
      {$`G \typei P : \Pi x : U'. \Prop$}
      {$`G \seq x : \subset{x}{U'}{P} \sub V$}
      {}
    \end{prooftree}
    
    On a $U `= \subset{x}{U'}{P}$ et $`G \subti x : U' \sub V$.
    Par induction, si $`G \typed u : U'$ alors $`G \subti u : U' \sub T$.
    Si $`G \typed u : \subset{x}{U'}{P}$ alors
    $`G \typed u : U'$ par application de \LetSubRule et donc par
    application de \SubLeftRule, $`G \subti u : \subset{x}{U'}{P}
    \sub V$.
    
  \item[\SubRightRule:] On a
    \begin{prooftree}
      \UAX{}
      {$`G \seq x : U \sub V'$}
      {$`G \seq x : U \sub \subset{x}{V'}{P}$}
      {}
    \end{prooftree}
    
    On a $V `= \subset{x}{V'}{P}$ et $`G \subti x : U \sub V'$.
    Par induction, si $`G \typed u : U$ alors $`G \subti u : U
    \sub V'$. Donc par application de \SubRightRule, $`G \subti u : U \sub V$.
    
  \end{itemize}
\end{proof}

\begin{lemma}[Substitutivit� du sous-typage]
  \label{substitutive-subtyping}

  Si $`G, x : V \subti t : T \sub U$, $G \typei u : V$ et 
  $`G \typei t[u/x] : T[u/x]$, alors $`G \typed t[u/x] : T[u/x] \sub U[u/x]$.
\end{lemma}
%% \begin{proof}
%%   Par induction sur la d�rivation de sous-typage:
%%   \def\seq{\subti}.
  
%%   \begin{itemize}
%%   \item[\SubConvRule:]
%%     On a $T \eqbi U$, $`G, x : V \subti y : T \sub U$ et
%%     $`G \typei y[u/x] : T[u/x]$.    
%%     Si $y = x$, alors on doit montrer $`G \subti u : T[u/x] \sub
%%     U[u/x]$ qui est vrai car $\eqbi$ est substitutive et $T[u/x], U[u/x]
%%     $ sont bien typ�s.
%%     Sinon, $`G \typei y : T[u/x] \sub U[u/x]$, vrai par
%%     substitutivit� de $\eqbi$.
    
%%   \item[\SubProdRule:] On a
%%     \begin{prooftree}
%%       \SubProd
%%     \end{prooftree}
    
    
%%   \item[\SubSigmaRule:] On a 
%%     \begin{prooftree}
%%       \SubSigma
%%     \end{prooftree}
    
%%     Par hypoth�se d'induction, $`G \typed t : T, U$, $`G \typed v : V[t/x], W[t/y]$.
%%     Par \SumRule, $(t, v) : \Sigma y : U, W$.

%%   \item[\SubLeftRule:] On a
%%     \begin{prooftree}
%%       \SubLeft
%%     \end{prooftree}
    
%%     Par induction, $`G \typed p : V$.

%%   \item[\SubRightRule:] On a
%%     \begin{prooftree}
%%       \SubRight
%%     \end{prooftree}
    
%%     Par induction, $`G \typed p : U$. Par \SubsetRule, $`G \typed p :
%%     \subset{x}{U}{P}$.
    
%%   \end{itemize}
%% \end{proof}

\begin{lemma}[Inversion du typage]
\label{inversion-typing}

On a les propri�t�s suivantes sur le jugement de typage:
\begin{enumerate}
\item Si $`G \type t : \Pi x : T.U$ alors $t = \lambda x : T.v$ et $`G, x : T
  \type v : U$.
\item Si $`G \type t : \Sigma x : T.U$ alors $t = (t', v)$, $`G, x : T \type U
  : s1$ et $`G \type v : U[t/x]$.
\item Si $`G \type t : \subset{x}{U}{P}$ alors $`G \type t : U$.
\end{enumerate}
\end{lemma}

\begin{lemma}[Inversion du sous-typage]
  \label{inversion-subtyping}
  Si $`G \subti t : \Pi x : T.U \sub \Pi x : V.W$ 
  alors $t = \lambda x : T. M$ et $`G, x : T \typei M : U$.
  
\end{lemma}

\begin{lemma}[Admissibilit� de la r�flexivit�, transitivit� du sous-typage]
  \quad
  \label{refl-trans-subtyping}
  \begin{enumerate}
  \item Pour tout $t, S$, si $`G \typed t : S$ alors $`G \subti t : S \sub S$.
  \item Pour tout $t, S, T, U$,  si $`G \subti t : S \sub T$ et $`G \subti t
    : T \sub U$ alors $`G \subti t : S \sub U$.
  \end{enumerate}
\end{lemma}

\begin{proof}  
  \begin{enumerate}
  \item $S \eqbi S$, donc pour tout $x : S$, $`G \subti x : S \sub
    S$. Par substivit� du sous-typage, la propri�t� est vraie pour tout
    $t$ tel que $`G \typed t : S$.
  \item Dans \cite{Pierce:TypeSystems}, voir p. 420.
  \end{enumerate}
\end{proof}

\begin{lemma}[Correction du sous-typage]
  \label{correct-subtyping}
  Si $`G \subti t : U \sub V$ alors $`G \typed t : U "=>" `G
  \typed t : V$.
\end{lemma}

\begin{proof}
  Par induction sur la d�rivation de sous-typage:
  
  \begin{itemize}
  \item[\SubConvRule:] On a $T \eqbi U$ et l'on suppose 
    $x \typed T$. On a donc bien $`G \typed x : U$.
    
  \item[\SubProdRule:] On a
    \begin{prooftree}
      \SubProd
    \end{prooftree}
    
    Par induction, $`G \typed x : U "=>" `G \typed x : T$, 
    $`G, x : U \typed v : V "=>" `G, x : U \typed v : W$.
    On suppose $`G \typed \lambda x : T.v : \Pi x : T.V$.
    Par inversion du typage, on a $`G, x : T \typed v : V$.
    
    Par \AbsRule, $`G \typed \lambda x : U.v : \Pi x : U.W$.
    
  \item[\SubSigmaRule:] On a 
    \begin{prooftree}
      \SubSigma
    \end{prooftree}
    
    Par hypoth�se d'induction, $`G \typed t : T, U$, $`G \typed v : V[t/x], W[t/y]$.
    Par \SumRule, $(t, v) : \Sigma y : U, W$.

  \item[\SubLeftRule:] On a
    \begin{prooftree}
      \SubLeft
    \end{prooftree}
    
    Par induction, $`G \typed p : V$.

  \item[\SubRightRule:] On a
    \begin{prooftree}
      \SubRight
    \end{prooftree}
    
    Par induction, $`G \typed p : U$. Par \SubsetRule, $`G \typed p :
    \subset{x}{U}{P}$.
    
  \end{itemize}
  
\end{proof}


\begin{lemma}[Correction du typage]
  \label{correct-typing}
  $`G \typei t : T "=>" `G \typed t : T$
\end{lemma}

\begin{proof}
  Par induction sur la d�rivation dans le syst�me algorithmique:

  \begin{description}
  \item[\WfAtomRule,\WfVarRule,\PropSetRule,\VarRule,\ProdRule,\AbsRule,
    \LetInRule, \SigmaRule, \SumRule, \LetSumRule:] r�gles inchang�es.

  \item[\AppRule:] On a
    \def\fCenter{\typei}
    \begin{prooftree}
      \AppI
    \end{prooftree}
    
    Par induction, $`G \typed f : \Pi x : V. W $.
    Par le lemme \ref{correct-subtyping}, et l'hypoth�se $`G \typed u : U$, 
    $`G \typed u : V$. Donc, par \AppRule, on a bien $`G \typed f u :
    W[u/x]$.
  \end{description}
  
\end{proof}

\begin{lemma}[Compl�tude du typage]
  \label{complete-typing}
  $`G \typed t : T "=>" `E U, `G \typei t : U `^ `G \subti t : U \sub T$
\end{lemma}

\begin{proof}
  Par induction sur la d�rivation dans le syst�me d�claratif:

  \begin{description}
  \item[\WfAtomRule,\WfVarRule,\PropSetRule,\VarRule,\ProdRule,\AbsRule,
    \LetInRule, \SigmaRule, \SumRule, \LetSumRule:] r�gles inchang�es.
    
  \item[\AppRule:] On a 
    \begin{prooftree}
      \App
    \end{prooftree}
    
    Par induction, $`E X, Y, `G \typei f : \Pi x : X. Y `^ 
    `G \subti f : \Pi x : X. Y \sub \Pi x : V. W$ et
    $`E U, `G \typei u : U `^ `G \subti u : U \sub V$.
    
    Si $`G \subti f : \Pi x : X.Y \sub \Pi x : V.W$, alors 
    $f$ est de la forme $\lambda x : X.v$ et par inversion (lemme
    \ref{inversion-subtyping}) $`G \subti x : V \sub X$.
    Par substitutivit� du sous-typage (lemme
    \ref{substitutive-term-subtyping}), on a donc $`G \subti u : V
    \sub X$. Par transitivit� du sous-typage \ref{refl-trans-subtyping}, 
    $`G \subti u : U \sub X$. On peut donc appliquer \AppRule{} pour
    obtenir $`G \typei f u : Y[u/x]$. 
    
    Par la covariance du produit en son codomaine, 
    $`G, x : V \subti v : Y \sub W$. Par substitivit� du
    sous-typage (lemme \ref{substitutive-subtyping}), on a donc
    $`G \subti v[u/x] : Y[u/x] \sub W[u/x]$, la propri�t� est
    donc bien v�rifi�e.
    
  \item[\SubsetRule:] On a
    \begin{prooftree}
      \Subset
    \end{prooftree}
    
    Par induction, $`E T, `G \typei x : T `^ `G \subti x : T \sub U$.
    

  \item[\LetSubRule:]
    
  \item[\ConvRule:]

  \end{description}
  
\end{proof}


%%% Local Variables: 
%%% mode: latex
%%% TeX-master: "subset-typing"
%%% End: 

\clearpage

\section*{Travaux connexes(?)}
Coercions: \cite{Chen:POPL-2003, saibi97inheritance}

\section*{Journal}

\subsection*{8 mars}
Nouvelle r�gle de produit fonctionel avec contravariance bien typ�e,
produit d�pendant (\SubSigmaRule) covariant.
Un exemple de produit avec contravariance se trouve dans \cite{cal00coherence}, p. 6.
Exemple d'utilisation int�ressante:

\BAX{}
{$`G \judgetypei f : \{ \phi : \even "->" `N `| `A x : `N, \phi~x ``<= x \}$}
{$`G \judgetypei g : `N "->" `N := \matht{pred}$}
{$`G \judgesubi g ``<= f$}
{}
\DP

Le sous-typage avec coercions: Luo, Callaghan, Sa�bi \cite{saibi97inheritance}... 
\begin{itemize}
\item Uniformit� du sous-typage: ne d�pend pas du contexte.
\item Coercions d�clar�es dans l'environement (ex: Coq).
\end{itemize}

Dans HOL, Joe Hurd simule le \emph{predicate subtyping} � la PVS avec
des \emph{predicate sets} \cite{hurd2001a}. Technique adaptable � Coq ?

%%% Local Variables: 
%%% mode: latex
%%% TeX-master: "~/research/coq/papers/subset-typing"
%%% End: 

\subsection*{9 mars}
Le \ps{} dans HOL n'est pas correct, il peut �tre subverti ais�ment �
cause d'une sorte de covariance des domaines de fonctions et de la
fonction de suppression des predicats (similaire � $\mu$):
$inv : `R^{\neq 0} "->" `R `: `R "->" `R$. Le d�veloppement dans \HOL{}
est fait directement dans le language, et il est argu� qu'il n'est pas
possible de le faire correctement � cause de l'exemple
pr�c�dent. L'algorithme de sous-typage est simplement une g�n�ration de
tout les sous-types possibles � partir d'un ensemble de r�gles de
sous-typage pour les constantes et constructions logiques ou fonctionelles.

Trouv� un article \cite{stumpsubset} sur les types sous-ensembles dans PF, logique d'ordre sup�rieur
avec fonctions partielles... Permet de traiter le cas
$\ifml 1 / i > 0 \thenml i \neq 0 \elseml `_$, qui g�n�re une obligation de
preuve $i \neq 0$ dans \PVS{}. Evidemment on ne risque pas de pouvoir
typer ce code en \Coq{} lorsque $`/ : `Z "->" \{ x : `Z `| x \neq 0 \} "->" `Z$ mais
certaines id�es peuvent �tre int�ressantes. 

%%% Local Variables: 
%%% mode: latex
%%% TeX-master: "~/research/coq/papers/subset-typing"
%%% End: 

\subsection*{11 mars}
Typage de la division euclidienne:
$\matht{div} : `A a : `N, `A b : \{ x : `N `| x \neq 0 \}, `S q : `N, `S r : \{ n : `N
`| n < b \}, a = bq + r := \funml a~b "=>" \ifml a < b \thenml (0, a) \elseml \letml (q, r) =
\matht{div}~(a - b)~b \inml (q + 1, r)$.

Soit $`t_{div} = `A a : `N, `A b : \{ x : `N `| x \neq 0 \}, `E q : `N, `E r : \{ n : `N
`| n < b \}, a = bq + r$ et $`G = \matht{div} : `t_{div}$:



\AXC{$1$}
\AXC{$2$}
\BIC{$`G, a : `N, b : `N^{*} \seq \ifml a < b \thenml (0, a) \elseml \letml (q, r) =
  \matht{div}~(a - b)~b \inml (q + 1, r) : `E q \dots$}
\doubleLine
\UIC{$`G \seq \funml a~b "=>" \ifml a < b \thenml (0, a) \elseml \letml (q, r) =
  \matht{div}~(a - b)~b \inml (q + 1, r) : `t_{div}$}
\DisplayProof

Soit $`G_{if} = `G, a : `N, b : `N^{*}, a ``/< b$.
\begin{prooftree}
  \AXC{$`G_{if} \seq b : `N^{*}$}
  \RightLabel{$`b = `N^{*}$}
  \UIC{$`G_{if} \seq b : `b$}

  \AXC{$`G_{if} \seq b : `g$}
  \AXC{$`G_{if} \seq (- a) : `g "->" `b''$}
  \AXC{$`G_{if} \judgetypea $}
  \TIC{$`G_{if} \seq (- a) b : `b''$}
  \AXC{$`G_{if} \seq \matht{div} : `b''' "->" `b' "->" `a$}
  \AXC{$`G_{if} \judgesubd a - b : `b'' ``<= `b''' "~>" t$}
  \UIC{$`G_{if} \seq \matht{div}~(a - b) "~>" \matht{div}~t : `b' "->" `a$}

  \TIC{$`G_{if} \seq \matht{div}~(a - b) : `b' "->" `a$}

  \AXC{$`G_{if} \judgesubd b : `b ``<= `b' "~>" p $}
  \TIC{$`G_{if} \seq \matht{div}~(a - b)~b "~>" \matht{div}~(a - b)~p : `a$}
  
  
  \AXC{$`G_{if}, (q, r) : `a \seq (q + 1, r) : `E q \dots$}
  \RightLabel{1}
  \BIC{$`G_{if} \seq \letml (q, r) = \matht{div}~(a - b)~b \inml (q + 1, r) : `E q \dots$}
\end{prooftree}  

\RightLabel{2}
\AXC{$`G, a : `N, b : `N^{*}, a < b \seq (0, a) : `E q \dots$}
\DisplayProof


%%% Local Variables: 
%%% mode: latex
%%% TeX-master: "~/research/coq/papers/subset-typing"
%%% End: 

\subsection*{14 mars}
R�gles d'introduction, d'�limination et de formation pour $\Pi$, $\Sigma$, sous-types
pr�dicats. Nouveau jugement de typage par r�ecriture. Nombreuses
questions � discuter avec Christine. Beaucoup de bruit dans le buro !


%%% Local Variables: 
%%% mode: latex
%%% TeX-master: "~/research/coq/papers/subset-typing"
%%% End: 

\subsection*{15 mars}
Distinction inf\'erence et typage. 
\begin{description}
\item[Inf\'erence ($"~>"$)] \`a la ML, on v\'erifie: $`G \typea p "~>" T "=>"
  `G \typed p : T$. 
\item[Typage ($:$)]. On a $`G \typed p : T$, on veut $`E U, `G \typea p
  "~>" U `^ `G \typea U "~>" T$. On utilise le sous-typage g\'en\`erant les
  obligations de preuve.
\end{description}

Eliminer \rname{LetSub}, inutilisable en pratique.

Quelques points \`a m\'editer:
\begin{itemize}
\item $`O \type 3 : `N "~>" `O \type 3 : `N^*$ ? D\'ependance envers le
  terme pour le sous-typage. De m\^eme, $2 : `N ``<= `N^*$, on devrait
  parler de renforcement.
\item On peut restreindre le sous-typage aux projections de types
  subsets avec l'\'egalit\'e syntaxique.
\item Je peux garder mes r\^egles de sous-typages, si elles sont syntax-directed!
\end{itemize}

Sous-typage \`a l'application et variable suffisante pour l'ad\'equation ?
On distingue les deux phases, pas de sous-typage \`a l'application.

V\'erifier Sub-{Left, Right}, l'application du sous-typage.

%%% Local Variables: 
%%% mode: latex
%%% TeX-master: "~/research/coq/papers/subset-typing"
%%% LaTeX-command: "TEXINPUTS=\"style:$TEXINPUTS\" latex"
%%% End: 

\subsection*{16 mars}
Il faut faire du sous-typage dans la sp�cification aussi:
$f : x : \subset{n}{`N}{0 \neq} "->" \subset{n}{`N}{x <}$.

Les deux phases:
\begin{description}
\item[Inf�rence] on donne les types impr�cis, ie: dans $x > n$, 
  $n "~>" `N$.
\item[Typage] on traverse la premi�re d�rivation de typage en ajoutant
  les coercions appropri�es, par exemple:
  $`G \type n : \subset{n}{`N}{0 \neq}$,  $\type_{inf} n "~>" `N$ est
  r�ecrit en: 
  $`G \type \pi_{1}~n : `N$.
\end{description}

\subsubsection*{Soir!}

Formalisation des trois jugements:
\begin{description}
\item[$\typed$] Typage d�claratif, syst�me ind�cidable, repr�sentant
  exactement ce qu'on veut ajouter comme fonctionnalit�.
\item[$\typei$] Version algorithmique, utilisant le dernier jugement
  pour r�aliser l'ad�quation avec la pr�sentation d�clarative.
\item[$\judgesubi$] ``Sous-typage'', sans obligations de preuves,
  d�cidable et � peu pr�s d�terministe.
\end{description}

Il faudra ensuite faire la traduction dans \Coq, avec de nouveaux
jugements r�ecrivant les d�rivations.

Propri�t�s � montrer:
\begin{itemize}
\item $`G \typed t : T "=>" `E U, `G \typei t : U `^ `G \judgesubi t : U \sub T$
\item $`G \typei t : T "=>" `G \typed t : T$
\end{itemize}

%%% Local Variables: 
%%% mode: latex
%%% TeX-master: "~/research/coq/papers/subset-typing"
%%% End: 

\subsection*{17 mars}

\begin{lemma}[Substitutivit� du sous-typage]
  \label{substitutive-subtyping}
  $`G \judgesub t : T \impsub U "=>" `G \typed t : T[u/x] \impsub U[u/x]$
\end{lemma}
\begin{proof}
  Par induction sur la d�rivation de sous-typage:

  \begin{itemize}
  \item[\SubConvRule:] On a $T \eqbi U$ et l'on suppose 
    $x \typed T$. On a donc bien $`G \typed x : U$.
    
  \item[\SubProdRule:] On a
    \begin{prooftree}
      \SubProd
    \end{prooftree}
    
    Par induction, $`G \typed x : U$, $`G \typed x : T$, 
    $`G, x : U \typed v : V$ et $`G, x : U \typed v : W$.
    Par \AbsRule, $`G \typed \lambda x : U.v : \Pi x : U.W$.
    
  \item[\SubSigmaRule:] On a 
    \begin{prooftree}
      \SubSigma
    \end{prooftree}
    
    Par hypoth�se d'induction, $`G \typed t : T, U$, $`G \typed v : V[t/x], W[t/y]$.
    Par \SumRule, $(t, v) : \Sigma y : U, W$.

  \item[\SubLeftRule:] On a
    \begin{prooftree}
      \SubLeft
    \end{prooftree}
    
    Par induction, $`G \typed p : V$.

  \item[\SubRightRule:] On a
    \begin{prooftree}
      \SubRight
    \end{prooftree}
    
    Par induction, $`G \typed p : U$. Par \SubsetRule, $`G \typed p :
    \subset{x}{U}{P}$.
    
  \end{itemize}
\end{proof}

\begin{lemma}[Inversion du sous-typage]
  \label{inversion-subtyping}
\end{lemma}

\begin{lemma}[D�rivabilit� de la r�flexivit�, transitivit� du sous-typage]
  \label{refl-trans-subtyping}
\end{lemma}

\begin{lemma}[Correction du sous-typage]
  \label{correct-subtyping}
  Si $`G \judgesub t : U \impsub V$ alors $`G \typed t : U "=>" `G
  \typed t : V$.
\end{lemma}

\setboolean{displayLabels}{false}

\begin{proof}
  Par induction sur la d�rivation de sous-typage:

  \begin{itemize}
  \item[\SubConvRule:] On a $T \eqbi U$ et l'on suppose 
    $x \typed T$. On a donc bien $`G \typed x : U$.
    
  \item[\SubProdRule:] On a
    \begin{prooftree}
      \SubProd
    \end{prooftree}
    
    Par induction, $`G \typed x : U$, $`G \typed x : T$, 
    $`G, x : U \typed v : V$ et $`G, x : U \typed v : W$.
    Par \AbsRule, $`G \typed \lambda x : U.v : \Pi x : U.W$.
    
  \item[\SubSigmaRule:] On a 
    \begin{prooftree}
      \SubSigma
    \end{prooftree}
    
    Par hypoth�se d'induction, $`G \typed t : T, U$, $`G \typed v : V[t/x], W[t/y]$.
    Par \SumRule, $(t, v) : \Sigma y : U, W$.

  \item[\SubLeftRule:] On a
    \begin{prooftree}
      \SubLeft
    \end{prooftree}
    
    Par induction, $`G \typed p : V$.

  \item[\SubRightRule:] On a
    \begin{prooftree}
      \SubRight
    \end{prooftree}
    
    Par induction, $`G \typed p : U$. Par \SubsetRule, $`G \typed p :
    \subset{x}{U}{P}$.
    
  \end{itemize}
  
\end{proof}


\begin{lemma}[Inversion du typage]
\label{inversion-typing}
\end{lemma}

\begin{lemma}[Correction du typage]
  \label{correct-typing}
  $`G \typei t : T "=>" `G \typed t : T$
\end{lemma}

\begin{proof}
  Par induction sur la d�rivation dans le syst�me algorithmique:

  \begin{description}
  \item[\WfAtomRule,\WfVarRule,\PropSetRule,\VarRule,\ProdRule,\AbsRule,
    \LetInRule, \SigmaRule, \SumRule, \LetSumRule:] r�gles inchang�es.

  \item[\AppRule:] On a
    \def\fCenter{\typei}
    \begin{prooftree}
      \AppI
    \end{prooftree}
    
    Par induction, $`G \typed f : \Pi x : V. W $.
    Par le lemme \ref{correct-sub}, et l'hypoth�se $`G \typed u : U$, 
    $`G \typed u : V$. Donc, par \AppRule, on a bien $`G \typed f u :
    W[u/x]$.
  \end{description}
  
\end{proof}

\begin{lemma}[Compl�tude du typage]
  \label{complete-typing}
  $`G \typed t : T "=>" `E U, `G \typei t : U `^ `G \judgesub t : U \impsub T$
\end{lemma}

\begin{proof}
  Par induction sur la d�rivation dans le syst�me d�claratif:

  \begin{description}
  \item[\WfAtomRule,\WfVarRule,\PropSetRule,\VarRule,\ProdRule,\AbsRule,
    \LetInRule, \SigmaRule, \SumRule, \LetSumRule:] r�gles inchang�es.
    
  \item[\AppRule:] On a 
    \begin{prooftree}
      \App
    \end{prooftree}
    
    Par induction, $`E X, Y, `G \typei f : \Pi x : X. Y `^ 
    `G \judgesub f : \Pi x : X. Y \impsub \Pi x : V. W$ et
    $`E U, `G \typei u : U `^ `G \judgesub u : U \impsub V$.
    
    Si $`G \judgesub f : \Pi X.Y \impsub \Pi x : V.W$, alors 
    $`G \judgesub x : V \impsub X$ (lemme \ref{inversion-subtyping}). 
    Par transitivit� du sous-typage \ref{refl-trans-subtyping}, 
    $`G \judgesub U \impsub X$. On peut donc appliquer \AppRule{} pour
    obtenir $`G \typei f u : Y[u/x]$. Par la covariance du
    produit en son codomaine et la substitivit� du sous-typage,
    on a $`G \judgesub f u : Y[u/x] \impsub W[u/x]$, la propri�t� est
    donc bien v�rifi�e.
    
  \item[\SubsetRule:] On a
    \begin{prooftree}
      \Subset
    \end{prooftree}
    
    Par induction, $`E T, `G \typei x : T `^ `G \judgesub x : T \impsub U$.
    

  \item[\LetSubRule:]
    
  \item[\ConvRule:]

  \end{description}
  
\end{proof}


%%% Local Variables: 
%%% mode: latex
%%% TeX-master: "~/research/coq/papers/subset-typing"
%%% End: 

\subsection*{22 mars}
Continue les preuves...
Trouver les bons lemmes de substitution!

Enlever la condition $A atomique$ dans \SubRightRule. Ca donne une
strat�gie d�terministe mais n'est pas indispensable dans cette pr�sentation.

%%% Local Variables: 
%%% mode: latex
%%% TeX-master: "~/research/coq/papers/subset-typing"
%%% End: 

\subsection*{23 mars}
Lu \cite{Chen:POPL-2003} en d�tail ainsi que
\cite{DBLP:journals/tcs/LuoS99} ou Zhaohui laisse en suspens la question
de la pertinence d'avoir des r�gles pour les produits reliant 
$\Pi x: A. \matht{list}~ `N$ et $\Pi x : A. \{ \matht{list}~ `N `| \ldots \}$ dans notre syst�me.
Une partie de \cite{DBLP:conf/csl/Luo96}. Plus int�ressant est
peut-�tre l'article sur les combinaisons incoh�rentes de coercions pour
les types sommes \cite{DBLP:conf/types/LuoL03}. Les objectifs de
coh�rence et d'�limination de la transitivit� ne sont pas loin des
notres: on veut que 'computationellement' les coercions soit
inessentielles (au contraire on accepte tout dans la partie logique)
et avoir un sous-typage avec de bonnes propri�t�s.

Merci Jean-Christophe pour
Mechanical Metatheory for the masses - The {\sc PoplMark} Challenge !

%%% Local Variables: 
%%% mode: latex
%%% TeX-master: "~/research/coq/papers/subset-typing"
%%% End: 
        
\subsection*{24 mars}
J'ai enlev� les termes du jugement de sous-typage d�claratif qui n'en a
pas besoin.

Preuves d'�quivalence entre syst�me algorithmique et d�claratif.
Plusieurs probl�mes:
\begin{itemize}
\item Besoin d'annotations de typage au let, ou alors restriction du
  type on a pas le droit d'utiliser le sous-typage au let. Ou alors on
  peut r�ecrire la d�rivation $x : S \sub t : T$ en $x : S' \sub t : T$
  lorsque $S' \sub S$ (narrowing) mais de fa�on � ce que l'on ajoute pas
  d'utilisation de $\sub$.
\item Application d'une fonction dans un type subset... on ne peut pas
  r�ecrire la d�rivation: soit on ajoute une fonction de
  'd�compr�hension' qu'on applique � gauche ou on autorise le
  sous-typage complet avec une r�gle du style:
  \begin{prooftree}
    \QAX{App}
    {$`G \seq f : T$}
    {$`G \subtd T \sub \Pi x : V. W$}
    {$`G \seq u : V' $}
    {$`G \subtd V' \sub V$}
    {$`G \seq (f u) : W [ u / x ]$}
    {$$}
  \end{prooftree}

  Mais on en revient � trouver une fonction pour d�cider de la deuxi�me
  pr�misse. On doit donc reprendre la fonction $\mu_0$ de \PVS{}.

  D�finition de $\mu_0$:
  \begin{eqnarray*}
    \subset{x}{U}{P} & "=>" & \mu_0~U \\
    x                & "=>" & x
  \end{eqnarray*}
  
  et l'on obtient la r�gle:
  \begin{prooftree}
    \QAX{App}
    {$`G \subtd f : T$}
    {$\mu_0~T = \Pi x : V. W$}
    {$`G \seq u : V' $}
    {$`G \subtd V' \sub V$}
    {$`G \seq (f u) : W [ u / x ]$}
    {$$}
  \end{prooftree}
\item ce n'est qu'une preuve informelle ;)
\end{itemize}

Comment traiter la conversion ? Voir la th�se de Chen.


%%% Local Variables: 
%%% mode: latex
%%% TeX-master: "~/research/coq/papers/subset-typing"
%%% End: 
        
\subsection*{25 mars}
Question de la transitivit� de notre "sous-typage".
On peut avoir/demander une forme restreinte de transitivit� du genre: 
$`G \subta t : A \sub B "~>" t' `^ `G \subta t' : B \sub C "~>" t'' "=>"
`G \subta t : A \sub C "~>" t''$. Mais dans un sens puisque notre
sous-typage d�pend des termes, il est clair que l'on a
$`G \subta t : A \sub B "~>" t' `^ `G \subta x : B \sub C "~>" t'' "=>"
`G \subta t : A \sub C "~>" t''$ si $x `; `G$ mais pas plus. Il faut
r�flechir � quelle est la solution la plus utile/souhaitable.
Lecture de la 3�me partie de \cite{ChenPhD} sur le sous-typage dans
$\lambda CC_{\leq}$.

Lecture d'articles sur la syntaxe abstraite pour le challenge {\sc
  PoplMark}\ldots


%%% Local Variables: 
%%% mode: latex
%%% TeX-master: "~/research/coq/papers/subset-typing"
%%% End: 
        
\subsection*{30 mars}
Quelques questions:
\begin{itemize} 
\item Unification pour \SumRule{} probl�matique ? Impl�mentation dans
  \Coq.
\item M�canisme d'annotations de type � ajouter ? Seulement aux lets ou
  partout ?
\end{itemize}

%%% Local Variables: 
%%% mode: latex
%%% TeX-master: "~/research/coq/papers/subset-typing"
%%% End: 
        
\subsection*{31 mars}
Corrections d'hier...
Inf�rence des sommes, plus de contextes au sous-typage.
V�rification des sortes. Quelques explications sur la conversion.
Probl�me d'enlever les termes pour \SubSigmaRule{}, on perd les
d�pendances.

D�finition de $\muterm$, exemple:
$\muterm~(\lambda x. x)~\subset{x}{\subset{y}{`N}{y > 0}}{x \neq 0} = (id `o \pi_1 `o \pi_1, `N)$

%%% Local Variables: 
%%% mode: latex
%%% TeX-master: "~/research/coq/papers/subset-typing"
%%% End: 

\bibliography{../bib/bib-joehurd,../bib/pvs-bib,../bib/bcp,../bib/Luo,subset-typing}
\bibliographystyle{plain}

\renewcommand{\thefootnote}{}
\footnotetext{Ce rapport a �t� pr�par� sous \LaTeX~avec la fonte 
  \texttt{Computer Modern Bright}}

\end{document}
