\documentclass[a4paper,11pt]{report}

\usepackage[francais]{babel} 
\usepackage[latin1]{inputenc}  %% les accents dans le fichier.tex
%\usepackage[T1]{fontenc}       %% Pour la c\'{e}sure des mots accentu\'{e}s
\usepackage{indentfirst}
\usepackage[dvips]{graphicx}
%\usepackage{coqdoc}
%\usepackage{cmbright}
\usepackage{amssymb}
\usepackage{amsmath}
\usepackage{amsthm}
\usepackage{amsfonts}
\usepackage{array}
\usepackage{abbrevs}
%\usepackage{bnf}
\usepackage{bussproofs}
\usepackage{hyperref}
\usepackage{fullpage}
\usepackage{concmath}
\usepackage{xspace}
\usepackage{fancyhdr}
\usepackage{ifthen}
\usepackage{ifmtarg}
\usepackage{pxfonts}
\usepackage[xdvi]{xy}
\xyoption{all}
%\CompileMatrices

\usepackage[rgb,xdvi]{xcolor}
\usepackage{subfigure}

% This is the "centered" symbol
\def\fCenter{\vdash}
\def\seq{\fCenter}
% Optional to turn on the short abbreviations
\EnableBpAbbreviations

\oddsidemargin -1cm
\topmargin -1cm
\headsep 5mm
\textwidth 18cm
\textheight 24cm

\newboolean{defineTheorem}
\setboolean{defineTheorem}{true}
\newtheorem{theorem}{Th�or�me}[section]
\newtheorem{lemma}[theorem]{Lemme}
\newtheorem{fact}[theorem]{Fait}
\newtheorem{proposition}[theorem]{Proposition}
\newtheorem{definition}[theorem]{D�finition}
\newtheorem{example}[theorem]{Exemple}
\newtheorem{remark}[theorem]{Remarque}
\newtheorem{corrolary}[theorem]{Corrolaire}

\makeatletter

\newcommand{\UR}[2]{\RightLabel{\rulelabel{#1}}\UIC{#2}}
\newcommand{\URL}[2]{\LeftLabel{\rulelabel{#1}}\UIC{#2}}

\newboolean{displayLabels}
\setboolean{displayLabels}{true}

\newcommand{\LeftRuleLabel}[1]{
  \@ifnotmtarg{#1}
  {\ifthenelse{\boolean{displayLabels}}{\LeftLabel{\rulelabel{#1}}}{}}
}

\newcommand{\UAX}[4]{\AXC{#2}
  \LeftRuleLabel{#1}
  \@ifnotmtarg{#4}{\RightLabel{#4}}
  \UIC{#3}}

\newcommand{\BAX}[5]{\AXC{#2}\AXC{#3}
  \LeftRuleLabel{#1}
  \@ifnotmtarg{#5}{\RightLabel{#5}}
  \BIC{#4}}

\newcommand{\TAX}[6]{\AXC{#2}\AXC{#3}\AXC{#4}
  \LeftRuleLabel{#1}
  \@ifnotmtarg{#6}{\RightLabel{#6}}
  \TIC{#5}}

\newcommand{\QAX}[7]{\AXC{#2}\noLine\UIC{#3}\AXC{#4}\AXC{#5}
  \LeftRuleLabel{#1}
  \@ifnotmtarg{#7}{\RightLabel{#7}}
  \TIC{#6}}

\newcommand{\BR}[2]{\RightLabel{\rulelabel{#1}}\BIC{#2}}
\newcommand{\BRL}[2]{\LeftLabel{\rulelabel{#1}}\BIC{#2}}

\makeatother

%%% Local Variables: 
%%% mode: latex
%%% TeX-master: t
%%% End: 


%\def\ifstreq#1#2{\ifEqString{#1}{#2}}

\def\impsub{\rightslice}

\def\judgewf{\vdash_{wf}}
\def\judgetyped{\vdash}
\def\judgetypea{\vdash_{\bullet}}
\def\judgetypei{\vdash_{\Box}}
\def\typed{\judgetyped}
\def\typedwf{\judgewf}
\def\typea{\judgetypea}
\def\typeawf{\judgetypea_{wf}}
\def\typei{\judgetypei}
\def\typeiwf{\judgetypei_{wf}}
\def\type{\typed}
\def\typewf{\typedwf}
\def\judgesubd{\vdash}
\def\judgesuba{\vdash_{\bullet}}
\def\judgesubi{\vdash_{\Box}}
\def\subtd{\judgesubd}
\def\subta{\judgesuba}
\def\subti{\judgesubi}
\def\subt{\subtd}
\def\subd{\rightslice}
\def\suba{\rightslice_{\bullet}}
\def\subi{\rightslice_{\Box}}
\def\sub{\subd}

\def\judgerw{~{\mbox{$"~>"$}}~}
\def\wf{\judgewf}

\newcommand{\matht}[1]{\text{{\tt #1}}}

\def\even{\matht{even}}
\def\odd{\matht{odd}}
\def\Set{\matht{Set}}
\def\Prop{\matht{Prop}}
\def\Type{\matht{Type}}
\def\ps{\emph{predicate subtyping}}

\def\WfAtomRule{Wf-Atom}
\def\WfVarRule{Wf-Var}
\def\PropSetRule{PropSet}
\def\TypeRule{Type}
\def\VarRule{Var}
\def\ProdRule{Prod}
\def\AbsRule{Abs}
\def\AppRule{App}
\def\LetInRule{Let-In}
\def\SigmaRule{Sigma}
\def\SumRule{Sum}
\def\LetSumRule{Let-Sum}
\def\SumInfRule{Sum-Inf}
\def\SumDepRule{Sum-Dep}
\def\SubsetRule{Subset}
\def\LetSubRule{Let-Sub}
\def\SubsumRule{Subsumption}
\def\ConvRule{Conv}

\def\SubReflRule{Sub-Refl}
\def\SubTransRule{Sub-Trans} 
\def\SubConvRule{Sub-Conv}
\def\SubProdRule{Sub-Prod}
\def\SubSigmaRule{Sub-Sigma}
\def\SubLeftRule{Sub-Left}
\def\SubRightRule{Sub-Right}
\def\SubProofRule{Sub-Proof}
\def\SubSubRule{Sub-Subset}
\def\SubTransRule{Sub-Trans}

\def\ifstreq#1#2{\def\testa{#1}\def\testb{#2}\ifx\testa\testb }

\def\inductionon#1{
  \ifstreq{#1}{typing-decl}{Par induction sur la d�rivation de typage.}
  \else\ifstreq{#1}{typing-algo}{Par induction sur la d�rivation de
    typage dans le syst�me algorithmique.}
  \else\ifstreq{#1}{typing-impl}{Par induction sur la d�rivation de
    typage.}
  \else\ifstreq{#1}{subtyping-decl}{Par induction sur la d�rivation de
    sous-typage.}
  \else\ifstreq{#1}{subtyping-algo}{Par induction sur la d�rivation de
    sous-typage dans le syst�me algorithmique.}
  \else\ifstreq{#1}{subtyping-impl}{Par induction sur la d�rivation de
    sous-typage.}
  \fi
}

\newenvironment{induction}[1][text=\empty]{
  \if#1\empty\else\inductionon{#1}
  \begin{list}{Unset default item}{}}
  {\end{list}}

%% Should be able to work with \else...
\newcommand{\inductionrule}[1]
{\ifstreq{#1}{WfAtom}{\WfAtomRule}
\fi\ifstreq{#1}{WfVar}\WfVarRule
\fi\ifstreq{#1}{PropSet}\PropSetRule
\fi\ifstreq{#1}{Type}\TypeRule
\fi\ifstreq{#1}{Var}\VarRule
\fi\ifstreq{#1}{Prod}\ProdRule
\fi\ifstreq{#1}{Abs}\AbsRule
\fi\ifstreq{#1}{LetIn}\LetInRule
\fi\ifstreq{#1}{Sigma}\SigmaRule
\fi\ifstreq{#1}{Sum}\SumRule
\fi\ifstreq{#1}{LetSum}\LetSumRule
\fi\ifstreq{#1}{App}\AppRule
\fi\ifstreq{#1}{SumInf}\SumInfRule
\fi\ifstreq{#1}{SumDep}\SumDepRule
\fi\ifstreq{#1}{LetSub}\LetSubRule
\fi\ifstreq{#1}{Subsum}\SubsumRule
\fi\ifstreq{#1}{Conv}\ConvRule
\fi\ifstreq{#1}{Subset}\SubsetRule

\fi\ifstreq{#1}{SubRefl}\SubReflRule
\fi\ifstreq{#1}{SubTrans}\SubTransRule
\fi\ifstreq{#1}{SubConv}\SubConvRule
\fi\ifstreq{#1}{SubProd}\SubProdRule
\fi\ifstreq{#1}{SubSigma}\SubSigmaRule
\fi\ifstreq{#1}{SubLeft}\SubLeftRule
\fi\ifstreq{#1}{SubRight}\SubRightRule
\fi\ifstreq{#1}{SubProof}\SubProofRule
\fi\ifstreq{#1}{SubSub}\SubSubRule
\fi}

\newcommand{\rulename}[1]{{\bf \inductionrule{#1}}}

\def\indrule{\rulename}

\def\case#1{\item[- \rulename{#1} :]}
\newcommand{\casetwo}[2]{\item[- \rulename{#1}, \rulename{#2} :]}
\def\casethree#1#2#3{\item[- \rulename{#1}, \rulename{#2},
  \rulename{#3} :]}
\def\casefour#1#2#3#4{\item[- \rulename{#1}, \rulename{#2},
  \rulename{#3}, \rulename{#4} :]}

\newcommand{\rname}[1]{{\bf \rulename{#1}}}
\newcommand{\rulelabel}[1]{{\bf (\rulename[#1])}}

%%% Local Variables: 
%%% mode: latex
%%% TeX-master: "subset-typing"
%%% LaTeX-command: "TEXINPUTS=\"style:$TEXINPUTS\" latex"
%%% End: 


\setboolean{displayLabels}{true}

\newcommand{\termgrammar}
{$\begin{array}{lcl}
    `a & \Coloneqq & x \\
    & | & \funml{}~x~:~`t "=>" `a \\
    & | & `a~`a \\ 
    & | & `a~`t \\
    & | & \text{\emph{constante}} \\
% & | & (`a,~`a) \\
    & | & (x \coloneqq `a,~`a~: `t) \\
    & | & \letml~x = `a ~\inml~`a \\
    & | & \letml~(x, y) = `a ~\inml~ `a

%    & | & \ifml~`a~\thenml~`a~\elseml~`a
  \end{array}$}

\newcommand{\typegrammarOrig}
{$\begin{array}{lcl}
    `t & \Coloneqq & x \\
    & | & `t "->" `t \\
    & | & `t~`t \\
    & | & \text{\emph{constante}} \\
    & | & `t * `t \\
    & | & \Sigma x : `t. `t \\
    & | & \lambda{}~x~:~s "=>" `t \\
    & | & `t~`a \\
    & | & \Pi x : `t. `t \\
    & | & \subset{x}{`t}{`t}
  \end{array}$}

\newcommand{\typegrammar}
{$\begin{array}{lcl}
    `t & \Coloneqq & x \\
    & | & `t~`t \\
    & | & `t~`a \\
    & | & \lambda{}~x~:~`t "=>" `t \\
    & | & \Pi x : `t. `t \\
    & | & \Sigma x : `t. `t \\
%    & | & `t * `t \\
    & | & \subset{x}{`t}{`t} \\
    & | & \Set \\
    & | & \Prop \\
    & | & \Type \\
    & | & \text{\emph{constante}} 
%    & & \\
%    \text{{\tt Inductive}} & \Coloneqq & ident
  \end{array}$}

\newcommand{\WfEmptyFull}[1]{
  \UAX{WfEmpty}
  {}
  {$\seq []$}
  {}
}  
\newcommand{\WfEmpty}[1][\Gamma]{\WfEmptyFull{#1}}
  
\newcommand{\WfVarFull}[4]{
  \UAX{WfVar}
  {$\tgen{#1}{#2}{#3}$}
  {$\wf #1, #4 : #2$}
  {$#3 `: \setproptype{} `^{} #4 `; #1$}
}
\newcommand{\WfVar}[1][\Gamma]{\WfVarFull{#1}{A}{s}{x}}

\newcommand{\PropSetFull}[2]{
  \UAX{PropSet}
  {$\wf #1$}
  {$\tgen{#1}{#2}{\Type}$}
  {$#2 `: \setprop$} 
}
\newcommand{\PropSet}[1][\Gamma]{\PropSetFull{#1}{s}}


\newcommand{\TypeTypeFull}[1]{
  \UAX{Type}
  {$\wf #1$}
  {$\tgen{#1}{\Type(i)}{\Type(i + 1)}$}
  {$i `: `N$}
}
\newcommand{\TypeType}[1][\Gamma]{\TypeTypeFull{#1}}

\newcommand{\VarFull}[3]{
  \BAX{Var}
  {$\wf #1$}
  {$#2 : #3 `: #1$}
  {$\tgen{#1}{#2}{#3}$}
  {}
}
\newcommand{\Var}[1][\Gamma]{\VarFull{#1}{x}{A}} 

\newcommand{\ProdFull}[7]{
  \BAX{Prod}
  {$\tgen{#1}{#2}{#3}$}
  {$\tgen{#1, #4 : #2}{#5}{#6}$}
  {$\tgen{#1}{\Pi #4 : #2.#5}{#7}$}
  {$(#3, #6, #7) `: \mathcal{R} `^{} #4 `; #1$}
}
\newcommand{\Prod}[1][\Gamma]{\ProdFull{#1}{T}{s_1}{x}{U}{s_2}{s_3}}

\newcommand{\AbsFull}[6]{
  \BAX{Abs}
  {$\tgen{#1}{\Pi #2 : #3.#4}{#5}$}
  {$\tgen{#1, #2 : #3}{#6}{#4}$}
  {$\tgen{#1}{\lambda #2 : #3. #6}{\Pi #2 : #3.#4}$}
  {$#2 `; #1$}
}

\newcommand{\Abs}[1][\Gamma]{\AbsFull{#1}{x}{T}{U}{s}{M}}

 \newcommand{\AppFull}[6]{
  \BAX{App}
  {$\tgen{#1}{#2}{\Pi #3 : #4. #5}$}
  {$\tgen{#1}{#6}{#4}$}
  {$\tgen{#1}{(#2 #6)}{#5 [ #6 / #3 ]}$}
  {$$}
}

\newcommand{\App}[1][\Gamma]{\AppFull{#1}{f}{x}{V}{W}{u}}

\newcommand{\SigmaRFull}[5]{
  \BAX{Sigma}
  {$\tgen{#1}{#2}{#3}$}
  {$\tgen{#1, #4 : #2}{#5}{#3}$}
  {$\tgen{#1}{\Sigma #4 : #2.#5}{#3}$}
  {$#3 `: \{ \Prop, \Set \} `^{} #4 `; #1$} 
}
\newcommand{\SigmaR}[1][\Gamma]{\SigmaRFull{#1}{T}{s}{x}{U}}

\newcommand{\SumDepFull}[7][\Gamma]{
  \TAX{SumDep}
  {$\tgen{#1}{\Sigma #2 : #5. #6}{#7}$}
  {$\tgen{#1}{#3}{#5}$}
  {$\tgen{#1}{#4}{#6[#3 / #2]}$}
  {$\tgen{#1}{\pair{\Sigma #2 : #5.#6}{#3}{#4}}{\Sigma #2 : #5.#6}$}
  {}
}

\newcommand{\SumDep}[1][\Gamma]{\SumDepFull[#1]{x}{t}{u}{T}{U}{s}}
 
\newcommand{\PiLeftFull}[5][\Gamma]{
  \UAX{PiLeft}
  {$\tgen{#1}{#2}{`S #3 : #4.#5}$}
  {$\tgen{#1}{\pi_1~#2}{#4}$}
  {}
}
\newcommand{\PiLeft}[1][\Gamma]{\PiLeftFull[#1]{t}{x}{T}{U}}

\newcommand{\PiRightFull}[5][\Gamma]{
  \UAX{PiRight}
  {$\tgen{#1}{#2}{`S #3 : #4.#5}$}
  {$\tgen{#1}{\pi_2~#2}{#5[\pi_1~#2/#3]}$}
  {}
}
\newcommand{\PiRight}[1][\Gamma]{\PiRightFull[#1]{t}{x}{T}{U}}


\newcommand{\SubsetFull}[4]{
  \BAX{Subset}
  {$\tgen{#1}{#3}{\Set}$}
  {$\tgen{#1, #2 : #3}{#4}{\Prop}$}
  {$\tgen{#1}{\mysubset{#2}{#3}{#4}}{\Set}$}
  {$#2 `; #1$}
}
\newcommand{\SubsetR}[1][\Gamma]{\SubsetFull{#1}{x}{U}{P}}


\newcommand{\SubsumFull}[5]{
  \TAX{Subsum}
  {$\tgen{#1}{#4}{#5}$}
  {$\tgen{#1}{#2}{#3}$}
  {$#5 \sub #2$} % #1 \subt 
  {$\tgen{#1}{#4}{#2}$}
  {}
}
\newcommand{\Subsum}[1][\Gamma]{\SubsumFull{#1}{T}{s}{t}{U}} 

\def\Coerce{\Subsum}

\newcommand{\ConvFull}[5]{
  \TAX{Conv}
  {$\tgen{#1}{#2}{#3}$}
  {$\tgen{#1}{#4}{#5}$}
  {$#5 \eqbr #2$}
  {$\tgen{#1}{#4}{#2}$}
  {}
}
\newcommand{\Conv}[1][\Gamma]{\ConvFull{#1}{T}{s}{t}{U}}

\newcommand{\typedRules}{
  \begin{center}
    \def\seq{\typed}
    \def\fCenter{\wf}
    \WfEmpty \DP\quad
    \WfVar \DP
    
    \def\fCenter{\typed}
    \vspace{\infvspace}
    \PropSet\DP

    \vspace{\infvspace}
    \Var\DP
    
    \vspace{\infvspace}
    \Prod\DP
    
    \vspace{\infvspace}
    \Abs\DP

    \vspace{\infvspace}
    \App\DP

    \vspace{\infvspace}
    \SigmaR\DP

    \vspace{\infvspace}
    \SumDep\DP
    
    \vspace{\infvspace}
    \PiLeft\DP
    \quad
    \PiRight\DP

    \vspace{\infvspace}
    \SubsetR\DP

    \vspace{\infvspace}
    \Subsum\DP
      
  \end{center}
}

\def\typedFig
{
\begin{figure}[tb]
  \typedRules
  \caption{Calcul de coercion par pr�dicats - version d�clarative}
  \label{fig:typing-decl-rules}
\end{figure}
}

%%% Local Variables: 
%%% mode: latex
%%% TeX-master: "subset-typing"
%%% LaTeX-command: "TEXINPUTS=\"style:$TEXINPUTS\" latex"
%%% End: 

\def\SubConv{
  \UAX{SubConv}
  {$T \eqbi U$}
  {$T \sub U$}
  {} 
}

\def\SubRefl{
  \UAX{SubRefl}
  {}
  {$S \seq S$}
  {}
}

\def\SubTrans{
\BAX{SubTrans}
{$S \seq T$}
{$T \seq U$}
{$S \seq U$}
{}
} 

\def\SubProd{
\BAX{SubProd}
{$U \seq T$} %"<|-|>"
{$V \seq W$}
{$\Pi x : T.V \seq \Pi x : U.W$}
{}
}

\def\SubSigma{
  \BAX{SubSigma}
  {$T \seq U$}
  {$V \seq W$}
  {$\Sigma x : T. V \seq \Sigma y : U. W$}
  {}
}

\def\SubProof{
  \UAX{SubProof}
  {$U \seq V$}
  {$U \seq \subset{x}{V}{P}$}
  {}
}

\def\SubSub{
  \UAX{SubSub}
  {$U \seq V$}
  {$\subset{x}{U}{P} \seq V$}
  {}
}

\def\subtdFig{
\begin{figure}[ht]
  \begin{center}
    \def\fCenter{\subd}
    
    \vspace{\infvspace}
    \SubConv\DP

    \vspace{\infvspace}
    \SubTrans\DP

    \vspace{\infvspace}
    \SubProd\DP

    \vspace{\infvspace}
    \SubSigma\DP
    
    \vspace{\infvspace}
    \SubProof\DP
    
    \vspace{\infvspace}
    \SubSub\DP
    
  \end{center}
  \caption{Coercion par pr�dicats - version d�clarative}
  \label{fig:subtyping-decl-rules}
\end{figure}
}

\def\subtdShort{
\begin{figure}[ht]
  \begin{center}
    \def\fCenter{\subd}
    \SubConv\DP
    \SubProof\DP
    \SubSub\DP

    \vspace{1cm}
    \SubProd\DP
    \SubSigma\DP
    %\SubTrans\DP
  \end{center}
  \caption{Coercion par pr�dicats - version d�clarative}
  \label{fig:subtyping-decl-rules-short}
\end{figure}
}

%%% Local Variables: 
%%% mode: latex
%%% TeX-master: "subset-typing"
%%% LaTeX-command: "TEXINPUTS=\"style:$TEXINPUTS\" latex"
%%% End: 


\def\AppA{
\TAX{App}
{$\talgo{`G}{f}{T} \quad \mualgo(T) = \Pi x : V. W : s$}
{$\talgo{`G}{u}{U} \quad \talgo{`G}{U, V}{s'}$}
{$\subalgo{`G}{U}{V} $}
{$\talgo{`G}{(f u)}{W [ u / x ]}$}
{}
}

\def\LetSumA{
  \TAX{LetSum}
  {$`G \seq t : S$}
  {$\mualgo(S) = `S x : T. U $}
  {$`G, x : T, u : U \seq v : V $}
  {$`G \seq \letml~(x, u) = t~\inml~v : V$}
  {$x,y `; `G$}
}

\def\PiLeftA{
  \BAX{PiLeft}
  {$\tgen{`G}{t}{S}$}
  {$\mualgo(S) = `S x : T.U$}
  {$\tgen{`G}{\pi_1~t}{T}$}
  {}
}

\def\PiRightA{
  \BAX{PiRight}
  {$\tgen{`G}{t}{S}$}
  {$\mualgo(S) = `S x : T.U$}
  {$\tgen{`G}{\pi_2~t}{U[\pi_1~t/x]}$}
  {}
}


\def\SumInfA{
  \TAX{SumInf}
  {$`G \seq t : T $}
  {$`G \seq u : U $}
  {$`G \seq \Sigma \_ : T.U : s$}
  {$`G \seq (t, u) : \Sigma \_ : T.U$}
  {}
}

\def\SumDepAold{
  \QAX{SumDep}
  {$`G \seq t : T$}
  {$`G \seq u : U'$}
  {$`G \seq \Sigma x : T.U : s$}
  {$`G \seq U' : s \quad U' \suba U[t/x]$}
  {$`G \seq (x \coloneqq~t, u : U) : \Sigma x : T.U$}
  {}
}

\def\SumDepA{
  \QAX{SumDep}
  {$`G \seq t : T$}
  {$`G \seq u : U'$}
  {$`G \seq \Sigma x : T.U : s$}
  {$`G \seq U' : s \quad U' \suba U[t/x]$}
  {$`G \seq \pair{\Sigma x : T.U}{t}{u} : \Sigma x : T.U$}
  {}
}

\def\typeaFig{
\begin{figure}[t]
  \begin{center}
    \def\fCenter{\wf}
    \def\type{\typea}
    \def\subt{\subta}
    \def\sub{\suba}
    
    \WfEmpty\DP
    \quad
    \WfVar\DP
    
    \def\fCenter{\typea}
    \vspace{\infvspace}
    \PropSet\DP
%    \quad
%    \TypeType\DP
    
    \vspace{\infvspace}
    \Var\DP
    
    \vspace{\infvspace}
    \Prod\DP
    
    \vspace{\infvspace}
    \Abs\DP

    \vspace{\infvspace}
    \AppA\DP

%    \vspace{\infvspace}
%    \LetIn\DP
    
    \vspace{\infvspace}
    \SigmaR\DP

%    \vspace{\infvspace}
%    \SumInfA\DP

    \vspace{\infvspace}
    \SumDepA\DP
    
    \vspace{\infvspace}
    \PiLeftA\DP
    \quad
    \PiRightA\DP
    %\LetSumA\DP
    
    \vspace{\infvspace}
    \Subset\DP
  \end{center}
  \caption{Calcul de coercion par pr�dicats - version algorithmique}
  \label{fig:typing-algo-rules}
\end{figure}
}

\def\typemuaFig{
  \begin{figure}[ht]
    \begin{eqnarray*}
      \mualgo'(\mysubset{x}{U}{P}) & "=>" & \mualgo'(\downarrow{U}) \\
      \mualgo'(x)                & "=>" & x \\
      \\
      \mualgo(x) & "=>" & \mualgo' (\downarrow{x})
    \end{eqnarray*}
    \caption{D�finition de $\mualgo$}
    \label{fig:mualgo-definition}
\end{figure}
}

%%% Local Variables: 
%%% mode: latex
%%% TeX-master: "subset-typing"
%%% LaTeX-command: "TEXINPUTS=\"style:$TEXINPUTS\" latex"
%%% End: 

%% \def\SubProd{
%%   \BAX{Sub-Prod}
%%   {$`G \seq U \sub T$} %"<|-|>"
%%   {$`G, x : U \seq V \sub W$}
%%   {$`G \seq \Pi x : T.V \sub \Pi x : U.W$}
%%   {}
%% }

%% \def\SubSigma{
%%   \BAX{Sub-Sigma}
%%   {$`G \seq T \sub U$}
%%   {$`G \seq V \sub W$}
%%   {$`G \seq \Sigma x : T. V \sub \Sigma x : U. W$}
%%   {}
%% }

\def\SubHnf{
  \UAX{SubHnf}
  {$\hnf{T}~\sub \hnf{U}$}
  {$T \sub U$}
  {} 
}

\def\SubLeft{
  \BAX{Sub-Left}
  {$`G \seq U \sub V$}
  {$`G \type P : \Pi x : U. \Prop$}
  {$`G \seq \subset{x}{U}{P} \sub V$}
  {}
}
    
\def\SubRight{
  \UAX{Sub-Right}
  {$`G \seq T \sub U$}
                                %{$`G \judgetype h : P~p$}
  {$`G \seq T \sub \subset{x}{U}{P}$}
  {}
}

\def\subtaFig{
\begin{figure}[ht]
  \begin{center}
    \def\fCenter{\suba}
    \def\type{\typea}
    \def\sub{\suba}

    \SubConv\DP
    \quad
    \SubHnf\DP
    \vspace{\infvspace}

    \SubProd\DP

    \vspace{\infvspace}
    \SubSigma\DP
    
    \vspace{\infvspace}
    \SubProof\DP

    \vspace{\infvspace}    
    \SubSub\DP
  \end{center}
  \caption{Coercion par pr�dicats - version algorithmique}
  \label{fig:subtyping-algo-rules}
\end{figure}
}

%%% Local Variables: 
%%% mode: latex
%%% TeX-master: "subset-typing"
%%% LaTeX-command: "TEXINPUTS=\"style:$TEXINPUTS\" latex"
%%% End: 

\def\typec{\vdash_{CCI}}

\newcommand{\timpl}[6]{#1 \typea #2 : #3 "~>" #4 \typec #5 : #6}
\newcommand{\subimpl}[4]{#1 \typec #2 : #3 \sub #4}

\def\muterm{\mu_{\mbox{term}}}

\def\WfAtomI{
  \UAX{Wf-Atom}
  {}
  {$\wf [] "~>"~\wf []$}
  {}
}  
  
\def\WfVarI{
  \UAX{Wf-Var}
  {$\timpl{`G}{A}{s}{`G'}{A'}{s}$}
  {$\wf `G, x : A "~>"~\wf `G', x : A'$}
  {$s `: \{ \Set, \Prop, \Type(i) \}$}
}

\def\PropSetI{
  \UAX{PropSet}
  {$\wf `G "~>" ~\wf `G'$}
  {$\timpl{`G}{s}{\Type(0)}{`G'}{s}{\Type(0)}$}
  {$s `: \{ \Prop, \Set \}$} 
}

\def\TypeTypeI{
  \UAX{Type}
  {$\wf `G "~>" \wf `G$}
  {$\timpl{`G}{\Type(i)}{\Type(i + 1)}{`G'}{\Type(i)}{\Type(i + 1)}$}
  {}
}

\def\VarI{
  \BAX{Var}
  {$\wf `G "~>"~\wf `G'$}
  {$x : A `: `G "~>" x : A' `: `G'$}
  {$\timpl{`G}{x}{A}{`G'}{x}{A'}$}
  {}
}

\def\ProdI{
  \BAX{Prod}
  {$\timpl{`G}{T}{s1}{`G'}{T'}{s1}$}
  {$\timpl{`G, x : T}{U}{s2}{`G', x : T'}{U'}{s2}$}
  {$\timpl{`G}{\Pi x : T.U}{s2}{`G'}{\Pi x : T'.U'}{s2}$}
  {$(s1, s2) `: \text{{\sc Sort}}$}
}

\def\AbsI{
  \BAX{Abs}
  {$\timpl{`G}{\Pi x : T. U}{s}{`G'}{\Pi x : T'. U'}{s}$}
  {$\timpl{`G, x : T}{M}{U}{`G', x : T'}{M'}{U'}$}
  {$\timpl{`G}{\lambda x : T. M}{\Pi x : T.U}
    {`G'}{\lambda x : T'. M'}{\Pi x : T'.U'}$}
  {}
}

\def\AppI{
  \QAX{App}
  {$\timpl{`G}{f}{T}{`G'}{f'}{T'}$}
  {$\mu~T' `= (\pi, \Pi x : V'. W')$}
  {$\timpl{`G}{u}{U}{`G'}{u'}{U'}$}
  {$\subimpl{`G'}{c}{U'}{V'}$}
  {$\timpl{`G}{f u}{W[u/x]}{`G'}{(\pi~f')~(c~u')}{W'[ c~u' / x ]}$}
  {}
}

\def\LetInI{
  \BAX{LetIn}
  {$\timpl{`G}{t}{T}{`G'}{t'}{T'}$}
  {$\timpl{`G, x : T}{v}{V}{`G', x : T'}{v'}{V'}$}
  {$\timpl{`G}{\letml x = t \inml v}{V[t / x]}
    {`G'}{\letml x = t' \inml v'}{V'[t' / x]}$}
  {}
}

\def\SigmaRI{
  \BAX{Sigma}
  {$\timpl{`G}{T}{s1}{`G'}{T'}{s1}$}
  {$\timpl{`G, x : T}{U}{s2}{`G', x : T'}{U'}{s2}$}
  {$\timpl{`G}{\Sigma x : T.U}{s2}{`G'}{\Sigma x : T'.U'}{s2} $}
  {$(s1, s2) `: \matht{Sort}$}
}


\def\SumInfI{
  \TAXWide{SumInf}
  {$\timpl{`G}{t}{T}{`G'}{t'}{T'}$}
  {$\timpl{`G}{u}{U}{`G'}{u'}{U'}$}
  {$\timpl{`G}{\Sigma \_ : T.U}{s}{`G'}{\Sigma \_ : T'.U'}{s}$}
  {$\timpl{`G}{(t, u)}{\Sigma \_ : T.U}{`G'}{(t', u')}{\Sigma \_ : T'.U'}$}
  {}
}

\def\SumDepI{
  \TAXWide{SumDep}
  {$\timpl{`G}{t}{T}{`G'}{t'}{T'}$}
  {$\timpl{`G}{u}{U[t/x]}{`G'}{u'}{U'[t'/x]}$}
  {$\timpl{`G}{\Sigma x : T.U}{s}{`G'}{\Sigma x : T'.U'}{s}$}
  {$\timpl{`G}{(t, u : U)}{\Sigma x : T.U}{`G'}{(t', u')}{\Sigma x : T'.U'}$}
  {}
}
 
\def\LetSumI{
  \BAX{LetSum}
  {$\timpl{`G}{t}{\Sigma x : T. U}{`G'}{t'}{\Sigma x : T'.U'}$}
  {$\timpl{`G, x : T, u : U}{v}{V}{`G, x : T', u : U'}{v'}{V'}$}
  {$\timpl{`G}{\letml (x, u) = t \inml v}{V}
    {`G'}{\letml (x, u) = t' \inml v'}{V'}$}
  {}
}

\def\SubsetI{
  \BAX{Subset}
  {$\timpl{`G}{U}{\Type}{`G'}{U'}{\Type}$}
  {$\timpl{`G, x : U}{P}{\Prop}{`G', x : U'}{P'}{\Prop} $}
  {$\timpl{`G}{\subset{x}{U}{P}}{\Type}{`G'}{\subset{x}{U'}{P'}}{\Type}$}
  {$$}
}

\def\marginleft{0em}

\def\typeiFig{
\begin{figure}[ht]
  \begin{center}
    \def\fCenter{\wf}
    \def\type{\typec}
    
    \WfAtomI\DP

    \vspace{\infvspace}
    \WfVarI\DP
    
    \vspace{\infvspace}
    \PropSetI\DP
    
    \vspace{\infvspace}
    \VarI\DP
    
    \vspace{\infvspace}
    \ProdI\DP
    
    \vspace{\infvspace}
    \AbsI\DP
    
    \vspace{\infvspace}
    \AppI\DP

    \vspace{\infvspace}
    \LetInI\DP
    
    \vspace{\infvspace}
    \SigmaRI\DP

    \vspace{\infvspace}
    \SumInfI\DP

    \vspace{\infvspace}
    \SumDepI\DP
    
    \vspace{\infvspace}
    \LetSumI\DP
    
    \vspace{\infvspace}
    \SubsetI\DP
  \end{center}
  \label{typing-impl-rules}
  \caption{R�ecriture du typage vers \Coq}
\end{figure}
}

\def\typemuiFig{
\begin{figure}
  \begin{eqnarray*}
    (f, \subset{x}{U}{P}))   & "=>" & \letml (f, t) = \muterm~(f, U) \inml (f `o
    \pi_1, t) \\
    x                        & "=>"  & x
  \end{eqnarray*}
  \label{muimpl-definition}
  \caption{D�finition de $\muterm$}
\end{figure}
}

%%% Local Variables: 
%%% mode: latex
%%% TeX-master: "subset-typing"
%%% LaTeX-command: "TEXINPUTS=\"style:$TEXINPUTS\" latex"
%%% End: 

\def\ctxdot{\text{\textbullet}}

\newcommand{\SubConvI}[1][\Gamma]{
\UAX{SubConv}
{$T \eqbr U$}
{$\subimpl{#1}{\ctxdot}{T}{U}$}
{$T = \hnf{T} `^ T "/=" \Pi, \Sigma, \{|\} `^ U = \hnf{U}$}
}

\newcommand{\SubConvIs}[1][\Gamma]{
\UAX{SubConv}
{$T \eqbr U$}
{$\subimpl{#1}{\ctxdot}{T}{U}$}
{} 
}

\newcommand{\SubHnfI}[1][\Gamma]{
\UAX{SubHnf}
{$\subimpl{#1}{c}{\hnf{T}}{\hnf{U}}$}
{$\subimpl{#1}{c}{T}{U}$}
{$T "/=" \hnf{T} `V U "/=" \hnf{U}$} 
}

\newcommand{\SubProdI}[1][\Gamma]{%
  \BAX{SubProd}
  {$\subimpl{#1}{c_1}{U}{T} : s$}
  {$\subimpl{#1, x : U}{c_2}{V}{W}$}  
  {$\subimpl{#1}{\lambda x : \ip{U}{#1}. c_2[\ctxdot~(c_1[x])]}
    {\Pi x : T.V}{\Pi x : U.W}$}
  {}
}

\newcommand{\SubSigmaI}[1][\Gamma]{%
\BAX{SubSigma}
{$\subimpl{#1}{c_1}{T}{U}$}
{$\subimpl{#1, x : T}{c_2}{V}{W}$}
{$\subimpl{#1}{\pair{\ip{\Sigma x : U.W}{`G}}{c_1[\pi_1~\ctxdot]}{c_2[\pi_2~\ctxdot][\pi_1~\ctxdot/x]}}
  {\Sigma x : T. V}{\Sigma x : U. W}$}
{}
}

\newcommand{\SubSubI}[1][\Gamma]{%
\UAX{SubSub}
{$\subimpl{#1}{c}{U}{T}$}
{$\subimpl{#1}{c[\eltpit~\ctxdot]}{\mysubset{x}{U}{P}}{T}$}
{$T = \hnf{T} `^ \hnf{T} "/=" \{|\}$}
}

\newcommand{\SubSubIs}[1][\Gamma]{%
\UAX{SubSub}
{$\subimpl{#1}{c}{U}{T}$}
{$\subimpl{#1}{c[\eltpit~\ctxdot]}{\mysubset{x}{U}{P}}{T}$}
{}
}
   
\newcommand{\SubProofI}[1][\Gamma]{%
  \BAX{SubProof}
  {$\subimpl{#1}{c}{T}{U}$}
  {$\timpl{#1}{\mysubset{x}{U}{P}}{\Set}$}
  {$\subimpl{#1}
    {\elt{\ip{U}{#1}}{\ip{\lambda x : U.P}{#1}}{c}
      {\ex{\ip{P}{#1, x : U}[c/x]}}}
  {T}{\mysubset{x}{U}{P}}$}
{$T = \hnf{T}$}
}

\def\subtiFig{
\begin{figure}[ht]
  \begin{center}
    \def\fCenter{\subtd}

    \vspace{\infvspace}
    \SubConvI\DP
    
    \vspace{\infvspace}
    \SubHnfI\DP

    \vspace{\infvspace}
    \SubProdI\DP

    \vspace{\infvspace}
    \SubSigmaI\DP
    
    \vspace{\infvspace}
    \SubSubI\DP
    
    \vspace{\infvspace}
    \SubProofI\DP
    
  \end{center}
  \caption{R��criture de la coercion vers \CCI}
  \label{fig:coerce-impl-rules}
\end{figure}
}

\def\subtisc{
  \begin{center}
    \def\fCenter{\subti}
    \SubConvIs\DP
    \quad
    \SubHnfI\DP

    \vspace{\infvspace}
    \SubProdI\DP
    
    \vspace{\infvspace}
    \SubSigmaI\DP

    \vspace{\infvspace}
    \SubSubIs\DP

    \vspace{\infvspace}
    \SubProofI\DP
  \end{center}}

\def\subtis{
\begin{figure}[ht]
  \subtisc
  \caption{R��criture de la coercion vers \CCI}
  \label{fig:coerce-impl-rules-short}
\end{figure}
}

%%% Local Variables: 
%%% mode: latex
%%% TeX-master: "subset-typing"
%%% LaTeX-command: "TEXINPUTS=\"style:.:\" latex"
%%% End: 


\newcommand{\src}[1]{\texttt{#1}}
\newcommand{\srcm}[1]{\text{\texttt{#1}}}

\def\thetitle{Coercion par pr�dicats en \Coq}

\pagestyle{plain}
\fancyhead[RO,L]{\thetitle}
\fancyfoot[C]{\thepage}

\title{\thetitle}

\author{Matthieu Sozeau \\ sous la direction de Christine Paulin \\
  \'Equipe {\sc Demons}, {\sc LRI}}

\date{\today}

%%% Local Variables: 
%%% mode: latex
%%% TeX-master: t
%%% End: 


\newboolean{showlog}
\setboolean{showlog}{false}

\title{\thetitle}

\author{Matthieu Sozeau}

\date{\today}

\begin{document}

%\maketitle
\thispagestyle{empty}
{\Large

\vfill
\begin{center}
Master Parisien de Recherche en Informatique\\
Universit� Paris VII\\
Rapport de stage
{\Huge \bf
\rule{\textwidth}{2pt}

\vspace*{0.8cm}
Coercion par pr�dicats en \Coq{}
\vspace*{0.8cm}

\rule{\textwidth}{2pt}}\\
\vspace{1cm}
Matthieu {\sc Sozeau}%\\
\end{center}

\vspace*{2cm}
\noindent Stage effectu� au \LRI{} dans l'�quipe \Demons{},
sous la direction de: \\
Christine {\sc Paulin-Mohring}
}
%%% Local Variables: 
%%% mode: latex
%%% TeX-master: "subset-typing"
%%% LaTeX-command: "TEXINPUTS=\"style:$TEXINPUTS\" latex"
%%% End: 



\chapter*{Remerciements}
Je remercie l'�quipe \Demons{} du \LRI{} pour son accueil chaleureux.
Je remercie en particulier Christine Paulin-Mohring pour son attention
et sa pr�sence constantes et Jean-Christophe Filli�tre pour son aide
pr�cieuse lors du d�veloppement, son dynamisme et sa passion pour le
domaine de la programmation qui n'ont pas �t� sans influence sur mon
orientation vers la recherche depuis mon arriv�e � la facult� d'Orsay.
Je suis reconnaissant envers mes colocataires th�sards pour m'avoir
support� jusqu'ici. Je remercie enfin Claire et Jo�lle sans qui ma vie
serait bien diff�rente.


\chapter*{R�sum�}
  \Coq{} est un assistant de preuve d'une grande expressivit� pour le
  d�veloppement de th�ories math�\-matiques et informatiques, ce qui permet
  de traiter un large �ventail de probl�mes. Le langage de \Coq{},
  constitu� d'un noyau fonctionnel de type \ML{} enrichi par des types
  d�pendants, permet de sp�cifier, v�rifier puis
  extraire des programmes corrects par construction. En contrepartie, les
  programmes sont plus difficiles � �crire et maintenir que dans un pur
  langage de programmation de type \ML, puisqu'ils m�langent les parties
  logiques et calculatoires. Pour rem�dier �
  ce probl�me, on propose le nouveau langage de description de
  programmes \lng{},
  s'int�grant parfaitement � l'environnement de d�veloppement existant,
  qui permet de donner dans un premier temps le code et la sp�cification
  du programme et d'engendrer ensuite ses conditions de correction. Nous
  allons tout d'abord �tudier le langage \lng{}, construire un
  algorithme de typage pour ses termes et montrer comment les traduire
  en termes � trous valides dans le syst�me \Coq{}. Notre contribution
  revient � int�grer un langage avec types d�pendant � la \PVS{} dans
  \Coq{}.

\newpage

\tableofcontents
\listoffigures
\newpage

\chapter{Introduction}
Nous nous pla\c cons dans le cadre du syst�me d'aide � la preuve \Coq{},
auquel nous souhaitons int\'egrer un langage de programmation plus souple
que le langage actuellement utilis\'e.

\section{Pr\'esentation de \Coq}

\Coq~est un assistant de preuve dont la premi�re version date de 1985,
et qui est aujourd'hui d\'evelopp\'e dans le projet \PCRI{} \LogiCal{} (INRIA, LIX,
LRI, CNRS). Originellement bas\'e sur le Calcul des Constructions (\CoC),
il a \'et\'e \'etendu au \CCIfull~(\CCI) et contient aujourd'hui de
nombreuses am\'eliorations telles qu'un syst�me sophistiqu\'e d'extraction
de programmes ou encore des proc\'edures de d\'ecision pour automatiser la
preuve.

Le d\'eveloppement de \Coq~est intimement li\'e � l'isomorphisme de {\sc
Curry-Howard} qui montre le lien entre logique intuitionniste et calcul. Cet
isomorphisme \'etablit qu'\'elaborer une preuve du calcul propositionnel
intuitionniste est \'equivalent � \'ecrire un terme du
\lc~simplement typ\'e (\lcst). Par exemple, montrer que $A "=>" A$ pour un
certain $A$ revient � \'ecrire la fonction identit\'e $\lambda x : A. x$ qui
a bien pour type $A "->" A$. Chaque logique constructive est donc
associ\'ee � un \lc{} particulier. Dans \Coq{}, on utilise cet
isomorphisme pour v\'erifier les preuves. Le noyau est simplement un
typeur pour \CCI{}. Si on peut typer un terme $t$ de type $T$, alors on est
assur\'e d'avoir trouv\'e une preuve constructive $t$ de la formule $T$.
Cette dualit\'e se refl�te aussi � l'utilisation de \Coq{} o\`u l'on a 
les deux visions: logique (d\'eveloppement math\'ematique, preuve) et 
calcul (d\'eveloppement informatique, programme).

\subsection{Preuve}
\Coq{} est utilis\'e le plus souvent pour \'elaborer des th\'eories
math\'ematiques prouv\'ees m\'ecaniquement. Dans cette optique, l'utilisateur 
mod\'elise un probl�me par des structures math\'ematiques et veut prouver
certaines propri\'et\'es sur ce mod�le (par exemple la preuve du th\'eor�me
des quatres couleurs r\'ecemment termin\'ee \cite{Gonthier4col} utilisait 
des r\'esultats de g\'eom\'etrie alg\'ebrique).

Pour prouver un but sous certaines hypoth�ses, on utilise des
tactiques qui simulent un raisonnement d\'eductif pour l'utilisateur.
Celles-ci permettent par exemple d'introduire une hypoth�se: pour le but
$A "=>" A$ on peut introduire l'hypoth�se $H : A$ pour obtenir le but
$A$ ; ou bien d'en appliquer une (ou tout autre r\'esultat d\'ej� \'etabli): 
en appliquant l'hypoth�se $H$ on prouve le but directement. 
Ces tactiques peuvent \^etre d'une complexit\'e arbitraire (r\'e\'ecritures,
proc\'edures de d\'ecision pour l'arithm\'etique, etc \ldots).

Les tactiques utilis\'ees pour cr\'eer des preuves ne sont en fait
qu'une interface au-dessus du noyau de \Coq{} qui se r\'eduit � un typeur
pour \CCI. A la fin d'une preuve, on a en effet construit un terme 
($\lambda x : A. x$ dans notre exemple) que l'on va soumettre au typeur
dont le but est de v\'erifier qu'il est bien de type $A "->" A$. Les
tactiques peuvent cependant �tre arbitrairement complexes (r�solution
d'�quations de l'arithm�tique, r�ecritures, etc...).

\subsection{Programmes}
D'un point de vue preuve de programmes, on a donc un environnement qui
permet de v\'erifier qu'un programme (un terme du calcul) v\'erifie une
certaine sp\'ecification (son type). Les types d\'ependants permettent de
sp\'ecifier fortement les termes. Par exemple, la fonction $\sdef{div} :
\nat "->" \nat "->" \nat * \nat$ de \ML{} est plus fortement sp\'ecifi\'ee en 
\Coq{} par $\sdef{div} : \nat "->" \subset{x}{\nat}{x \neq 0} "->" \nat
* \nat$.
Seulement, on ne peut pas \'ecrire simplement un programme \ML{} et
donner sa sp\'ecification forte. Comme on a enrichi les types, on doit 
aussi enrichir les termes avec des termes de preuve, inutiles au 
calcul mais n\'ecessaires pour garantir la
correction logique du programme et le fait que la machine puisse
v\'erifier m\'ecaniquement la correction (annotations,...). Par exemple,
 si l'on veut appeler $\sref{div}$ sur $1$ et $n$ (pour $n : \nat$), il
faut construire un terme $\sref{div}~1~(\sref{elt}~(\lambda x : \nat
"->" x \neq 0)~x~p)$ o\`u $p$ est une preuve de $n \neq 0$. 
En effet, en \Coq~on a:
\begin{definition}[Type sous-ensemble]
  \label{subset-type-def}
  $\subset{x}{T}{P}$ est le type des termes de type $T$ v\'erifiant la
  propri\'et\'e $P$.
\end{definition}

Un objet de type $\subset{x}{T}{P}$ peut �tre vu comme une paire
$(x,p)$ ou $x$ est un objet de type $T$ (le t�moin) et $p$ une preuve
de $P[t/x]$. Seulement, le typeur a besoin de plus d'information,
l'annotation $\lambda x : \nat "->" x \neq 0$ est n�cessaire pour avoir
un syst�me d'inf�rence d�cidable (on ne peut
pas inf\'erer la propri�t� $P$ � partir de sa preuve $p$ puisqu'il est de type
$P[t/x]$). On voit donc ici que l'on doit rajouter de nombreuses
informations d'ordre logique � nos termes.

A l'inverse, on peut extraire un programme de toute preuve en \'eliminant les
parties logiques et en ne conservant que la partie calculatoire d'un terme.

\section{Motivation}

\Coq{} permet de d\'evelopper des programmes complexes,
de leur donner des sp\'ecifications fortes et de les v\'erifier
automatiquement. On peut m\^eme extraire de ces d\'eveloppements des
programmes corrects par construction. Il y a cependant certaines
difficult\'es � d\'evelopper en \Coq{} que nous allons \'etudier maintenant.

\subsection{Un langage trop expressif?} 

Le langage de \CCI{} permet de bien sp\'ecifier des fonctions non
triviales, par exemple, si l'on d\'efinit une fonction de division
euclidienne en \ML{} elle aurait le type: $  \valml~\sdef{div} : \nat "->" \nat "->"
\nat * \nat$. En \Coq, on peut d\'efinir:
\[\Definition~\sdef{div} : \forall a : \nat, \forall b : \nat,
b \neq 0 "->" \{~q : \nat~\&~\{~r : \nat `| r < b `^ a = b * q + r~\} \}\]

Les types d\'ependants permettent de bien relier les entr\'ees aux sorties
et donc de sp\'ecifier les programmes aussi fortement que l'on d\'esire, 
mais aussi de fa\c con concise. En revanche, le terme de preuve 
correspondant � \sref{div} est nettement plus long (de l'ordre de 60 lignes), 
et ne peut simplement pas \^etre \'ecrit d'une traite 
sans une expertise approfondie. Pour rem\'edier � ce
probl�me, on utilise des tactiques qui permettent d'\'ecrire la preuve/programme
incr\'ementalement (\note{figure ici?}voir figure \ref{fig:euclid-script}
page \pageref{fig:euclid-script}). L'inconv\'enient de cette
m\'ethode est que l'on n'obtient pas toujours le programme d\'esir\'e
au d\'epart, puisque les tactiques cachent profond\'ement leur effet sur le
terme de preuve. Certaines techniques de r\'e\'ecriture peuvent aussi
g\'en\'erer des termes de complexit\'e algorithmique bien moins optimale que
ce que l'utilisateur \'ecrirait. Cependant ce mode de fonctionnement est
utile et utilis\'e par la majorit\'e des utilisateurs de \Coq{} avec succ�s
(certification d'un compilateur \C, th\'eor�me des quatres couleurs
\cite{Gonthier4col}, \ldots).

\subsection{M\'elange logique et calcul}
Une difficult\'e essentielle lorsque l'on veut permettre � des
utilisateurs non experts de d\'evelopper dans \Coq{} est le ``m\'elange
des genres'' permanent entre logique et calcul. Pour appliquer une
fonction division qui attend un d\'enominateur diff\'erent de $0$ par
exemple, il faut passer � la fois l'argument lui m\^eme, mais aussi une
preuve de sa non-nullit\'e. Lorsque l'on a l'habitude de programmer, \c ca
n'est pas la chose la plus naturelle et l'on aimerait pouvoir d\'ecoupler
les parties codage et preuve pour simplement diviser le probl�me. Les
parties logiques pourront souvent \^etre r\'esolues automatiquement par des
tactiques.

\subsection{Objectif}
A long terme, on souhaite permettre � un utilisateur de programmer dans
un langage proche de \ML{} et de prouver ses programmes dans un deuxi�me
temps � l'aide de \Coq{} et ses tactiques. Une fois les preuves
termin\'ees, on peut extraire un programme correct par construction et
essentiellement \'equivalent � celui de d\'epart ou le r\'eutiliser facilement
dans l'environnement \Coq{}.

\section{Travaux Connexes}

La preuve de programmes fonctionnels est un domaine de recherche
actif. L'id\'ee d'\'etendre les langages \ML{} avec des types d\'ependants a
\'et\'e d\'evelopp\'ee dans \DML{} \cite{XiPfenning1999DTP}, \Cayenne{}
\cite{Augustsson99} et \Omegapdx{} \cite{Omega}. Il
s'agit dans ces travaux de faire un langage dont l'inf\'erence est d\'ecidable, donc
de restreindre les types d\'ependants � des domaines o\`u l'on peut faire de
la preuve automatique (\DML{}) ou bien d'accro�tre la puissance du langage
pour rendre l'utilisation des types d\'ependants plus ais\'ee (\Cayenne{}
a la r\'ecursivit\'e g\'en\'erale par exemple) mais en perdant l'id�e de
correction (et en perdant m�me la d�cidabilit� du typage pour
\Cayenne{}). 

Nous prenons le
contre-pied de ces travaux en acceptant de g\'en\'erer des obligations de
preuve et en essayant de trouver un langage le plus proche de \ML{}
possible tout en retenant la puissance de \Coq{} et des types
d\'ependants. Nous pr\'esentons maintenant des travaux directement li\'es �
notre contribution.


\subsection{La tactique \Program}
Il existe un travail r\'ealis\'e dans \Coq{} couvrant une partie de nos objectifs.
D\'evelopp\'ee par Catherine Parent \cite{conf/mpc/Parent95}, 
la tactique \Program{} permettait de synth\'etiser des preuves � partir de
programmes. L'id\'ee \'etait de trouver un langage de programmation
suffisamment restrictif pour r\'ealiser une inversion de l'extraction, 
c'est-�-dire, � partir d'un terme essentiellement calculatoire
(des annotations \'etaient n\'ecessaires), retrouver un terme de preuve
r\'ealisant la sp\'ecification donn\'ee. 
A partir de l�, on \'etait assur\'e que le programme extrait serait
identique � celui que l'on \'ecrivait pour sa partie informative. Cette
m\'ethode g\'en\'erale avait l'inconv\'enient d'\^etre peu intuitive et de ne pas
s'int\'egrer � l'environnement \Coq. En particulier, appeler une fonction
d�finie avec \Program{} est aussi difficile qu'avec n'importe quelle
d�finition \Coq{}. Il n'existe pas de m�canisme permettant de faire la
distinction de phase codage/preuve, qui permettrait de faire de simples
appels et de v�rifier ensuite que les arguments sont valides, ce qui est
beaucoup plus naturel lorsque l'on programme.
Li\'ee � l'extraction interne qui
a disparu dans les derni�res versions de \Coq{} (remplac\'ee par la
contribution de Pierre Letouzey \cite{LetouzeyPhD}), elle n'est plus 
maintenue aujourd'hui.

\subsection{Types sous-ensemble}
Plut\^ot que de continuer dans la m\^eme direction, nous avons 
cherch\'e � assouplir le syst�me. L'assistant de preuve \PVS{}
\cite{PVS-Semantics:TR} aux capacit\'es similaires � \Coq{}, int�gre un
m\'ecanisme d\'enomm\'e \ps{} que nous allons pr\'esenter maintenant.

Les types sous-ensembles (d�finition \ref{subset-type-def}) sont d'une
grande utilit\'e pour la sp\'ecification
de programmes, par exemple pour les pr\'e-conditions:
$\Definition~\sref{div} : \nat "->" \subset{x}{nat}{x \neq 0} "->" \nat
* \nat$.

L'id\'ee du \ps{} impl\'ement\'e dans \PVS{}
\cite{Shankar&Owre:WADT99,Rushby98:TSE} est de consid\'erer tout objet de
type $T$ comme un objet de type $\subset{x}{T}{P}$ pour $P$ vraie et
vice-versa. Comme tout objet $t$ de type $T$ ne v\'erifie pas forc\'ement la
propri\'et\'e $P$, on g\'en�re des ``\emph{Type-checking conditions}'' (\TCC), c'est
� dire que l'on demande � l'utilisateur de prouver $P[t/x]$ pour assurer
que le programme est correct.

\subsection{Coercions}
\PVS{} n'a pas la m\^eme architecture que \Coq{}, en particulier il n'y a
pas de termes de preuve et de noyau pour v\'erifier ces termes. Il faut
donc faire confiance � la quasi-totalit\'e du code pour croire en la
correction des programmes v\'erifi\'es. Le crit�re de {\sc De Bruijn},
qui dit en substance qu'un petit noyau est plus s\^ur n'est pas
respect\'e, alors que celui de \Coq~a m\^eme \'et\'e
formellement v\'erifi\'e \cite{Barras96a}.
 
Dans notre cas, il faut g\'en\'erer des termes de preuve et donc le code 
correspondant � ce ``sous-typage''. Une litt\'erature importante
\cite{conf/popl/Chen03,conf/csl/Luo96} existe
autour des syst�mes � coercions explicites dont nous nous sommes
inspir\'es pour r\'ealiser la g\'en\'eration des termes. Dans un syst�me �
coercions explicites, on peut faire des abus de notations tels que
utiliser un objet de type $T$ � la place d'un de type $U$, mais
on applique une coercion qui am�ne l'objet vers le type $U$ avant 
de retyper dans un syst�me sans coercions. G\'en\'eralement les coercions
sont tr�s similaires � des identit\'es, c'est-�-dire qu'elles sont calculatoirement
insignifiantes mais leur utilisation facilite le d\'eveloppement. Dans
\Coq{} par exemple le syst�me de coercions \cite{saibi97inheritance} a
permis de d\'evelopper des th\'eories alg\'ebriques r\'eutilisables sur
plusieurs structures instantan\'ement (un th\'eor�me sur les corps pouvant s'appliquer aux
anneaux).

Les extensions du Calcul des Constructions avec des notions de
sous-typage comme $\lambda C_\leq$ de Chen ne sont cependant pas
dans la m\^eme cat\'egorie que notre travail. En particulier, nous ne
nous int\'eressons pas aux propri\'et\'es de normalisation, pr\'eservation du
typage ou encore au fait d'avoir des sous-types minimaux dans notre
syst�me. On peut le voir plut\^ot comme un syst�me syntaxique intelligent
au-dessus du Calcul des Constructions.

%%% Local Variables: 
%%% mode: latex
%%% TeX-master: "subset-typing"
%%% LaTeX-command: "TEXINPUTS=\"style:$TEXINPUTS\" latex"
%%% End: 

\chapter{Le calcul de coercion par pr�dicats}
Nous avons d�velopp� un langage supportant le \ps{} utilisable dans
\Coq. L'utilisateur peut d�finir des programmes dans un langage plus
souple puis prouver certains buts pour obtenir finalement un terme de
\CCI{} complet v�rifiable par le noyau. On peut finalement utiliser 
les types d�pendants comme des types
simples et s'occuper des d�pendances dans un deuxi�me temps (pour la preuve).
L'architecture de notre syst�me est la suivante:
on type le programme dans notre langage \lng{} o� l'on peut faire des
abus de notations avec les objets de type sous-ensemble, puis l'on r��crit le terme typ�
dans \CCI{} en laissant des ``trous'' dans les termes qui d�sambig�ent
les abus et enfin \Coq{} se charge de g�n�rer les obligations correspondant � ces trous.


On va donc tout d'abord pr�senter le langage \lng{}, puis un algorithme
de typage correct et complet pour les programmes �crits en \lng{}. Ensuite on montrera comment
plonger ce langage dans \CCI{} en ajoutant les coercions ad�quates et
enfin on expliquera comment se d�roule la g�n�ration des obligations de
preuves � partir des termes engendr�s par le plongement.

\setboolean{displayLabels}{true}

\section{Le langage \lng{}}
\label{section:russel}
Le langage que nous voulons est tr�s proche de \ML{}, plus les annotations
n�cessaires pour avoir un typage pr�cis et d�cidable. On �tudie ici une
restriction de \ML{}, purement fonctionnelle et sans filtrage, qu'on
�tendra dans la suite de notre travail. On n'a donc pas de types
inductifs mais on consid�re les types $\Sigma$, g�n�ralisation des
tuples de \ML{} form�s par l'op�rateur $*$.

\subsection{Syntaxe}
La syntaxe (figure \ref{fig:syntax}) est directement inspir�e des langages fonctionnels.
On part du \lc{} (variables, abstraction et application) puis l'on
ajoute des constantes (pour les entiers, bool�ens, etc...) ainsi que les
couples. La syntaxe $(x := `a, t : `t)$ permet de
cr�er des paires d�pendantes, de type $\Sigma x : `t. `t$. On peut aussi
appliquer un terme � un type pour instancier une fonction polymorphe par exemple.

Du c�t� des types, on a tout d'abord les types simples (constantes,
fl�che, produit cart�sien) qui sont des cas particuliers du produit ($\Pi$) et
de la somme ($\Sigma$) d�pendants. Les variables introduites par ces
types peuvent �tre utilis�es lors des applications de types. On
peut de plus abstraire sur les types avec le $\lambda$ (polymorphisme)
et les sortes.
Enfin on peut appliquer un type � un terme ($`t~`a$). 

%\vspace{-0.5cm}
\begin{figure}[ht]
  \begin{center}
    \subfigure[Termes]{\termgrammar}\quad
    \subfigure[Types]{\typegrammar}
  \end{center}
  \caption{Syntaxe}
  \label{fig:syntax}
\end{figure}
% \vspace{-1cm}

\subsection{S�mantique}
\typenvd

\typedFig
\subtdFig

La s�mantique du langage nous est donn�e par un syst�me de typage
(figure \ref{fig:typing-decl-rules} page \pageref{fig:typing-decl-rules}). Le
jugement de typage est d�fini inductivement par un ensemble de r�gles
d'inf�rence. 
Dans notre cas ce sont les r�gles du \CCfull{} (\CC{})
�tendu avec les $\Sigma$-types auxquelles on a ajout� une r�gle de
coercion (\irule{Coerce}, figure \ref{fig:typing-decl-rules}) que l'on trouve classiquement dans les syst�mes avec
sous-typage avec le nom de subsumption. 
Le jugement $\Gamma \typed t : T$ se lit: dans l'environnement $\Gamma$,
$t$ est de type $T$.

L'�quivalence utilis�e est $\eqbr$, soit la $\beta$-r�duction
classique ainsi que les projections $\pi_i$ pour les paires.

% \begin{remark}
%   En pratique, les types du Calcul des Constructions ne sont pas
%   toujours en forme normale et il peut donc �tre n�cessaire de les
%   r�duire (en t�te seulement) pour v�rifier des jugements du genre: 
%   $`G \seq t : \Pi x : T.V$.
% \end{remark}

La relation $\mathcal{R}$ d�finissant les produits formables
dans le syst�me est d�finie par les r�gles suivantes:
\begin{figure}
  \[\begin{array}{cccll}
    s_1 & s_2 & s_3 & \text{Habitants} & \text{Exemple} \\
    \hline
    \Prop & \Prop & \Prop & \text{Implication logique} & x ``<= 0 "->" x = 0  \\
    \Set & \Set & \Set & \text{Fonctions} & \Pi x : \nat. \nat \\
    \Type & \Set & \Type & \text{Fonctions polymorphes} & \Pi A : \Set, A
    "->" A \\
    \Set & \Type & \Type & \text{Types d�pendants} & \sref{vector} : \nat "->" \Set : \Set "->" \Type \\
    \Set & \Prop & \Prop & \text{Termes dans les propositions} & 
    \Pi n : \nat. \Pi l : \text{list}~n. \text{length}~l = n \\
    \Type & \Prop & \Prop & \text{Impr�dicativit� de } \Prop & \Pi x : Prop. x `V `! x \\
    % \Type(i) & \Type(j) & \Type(max~i~j)  & \text{Connecteurs logiques,
    %   \ldots} & \Pi A : \Prop. \Pi B : \Prop. A `^ B "->" B `^ A
  \end{array}\]
  \caption{D�finition de $\mathcal{R}$}
  \label{R-definition}
\end{figure}
On a un syst�me proche du \CCfull{} avec types $\Sigma$, mais
avec \Set{} pr�dicatif (comme dans \Coq{}).
On n'a pas $(\Prop,\Set,\Set)$ dans notre relation $\mathcal{R}$ pour
une bonne raison. Cela permet de cr�er des fonctions d�pendant de
propositions, par exemple $\Pi n : \nat, n > 0 "->" \Pi l :
\text{list}~A~n "->" A$. Or on veut � tout prix �viter d'introduire des
termes de preuve dans notre langage, et l'on voit que
cette fonction pourrait naturellement s'�crire $\Pi n : \mysubset{n}{\nat}{n > 0} "->" \Pi l :
\text{list}~A~n "->" A$. Encore une fois le type sous-ensemble nous
permet d'�viter d'avoir � passer des termes de preuve directement. 

Les sommes formables dans le syst�me sont r�duites au couples d'objets de
types de m�me sorte $s `: \{ \Prop, \Set \}$.
Dans le premier cas les habitants sont les couples de propositions
(codage du $`^$), dans le second ce sont les couples d'objets, soit les
paires de \ML.
Intuitivement, c'est le type sous-ensemble $\mysubset{x}{T}{P}$ qui permet
de faire des couples $\Set,\Prop$ habitant $\Set$. Les types $\Sigma x : U.V$ o� $U
: \Prop$ et $V : \Set$ n'ont pas d'int�r�t dans notre cas puisqu'ils
repr�sentent des objets de type $U `^ V$ mais on ne peut
pas utiliser $U$ dans notre syst�me. On pr�f�re coder ces objets par des
objets de type $\mysubset{x}{V}{U}$ (on n'est pas int�ress� par la preuve
de $U$ pour programmer).


La r�gle \irule{Coerce} formalise l'id�e que 
l'on peut utiliser un terme de type $T$ � la place d'un terme de type
$U$ si $T$ et $U$ sont dans une certaine relation. C'est l�
qu'interviendront les types sous-ensemble. \CC{} contient une r�gle de typage
similaire � \irule{Coerce}, la r�gle de conversion (\irule{Conv}), qui
dit essentiellement que deux types
$`b$-convertibles (on rappelle que l'on peut calculer dans les types
puisqu'on a l'abstraction, l'application, etc...) sont �quivalents.
On peut directement int�grer cette relation de $`b$-convertibilit� � notre
syst�me de coercion comme montr� figure \ref{fig:subtyping-decl-rules}
(\irule{SubConv}), � condition d'avoir l'inclusion
$\eqbr~\subseteq~\subd + \text{\irule{SubConv}}$.
En fait notre notion de r�duction est un peu plus large que $\beta$
puisqu'on peut r�duire les $\sref{let}$:
$\letml~(x,y) = (u, v)~\inml~t$ se r�duit en $t[u/x][v/y]$. En
pratique cette constructions est du sucre syntaxique pratique au niveau
du typage (on peut inf�rer le type de $t$), mais elle est inessentielle au
niveau du calcul.
On peut ais�ment rajouter un $\letml~x=t~\inml~v$ � notre langage de
fa�on similaire: c'est �quivalent � $(\lambda x : T.v)~t$, mais $T$ peut
est inf�r� plut�t que donn� par l'utilisateur.

On consid�re les constantes comme des variables pr�d�finies 
dans nos contextes, par exemple on a la constante $\sref{list} : \Pi x :
nat. \Set$. 
L'ajout d'une constante � un contexte ne doit pas alt�rer sa
bonne formation comme pour le cas des variables, donc son type doit �tre
bien form� (en g�n�ral, toute d�finition de \Coq~donne lieu � une
constante dans notre syst�me si elle est bien typ�e).

\subsubsection{Jugement de coercion}
Notre syst�me de coercion par pr�dicats permet � l'utilisateur
d'utiliser une valeur de type $U$ l� o� l'on attend une valeur de type
$\mysubset{x}{V}{P}$ (\irule{SubProof}) si $U$ est lui-m�me coercible en $V$.
A l'inverse, on permet aussi d'utiliser une valeur de type
$\mysubset{x}{U}{P}$ (\irule{SubSub}) � la place d'une valeur de type
$V$ si $U$ est coercible vers $V$. Notre jugement de coercion est donc
sym�trique et laisse beaucoup de libert� � l'utilisateur au moment du
codage. Par exemple on peut d�river $u : \nat \type u : \mysubset{x}{\nat}{`_}$
Seulement, lors de la traduction de la d�rivation de coercion $\nat
\subd \mysubset{x}{\nat}{`_}$ (n�cessaire pour traduire l'abus de notation
$x : \mysubset{x}{\nat}{`_}$), l'utilisateur aura � r�soudre une obligation
de preuve de $`_$. On repose donc toujours sur la coh�rence du Calcul
des Constructions. 
Les r�gles \irule{SubProd} et \irule{SubSigma} permettent de faire des
coercions dans les types composites. Classiquement, la r�gle pour le 
produit fonctionnel est contravariante (une fonction sous-type d'une
autre accepte plus d'entr�es mais donne une sortie plus fine, voir
 \cite{journals/toplas/Castagna95}) et la r�gle pour le 
produit cart�sien covariante (une paire est coercible en une autre si 
leurs composantes sont coercibles deux-�-deux). Le sens des coercions
n'a pas d'importance dans le syst�me d�claratif puisqu'il est sym�trique
mais il est essentiel lors de la cr�ation des coercions que nous
d�crirons plus tard.

La r�gle \irule{SubTrans} assure que l'on a un syst�me compositionnel. Il y a ici une
analogie avec l'�limination des coupures dans les syst�mes logiques, o�
l'on montre que toute d�rivation utilisant la r�gle de \emph{modus ponens} ($A "=>" B$ et $B "=>" C$ implique
$A "=>" C$) peut se r��crire en une d�rivation ne l'utilisant
jamais. Dans les syst�mes � sous-typage, on montre de fa�on �quivalente
que l'on peut �liminer la r�gle de transitivit� ; premi�re �tape vers un
syst�me d�cidable.


Notre jugement de coercion identifie les types $U$ et $\mysubset{x}{U}{P}$
mais notre syst�me de typage ne permet pas d'�liminer (prendre la partie
preuve) ou d'introduire (cr�er un couple t�moin,preuve) des objets de
type sous-ensemble. Cela nous assure une certaine coh�rence, puisque
m�me si l'on ne v�rifie pas qu'un objet de type $U$ a bien la propri�t�
$P$, on ne peut pas raisonner sur le fait que $U$ a la propri�t� dans le
langage.


On ne fera pas la m�tath�orie du syst�me d�claratif ici, puisque
c'est une extension conservative du Calcul des Constructions et l'on
�tudiera en d�tail le syst�me algorithmique. Notre preuve de conservativit�
est simple: si l'on oublie les utilisations des types sous-ensemble de
notre syst�me de typage (\irule{Subset}) et de coercion
(\irule{SubProof} et \irule{SubSub}), alors le jugement de coercion est 
juste la $\beta$-convertibilit� et donc \irule{Coerce} et \irule{Conv} 
sont �quivalentes. Comme les autres r�gles de notre syst�me d�claratif
proviennent directement de \CC{}, on arrive � un syst�me strictement
�gal au syst�me du calcul des constructions. On peut donc s'appuyer sur
les r�sultats connus pour \CC{} pour une partie de notre syst�me.

Pour une �tude compl�te du \CCfull{}, se r�f�rer �
\cite{Barras99,Luo90}.
On va plut�t s'int�resser � la construction d'un algorithme de typage
correspondant � notre syst�me d�claratif.

%%% Local Variables: 
%%% mode: latex
%%% TeX-master: "subset-typing"
%%% LaTeX-command: "TEXINPUTS=\"style:$TEXINPUTS\" latex"
%%% End: 

\newpage
\subsection{�laboration du syst�me algorithmique \& propri�t�s}
\typenva

Pour pouvoir impl�menter le typeur, il nous faut un syst�me dirig� par la
syntaxe. 
C'est presque le cas pour la coercion, il y a juste la r�gle
\irule{SubConv} qu'on peut appliquer � n'importe quel moment. 


\subsubsection*{Conversion}
On note $\suba$ le m�me syst�me que figure \ref{subtyping-decl-rules} mais
o� l'on n'applique \irule{SubConv} seulement si aucune autre r�gle ne
s'applique.

On montre que les deux syst�mes de coercion sont �quivalents: 

Il nous faut tout d'abord un lemme sur la conversion:
\begin{lemma}
  \label{conversion-pi}
  Si $\Pi T U \eqbi S$ alors $S \eqbi \Pi T' U'$ avec $T \eqbi T'$ et $U
  \eqbi U'$.
\end{lemma}

Le seul cas int�ressant est si les deux termes sont dans la relation de $\beta$-�quivalence:
\begin{lemma}[Conservation de la conversion par sous-typage]
  \label{conversion-coercion}
  Si $`G \typea T, U : s$ et $T \eqbi U$ alors $T \suba U$.
\end{lemma}

\begin{proof}
  Par induction sur la forme de $T$.
  
  \def\seq{\suba}.
  
  \begin{itemize}
  \item[$T$ atomique:]
    On a alors $U = T$, trivial.
    
  \item[$T `= \Pi X.Y$:]
    Alors $U `= \Pi V.W$ et $X \eqbi V$, $Y \eqbi W$
    d'apr�s le lemme \ref{conversion-pi}.
    Par induction $X \sub Y$ et $V \sub W$. 
    On applique alors \irule{SubProd} � ces deux pr�misses.
    
  \item[$T `= \Sigma X.Y$:]
    $U$ est de la forme $\Sigma V.W$, avec $X \eqbi V$ et $Y \eqbi
    W$. Par induction et application de \irule{SubSigma}.
    
  \item[$T `= \subset{x}{X}{P}$:] 
    On a alors $U `= \subset{x}{X'}{P'}$ avec $X \eqbi X'$, $P \eqbi
    P'$, et la propri�t� est vraie par \irule{SubLeft} et \irule{SubRight}:
    
    \begin{prooftree}
      \AXC{$X \sub X'$}
      \LeftLabel{\SubLeftRule}
      \UIC{$\subset{x}{X}{P} \sub X'$}
      \LeftLabel{\SubRightRule}
      \UIC{$\subset{x}{X}{P} \sub \subset{x}{X'}{P'}$}
    \end{prooftree}
  \end{itemize}
\end{proof}


Il faut cependant s'assurer que deux types de sortes
diff�rentes ne peuvent �tre identifi�s. En effet dans notre syst�me
la $\beta$-convertibilit� n'assure pas que deux termes ont la m�me
sorte, par exemple:
$\typed (\lambda x : \Type. x)~(\nat:\Set) : \Type "->"_\beta \nat : \Set$.
\TODO{Dans mon syst�me, on ne peut pas faire la coercion de nat � Set}

Le fait que les arguments sont tout deux sort�s avec la m�me sorte
avant de d�river le jugement de coercion nous assure que l'on ne fera
pas d'identification erron�e dans les autres r�gles que \irule{SubConv}.

En cons�quence $\subd$ et $\suba$ sont �quivalentes. Le syst�me
d'inf�rence de $\suba$ donne donc un algorithme pour d�cider de la relation
de coercion. L'ind�terminisme entre les r�gles \irule{SubProof} et
\irule{SubSub} ne pose pas de probl�me: on peut laisser le choix � 
l'impl�mentation puisque le syst�me est confluent.

Cependant, il reste une source importante d'ind�cidabilit� dans le
syst�me de typage, c'est la r�gle de coercion. On va l'�liminer du
syst�me algorithmique apr�s avoir montr� que
toute d�rivation de typage utilisant \irule{Coerce} peut se r��crire en
une d�rivation n'utilisant cette r�gle qu'� sa racine.
\def\subs{\subset{x}{U}{P}}
Il nous faut changer quelque peu les r�gles pour obtenir le syst�me
algorithmique. En particulier, on va utiliser la fonction $\mu0$ de \PVS{}
\cite{PVS-Semantics:TR} renom�e $\mu$ ici pour op�rer des
\emph{d�compr�hension}. Cette fonction efface les constructeurs de type
sous-ensemble en t�te d'un type, par exemple: $\mualgo(\subset{f}{\nat
  "->" \nat}{f~0 = 0}) = \nat "->" \nat$. Cela va nous permettre de
restreindre l'utilisation du jugement du sous-typage � l'application.

\TODO{bof}
\begin{remark}
  En pratique, les types du Calcul des Constructions ne sont pas
  toujours en forme normale et il peut donc �tre n�cessaire de les
  r�duire pour v�rifier des jugements du genre: 
  $`G \seq t : \Pi x : T.V$. On pourrait voir $\mu$ comme une
  extension de la relation de $\beta$-�quivalence avec la r�duction
  $\subset{x}{U}{P} "->"_\mu U$. 
\end{remark}

\subsection*{Sommes d�pendantes}
En inspectant la r�gle \irule{Sum}, on remarque qu'il n'est pas possible
d'inf�rer le type $U$ � partir du seul terme $(t, u)$. Cela
n�cessiterait de r�soudre un probl�me d'unification d'ordre sup�rieur
auquel il n'y a pas de solution la plus g�n�rale. On introduit donc dans
le syst�me algorithmique deux nouvelles r�gles, dont une (\irule{SumDep})
permettant d'annoter le terme avec le type $U$ recherch�. On consid�re
que c'est � l'utilisateur d'annoter suffisament les termes. Par d�faut,
si le terme n'est pas annot� on consid�re l'objet $(u, v)$ comme une
paire non-d�pendante (\irule{SumInf}).

\typenva
\typeaFig
\typemuaFig
\subtaFig

\subsection*{Subsumption}
\typenva

% On enl�ve \irule{Coerce} du syst�me et on change la r�gle 
% d'application \irule{App} de la figure \ref{typing-decl-rules}.
% On note $\typea$ le syst�me de typage obtenu. 

\begin{proposition}[Passage de la subsumption � l'application]
  \label{subsum-elim}
  On peut r�ecrire toute d�rivation $`G \typea t : T$ utilisant la
  subsumption ailleurs qu'� sa racine vers une d�rivation $`G \type t :
  U$ avec $U \suba T$.
\end{proposition}

On a besoin de quelques lemmes auparavant:

\begin{lemma}[$\beta$-equivalence et $\mu$]
  \label{beta-mu}
  Si $X \sub Y$ et $\mu~Y \eqbi \Sigma x : T.U$ alors $\mu X \eqbi \Sigma x : T'.U'$
  et $T' \sub T$, $U' \sub U$.
  Si $X \sub Y$ et $\mu~Y \eqbi \Pi x : T.U$ alors $\mu X \eqbi \Pi x :
  T'.U'$ et $T \sub T'$, $U' \sub U$.
\end{lemma}
\begin{proof}
  \begin{induction}
    Par induction sur la d�rivation de coercion, on fait le cas pour $\Sigma$.
    
    \case{SubConv} Trivial, puisqu'on aura $\mu~X = \mu~Y$.

    \case{SubProd} Impossible, $\mu$ ne traversant pas les produits.

    \case{SubSigma} Direct, on a une d�rivation de $\Sigma x : T'. U'
    \sub \Sigma x : T.U$.
    
    \case{SubLeft} Ici, $Y `= \subset{x}{V}{P}$, on peut donc d�duire que
    $\mu~Y = \mu~V \eqbi \Sigma x : T.U$. On 
    applique l'hypoth�se de r�curence avec $X \sub V$ et on obtient:
    $\mu~X \eqbi \Sigma x : T'.U' `^ T' \sub T `^ U' \sub U$.

    \case{SubRight} Ici, $X `= \subset{x}{V}{P}$. Par induction, 
    $\mu~V = \mu~X \eqbi \Sigma x : T'.U' `^ T' \sub T `^ U' \sub U$.
  \end{induction}
\end{proof}

\begin{lemma}[Bonne formation des contextes]
  \label{wf-contexts-a}
  Si $`G \type t : T$ alors $\typewf `G$.
\end{lemma}
\begin{proof}
  \inductionon{typing-decl}
\end{proof}

\begin{fact}[Inversion du jugement de bonne formation]
  \label{inversion-wf-a}
  Si $\typewf `G, x : U$ alors $`G \type U : s$ et $s `: \{ \Set, \Prop, \Type(i) \}$.
\end{fact}

\begin{lemma}[Affaiblissement]
  \label{weakening-a}
  Si $`G, `D \type t : T$ alors pour tout $x : S `; `G, `D$ tel que
  $\wf `G, x : S, `D$, $`G, x : S, `D \type t : T$
\end{lemma}

\begin{proof}
  \begin{induction}[typing-decl]
    \casetwo{PropSet}{Type} Trivial.

    \case{Var}
    On a $x : S `; `G, `D$, donc $`G, x : S, `D \type y : T$ est toujours d�rivable.
    
    \case{Prod}
    Par induction $`G, x : S, `D \type T : s1$ et $`G, x : S, `D,
    y : T \seq U : s2$. On applique \irule{Prod} pour obtenir 
    $`G, x : S, `D \type \Pi x : T.U : s2$. De m�me pour le reste des r�gles.
  \end{induction}
\end{proof}  

Le renforcement montre que notre notion de sous-typage est correcte
vis-�-vis du typage. On peut d�river les m�mes jugements dans des
contextes o� les variables ont des types plus pr�cis.

\begin{lemma}[Renforcement]
  \label{narrowing-a}
  \[ `G \seq S, S' : s, S' \sub S "=>" 
  \left\{ \begin{array}{lcl}
      \typewf `G, x : S, `D & "=>" & \typewf `G, x : S', `D \\
      & `^{} & \\
      `G, x : S, `D \seq t : T & "=>" & `G, x : S', `D \seq t : T
    \end{array}
  \right. \]
\end{lemma}

\begin{proof}
  Par induction sur la taille de la d�rivation de typage ou de bonne formation.
    
  \begin{induction}
    \case{WfEmpty} Trivial.
    
    \case{WfVar} 
    La conclusion est $\typewf `G, x : S, `D$
    
    \begin{induction}[text=Par induction sur la taille de $`D$]
    \item[\protect{$`D = []$}]
      La racine de la d�rivation est de la forme:
      \begin{prooftree}
        \UAX{WfVar}
        {$`G \type S : s$}
        {$\wf `G, x : S$}
        {$s `: \{ \Set, \Prop, \Type(i) \}$}
      \end{prooftree}
      On a $`G \type S' : s$, donc par \irule{WfVar}, $\typewf `G, x : S'$.  
      
    \item[\protect{$`D `= `D', y : U$}]
      La racine de la d�rivation est de la forme:
      \begin{prooftree}
        \UAX{WfVar}
        {$`G, x : S, `D' \type U : t$}
        {$\wf `G, x : S, `D', y : U$}
        {$s `: \{ \Set, \Prop, \Type(i) \}$}
      \end{prooftree}
      Par induction sur la d�rivation de typage $`G, x : S', `D' \seq U : t$,
      on a donc bien $\typewf `G, x : S', `D', y : U$ par \irule{WfVar}.
    \end{induction}
    
    \casetwo{PropSet}{Type} 
    Par induction, $\typewf `G, x : S', `D$, on applique simplement la r�gle.
    
    \case{Var}
    Par induction, $\typewf `G, x : S', `D$. La seule diff�rence avec le
    contexte pr�cedent est le type associ� � $x$, donc si $t \not= x$, on
    peut simplement r�appliquer \irule{Var}. Si $t `= x$ on construit la
    d�rivation:

    \begin{prooftree}
      \BAX{Var}
      {$\wf `G, x : S', `D$}
      {$x : S' `: `G$}
      {$`G, x : S', `D \seq x : S'$}
      {}
      \AXC{$`G, x : S', `D \type S,S' : s$}
      \AXC{$S' \sub S$} % `G \subt 
      \TIC{$`G, x : S', `D \type x : S$}
    \end{prooftree}
    
    Par l'affaiblissement (lemme \ref{weakening-a}) et $`G \type S,S' : s$,
    on obtient la pr�misse $`G, x : S', `D \type S,S' : s$.
    
    \case{Prod} 
    Par induction, $`G, x : S', `D \type T : s1$ et $`G, x : S', `D
    y : T \seq U : s2$. On applique \irule{Prod} pour obtenir 
    $`G, x : S' \type \Pi x : T.U : s2$. De m�me pour le reste des r�gles.
  \end{induction}
\end{proof}

On peut maintenant montrer notre proposition:
\begin{proof}
  On inspecte les d�rivations possibles utilisant \irule{Coerce} juste avant
  une autre r�gle.
  
  \typenva
  \begin{induction}
    \case{Var} Par de pr�misse de la forme $`G \type t : T$, donc
    pas d'application de \irule{Subsum} possible.
    
    \case{Abs} \quad
    
    \begin{prooftree}
      \AXC{\vdots}
      \UIC{$`G \seq \Pi x : T. U : s $}
      \AXC{\vdots}
      \UIC{$`G, x : T \seq M : U'$}
      \AXC{\vdots}
      \UIC{$`G, x : T \seq U', U : s$}
      \AXC{\vdots}
      \UIC{$U' \sub U$}
      \TIC{$`G, x : T \seq M : U $}
      \BIC{$`G \seq \lambda x : T. M : \Pi x : T.U$}
    \end{prooftree}
    
    Comme $`G, x : T \typed U' : s$ on peut former le produit 
    $`G \typea \Pi x : T. U' : s$.
    On r�ecrit donc la d�rivation en:     
    
    \begin{prooftree}
      \AXC{\vdots}
      \UIC{$`G \seq \Pi x : T. U' : s $}
      \AXC{\vdots}
      \UIC{$`G, x : T \seq M : U'$}
      \BIC{$`G \seq \lambda x : T. M : \Pi x : T.U'$}
      \AXC{$T \eqbi T$}
      \UIC{$T \sub T$}
      \AXC{\vdots}
      \UIC{$U' \sub U$}
      \BIC{$\Pi x : T.U' \sub \Pi X : T.U$}        
      \BIC{$`G \seq \lambda x : T. M : \Pi x : T.U$}
    \end{prooftree}
    
    \case{Sum}\quad
    
    \begin{prooftree}
      \AXC{\vdots}
      \UIC{$`G \seq \Sigma x : T.U : s $}
      \AXC{\vdots}
      \UIC{$`G \seq t : S$}
      \AXC{\vdots}
      \UIC{$S \sub T$}
      \BIC{$`G \seq t : T $}
      \AXC{\vdots}
      \UIC{$`G \seq u : U[t/x]$}
      \TIC{$`G \seq (t,u) : \Sigma x : T.U$}
    \end{prooftree}
    
    Comme $S \sub T$, on a bien $\Sigma x : S.U \sub \Sigma x : T.U$
    par \irule{SubSigma}. On sait de plus que $\Sigma x : S.U$ est bien
    sort� de sorte $s$. En effet, par inversion de $`G \seq \Sigma x :
    T.U : s$ on a $`G, x : T \seq U : s$ et par renforcement ($S \sub T$), $`G, x : S \seq
    U : s$. On peut donc d�river:

    \begin{prooftree}
      \AXC{\vdots}
      \UIC{$`G \seq \Sigma x : S.U : s $}
      \AXC{\vdots}
      \UIC{$`G \seq t : S$}
      \AXC{\vdots}
      \UIC{$`G \seq u : U[t/x]$}
      \TIC{$`G \seq (t,u) : \Sigma x : S.U$}
      \AXC{\vdots}
      \UIC{$\Sigma x : S.U \sub \Sigma x : T.U$}
      \BIC{$`G \seq (t,u) : \Sigma x : T.U$}
    \end{prooftree}
    
    \case{LetSum}\quad
    \begin{prooftree}
      \AXC{\vdots}        
      \UIC{$`G \seq t : S $}
      \AXC{\vdots}
      \UIC{$S \sub V$}
      \BIC{$`G \seq t : V$}
      \AXC{$\mu~V \eqbi \Sigma x : T.U$}
      \AXC{\vdots}
      \UIC{$`G, x : T, u : U \seq v : V $}
      \TIC{$`G \seq \letml~(x, u) = t~\inml~v : V$}
    \end{prooftree}
    
    Par le lemme \ref{beta-mu}, on a $\mu~S \eqbi \Sigma x : T'.U'$,
    $T' \suba T$ et $U' \suba U$. Par renforcement (lemme
    \ref{narrowing-a}) on a:

    \begin{prooftree}
      \AXC{\vdots}        
      \UIC{$`G \seq t : S $}
      \AXC{$\mu~S \eqbi \Sigma x : T'.U'$}
      \AXC{\vdots}
      \UIC{$`G, x : T', u : U' \seq v : V $}
      \TIC{$`G \seq \letml~(x, u) = t~\inml~v : V$}
    \end{prooftree}

    Si l'on applique la subsumption � droite c'est direct ($V' \sub V$).
    
    \case{LetIn}
    De fa�on similaire, par renforcement on obtient la d�rivation sans \irule{Coerce}.
    
    \case{App}\quad    
    \begin{prooftree}
      \AXC{\vdots}
      \UIC{$`G \seq f : T $}
      \AXC{\vdots}
      \UIC{$ T \sub \Pi x : V. W$}
      \BIC{$`G \seq f : \Pi x : V. W $}
      \noLine
      \UIC{$\mualgo(\Pi x : V.W) \eqbi \Pi x : V.W$}
      \AXC{\vdots}
      \UIC{$`G \seq u : U $}
      \AXC{\vdots}
      \UIC{$U \sub V$}
      \TIC{$`G \seq f u : W [ u / x ]$}
    \end{prooftree}
    
    Par le lemme \ref{beta-mu} on a $\mu(T) \eqbi \Pi x : V'.W'$ avec
    $V \suba V'$ et $W' \suba W$. Par la transitivit� de la coercion on
    a: $U \suba V `^ V \suba V' "=>" U \suba V'$.
    On peut donc d�river:
    
    \begin{prooftree}
      \AXC{\vdots}
      \UIC{$`G \seq f : T $}
      \noLine
      \UIC{$\mualgo(T) \eqbi \Pi x : V'.W'$}
      \AXC{\vdots}
      \UIC{$`G \seq u : U $}
      \AXC{\vdots}
      \UIC{$U \sub V'$}
      \TIC{$`G \seq f u : W [ u / x ]$}
    \end{prooftree}
    
    \case{Subsum}\quad

    Dans le cas o� l'on a un enchainement de r�gles \irule{Coerce}:
    \begin{prooftree}
      \AXC{\vdots}
      \UIC{$`G \seq t : S$}
      \AXC{\vdots}
      \UIC{$S \sub T$}
      \BIC{$`G \seq t : T$}
      \AXC{\vdots}
      \UIC{$T \sub U$}
      \BIC{$`G \seq t : U$}
    \end{prooftree}
    
    En utilisant la transitivit� de la coercion on obtient:
    \begin{prooftree}
      \AXC{\vdots}
      \UIC{$`G \seq t : S$}
      \AXC{\vdots}
      \UIC{$S \sub U$}
      \BIC{$`G \seq t : U$}
    \end{prooftree}

  \end{induction}
  
\end{proof}

Par induction, on a donc montr� que l'on peut r�duire l'utilisation de la r�gle de
subsumption � la racine d'une d�rivation. On peut ignorer sans perte de g�n�ralit� 
l'utilisation de la coercion � la racine de la d�rivation, 
on fera de toute fa�on un test de coercion entre le type inf�r� et le
type sp�cifi� juste avant la r��criture. On consid�re maintenant le
syst�me algorithmique comme pr�sent� figure \ref{fig:typing-algo-rules}.

\begin{lemma}[Substitutivit� de $\mualgo$]
  \label{substitutive-mu}
  Si $\mualgo(T) = U$ alors $\mualgo(T[u/x]) = U[u/x]$.
\end{lemma}

\begin{proof}
  Il suffit de suivre la d�finition de $\mualgo$.
  Si $T$ est de la forme $\sub{y}{V}{P}$ alors $\mualgo(T) =
  \mualgo(V)$ et $T[u/x]$ est de la forme
  $\sub{y}{V[u/x]}{P[u/x]}$. Par induction, $\mualgo(T[u/x]) =
  \mualgo(V[u/x]) = \mualgo(V)[u/x] = \mualgo(T)[u/x]$.
  Sinon c'est trivial.
\end{proof}

\begin{lemma}[Substitutivit� de la coercion]
  \label{substitutive-coercion}
  Si $U \suba T$ alors pour tout $u$, $U[u/x] \suba T[u/x]$.
\end{lemma}

\begin{proof}
  \begin{induction}[subtyping-algo]
    \case{SubConv}
    Direct par pr�servation de l'�quivalence $\eqbi$ par substitution.
    
    \case{SubProd}
    Par induction $U[u/x] \suba T[u/x]$ et $V[u/x] \suba W[u/x]$, donc
    $\Pi y : T[u/x].V[u/x] \suba \Pi y : U[u/x].W[u/x]$. La propri�t�
    est donc bien v�rifi�e.
    
    \case{SubSigma} Direct par induction.
    
    \case{SubSub} Par induction, $U'[u/x] \suba V[u/x]$. On applique
    \irule{SubLeft} pour obtenir $\sub{y}{U'[u/x]}{P} \suba V[u/x]$. 
    
    \case{SubRight} Direct par induction.
  \end{induction}
\end{proof}

\begin{lemma}[Substitutivit� du typage]
  \label{substitutive-typing}
  Si $`G \typea u : U$ alors
  \[ \left\{ \begin{array}{lcl}
      `G, x : U, `D \typea t : T & "=>" & `G, `D[u/x] \typea t[u/x] :
      T[u/x] \\
      \wf `G, x : U, `D & "=>" & \wf `G, `D[u/x]
    \end{array}\right. \]
\end{lemma}

\begin{proof}
  \typenva
  Par induction mutuelle sur la d�rivation de typage $`G, x : U, `D
  \typea t : T$ ou $\wf `G, x : U, `D$.
  
  \begin{induction}
    \case{WfEmpty} Trivial.

    \case{WfVar}
    Par induction sur $`D$.
    \begin{itemize}
    \item[\protect{$`D = []$}]
      On a alors $`G \typea U : s$ donc $\wf `G$ et triviallement, $\wf
      `G, `D[u/x]$.

    \item[\protect{$`D = `D', y : T$}]
      On a alors $`G, x : U, `D' \typea T : s$ et par induction
      $`G, `D'[u/x] \typea T[u/x] : s[u/x]$. Donc on peut appliquer
      \irule{WfVar} pour obtenir $\wf `G, `D'[u/x], T[u/x]$ soit
      $\wf `G, `D[u/x]$
    \end{itemize}
    
    \casetwo{PropSet}{Type} 
    La substitution n'a aucun effet et $`G, `D[u/x]$ est bien
    form� par induction.
    
    \case{Var}
    Par induction, $\wf `G, `D[u/x]$.
    Si $t `= x$ alors on a $T = U$ et $T[u/x] = U$ puisque $x$
    n'apparait pas dans $U$. On a donc bien $`G, `D[u/x] \typea t[u/x] = u :
    T[u/x] = U$. 
    Si $y : T `: `G$ alors on applique simplement \irule{Var}.
    Si $y : T `: `G$ alors $y : T[u/x] `: `D[u/x]$ et on obtient
    $`G, `D[u/x] \typea y[u/x] :  T[u/x]$ par \irule{Var}.
    
    \case{Prod}
    Par induction  $`G, `D[u/x] \typea T[u/x] : s_1[u/x]$ et
    $`G, `D[u/x], y : T[u/x] \typea M[u/x] : s_2[u/x]$. 
    On peut appliquer \irule{Prod} pour obtenir $`G, `D[u/x] \typea \Pi
    y : T[u/x].M[u/x] : s_2[u/x]$ soit $`G, `D[u/x] \typea (\Pi y :
    T.M)[u/x] : s_2[u/x]$.
    De fa�on similaire pour les autres cas.

    \case{App}
    On �tudie le cas de l'application qui requiert un lemme suppl�mentaire.
    Par induction, $`G, `D[u/x] \typea f[u/x] : T[u/x]$ et
    $`G, `D[u/x] \typea a[u/x] : A[u/x]$. Si $\mualgo(T) \eqbi \Pi y :
    V.W$ alors $\mualgo(T[u/x]) \eqbi \Pi y : V[u/x].W[u/x]$ (lemme
    \ref{substitutive-mu}). Par induction, on a aussi $`G, `D[u/x] \typea
    A[u/x],V[u/x] : s$. Enfin, par substitutivit� de la coercion on a $A[u/x]
    \suba V[u/x]$. On peut donc appliquer \irule{App} pour obtenir 
    $`G, `D[u/x] \typea (f[u/x] a[u/x]) : W[u/x][a[u/x]/y]$. Or $W[u/x][a[u/x]/y] =
    W[a/y][u/x]$. On a donc bien $`G, `D[u/x] \typea (f a)[u/x] :
    (W[a/y])[u/x]$.
    On a un raisonnement similaire pour \irule{LetSum}.


  \end{induction}
  
\end{proof}

\begin{lemma}[Substitutivit� du typage avec coercion]
  \label{substitutive-typing-coercion}
  Si $`G, x : V \typea t : T \sub U$, $G \typea u : V$
  alors $`G \typed t[u/x] : T[u/x] \sub U[u/x]$.
\end{lemma}

\begin{proof}
  Par substitutivit� du typage (\ref{substitutive-typing}) on a $`G \typed t[u/x] : T[u/x]$.
  Par le lemme pr�c�dent $T[u/x] \suba U[u/x]$.
\end{proof}

\begin{lemma}[Inversion de la coercion]
  \label{inversion-coercion}
  Si $`G \subta \lambda x : T. M : \Pi x : T.U \sub \Pi x : V.W$ 
  alors $`G, x : T \typea M : U$.
\end{lemma}
\begin{proof}
  \TODO{pas utilis� ?}
\end{proof}

\begin{fact}[Sym�trie de la coercion]
  La relation $\sub$ est sym�trique.
\end{fact}

\begin{lemma}[Coercion et $\mu$]
  \label{coercion-mu}
  Si $\Pi x : X.Y \sub U$ alors $\mu(U) \eqbi \Pi x : X'.Y'$ et $X' \sub
  X$, $Y \sub Y'$.
  Si $\Sigma x : X.Y \sub U$ alors $\mu(U) \eqbi \Sigma x : X'.Y'$ et $X \sub
  X'$, $Y \sub Y'$.
  Pour tout $U$, $U \sub \mu(U)$.
\end{lemma}
\begin{proof}
  Par induction sur les d�rivations de $\suba$ et la d�finition de $\mu$.
\end{proof}

\begin{lemma}[Coercion et conversion]
  \label{coercion-conversion}
  Si $S \eqbi T$ et $T \sub U$ alors $S \sub U$
\end{lemma}

\begin{proof}
  Par simple inspection des r�gles on voit que le jugement ne peut
  distinguer deux termes $\beta$-�quivalents.
\end{proof}

\begin{lemma}[Transitivit� de la coercion]
  \label{transitive-coercion}
  Pour tout $S, T, U$,  si $S \sub T$ et $T \sub U$ alors $S \sub U$.
\end{lemma}

\begin{proof}  
  \TODO{Dans \cite{Pierce:TypeSystems}, voir p. 420}
  On proc�de par �limination de la r�gle \irule{SubTrans} dans toute
  d�rivation de $S \sub U$.
  
  \begin{induction}[subtyping-decl]

    \case{SubConv}\quad
    \begin{prooftree}
      \AXC{$S \eqbi T$}
      \UIC{$S \sub T$}
      \AXC{$T \sub U$}
      \BIC{$S \sub U$}
    \end{prooftree}
    
    Par le lemme pr�c�dent, on �limine trivialement \irule{SubTrans}.

    \case{SubProd}\quad
    \begin{prooftree}
      \AXC{$X' \sub X$}
      \AXC{$Y \sub Y'$}
      \BIC{$\Pi~X~Y \sub \Pi~X'~Y'$}
      \AXC{$\Pi~X'~Y' \sub U$}
      \BIC{$\Pi~X~Y \sub U$}
    \end{prooftree}
    
    Si $\Pi~X'~Y' \sub U$, alors il existe $S$, $T$, tel que $\mu(U)
    \eqbi \Pi~S~T$ et $S \sub X'$, $Y' \sub T$.
    Par induction, $S \sub X$ et $Y \sub T$ donc $\Pi~X~Y \sub \Pi~S~T$ 
    et enfin $\Pi~X~Y \sub U$.

    De facon �quivalente pour le second cas.

    \case{SubSigma}\quad
    \begin{prooftree}
      \AXC{$X \sub X'$}
      \AXC{$Y \sub Y'$}
      \BIC{$\Sigma~X~Y \sub \Sigma~ X'~Y'$}
      \AXC{$\Sigma~X'~Y' \sub U$}
      \BIC{$\Sigma~X~Y \sub U$}
    \end{prooftree}
    
    Si $\Sigma~X'~Y' \sub U$, alors il existe $S$, $T$, tel que $\mu(U)
    \eqbi \Sigma~S~T$ et $X' \sub S$, $Y' \sub T$.
    Par induction, $X \sub S$ et $Y \sub T$ donc $\Sigma~X~Y \sub
    \Sigma~S~T$ et enfin $\Sigma~X~Y \sub U$ (lemme \ref{coercion-mu}).
    
    De fa�on �quivalente pour le second cas.

    \case{SubProof}\quad
    \begin{prooftree}
      \AXC{$S \seq V$}
      \UIC{$S \seq \subset{x}{V}{P}$}
      \AXC{$\subset{x}{V}{P} \sub U$}
      \BIC{$S \sub U$}
    \end{prooftree}
    
    Si $\subset{x}{V}{P} \sub U$ alors $\mu(U) \eqbi V$.
    Comme $S \seq V$, par induction, $S \seq \mu(U)$ donc
    $S \sub U$.
    
    De fa�on �quivalente pour le second cas.

    \case{SubSub}
    Cas identique � \irule{SubProof}
    
  \end{induction}
\end{proof}

\begin{corrolary}[Compl�tude de la coercion]
  \label{complete-coercion}
  Si $U \subd V$ alors $U \suba V$.
\end{corrolary}

\begin{proof}
  Les r�gles des deux syst�mes sont les m�mes except� \irule{SubTrans}
  qu'on peut �liminer dans le syst�me algorithmique. De plus
  l'application restreinte de la conversion ne change pas les jugements
  d�rivables (lemme \label{conversion-sub}).
\end{proof}

\begin{theorem}[Correction de la coercion]
  \label{correct-coercion}
  Si $U \suba V$ alors $U \subd V$.
\end{theorem}

\begin{proof}
  Les r�gles du syst�me algorithmique sont un sous-ensemble des r�gles
  du syst�me d�claratif.
\end{proof}

\begin{theorem}[Correction du typage]
  \label{correct-typing}
  Si $`G \typea t : T$ alors $`G \typed t : T$
\end{theorem}

\setboolean{displayLabels}{false}
\begin{proof}
  \begin{induction}[typing-algo]
  \item[\irule{WfEmpty},\irule{WfVar},\irule{PropSet},\irule{Var},\irule{Prod},\irule{Abs},
    \irule{LetIn}, \irule{Sigma}, \irule{Sum}:] r�gles inchang�es.
    
    \case{LetSum}
    On a 
    \begin{prooftree}
      \LetSumA
    \end{prooftree}
    
    Par induction, $`G \typed t : S$, et par correction de la coercion $S \subd \Sigma x : T. U$.
    On peut donc d�river $`G \typed t : \Sigma x : T.U$ � l'aide de \irule{Coerce}.
    On peut directement appliquer \irule{LetSum} � cette pr�misse et �
    l'hypoth�se d'induction $`G, x : T, y : U \typed v : V$.
    
    \case{App} On a:
    \def\fCenter{\typea}
    \begin{prooftree}
      \AppA
    \end{prooftree}
    
    Par induction, $`G \typed f : T$, et $T \subd \Pi x : V. W$.
    On peut donc d�river $`G \typed f : \Pi x : V.W$ � l'aide de la
    subsumption.
    Par le lemme \ref{correct-coercion}, et l'hypoth�se $`G \typed u :
    U$, on obtient $`G \typed u : V$ par \irule{Subsum}.
    Donc, par \irule{App}, on a bien $`G \typed f u : W[u/x]$.
  \end{induction}  
\end{proof}

\setboolean{displayLabels}{true}
\begin{lemma}[Compl�tude du typage]
  \label{complete-typing}
  $`G \typed t : T "=>" `E U, `G \subta t : U \sub T$
\end{lemma}

\begin{proof}
  \begin{induction}[typing-decl]
  \item[\irule{WfEmpty},\irule{WfVar},\irule{PropSet},\irule{Var},\irule{Prod},\irule{Abs},
    \irule{LetIn}, \irule{Sigma}, \irule{Sum}, \irule{Subset}:] r�gles inchang�es.
    
    \case{App} On a 
    \typenvd
    \begin{prooftree}
      \App
    \end{prooftree}
    
    \typenva
    Par induction, $`E T, `G \typea f : T \suba \Pi x : V. W$ et
    $`E U, `G \subta u : U \sub V$.
    
    Si $T \suba \Pi x : V.W$ alors $\mualgo(T) \eqbi \Pi x : V'.W'$ avec
    $V \suba V'$ et $W' \suba W$.

    Par transivit� de la coercion: $U \suba V'$, on peut donc d�river 
    \begin{prooftree}
      \TAX{App}
      {$`G \seq f : T \quad \mualgo(T) \eqbi \Pi x : V'. W'$}
      {$`G \seq u : U \quad `G \seq U, V' : s$}
      {$U \suba V'$}
      {$`G \seq (f u) : W' [ u / x ]$}
      {}
    \end{prooftree}
    
    Par substitutivit� de la coercion (lemme
    \ref{substitutive-coercion}), $W'[u/x] \suba W[u/x]$, la propri�t�
    est donc bien v�rifi�e.

    \case{LetSum} On a
    \typenvd
    \begin{prooftree}
      \LetSum
    \end{prooftree}
    
    \typenva
    Par induction, $`E S, `G \typea t : S \suba \Sigma x : T.U$ et 
    $`E V', `G, x: T, y : U \typea v : V' \suba V$.
    On a $\mualgo(S) \eqbi \Sigma x : T'.U'$ avec $T' \suba T$ et $U'
    \suba U$. Par renforcement on peut donc d�river $`G, x : T', y : U'
    \seq v : V'$.
    
    On a donc la d�rivation suivante dans le syst�me algorithmique:
    \begin{prooftree}
      \TAX{LetSum}
      {$`G \seq t : S$}
      {$\mualgo(S) \eqbi `S x : T'. U'$}
      {$`G, x : T', y : U' \seq v : V'$}
      {$`G \seq \letml~(x, u) = t~\inml~v : V'$}
      {}
    \end{prooftree}
    
    Comme $V' \suba V$, la propri�t� est vraie.

    \casetwo{Conv}{Subsum}
    Dans les deux cas on a inductivement $`E T', `G \typea t : T'
    \suba T$. Avec \irule{Conv} on a $T \eqbi S$, donc $T' \suba S$ par
    le lemme \ref{coercion-conversion}. Pour \irule{Subsum} on a $T \subd S$.
    Par compl�tude de la coercion, $T \suba S$ et par transitivit� de la
    coercion, $T' \suba S$. La propri�t� est donc bien v�rifi�e dans les
    deux cas.
    
  \end{induction}
  
\end{proof}

On combine les th�or�mes de correction et compl�tude pour obtenir la propri�t� suivante entre les deux syst�mes:
\begin{corrolary}[�quivalence des syst�mes d�claratifs et algorithmiques]
  $`G \typed t : T$ \ssi{} il existe $U$ tel que $`G \typea t : U$ et $U \suba T$.
\end{corrolary}

On a maintenant un syst�me raffin� d�rivant les m�me jugements (�
coercion pr�s) que le syst�me d�claratif. On veut en extraire un
algorithme de typage. Pour cela on doit pouvoir r�soudre deux probl�mes:
\begin{itemize}
\item\textbf{V�rification de type.} On donne $`G$,$t$ et $T$ et l'on doit
  d�cider si $`G \typea t : T$ ;
\item\textbf{Inf�rence de type.} On donne $`G$,$t$ et l'on doit trouver $T$ tel
  que $`G \typea t : T$ si c'est d�rivable, sinon on �choue.
\end{itemize}
En pratique, la v�rification a besoin de l'inf�rence puisque lorsqu'on
v�rifie une application $f u : T$ on doit inf�rer le type de $f$.
On montre donc les th�or�mes suivants:

\begin{theorem}[D�cidabilit� de l'inf�rence dans le syst�me algorithmique]
  Le probl�me d'inf�rence $`G \typea t : ?$ est d�cidable.
\end{theorem}

\begin{proof}
  Il suffit d'observer que les r�gles de typage sont dirig�es par la
  syntaxe du deuxi�me argument et permettent donc d'inf�rer un type pour
  tout terme. En lisant les pr�misses de chaque r�gle de gauche �
  droite, on voit que l'inf�rence est d�cidable.
\end{proof}

\begin{theorem}[D�cidabilit� de $\typea$]
  La relation de typage $`G \typea t : T$ est d�cidable.
\end{theorem}
\begin{proof}
  Direct. On utilise le th�or�me pr�c�dent pour le cas de l'application.
\end{proof}

%%% Local Variables: 
%%% mode: latex
%%% TeX-master: "subset-typing"
%%% LaTeX-command: "TEXINPUTS=\"style:$TEXINPUTS\" latex"
%%% End: 

\section{G�n�ration des obligations de preuve}
On veut d�sormais traduire les d�rivations du syst�me algorithmique
dans \CCI{} dont le jugement de typage est $\typec$. 
Les termes de \lng{} ne sont pas directement typables dans \CCI{}
puisque nous avons permis d'utiliser des objets comme s'ils avaient des
types diff�rents de leurs types originaux avec la r�gle de coercion. Il
va donc falloir maintenant expliciter ces coercions pour obtenir des
termes typables dans \CCI{}. Cependant, on ne peut pas cr�er un terme
complet � partir de notre d�rivation, puisqu'on ne peut pas inf�rer des
preuves arbitraires. On utilise donc des existentielles (intuitivement
des trous dont on ne connait que le type des habitants) pour traduire le
fait qu'il est de la responsabilit� de l'utilisateur de prouver que son
utilisation de la coercion n'�tait pas incorrecte.

\subsection{Interpr�tation}
On d�finit l'interpr�tation $\ip{t}{`G}$ par r�currence sur la forme des
termes (figure \ref{fig:interp}). Cette interpr�tation renvoie
un terme $t'$ r�ecrit que l'on montrera bien typ� dans l'environnement \CCI{} $\ipG{`G}$.

\begin{definition}[Interpr�tation des contextes]
  \label{ctx-interp}
  On fait l'extension aux contextes de la fa�on suivante:
  \begin{itemize}
  \item $\ipG{[]} = []$
  \item $\ipG{`G, x : T} =  \ipG{`G}, x : \ip{T}{`G}$
  \end{itemize}
\end{definition}

\def\typeafn#1#2{\typeml_{#1}(#2)}
\begin{figure}
  \[\begin{array}{lcll}
    \ip{x}{`G} & = & x & \\
    \\
    \ip{s}{`G} & = & s & s `: \setproptype\\
    \\
    \ip{\Pi x : T.U}{`G} 
    & = & \Pi x : \ip{T}{`G}.\ip{U}{`G, x : T} & \\
    & \\
    \ip{\lambda x : `t.v}{`G} 
    & = & \letml~`t' = \ip{`t}{`G}~\inml & \\
    & & \letml~v' = \ip{v}{`G, x : `t}~\inml & \\
    & & (\lambda x : `t'. v') & \\
    & \\
    \ip{f~u}{`G} 
    & = & \letml~F~=\typeafn{`G}{f}~\andml~U = \typeafn{`G}{u}~\inml & \\
    & & \letml~(\Pi x : V.W) = \mualgo{F}~\inml & \\
    & & \letml~\pi = \coerce{`G}{F}{(\Pi x : V.W)} & \\
    & & \letml~c = \coerce{`G}{U}{V}\inml & \\
    & & (\pi~\ip{f}{`G})~(c~\ip{u}{`G}) & \\
    & \\
    \ip{\Sigma x : T.U}{`G} 
    & = & \Sigma x : \ip{T}{`G}.\ip{U}{`G, x : T} & \\
    & \\
    \ip{\pair{\Sigma x : T.U}{t}{u}}{`G}
    & = & \letml~t' = \ip{t}{`G}~\inml & \\
    & & \letml~T' = \typeafn{`G}{t}~\inml& \\
    & & \letml~ct = \coerce{`G}{T'}{T}~\inml& \\
    & & \letml~U' = \typeafn{`G}{u}~\inml& \\
    & & \letml~u' = \ip{u}{`G}~\inml & \\
    & & \letml~cu = \coerce{`G}{U'}{U[t/x]}~\inml & \\
    & & \pair{\ip{\Sigma x : T.U}{`G}}{ct[t']}{cu[u']} & \\
    & \\
    \ip{\pi_i~t}{`G} 
    & = & \letml~t' = \ip{t}{`G}~\inml & i `: \{ 1, 2 \} \\
    & & \letml~T = \typeafn{`G}{t}~\inml & \\
    & & \letml~\Sigma x : V.W = \mualgo{T}~\inml & \\
    & & \letml~c = \coerce{`G}{T}{(\Sigma x : V.W)}~\inml & \\
    & & \pi_i~c[t'] & \\
    & \\
    \ip{\mysubset{x}{U}{P}}{`G}
    & = & \mysubset{x}{\ip{U}{`G}}{\ip{P}{`G, x : U}}
  \end{array}\]
  \caption{Interpr�tation dans \CCI{}}
  \label{fig:interp}
\end{figure}

Chaque jugement de coercion du syst�me algorithmique permet de d�river
une coercion explicite qui sera directement appliqu�e � un objet.

On formalise donc les coercions par des contextes d'�valuation classiques.
\begin{definition}[Contextes d'�valuation]
  \label{eval-ctx}
  Un contexte d'�valuation est un terme form� � partir de la grammaire
  originale des termes � laquelle on ajoute des terminaux $\ctxdot$ dans
  chacune de r�gles.
\end{definition}

\begin{definition}[Substitution et composition de coercions]
  La substitution (l'application) dans un contexte d'�valuation est not�e $c[d]$, elle
  remplace toutes les occurrences de $\ctxdot$ dans $c$ par $d$.

  Le composition de deux coercions not�e $c `o d$ est �gale � $c[d]$,
  son �l�ment neutre est $\ctxdot$.
  
  La substitution d'un terme pour une variable dans un contexte
  d'�valuation est not�e $c[t/x]$ comme pour les termes.
\end{definition}

\subsubsection{Coercions explicites}
On d�finit le syst�me $\suba$ (figure \ref{fig:coerce-impl-rules})
qui d�rive une coercion � partir de deux types $S$ et $T$ dans un environnement $`G$.
On a introduit du d�terminisme par rapport au jugement de
coercion algorithmique puisqu'on donne priorit� � la r�gle
\irule{SubSub} par rapport � la r�gle \irule{SubProof} (ces r�gles sont
confluentes comme nous le monterons lemme \ref{coerce-unicity}). On explicite aussi la priorit� donn�e � la mise en forme
normale de t�te (figure \ref{fig:hnfdef}) puis � la d�rivation
par rapport au test de conversion dans la pr�misse de \irule{SubConv}.

Notre op�ration de mise en forme normale de t�te est d�finie de la fa�on suivante:
\begin{figure}[ht]
  \[\begin{array}{lcll}
    \hnf{((\lambda x : T.e)~v)} & = & \hnf{e[v/x]} & \\
    \hnf{\pi_1~(x, y)} & = & \hnf{x} & \\
    \hnf{\pi_2~(x, y)} & = & \hnf{y} & \\
    \hnf{e} & = & e & \{\text{si $e$ est d'une autre forme}\}
  \end{array}\]
  \caption{D�finition de la r�duction de t�te}
  \label{fig:hnfdef}
\end{figure}

On note donc $\hnf{T}$ la forme normale de t�te $T$ et $\nf{T}$ la forme normale
de $T$.

\begin{figure}[ht]
  \[\begin{array}{llcll}
    (\beta) & (\lambda x : X.e)~v & = & e[v/x] & \\
    (\pi_i) & \pi_i~\pair{T}{e_1}{e_2} & = & e_i & \\
    (\sigma_i) & \sigma_i~(\elt{E}{P}{e_1}{e_2}) & = & e_i & \\
    (\eta) & (\lambda x : X.e~x) & = & e & \text{si $x `; FV(e)$} \\ % et $e : \Pi x : X.Y$} \\
    (\rho) & \pair{\Sigma x : X.Y}{\pi_1~e}{\pi_2~e} & = & e & \text{si $e : \Sigma x : X. Y$} \\
    & \elt{E}{P}{(\eltpit~e)}{(\eltpip~e)} & = & e & \text{si $e : \mysubset{x}{E}{P}$} \\
    (\sigma) & \elt{E}{P}{t}{p} & = & \elt{E}{P}{t'}{p'} & \text{si $t
      `= t'$}
  \end{array}\]
  \caption{Th�orie �quationnelle de \CCI{}}
  \label{fig:eqcci}
\end{figure}

On utilise l'�quivalence $\eqbpers$ d�finie comme la cl�ture r�flexive,
sym�trique et transitive de la relation d�finie figure
\ref{fig:eqcci}. Cette relation sera d�not�e par $`=$ pour plus de
clart�. Cette relation contient la $\beta$-r�duction et les projections
pour les sommes d�pendantes, mais aussi des relations n�cessaires pour
supporter l'interpr�tation de termes de \lng{} dans le langage. On a
donc la r�gle $\eta$ pour l'abstraction et $\rho$ pour le
\emph{surjective pairing} qui s'applique aux sommes d�pendantes et aux
objets de type sous-ensemble. Enfin on a une forme limit�e
d'indiff�rence aux preuves pour les objets de type sous-ensemble.
On ajoute une r�gle de typage au syst�me de \CCI{} pour typer les
existentielles:
\begin{prooftree}
  \AXC{$\tcoq{`G}{P}{\Prop}$}
  \UIC{$\tcoq{`G}{\ex{P}}{P}$}
\end{prooftree}

% enrichie avec les
%existentielles, on �tend donc la relation aux existentielles de la fa�on
%suivante: \[\ex{`G}{P} \eqbres \ex{`G'}{P'} `= `G \eqbres `G' `^ P \eqbres P'\]

\subtiFig

Le syst�me figure \ref{fig:coerce-impl-rules} d�rive les termes de
coercion. Il a de bonnes propri�t�s pour la preuve et l'impl�mentation
telles que l'unicit� et l'admissibilit� de la transitivit� que nous
montrerons plus tard. 

\subsection{Propri�t�s}
On veut montrer que si l'on a un jugement valide dans notre syst�me
algorithmique, alors son image par l'interpr�tation est un jugement
valide de \CCI{}. On rappelle que \CCI{} est �quivalent au premier
calcul pr�sent� o� la r�gle de coercion est remplac�e par la r�gle 
de conversion.

\typenvi

\subsubsection{Correction}
Notre probl�me se ram�ne � montrer le th�or�me suivant: 
\[`G \typea t : T "=>" \iG \typec \ip{t}{`G} : \ip{T}{`G}\]
Ce r�sultat ne se montre pas ais�ment.
En effet le jugement de coercion rend la preuve
tr�s difficile � cause de son caract�re
non local. Pour mieux comprendre ce probl�me, consid�rons l'exemple
suivant:

\paragraph{Exemple}
Dans le syst�me algorithmique, on peut tr�s bien d�river
$\Pi n : \nat. \sref{list}~n \suba \Pi n :
\mysubset{x}{\nat}{P}.\listml~n$ puisque $\mysubset{x}{\nat}{P} \suba
\nat$ et $\listml~n \eqbr \listml~n$.
Si l'on interpr�te ces deux types, une coercion va �tre ins�r�e dans le
second type: $\ip{\Pi n : \mysubset{x}{\nat}{P}.\listml~n}{`G} = \Pi n :
\mysubset{x}{\nat}{P}.\listml~(\pi_1~n)$. La coercion g�n�r�e doit donc
avoir pour type: $\Pi n : \nat. \listml~n "->" \Pi n :
\mysubset{x}{\nat}{P}.\listml~(\pi_1~n)$, mais elle est d�riv�e en se
basant seulement sur les types algorithmiques. On peut v�rifier ici
que l'intuition de la coercion par pr�dicats est bonne, puisqu'on peut
d�river ce jugement:

\begin{prooftree}
  \AXC{$\nat \eqbr \nat$}
  \UIC{$\subimpl{`G'}{\ctxdot}{\nat}{\nat}$}
  \UIC{$\subimpl{`G'}{\pi_1~\ctxdot}{\mysubset{x}{\nat}{P}}{\nat}$}

  \AXC{$\listml~n \eqbr \listml~n$}
  \UIC{$\subimpl{`G, n : \mysubset{x}{\nat}{P}}{\ctxdot}
    {\listml~n}{\listml~n}$}

  \BIC{$\subimpl{`G'}{\lambda x : \ip{\mysubset{x}{\nat}{P}}{`G}.
      \ctxdot~[\ctxdot~(\pi_1~x)] = \ctxdot~(\pi_1 x)}{\Pi n : \nat. \listml~n}
    {\Pi n : \mysubset{x}{\nat}{P}.\listml~n}$}
\end{prooftree}

Supposons $`G \typec t : \Pi n : \nat. \listml~n$ alors on a la d�rivation
de typage suivante:
\begin{prooftree}
  
  \AXC{$\timpl{`G, x : \mysubset{x}{\nat}{\ip{P}{`G}}}{t}
    {\Pi n : \nat. \listml~n}$}

  \AXC{$\timpl{`G, x : \mysubset{x}{\nat}{\ip{P}{`G}}}{\pi_1~x}{\nat}$}
  
  \BIC{$\timpl{`G, x : \mysubset{x}{\nat}{\ip{P}{`G}}}
    {t~(\pi_1~x)}
    {\listml~(\pi_1~x)}$}

  \UIC{$\timpl{`G}
    {\lambda x : \ip{\mysubset{x}{\nat}{P}}{`G}. t~(\pi_1~x)}
    {\Pi n : \ip{\mysubset{x}{\nat}{P}}{`G}.\listml~(\pi_1~x)}$}
\end{prooftree}

On cr�e donc bien un terme de type $\ip{\Pi n :
  \mysubset{x}{\nat}{P}.\listml~n}{`G}$ en appliquant la coercion a un
terme de type $\ip{\Pi n : \nat. \listml~n}{`G}$, c'est l'effet recherch�.

\subsection{Coercions explicites}
On d�finit le syst�me $\subi$ (figure \ref{fig:subtyping-impl-rules})
qui d�rive une coercion � partir de deux types $S$ et $T$ dans un environnement $`G$.
On a introduit du d�terminisme par rapport au jugement de
coercion algorithmique puisqu'on donne priorit� � la r�gle
\irule{SubSub} par rapport � la r�gle \irule{SubProof} (ces r�gles sont
confluentes). On explicite aussi la priorit�e donn�e � la mise en forme
normale de t�te (figure \ref{fig:hnfdef}) puis � la d�rivation
par rapport au test de
conversion dans la pr�misse de \irule{SubConv}.

Notre op�ration de mise en forme normale de t�te est d�finie de la fa�on suivante:
\begin{figure}[h]
  \[\begin{array}{lcl}
    \hnf{((\lambda x : T.e)~v)} & = & \hnf{e[v/x]} \\
    \hnf{x} & = & x
  \end{array}\]
  \caption{D�finition de la forme normale de t�te}
  \label{fig:hnfdef}
\end{figure}

On utilise la $\beta$-�quivalence de $CCI$ enrichi avec les
existentielles, on �tend donc la relation aux existentielles de la fa�on
suivante: $\ex{`G}{P} \eqbi \ex{`G'}{P'} `= `G \eqbi `G' `^ P \eqbi P'$.

\coerceFig
\subticFig


\begin{lemma}[Coercion et conversion dans le contexte]
  \label{coercion-conversion-ctx-impl}
  Si $`G, x : V, `D \typec c : S \subi T$ et $V \eqbi V'$ alors
  $`G, x : V', `D \typec c' : S \subi T$ et $c \eqbi c'$.
\end{lemma}
\begin{proof}
  Par induction sur la d�rivation de coercion. Tout les cas passent par
  induction sauf \irule{SubConv} qui n'utilise pas $`G$ en pr�misse donc
  ne pose pas de probl�me. Le seul cas int�ressant est \irule{SubProof}
  qui utilise la $\beta$-�quivalence sur les existentielles.
\end{proof}

On conserve les propri�t�s suivantes sur la coercion:
\def\subhnf{\subihnf}
\begin{lemma}[Coercion et conversion]
  \label{coercion-conversion-impl}
  Si $`G \typea S,T,U,V : s$, $S \eqbi T$, $U \eqbi V$ et
  $`G \typec c : T \sub U$ alors $`G \typec c' : S \sub V$ avec 
  $c \eqbi c'$. 
\end{lemma}

\begin{proof}
  Par simple inspection des r�gles on voit que le jugement ne peut
  distinguer deux termes $\beta$-�quivalents (ils ont forc�ment des
  formes normales de t�te �quivalentes). On en d�duit que la derni�re
  r�gle appliqu�e dans le jugement $T \sub U$ est la seule r�gle
  applicable � $S \sub V$.

  Par induction sur la d�rivation dans $\sub$.
   
  \begin{induction}
    \case{SubHnf}\quad
    Comme $S \eqbi T$, $S \eqbi \hnf{T}$. 
    De m�me comme $U \eqbi V$, $\hnf{U} \eqbi V$.
    On a l'hypoth�se $\hnf{T} \sub \hnf{U}$ donc
    par induction on a $c' : \hnf{S} \sub \hnf{V}$ avec $c' \eqbi c$. 
    On d�rive bien $`G \typec c' : S \sub V$ avec $c' \eqbi c$ par \irule{SubHnf}. 
    
    \case{SubConv}\quad
    On a  $S \eqbi T$, $T \eqbi U$ et $U \eqbi V$ donc $S \eqbi V$ par transitivit�
    de la $\beta$-�quivalence. On a donc $c' = \lambda x : S.x
    \eqbi \lambda x : T.x$ d�rivable par une application optionnelle de
    \irule{SubHnf} puis \irule{SubConv}. 
    C'est la seule r�gle applicable pour d�river
    le jugement $S \sub V$, sinon on aurait des symboles de t�te
    diff�rents pour deux termes $\beta$-�quivalents. 

    \case{SubProd}\quad
    Soit $T = \Pi x : A.B$ et $U = \Pi x : C.D$.
    On a $\subimpl{`G}{c_1}{A}{C}$ et $\subimpl{`G, x : C}{c_2}{B[c_1~x/x]}{D}$
    Or si $S \eqbi \Pi x : A.B$ et $\Pi x : C.D \eqbi V$ alors $\hnf{S}
    = \Pi x : A'.B'$ avec $A \eqbi A'$ et $B \eqbi B'$ et $\hnf{V} =
    \Pi x : C'.D'$ avec $C \eqbi C'$ et $D \eqbi D'$.
      
    Par induction on a $`G \typec c_1' : C' \sub A'$ avec $c_1'
    \eqbi c_1$. Par hypoth�se on a $`G, x : C \typec c_2 : B[c_1~x/x]
    \sub D$. Par induction: $`G, x : C \typec c_2' : B'[c_1'~x/x] \sub
    D'$ avec $c_2' \eqbi c_2$ car $B[c_1~x/x] \eqbi B'[c_1~x/x]$.
    On peut donc d�duire $`G, x : C' \typec c_2' : B'[c_1'~x/x] \sub D'$
    par le lemme \ref{coercion-conversion-ctx-impl}. 
    On d�duit par
    \irule{SubProd} que $c' : \hnf{S} \sub \hnf{V}$ est
    d�rivable. On a bien $c \eqbi c'$ car $c_1' \eqbi c_1$ et $c_2'
    \eqbi c_2$, donc $\lambda f.\lambda x.c_2~(f~(c_1~x)) \eqbi
    \lambda f.\lambda x.c_2'~(f~(c_1'~x))$.

    \case{SubSigma}\quad
    De m�me par \irule{SubSigma}:
    
    \begin{prooftree}      
      \AXC{$\subimpl{`G}{c_1'}{A'}{C'}$}
      \AXC{$\subimpl{`G, x : A'}{c_2'}{B'}{D'[c_1'~x/x]}$}
      \BIC{$\subimplhnf{`G}{\lambda t : \Sigma x : A'. B'.~\letml~(x, y) = t~\inml~(c_1'~x, c_2'~y)}
        {\Sigma x : A'. B'}{\Sigma x : C'. D'}$}
    \end{prooftree}
 
    \case{SubSub}\quad  
    On a $\hnf{T} = \mysubset{x}{T'}{P}$ donc $\hnf{S} = \mysubset{x}{S'}{P'}$
    avec $T' \eqbi S'$ et $P \eqbi P'$.
    Par hypoth�se, $\subimpl{`G}{c}{T'}{U}$, donc par induction,
    $\subimpl{`G}{c'}{S'}{\hnf{V}}$ ($U \eqbi \hnf{V}$) et par \irule{SubSub}, 
    $\subimpl{`G}{c' `o \pi_1}{\mysubset{x}{S'}{P'} = \hnf{S}}{\hnf{V}}$.
    Clairement, $c' `o \pi_1 \eqbi c `o \pi_1$, la propri�t� est donc
    bien v�rifi�e.
    
    \case{SubProof}\quad
    On a $U = \mysubset{x}{U'}{P}$.
    De m�me on obtient:
    \begin{prooftree}
      \UAX{SubProof}
      {$\subimpl{`G}{c'}{\hnf{S}}{V' \eqbi U'}$}
      {$\subimplhnf{`G}
        {\lambda t : \hnf{S}.~\elt{V'}{(\lambda x : V'.P')}{(c'~t)}{?_{P'[c'~t/x]}}}
        {\hnf{S}}{\hnf{V} = \mysubset{x}{V'}{P'}}$}
      {}
    \end{prooftree}

    Encore une fois $c \eqbi c'$ est v�rifi� car $\ex{`G, t :
      \hnf{S}}{P'[c'~t/x]} \eqbi \ex{`G, x : T}{P[c~t/x]}$.
  \end{induction}
\end{proof}

\begin{lemma}[Coercion et formes normales de t�te]
  \label{substi-coercion-hnf}
  Si $`G \typec c : T \suba U$ alors $`G \typec c' : \hnf{T}~\suba
  \hnf{U}$ avec $c \eqbi c'$ est d�rivable par une
  d�rivation plus petite ou �gale.
\end{lemma}

\begin{proof}
  Par idempotence de la mise en forme normale de t�te, on a la m�me
  d�rivation dans le cas ou la derni�re r�gle appliqu�e �tait
  \irule{SubHnf}, sinon c'est trivial.
\end{proof}

\begin{lemma}[Coercion de termes convertibles]
  \label{subti-eqb-coercion-eqbe-id}
  Si $`G \typea T, U : s$ et $T \eqb U$ alors il existe $c$,
  $`G \typec c : T \suba U$ avec $c \eqbe \lambda x : T.x$.
\end{lemma}

\begin{proof}
  Par induction sur la forme normale de $T$ not�e $\nf{T}$.

  \begin{itemize}
  \item $\nf{T} = \Pi y : A.B$ Alors, $\nf{U} = \Pi y : A'.B'$ avec 
    $A \eqb A'$, $B \eqb B'$. Par induction, $c_1 \eqbe \lambda x : \ip{A'}{`G}.x
    : A' \sub A$ et $c_2 \eqbe \lambda y : \ip{B'}{`G, x : A'}.y : B
    \sub B'$. On a donc:
    \begin{eqnarray*}
      c & = & \lambda f : \ip{\Pi x : A.B}{`G}.\lambda x :
      \ip{A'}{`G}. c_2~f~(c_1~x) \\
      & \eqbe & \lambda f : \ip{\Pi x : A.B}{`G}.\lambda x :
      \ip{A'}{`G}. f~x \\
      & \eqbe & \lambda f : \ip{\Pi x : A.B}{`G}. f
    \end{eqnarray*}
    De fa�on �quivalente pour $\Sigma, \{|\}$.
    
  \item Si $\nf{T}$ n'a pas pour symbole de t�te, $\Pi, \Sigma$ ou
    $\{|\}$, alors il est possible que $U$ est pour symbole de t�te un
    type sous-ensemble auquel cas le m�canisme est similaire. 
    Dans tout les autres cas, \irule{SubConv} est la seule r�gle
    applicable et on a $c = \lambda x : T.x$.
  \end{itemize}
  
\end{proof}


\input{subset-typing-impl-subtyping-subst}

\input{subset-typing-impl-subtyping-trans}

\begin{lemma}[Coercion identit�]
  Si $`G \typec T : s$ alors il existe $cid_T$,
  $\subimpl{`G}{cid_T}{T}{T}$ et pour tout $t$, $`G \typec t : T$,
  $cid_T t \eqbi t$.
\end{lemma}

\begin{proof}
  Par induction sur la forme de $T$.

  

\end{proof}


On peut maintenant �noncer le lemme de symm�trie:

\begin{lemma}[Sym�trie de la coercion]
  \label{subi-sym}
  S'il existe $c$ tel que $`G \typec c : A \subi B$
  alors $`E!c^{-1}, `G \typec c^{-1} : B \subi A$ et $c^{-1} `o c \eqbei
  \sref{id}~A$ et $c `o c^{-1} \eqbei \sref{id}~B$, o�
  $\sref{id} \coloneqq \lambda X : Set.\lambda x : X. x$.
\end{lemma}
\begin{proof}
  On sait que le jugement $\subi$ est symm�trique, c'est � dire qu'on a
  l'existence des inverses. On utilise la transitivit� pour montrer le
  reste du lemme. Si $\subimpl{`G}{c}{A}{B}$ et
  $\subimpl{`G}{c^{-1}}{B}{A}$ alors il existe $f$ et $f^{-1}$ tel que
  $\subimpl{`G}{f \eqbi c^{-1} `o c}{A}{A}$ et $\subimpl{`G}{f^{-1}
    \eqbi c `o c^{-1}}{B}{B}$. Par unicit� des coercions, $f \eqbei \lambda
  x : A.x$ et $f^{-1} \eqbei \lambda x : B.x$. En g�n�ral $f$ et $f'$
  sont en forme $\eta$-longue et pas l'identit�.
\end{proof}

%%% Local Variables: 
%%% mode: latex
%%% TeX-master: "subset-typing"
%%% LaTeX-command: "TEXINPUTS=\"style:$TEXINPUTS\" latex"
%%% End: 


%%% Local Variables: 
%%% mode: latex
%%% TeX-master: "subset-typing"
%%% LaTeX-command: "TEXINPUTS=\"style:$TEXINPUTS\" latex"
%%% End: 


\chapter{\Subtac}
Nous avons d\'evelopp\'e la contribution \Subtac{} (pour
``subset-tactics'') disponible dans la version
\CVS{}~courante de \Coq{} (\url{http://coq.inria.fr}). Elle permet de
typer un programme en \lng{} et g\'en\'erer un terme incomplet
correspondant (voir annexe \ref{fig:euclid-subtac}). 

\section{Existentielles}
La g\'en\'eration des buts correspondant aux variables existentielles et la
formation du terme final devaient originellement \^etre laiss\'ees \`a la
tactique \Refine~et au syst\`eme de gestion des existentielles de \Coq. Certaines limitations 
dans l'implantation du raffinement (le m\'ecanisme permettant de manipuler
des termes ``\`a trous'') nous ont emp\^ech\'e d'utiliser \Refine. En
particulier, la gestion des d\'efinitions r\'ecursives et la pr\'esence de
variables existentielles dans les types d'autres existentielles
 n'\'etaient pas support\'ees. En
cons\'equence, nous avons d\'evelopp\'e une nouvelle tactique \Coq{}
permettant de g\'erer les termes avec existentielles de fa\c con plus
g\'en\'erale. 

\subsection{La tactique \eterm}
L'id\'ee de d\'epart de la tactique \Refine{} est de prendre un terme \`a
trous et d'en faire une traduction en une s\'equence de tactiques. Par
exemple, lorsque \Refine{} rencontre une abstraction, il fait une
introduction, lorsque c'est un cast, on applique l'identit\'e et ainsi de
suite. Intuitivement, la s\'equence de tactiques engendr\'ee va construire
le terme de d\'epart implicitement. 

La tactique \eterm{} fonctionne diff\'eremment. \`A partir d'un terme $t$
contenant des existentielles, \eterm{} va g\'en\'eraliser le terme par
rapport \`a celles-ci, et g�n�raliser chaque existentielle par rapport �
son contexte, cr\'eant ainsi un objet $(\lambda ex_1 : T_1, \ldots, ex_n :
T_n, t[?_1 := ex_1, \ldots, ?_n := ex_n])$, o� chaque $ex_i$ est appliqu�
aux variables introduites dans son contexte par les abstractions.
Habituellement, on propose $t$ comme habitant d'un type $T$ donn\'e (le
but), par exemple on peut proposer $\lambda x : \nat.x$ comme preuve du
but $\nat "->" \nat$.
Plut\^ot que de donner directement $t$, on applique le nouveau
terme, et \Coq{} va automatiquement nous demander d'instancier les
arguments $ex_1 \ldots ex_n$ correspondant aux existentielles du terme 
$t$. Cette technique permet d'avoir des d\'ependances entre existentielles
(par exemple, $ex_1$ peut appara�tre dans tous les types $T_2 \ldots
T_n$) et de ne pas reposer enti�rement sur la gestion des existentielles
de \Coq{} qui n'est pas tr\`es flexible \`a l'heure actuelle. En
particulier, si l'on applique une produits o� il y a des d�pendances
entre arguments, \Coq{} va recr�er des existentielles.

Il nous faut nous pencher un peu plus avant sur la g\'en\'eralisation des
existentielles pour comprendre le m\'ecanisme d'\eterm.
Puisqu'on veut pouvoir avoir des d\'ependances entres les $n$
existentielles d'un terme et qu'on s\'erialise celles-ci en un produit
$n$-aire, il nous faut \^etre tr\`es attentifs \`a l'ordre dans lequel on
g\'en\'eralise les variables existentielles. Si $?_3 : T_3$ o\`u $T_3$
r\'ef\'erence $?_4$, il faut que l'existentielle $ex_4$ apparaisse
\emph{avant} $ex_3$ dans notre produit. Il est toujours possible de
trouver un ordre compatible avec ces d\'ependances puisqu'il est
impossible de cr\'eer un cycle o\`u $?_i$ r\'ef\'erencerait $?_j$ et vice-versa
(ceci est assur\'e par le caract\`ere fonctionnel des objets implantant
les existentielles dans \Coq). La tactique actuelle est cod\'ee avec
l'hypoth\`ese que toute existentielle $?_i$ ne d\'epend pas des
existentielles d'indice sup\'erieur \`a $i$. Il est cependant envisageable de
r��crire tout terme contenant des existentielles comme un terme
\'equivalent avec des indices respectant cet ordre.

\section{Traitement de la r\'ecursion}
Lorsque l'on d\'eveloppe un programme r\'ecursif dans un syst\`eme tel que
\Coq, on est forc\'e de fournir une preuve de sa terminaison. Pour cela,
on montre g\'en\'eralement qu'on a un ordre bien
fond\'e sur le type de l'argument de r\'ecursion et que chaque appel respecte
cet ordre. Nous avons ajout\'e des facilit\'es d'\'ecriture de fonctions
r\'ecursives \`a notre langage ; on ajoute les existentielles
correspondant aux preuves que l'ordre est bien fond\'e ou qu'il est bien
respect\'e par les termes. Ainsi lors du raffinement on obtient naturellement
les buts correspondants \`a prouver.


La possibilit� de faire des fonctions r�cursives �
l'int�rieur des termes devraient �tre facile � introduire, il suffit
d'avoir un combinateur $\sref{fix}$ comme constante avec une forme
pratique pour le langage: par exemple la fonction utilisable pour
l'appel r�cursif devrait avoir un type de la forme:
$\mysubset{x}{T}{R~x~a} "->" B$ o� $R$ est la relation bien fond�e et $T$ le type de
l'argument de r�cursion ainsi on g�n�rera les obligations automatiquement
lors des appels r�cursifs.

\section{Traitement des inductifs}
Notre langage ne prend pas encore en compte les d\'efinitions inductives g\'en\'erales.
Au-del\`a du traitement des types sous-ensemble, on a un support minimal
pour les inductifs \`a deux constructeurs qui correspondent \`a des bool\'eens
annot\'es par des propri\'et\'es logiques (voir traitement de la
conditionnelle figure \ref{fig:euclid-subtac}). A long terme on devrait
pouvoir traiter les inductifs dans $\Set$ pr\'edicatif, qui ne peuvent 
embarquer des propositions qu'en utilisant des types sous-ensemble avec
le m\^eme m\'ecanisme de coercion et conserver l'inf\'erence.

\section{\texttt{Program} et \texttt{Recursive program}}
On peut utiliser notre contribution � l'aide des deux tactiques
\texttt{Program} et \texttt{Recursive program}. 

\paragraph{\texttt{Program}}
La syntaxe pour l'appel de cette tactique est la suivante:
$\texttt{Program}~\sref{name}~:~`t := `a.$ o� $`t, `a$ d�nottent les
cat�gories syntaxiques des types et des termes respectivement.
La tactique fonctionne ainsi:
On inf�re le type du terme, on applique la coercion du terme vers le
type sp�cifi� puis on r��crit le terme obtenu dans \CCI{}. On r��crit
ensuite le type sp�cifi� dans \CCI{} et on le pose comme but. On utilise
\eterm{} qui va s'occuper du terme � trous et l'appliquer au but. On
obtient alors automatiquement les obligations de preuves. 

\paragraph{\texttt{Recursive program}}
Cette deuxi�me tactique permet d'�crire des d�finitions r�cursives, on va donc demander
plus d'informations � l'utilisateur lors de la d�finition:
\[\texttt{Recursive program}~\sref{name}~(a : `t_a) \{ \sref{wf}~R~(\sref{auto} `|
  \sref{proof}~p)? \} : ~`t := `a.\]
L'argument $a$ de type $`t_a$ est l'argument de r�cursion, $R$ est la
relation d'ordre qu'on doit montrer bien fond�e. Si l'on utilise
$\sref{auto}$ la tactique va chercher dans une base une preuve de bonne
fondation. Si l'on utilise $\sref{proof}$ alors $p$ doit r�f�rencer une
preuve de la bonne fondation de $R$ dans l'environnement courant. Sinon
on g�n�re une obligation de preuve que $R$ est bien fond�.
La tactique fait ensuite � peu pr�s la m�me chose que \texttt{Program}.
Le typage et la r��criture se font dans un contexte o� la
fonction $name : \Pi a : `t_a.`t$ appara�t et
l'on v�rifie � l'application si l'on fait un appel r�cursif auquel cas on
ins�re une existentielle correspondant � la preuve que l'ordre est bien
respect� (dans \Coq, la fonction permettant de faire la r�cursion aura
un type de la forme $\Pi x : `t_a. R~x~a "->" `t$). Un exemple de
l'utilisation de \texttt{Recursive program} est donn� figures
\ref{fig:euclid-subtac} et \ref{fig:euclid-subtac-script}.


%%% Local Variables: 
%%% mode: latex
%%% TeX-master: "subset-typing"
%%% LaTeX-command: "TEXINPUTS=\"style:$TEXINPUTS\" latex"
%%% End: 


%%% Local Variables: 
%%% mode: latex
%%% TeX-master: "subset-typing"
%%% LaTeX-command: "TEXINPUTS=\"style:$TEXINPUTS\" latex"
%%% End: 

\chapter{Conclusion}
Nous avons d�velopp� un langage de programmation plus souple que le
langage de \Coq{} mais conservant sa richesse d'expression (types
d�pendants). Il permet de d�coupler la description algorithmique de la
v�rification. La correction des termes engendr�s est
garantie par le syst�me sous-jacent qui offre ensuite la possibilit�
d'extraire un programme correct par construction dans un langage de type
\ML. D'autre part, cette m�thode s'int�gre bien dans l'environnement
\Coq{} et ouvre la voie � la r�alisation de travaux plus complexes par
des utilisateurs non-experts. Cela constitue la premi�re �tape vers un
environnement de programmation s�re utilisable dans \Coq. 

\section*{Travaux futurs}
Nous avons de multiples directions dans lesquelles �tendre notre
syst�me. Le support des inductifs, de la r�cursion interne, des univers cumulatifs
et des d�finitions dans les contextes semblent des probl�mes techniques
abordables. 
La possibilit� d'utiliser \lng{} dans
d'autres parties de \Coq{} (�nonc�s de lemmes, termes de tactiques) est
aussi un objectif int�ressant. Modifier \Coq{} pour qu'il g�re mieux les
existentielles est aussi un probl�me crucial dont nous avons fait
l'exp�rience durant le d�veloppement de \Subtac{}.

%%% Local Variables: 
%%% mode: latex
%%% TeX-master: "subset-typing"
%%% LaTeX-command: "TEXINPUTS=\"style:$TEXINPUTS\" latex"
%%% End: 


\ifthenelse{\boolean{showlog}}{
\clearpage
\section*{Journal}

\subsection*{8 mars}
Nouvelle r�gle de produit fonctionel avec contravariance bien typ�e,
produit d�pendant (\SubSigmaRule) covariant.
Un exemple de produit avec contravariance se trouve dans \cite{cal00coherence}, p. 6.
Exemple d'utilisation int�ressante:

\BAX{}
{$`G \judgetype f : \{ \phi : \even "->" `N `| `A x : `N, \phi~x ``<= x \}$}
{$`G \judgetype g : `N "->" `N := \matht{pred}$}
{$`G \judgesub g ``<= f$}
{}
\DP

Le sous-typage avec coercions: Luo, Callaghan, Sa�bi \cite{saibi97inheritance}... 
\begin{itemize}
\item Uniformit� du sous-typage: ne d�pend pas du contexte.
\item Coercions d�clar�es dans l'environement (ex: Coq).
\end{itemize}

Dans HOL, Joe Hurd simule le \emph{predicate subtyping} � la PVS avec
des \emph{predicate sets} \cite{hurd2001a}. Technique adaptable � Coq ?

%%% Local Variables: 
%%% mode: latex
%%% TeX-master: "~/research/coq/papers/subset-typing"
%%% End: 

\subsection*{9 mars}
Le \ps{} dans HOL n'est pas correct, il peut �tre subverti ais�ment �
cause d'une sorte de covariance des domaines de fonctions et de la
fonction de suppression des predicats (similaire � $\mu$):
$inv : `R^{\neq 0} "->" `R `: `R "->" `R$. Le d�veloppement dans \HOL{}
est fait directement dans le language, et il est argu� qu'il n'est pas
possible de le faire correctement � cause de l'exemple
pr�c�dent. L'algorithme de sous-typage est simplement une g�n�ration de
tout les sous-types possibles � partir d'un ensemble de r�gles de
sous-typage pour les constantes et constructions logiques ou fonctionelles.

Trouv� un article \cite{stumpsubset} sur les types sous-ensembles dans PF, logique d'ordre sup�rieur
avec fonctions partielles... Permet de traiter le cas
$\ifml 1 / i > 0 \thenml i \neq 0 \elseml `_$, qui g�n�re une obligation de
preuve $i \neq 0$ dans \PVS{}. Evidemment on ne risque pas de pouvoir
typer ce code en \Coq{} lorsque $`/ : `Z "->" \{ x : `Z `| x \neq 0 \} "->" `Z$ mais
certaines id�es peuvent �tre int�ressantes. 

%%% Local Variables: 
%%% mode: latex
%%% TeX-master: "~/research/coq/papers/subset-typing"
%%% End: 

\subsection*{11 mars}
Typage de la division euclidienne:
$\matht{div} : `A a : `N, `A b : \{ x : `N `| x \neq 0 \}, `S q : `N, `S r : \{ n : `N
`| n < b \}, a = bq + r := \funml a~b "=>" \ifml a < b \thenml (0, a) \elseml \letml (q, r) =
\matht{div}~(a - b)~b \inml (q + 1, r)$.

Soit $`t_{div} = `A a : `N, `A b : \{ x : `N `| x \neq 0 \}, `E q : `N, `E r : \{ n : `N
`| n < b \}, a = bq + r$ et $`G = \matht{div} : `t_{div}$:



\AXC{$1$}
\AXC{$2$}
\BIC{$`G, a : `N, b : `N^{*} \seq \ifml a < b \thenml (0, a) \elseml \letml (q, r) =
  \matht{div}~(a - b)~b \inml (q + 1, r) : `E q \dots$}
\doubleLine
\UIC{$`G \seq \funml a~b "=>" \ifml a < b \thenml (0, a) \elseml \letml (q, r) =
  \matht{div}~(a - b)~b \inml (q + 1, r) : `t_{div}$}
\DisplayProof

Soit $`G_{if} = `G, a : `N, b : `N^{*}, a ``/< b$.
\begin{prooftree}
  \AXC{$`G_{if} \seq b : `N^{*}$}
  \RightLabel{$`b = `N^{*}$}
  \UIC{$`G_{if} \seq b : `b$}

  \AXC{$`G_{if} \seq b : `g$}
  \AXC{$`G_{if} \seq (- a) : `g "->" `b''$}
  \AXC{$`G_{if} \judgetypea $}
  \TIC{$`G_{if} \seq (- a) b : `b''$}
  \AXC{$`G_{if} \seq \matht{div} : `b''' "->" `b' "->" `a$}
  \AXC{$`G_{if} \judgesubd a - b : `b'' ``<= `b''' "~>" t$}
  \UIC{$`G_{if} \seq \matht{div}~(a - b) "~>" \matht{div}~t : `b' "->" `a$}

  \TIC{$`G_{if} \seq \matht{div}~(a - b) : `b' "->" `a$}

  \AXC{$`G_{if} \judgesubd b : `b ``<= `b' "~>" p $}
  \TIC{$`G_{if} \seq \matht{div}~(a - b)~b "~>" \matht{div}~(a - b)~p : `a$}
  
  
  \AXC{$`G_{if}, (q, r) : `a \seq (q + 1, r) : `E q \dots$}
  \RightLabel{1}
  \BIC{$`G_{if} \seq \letml (q, r) = \matht{div}~(a - b)~b \inml (q + 1, r) : `E q \dots$}
\end{prooftree}  

\RightLabel{2}
\AXC{$`G, a : `N, b : `N^{*}, a < b \seq (0, a) : `E q \dots$}
\DisplayProof


%%% Local Variables: 
%%% mode: latex
%%% TeX-master: "~/research/coq/papers/subset-typing"
%%% End: 

\subsection*{14 mars}
R�gles d'introduction, d'�limination et de formation pour $\Pi$, $\Sigma$, sous-types
pr�dicats. Nouveau jugement de typage par r�ecriture. Nombreuses
questions � discuter avec Christine. Beaucoup de bruit dans le buro !


%%% Local Variables: 
%%% mode: latex
%%% TeX-master: "~/research/coq/papers/subset-typing"
%%% End: 

\subsection*{15 mars}
Distinction inf\'erence et typage. 
\begin{description}
\item[Inf\'erence ($"~>"$)] \`a la ML, on v\'erifie: $`G \typei p "~>" T "=>"
  `G \typed p ":" T$. 
\item[Typage ($:$)]. On a $`G \typed p : T$, on veut $`E U, `G \typei p
  "~>" U `^ `G \typei U "~>" T$. On utilise le sous-typage g\'en\`erant les
  obligations de preuve.
\end{description}

Eliminer \rname{LetSub}, inutilisable en pratique.

Quelques points \`a m\'editer:
\begin{itemize}
\item $`O \type 3 : `N "~>" `O \type 3 : `N^*$ ? D\'ependance envers le
  terme pour le sous-typage. De m\^eme, $2 : `N ``<= `N^*$, on devrait
  parler de renforcement.
\item On peut restreindre le sous-typage aux projections de types
  subsets avec l'\'egalit\'e syntaxique.
\item Je peux garder mes r\^egles de sous-typages, si elles sont syntax-directed!
\end{itemize}

Sous-typage \`a l'application et variable suffisante pour l'ad\'equation ?
On distingue les deux phases, pas de sous-typage \`a l'application.

V\'erifier Sub-{Left, Right}, l'application du sous-typage.

%%% Local Variables: 
%%% mode: latex
%%% TeX-master: "~/research/coq/papers/subset-typing"
%%% End: 

\subsection*{16 mars}
Il faut faire du sous-typage dans la sp�cification aussi:
$f : x : \subset{n}{`N}{0 \neq} "->" \subset{n}{`N}{x <}$.

Les deux phases:
\begin{description}
\item[Inf�rence] on donne les types impr�cis, ie: dans $x > n$, 
  $n "~>" `N$.
\item[Typage] on traverse la premi�re d�rivation de typage en ajoutant
  les coercions appropri�es, par exemple:
  $`G \type n : \subset{n}{`N}{0 \neq}$,  $\type_{inf} n "~>" `N$ est
  r�ecrit en: 
  $`G \type \pi_{1}~n : `N$.
\end{description}

\subsubsection*{Soir!}

Formalisation des trois jugements:
\begin{description}
\item[$\typed$] Typage d�claratif, syst�me ind�cidable, repr�sentant
  exactement ce qu'on veut ajouter comme fonctionnalit�.
\item[$\typei$] Version algorithmique, utilisant le dernier jugement
  pour r�aliser l'ad�quation avec la pr�sentation d�clarative.
\item[$\judgesubi$] ``Sous-typage'', sans obligations de preuves,
  d�cidable et � peu pr�s d�terministe.
\end{description}

Il faudra ensuite faire la traduction dans \Coq, avec de nouveaux
jugements r�ecrivant les d�rivations.

Propri�t�s � montrer:
\begin{itemize}
\item $`G \typed t : T "=>" `E U, `G \typei t : U `^ `G \judgesubi t : U \sub T$
\item $`G \typei t : T "=>" `G \typed t : T$
\end{itemize}

%%% Local Variables: 
%%% mode: latex
%%% TeX-master: "~/research/coq/papers/subset-typing"
%%% End: 

\subsection*{17 mars}
Preuves de substitutivit� \ref{substitutive-subtyping}, 
inversion \ref{inversion-subtyping}, admissibilit� de refl, trans dans
le sous-typage \ref{refl-trans-subtyping}, correction et compl�tude du typage.


%%% Local Variables: 
%%% mode: latex
%%% TeX-master: "~/research/coq/papers/subset-typing"
%%% End: 

\subsection*{22 mars}
Continue les preuves...
Trouver les bons lemmes de substitution!

Enlever la condition $A atomique$ dans \SubRightRule. Ca donne une
strat�gie d�terministe mais n'est pas indispensable dans cette pr�sentation.

%%% Local Variables: 
%%% mode: latex
%%% TeX-master: "~/research/coq/papers/subset-typing"
%%% End: 

\subsection*{23 mars}
Lu \cite{Chen:POPL-2003} en d�tail ainsi que
\cite{DBLP:journals/tcs/LuoS99} ou Zhaohui laisse en suspens la question
de la pertinence d'avoir des r�gles pour les produits reliant 
$\Pi x: A. \matht{list}~ `N$ et $\Pi x : A. \{ \matht{list}~ `N `| \ldots \}$ dans notre syst�me.
Une partie de \cite{DBLP:conf/csl/Luo96}. Plus int�ressant est
peut-�tre l'article sur les combinaisons incoh�rentes de coercions pour
les types sommes \cite{DPLB:conf/types/LuoL03}. Les objectifs de
coh�rence et d'�limination de la transitivit� ne sont pas loin des
notres: on veut que 'computationellement' les coercions soit
inessentielles (au contraire on accepte tout dans la partie logique)
et avoir un sous-typage avec de bonnes propri�t�s.


%%% Local Variables: 
%%% mode: latex
%%% TeX-master: "~/research/coq/papers/subset-typing"
%%% End: 

\subsection*{24 mars}
J'ai enlev� les termes du jugement de sous-typage d�claratif qui n'en a
pas besoin.

Preuves d'�quivalence entre syst�me algorithmique et d�claratif.
Plusieurs probl�mes:
\begin{itemize}
\item Besoin d'annotations de typage au let, ou alors restriction du
  type on a pas le droit d'utiliser le sous-typage au let. Ou alors on
  peut r�ecrire la d�rivation $x : S \sub t : T$ en $x : S' \sub t : T$
  lorsque $S' \sub S$ (narrowing) mais de fa�on � ce que l'on ajoute pas
  d'utilisation de $\sub$.
\item Application d'une fonction dans un type subset... on ne peut pas
  r�ecrire la d�rivation: soit on ajoute une fonction de
  'd�compr�hension' qu'on applique � gauche ou on autorise le
  sous-typage complet avec une r�gle du style:
  \begin{prooftree}
    \QAX{App}
    {$`G \seq f : T$}
    {$`G \subtd T \sub \Pi x : V. W$}
    {$`G \seq u : V' $}
    {$`G \subtd V' \sub V$}
    {$`G \seq (f u) : W [ u / x ]$}
    {$$}
  \end{prooftree}

  Mais on en revient � trouver une fonction pour d�cider de la deuxi�me
  pr�misse. On doit donc reprendre la fonction $\mu_0$ de \PVS{}.

  D�finition de $\mu_0$:
  \begin{eqnarray*}
    \subset{x}{U}{P} & "=>" & \mu_0~U \\
    x                & "=>" & x
  \end{eqnarray*}
  
  et l'on obtient la r�gle:
  \begin{prooftree}
    \QAX{App}
    {$`G \subtd f : T$}
    {$\mu_0~T = \Pi x : V. W$}
    {$`G \seq u : V' $}
    {$`G \subtd V' \sub V$}
    {$`G \seq (f u) : W [ u / x ]$}
    {$$}
  \end{prooftree}
\item ce n'est qu'une preuve informelle ;)
\end{itemize}

Comment traiter la conversion ? Voir la th�se de Chen.


%%% Local Variables: 
%%% mode: latex
%%% TeX-master: "~/research/coq/papers/subset-typing"
%%% End: 
        
\subsection*{25 mars}
Question de la transitivit� de notre "sous-typage".
On peut avoir/demander une forme restreinte de transitivit� du genre: 
$`G \subta t : A \sub B "~>" t' `^ `G \subta t' : B \sub C "~>" t'' "=>"
`G \subta t : A \sub C "~>" t''$. Mais dans un sens puisque notre
sous-typage d�pend des termes, il est clair que l'on a
$`G \subta t : A \sub B "~>" t' `^ `G \subta x : B \sub C "~>" t'' "=>"
`G \subta t : A \sub C "~>" t''$ si $x `; `G$ mais pas plus. Il faut
r�flechir � quelle est la solution la plus utile/souhaitable.
Lecture de la 3�me partie de \cite{ChenPhD} sur le sous-typage dans
$\lambda CC_{\leq}$.

Lecture d'articles sur la syntaxe abstraite pour le challenge {\sc
  PoplMark}\ldots


%%% Local Variables: 
%%% mode: latex
%%% TeX-master: "~/research/coq/papers/subset-typing"
%%% End: 
        
\subsection*{30 mars}
Quelques questions:
\begin{itemize} 
\item Unification pour \SumRule{} probl�matique ? Impl�mentation dans
  \Coq.
\item M�canisme d'annotations de type � ajouter ? Seulement aux lets ou
  partout ?
\end{itemize}

%%% Local Variables: 
%%% mode: latex
%%% TeX-master: "~/research/coq/papers/subset-typing"
%%% End: 
        
\subsection*{31 mars}
Corrections d'hier...


%%% Local Variables: 
%%% mode: latex
%%% TeX-master: "~/research/coq/papers/subset-typing"
%%% End: 
        
\subsection*{1er avril}
On enl�ve les contextes au sous-typage algorithmique d�finitivement.
Pour faciliter les preuves on consid�re que le sous-typage contient la
$\eqbi$ d�s le d�part.
Ecriture de la r�ecriture du sous-typage pour Coq. 
On fait du sous-typage sur les traductions et l'on devra montrer que $U
\suba V "=>" `E t, t : U' \subi V'$.
R�ecriture: $\subimpl{`G}{U}{V}{`G'}{t}{U'}{V'}$. On connait $`G, U, V,
`G', U', V'$ et l'on cherche la coercion $t$.
        
%%% Local Variables: 
%%% mode: latex
%%% TeX-master: "~/research/coq/papers/subset-typing"
%%% End: 
\subsection*{5 avril}
Correction r��criture du sous-typage.
Ajout des r�gles de formation de subset manquantes.
On change le jugement de typage de $\subset{x}{U}{P}$ pour $P$ ayant une
variable libre $x$.

Id�e de terme � trous instanci� ? (remplacerait $c u'$ dans la
conclusion de l'application). 

%%% Local Variables: 
%%% mode: latex
%%% TeX-master: "~/research/coq/papers/subset-typing"
%%% End: 
\subsection*{1\textsuperscript{er} Juin}
Fin du codage: il reste des probl�mes avec les existentielles mais �a
n'est pas ma partie.

Corrections du papier, �criture de la grammaire du langage.
Mon sujet de th�se est pr�t! Mon CV aussi.

%%% Local Variables: 
%%% mode: latex
%%% TeX-master: "~/research/coq/papers/subset-typing"
%%% End: 
\subsection*{6 Juin}
Fini la fiche SIREDO, fini le curriculum!



%%% Local Variables: 
%%% mode: latex
%%% TeX-master: "~/research/coq/papers/subset-typing"
%%% End: 
\subsection*{10 Juin}
Probl�mes du shadowing/masquage des variables dans l'env. (n�cessaire
pour renforcement...)

Ne devrait ont pas avoir \Set{} plut�t que \Type{} pour le type de la
donn�e des subsets ?? (utile pour inversion du produit).

Utilisation des formes normales pour les sortes (inversion du produit?).

Pour quoi ai-je prouv� l'inversion du produit ???

%%% Local Variables: 
%%% mode: latex
%%% TeX-master: "~/research/coq/papers/subset-typing"
%%% End: 
}{}

\newpage
\addcontentsline{toc}{chapter}{Bibliographie}
\bibliography{../bib/bib-joehurd,../bib/barras,../bib/pvs-bib,../bib/bcp,../bib/Luo,subset-typing,../bib/cparent/cparent}
\bibliographystyle{plain-url}

\renewcommand{\thefootnote}{}
\footnotetext{Ce rapport a �t� pr�par� sous \LaTeX~avec la fonte 
  \texttt{Computer Modern Bright}}

\appendix
\chapter{Exemples}
\begin{figure}[ht]
\begin{verbatim}
Definition div : forall a : nat, forall b : nat, 
  b <> 0 -> { q : nat & { r : nat | r < b /\ a = b * q + r } }.
Proof.
intros a ; pattern a ; apply lt_wf_rec ; intros. (* R�cursion *)
elim (lt_ge_dec n b). (* If then else *)
intros. (* Premi�re branche *)
(* Structure du terme *)
refine (existS _ 0 _) ; refine (exist _ n _) ; refine (conj _ _) ;
[ assumption | rewrite mult_0_r ; rewrite plus_0_l ; reflexivity ]. (* Preuve *)
(* Seconde branche *)
intros ; assert (n - b < n). (* Preuve pour l'appel *)
apply lt_minus ; [ apply (ge_le _ _ b0) | apply (nat_neq_0_gt_0 b H0) ].
induction (H (n - b) H1 b H0). (* Appel r�cursif *)
induction p ; induction p. (* Destruction du r�sultat *)
refine (existS _ (S x) _) ; refine (exist _ x0 _). (* Structure du terme *)
(* Preuve *)
split.
assumption.
pose (eq_plus_eq _ _ H3 b).
assert (n - b + b = n) ; try omega.
rewrite <- H4 ; rewrite e ; rewrite plus_comm ; rewrite plus_assoc.
replace (b + b * x) with (b * S x).
reflexivity.
rewrite mult_comm ; simpl ; pattern (x * b) ; rewrite mult_comm.
reflexivity.
Qed.
\end{verbatim}
  \caption{Script de preuve de la division euclidienne}
  \label{fig:euclid-script}
\end{figure}

\begin{figure}[ht]
\begin{verbatim}
(* Subtac ne g�re pas encore les notations de Coq *)
Definition neq (A : Type) (x y : A) : Prop := x <> y.
Definition div_prop (a b q r : nat) := a = (b * q) + r /\ r < b. 
Definition lt_ge_dec (x y : nat) : { x < y } + { x >= y }.
Proof.
  intros ; elim (le_lt_dec y x) ; intros ; auto with arith.
Defined.

Recursive program mydiv (a : nat) { well_founded lt a lt_wf } : 
  { b : nat | neq nat b O } ->
  [ q : nat ] { r : nat | div_prop a b q r } :=
  fun { y : nat | neq nat y O } =>
    if lt_ge_dec a y
    then (q := O, a : { r : nat | div_prop a y q r })
    else let (q', r) = mydiv (minus a y) y in 
        (q := S q', r : { r : nat | div_prop a y q r }).

(* Dans Coq, mydiv aura le type:
forall a : nat, forall b : { b : nat | b <> 0 },
{ q : nat & { r : nat | div_prop a (proj1_sig b) q r } } *)

(* Obligations de preuves engendr�es *)
(* Hypoth�ses communes: *)
a : nat
mydiv : (n : nat) n < a -> forall b : { b : nat | b <> 0 },
 { q : nat & { r : nat | div_prop n (proj1_sig b) q r } }
y : { b : nat | b <> 0 }

(* (q := 0, a ...)
[ H : a < proj1_sig y, |- div_prop a (proj1_sig y) 0 a]

(* Argument de r�cursion *)
[H : a >= proj1_sig y |- a - proj1_sig y < a]

(* (q := S q', r) *)
[ H : a >= proj1_sig y, q' : nat,
  r : { r : nat | div_prop (a - proj1_sig y) (proj1_sig y) q' r }
|- div_prop a (proj1_sig y) (S q')  (proj1_sig r)]
\end{verbatim}
  \caption{La division euclidienne avec \Subtac}
  \label{fig:euclid-subtac}
\end{figure}

\begin{figure}[ht]
\begin{verbatim}
Recursive program mydiv (a : nat) using lt proof lt_wf  :
  { b : nat | neq nat b O } -> [ q : nat ] { r : nat | div_prop a b q r } :=
  fun { b : nat | neq nat b O } =>
    if lt_ge_dec a b
      then (q := O, a : { r : nat | div_prop a b q r })
      else let (q', r) = mydiv (minus a b) b in
        (q := S q', r : { r : nat | div_prop a b q r }).
unfold neq ; simpl ; intros.
induction b ; simpl ; simpl in H ; omega.

unfold neq, div_prop ; simpl ; intros.
induction b ; induction r ; simpl ; simpl in H, p0 ; intuition.
rewrite mult_comm ; simpl.
rewrite mult_comm ; simpl.
omega.

unfold neq, div_prop ; simpl ; induction b ; simpl ; intros.
intuition.
Qed.
\end{verbatim}
  \caption{La division euclidienne avec \Subtac: script}
  \label{fig:euclid-subtac-script}
\end{figure}


%%% Local Variables: 
%%% mode: latex
%%% TeX-master: "subset-typing"
%%% LaTeX-command: "TEXINPUTS=\"..:style:$TEXINPUTS\" latex"
%%% End: 


\end{document}

%%% Local Variables: 
%%% mode: latex
%%% TeX-master: "subset-typing"
%%% LaTeX-command: "TEXINPUTS=\"style:$TEXINPUTS\" latex"
%%% End: 
