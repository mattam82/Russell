\documentclass[a4paper,11pt]{report}

\usepackage[francais]{babel} 
\usepackage[latin1]{inputenc}  %% les accents dans le fichier.tex
%\usepackage[T1]{fontenc}       %% Pour la c\'{e}sure des mots accentu\'{e}s
\usepackage{indentfirst}
\usepackage[dvips]{graphicx}
%\usepackage{coqdoc}
\usepackage{amssymb}
\usepackage{amsmath}
\usepackage{amsthm}
\usepackage{amsfonts}
\usepackage{array}
\usepackage{myabbrevs}
\usepackage{abbrevs}
%\usepackage{bnf}
\usepackage{bussproofs}
\usepackage{hyperref}
\usepackage{fullpage}
\usepackage{concmath}
%\usepackage{cmbright}
\usepackage{fancyhdr}
\usepackage{ifthen}
\usepackage{ifmtarg}
\usepackage{pxfonts}
\usepackage[xdvi]{xy}
\xyoption{all}
%\CompileMatrices

\usepackage[rgb,xdvi]{xcolor}
\usepackage{subfigure}

% This is the "centered" symbol
\def\fCenter{\vdash}
\def\seq{\fCenter}
% Optional to turn on the short abbreviations
\EnableBpAbbreviations

\oddsidemargin -1cm
\topmargin -1cm
\headsep 5mm
\textwidth 18cm
\textheight 24cm

\newboolean{defineTheorem}
\setboolean{defineTheorem}{true}
\ifthenelse{\boolean{defineTheorem}}{
  \newtheorem{theorem}{Th�or�me}[section]
  \newtheorem{lemma}[theorem]{Lemme}
  \newtheorem{fact}[theorem]{Fait}
  \newtheorem{proposition}[theorem]{Proposition}
  \newtheorem{definition}[theorem]{D�finition}
  \newtheorem{example}[theorem]{Exemple}
  \newtheorem{remark}[theorem]{Remarque}
  \newtheorem{corrolary}[theorem]{Corollaire}}{}

\makeatletter

\newcommand{\UR}[2]{\RightLabel{\rulelabel{#1}}\UIC{#2}}
\newcommand{\URL}[2]{\LeftLabel{\rulelabel{#1}}\UIC{#2}}

\newboolean{displayLabels}
\setboolean{displayLabels}{true}

\newcommand{\LeftRuleLabel}[1]{
  \@ifnotmtarg{#1}
  {\ifthenelse{\boolean{displayLabels}}{\LeftLabel{\rulelabel{#1}}}{}}
}

\newcommand{\UAX}[4]{\AXC{#2}
  \LeftRuleLabel{#1}
  \@ifnotmtarg{#4}{\RightLabel{#4}}
  \UIC{#3}}

\newcommand{\BAX}[5]{\AXC{#2}\AXC{#3}
  \LeftRuleLabel{#1}
  \@ifnotmtarg{#5}{\RightLabel{#5}}
  \BIC{#4}}

\newcommand{\TAX}[6]{\AXC{#2}\AXC{#3}\AXC{#4}
  \LeftRuleLabel{#1}
  \@ifnotmtarg{#6}{\RightLabel{#6}}
  \TIC{#5}}

\newcommand{\QAX}[7]{\AXC{#2}\noLine\UIC{#3}\AXC{#4}\AXC{#5}
  \LeftRuleLabel{#1}
  \@ifnotmtarg{#7}{\RightLabel{#7}}
  \TIC{#6}}

\newcommand{\BR}[2]{\RightLabel{\rulelabel{#1}}\BIC{#2}}
\newcommand{\BRL}[2]{\LeftLabel{\rulelabel{#1}}\BIC{#2}}

\makeatother

%%% Local Variables: 
%%% mode: latex
%%% TeX-master: t
%%% End: 


%\def\ifstreq#1#2{\ifEqString{#1}{#2}}

\def\impsub{\rightslice}

\def\judgewf{\vdash_{wf}}
\def\judgetyped{\vdash}
\def\judgetypea{\vdash_{\bullet}}
\def\judgetypei{\vdash_{\Box}}
\def\typed{\judgetyped}
\def\typedwf{\judgewf}
\def\typea{\judgetypea}
\def\typeawf{\judgetypea_{wf}}
\def\typei{\judgetypei}
\def\typeiwf{\judgetypei_{wf}}
\def\type{\typed}
\def\typewf{\typedwf}
\def\judgesubd{\vdash}
\def\judgesuba{\vdash_{\bullet}}
\def\judgesubi{\vdash_{\Box}}
\def\subtd{\judgesubd}
\def\subta{\judgesuba}
\def\subti{\judgesubi}
\def\subt{\subtd}
\def\subd{\rightslice}
\def\suba{\rightslice_{\bullet}}
\def\subi{\rightslice_{\Box}}
\def\sub{\subd}

\def\judgerw{~{\mbox{$"~>"$}}~}
\def\wf{\judgewf}

\newcommand{\matht}[1]{\text{{\tt #1}}}

\def\even{\matht{even}}
\def\odd{\matht{odd}}
\def\Set{\matht{Set}}
\def\Prop{\matht{Prop}}
\def\Type{\matht{Type}}
\def\ps{\emph{predicate subtyping}}

\def\WfAtomRule{Wf-Atom}
\def\WfVarRule{Wf-Var}
\def\PropSetRule{PropSet}
\def\TypeRule{Type}
\def\VarRule{Var}
\def\ProdRule{Prod}
\def\AbsRule{Abs}
\def\AppRule{App}
\def\LetInRule{Let-In}
\def\SigmaRule{Sigma}
\def\SumRule{Sum}
\def\LetSumRule{Let-Sum}
\def\SumInfRule{Sum-Inf}
\def\SumDepRule{Sum-Dep}
\def\SubsetRule{Subset}
\def\LetSubRule{Let-Sub}
\def\SubsumRule{Subsumption}
\def\ConvRule{Conv}

\def\SubReflRule{Sub-Refl}
\def\SubTransRule{Sub-Trans} 
\def\SubConvRule{Sub-Conv}
\def\SubProdRule{Sub-Prod}
\def\SubSigmaRule{Sub-Sigma}
\def\SubLeftRule{Sub-Left}
\def\SubRightRule{Sub-Right}
\def\SubProofRule{Sub-Proof}
\def\SubSubRule{Sub-Subset}
\def\SubTransRule{Sub-Trans}

\def\ifstreq#1#2{\def\testa{#1}\def\testb{#2}\ifx\testa\testb }

\def\inductionon#1{
  \ifstreq{#1}{typing-decl}{Par induction sur la d�rivation de typage.}
  \else\ifstreq{#1}{typing-algo}{Par induction sur la d�rivation de
    typage dans le syst�me algorithmique.}
  \else\ifstreq{#1}{typing-impl}{Par induction sur la d�rivation de
    typage.}
  \else\ifstreq{#1}{subtyping-decl}{Par induction sur la d�rivation de
    sous-typage.}
  \else\ifstreq{#1}{subtyping-algo}{Par induction sur la d�rivation de
    sous-typage dans le syst�me algorithmique.}
  \else\ifstreq{#1}{subtyping-impl}{Par induction sur la d�rivation de
    sous-typage.}
  \fi
}

\newenvironment{induction}[1][text=\empty]{
  \if#1\empty\else\inductionon{#1}
  \begin{list}{Unset default item}{}}
  {\end{list}}

%% Should be able to work with \else...
\newcommand{\inductionrule}[1]
{\ifstreq{#1}{WfAtom}{\WfAtomRule}
\fi\ifstreq{#1}{WfVar}\WfVarRule
\fi\ifstreq{#1}{PropSet}\PropSetRule
\fi\ifstreq{#1}{Type}\TypeRule
\fi\ifstreq{#1}{Var}\VarRule
\fi\ifstreq{#1}{Prod}\ProdRule
\fi\ifstreq{#1}{Abs}\AbsRule
\fi\ifstreq{#1}{LetIn}\LetInRule
\fi\ifstreq{#1}{Sigma}\SigmaRule
\fi\ifstreq{#1}{Sum}\SumRule
\fi\ifstreq{#1}{LetSum}\LetSumRule
\fi\ifstreq{#1}{App}\AppRule
\fi\ifstreq{#1}{SumInf}\SumInfRule
\fi\ifstreq{#1}{SumDep}\SumDepRule
\fi\ifstreq{#1}{LetSub}\LetSubRule
\fi\ifstreq{#1}{Subsum}\SubsumRule
\fi\ifstreq{#1}{Conv}\ConvRule
\fi\ifstreq{#1}{Subset}\SubsetRule

\fi\ifstreq{#1}{SubRefl}\SubReflRule
\fi\ifstreq{#1}{SubTrans}\SubTransRule
\fi\ifstreq{#1}{SubConv}\SubConvRule
\fi\ifstreq{#1}{SubProd}\SubProdRule
\fi\ifstreq{#1}{SubSigma}\SubSigmaRule
\fi\ifstreq{#1}{SubLeft}\SubLeftRule
\fi\ifstreq{#1}{SubRight}\SubRightRule
\fi\ifstreq{#1}{SubProof}\SubProofRule
\fi\ifstreq{#1}{SubSub}\SubSubRule
\fi}

\newcommand{\rulename}[1]{{\bf \inductionrule{#1}}}

\def\indrule{\rulename}

\def\case#1{\item[- \rulename{#1} :]}
\newcommand{\casetwo}[2]{\item[- \rulename{#1}, \rulename{#2} :]}
\def\casethree#1#2#3{\item[- \rulename{#1}, \rulename{#2},
  \rulename{#3} :]}
\def\casefour#1#2#3#4{\item[- \rulename{#1}, \rulename{#2},
  \rulename{#3}, \rulename{#4} :]}

\newcommand{\rname}[1]{{\bf \rulename{#1}}}
\newcommand{\rulelabel}[1]{{\bf (\rulename[#1])}}

%%% Local Variables: 
%%% mode: latex
%%% TeX-master: "subset-typing"
%%% LaTeX-command: "TEXINPUTS=\"style:$TEXINPUTS\" latex"
%%% End: 


\setboolean{displayLabels}{true}

\newcommand{\termgrammar}
{$\begin{array}{lcl}
    `a & \Coloneqq & x \\
    & | & \lambda x : `t "=>" `a \\
    & | & `a~`a \\ 
    & | & `a~`t \\
    & | & \text{\emph{constante}} \\
% & | & (`a,~`a) \\
    & | & \pair{\Sigma x : `t.`t}{`a}{`a} \\
%    & | & \letml~x = `a ~\inml~`a \\
    & | & \pi_1~`a `| \pi_2~`a

%    & | & \ifml~`a~\thenml~`a~\elseml~`a
  \end{array}$}

\newcommand{\typegrammarOrig}
{$\begin{array}{lcl}
    `t & \Coloneqq & x \\
    & | & `t "->" `t \\
    & | & `t~`t \\
    & | & \text{\emph{constante}} \\
    & | & `t * `t \\
    & | & \Sigma x : `t. `t \\
    & | & \lambda x : s "=>" `t \\
    & | & `t~`a \\
    & | & \Pi x : `t. `t \\
    & | & \mysubset{x}{`t}{`t}
  \end{array}$}

\newcommand{\typegrammar}
{$\begin{array}{lcl}
    `t & \Coloneqq & x \\
    & | & `t~`t \\
    & | & `t~`a \\
    & | & \lambda x : `t "=>" `t \\
    & | & \Pi x : `t. `t \\
    & | & \Sigma x : `t. `t \\
%    & | & `t * `t \\
    & | & \mysubset{x}{`t}{`t} \\
    & | & \Set \\
    & | & \Prop \\
    & | & \Type \\
    & | & \text{\emph{constante}} 
%    & & \\
%    \text{{\tt Inductive}} & \Coloneqq & ident
  \end{array}$}



\newcommand{\WfEmptyFull}[1]{
  \UAX{WfEmpty}
  {}
  {$\seq []$}
  {}
}  
\newcommand{\WfEmpty}[1][\Gamma]{\WfEmptyFull{#1}}
  
\newcommand{\WfVarFull}[4]{
  \UAX{WfVar}
  {$\tgen{#1}{#2}{#3}$}
  {$\wf #1, #4 : #2$}
  {$#3 `: \setproptype{} `^{} #4 `; #1$}
}
\newcommand{\WfVar}[1][\Gamma]{\WfVarFull{#1}{A}{s}{x}}

\newcommand{\PropSetFull}[2]{
  \UAX{PropSet}
  {$\wf #1$}
  {$\tgen{#1}{#2}{\Type}$}
  {$#2 `: \setprop$} 
}
\newcommand{\PropSet}[1][\Gamma]{\PropSetFull{#1}{s}}


\newcommand{\TypeTypeFull}[1]{
  \UAX{Type}
  {$\wf #1$}
  {$\tgen{#1}{\Type(i)}{\Type(i + 1)}$}
  {$i `: `N$}
}
\newcommand{\TypeType}[1][\Gamma]{\TypeTypeFull{#1}}

\newcommand{\VarFull}[3]{
  \BAX{Var}
  {$\wf #1$}
  {$#2 : #3 `: #1$}
  {$\tgen{#1}{#2}{#3}$}
  {}
}
\newcommand{\Var}[1][\Gamma]{\VarFull{#1}{x}{A}} 

\newcommand{\ProdFull}[7]{
  \BAX{Prod}
  {$\tgen{#1}{#2}{#3}$}
  {$\tgen{#1, #4 : #2}{#5}{#6}$}
  {$\tgen{#1}{\Pi #4 : #2.#5}{#7}$}
  {$(#3, #6, #7) `: \mathcal{R} `^{} #4 `; #1$}
}
\newcommand{\Prod}[1][\Gamma]{\ProdFull{#1}{T}{s_1}{x}{U}{s_2}{s_3}}

\newcommand{\AbsFull}[6]{
  \BAX{Abs}
  {$\tgen{#1}{\Pi #2 : #3.#4}{#5}$}
  {$\tgen{#1, #2 : #3}{#6}{#4}$}
  {$\tgen{#1}{\lambda #2 : #3. #6}{\Pi #2 : #3.#4}$}
  {$#2 `; #1$}
}

\newcommand{\Abs}[1][\Gamma]{\AbsFull{#1}{x}{T}{U}{s}{M}}

 \newcommand{\AppFull}[6]{
  \BAX{App}
  {$\tgen{#1}{#2}{\Pi #3 : #4. #5}$}
  {$\tgen{#1}{#6}{#4}$}
  {$\tgen{#1}{(#2 #6)}{#5 [ #6 / #3 ]}$}
  {$$}
}

\newcommand{\App}[1][\Gamma]{\AppFull{#1}{f}{x}{V}{W}{u}}

\newcommand{\SigmaRFull}[5]{
  \BAX{Sigma}
  {$\tgen{#1}{#2}{#3}$}
  {$\tgen{#1, #4 : #2}{#5}{#3}$}
  {$\tgen{#1}{\Sigma #4 : #2.#5}{#3}$}
  {$#3 `: \{ \Prop, \Set \} `^{} #4 `; #1$} 
}
\newcommand{\SigmaR}[1][\Gamma]{\SigmaRFull{#1}{T}{s}{x}{U}}

\newcommand{\SumDepFull}[7][\Gamma]{
  \TAX{SumDep}
  {$\tgen{#1}{\Sigma #2 : #5. #6}{#7}$}
  {$\tgen{#1}{#3}{#5}$}
  {$\tgen{#1}{#4}{#6[#3 / #2]}$}
  {$\tgen{#1}{\pair{\Sigma #2 : #5.#6}{#3}{#4}}{\Sigma #2 : #5.#6}$}
  {}
}

\newcommand{\SumDep}[1][\Gamma]{\SumDepFull[#1]{x}{t}{u}{T}{U}{s}}
 
\newcommand{\PiLeftFull}[5][\Gamma]{
  \UAX{PiLeft}
  {$\tgen{#1}{#2}{`S #3 : #4.#5}$}
  {$\tgen{#1}{\pi_1~#2}{#4}$}
  {}
}
\newcommand{\PiLeft}[1][\Gamma]{\PiLeftFull[#1]{t}{x}{T}{U}}

\newcommand{\PiRightFull}[5][\Gamma]{
  \UAX{PiRight}
  {$\tgen{#1}{#2}{`S #3 : #4.#5}$}
  {$\tgen{#1}{\pi_2~#2}{#5[\pi_1~#2/#3]}$}
  {}
}
\newcommand{\PiRight}[1][\Gamma]{\PiRightFull[#1]{t}{x}{T}{U}}


\newcommand{\SubsetFull}[4]{
  \BAX{Subset}
  {$\tgen{#1}{#3}{\Set}$}
  {$\tgen{#1, #2 : #3}{#4}{\Prop}$}
  {$\tgen{#1}{\mysubset{#2}{#3}{#4}}{\Set}$}
  {$#2 `; #1$}
}
\newcommand{\SubsetR}[1][\Gamma]{\SubsetFull{#1}{x}{U}{P}}


\newcommand{\SubsumFull}[5]{
  \TAX{Subsum}
  {$\tgen{#1}{#4}{#5}$}
  {$\tgen{#1}{#2}{#3}$}
  {$#5 \sub #2$} % #1 \subt 
  {$\tgen{#1}{#4}{#2}$}
  {}
}
\newcommand{\Subsum}[1][\Gamma]{\SubsumFull{#1}{T}{s}{t}{U}} 

\def\Coerce{\Subsum}

\newcommand{\ConvFull}[5]{
  \TAX{Conv}
  {$\tgen{#1}{#2}{#3}$}
  {$\tgen{#1}{#4}{#5}$}
  {$#5 \eqbr #2$}
  {$\tgen{#1}{#4}{#2}$}
  {}
}
\newcommand{\Conv}[1][\Gamma]{\ConvFull{#1}{T}{s}{t}{U}}

\def\typedFig
{
\begin{figure}[tb]
  \begin{center}
    \def\fCenter{\wf}
    \WfEmpty \DP\quad
    \WfVar \DP
    
    \def\fCenter{\typed}
    \vspace{\infvspace}
    \PropSet\DP

%    \vspace{\infvspace}
%    \TypeType\DP

    \vspace{\infvspace}
    \Var\DP
    
    \vspace{\infvspace}
    \Prod\DP
    
    \vspace{\infvspace}
    \Abs\DP

    \vspace{\infvspace}
    \App\DP

    \vspace{\infvspace}
    \SigmaR\DP

    \vspace{\infvspace}
    \SumDep\DP
    
    \vspace{\infvspace}
    \PiLeft\DP
    \quad
    \PiRight\DP

    \vspace{\infvspace}
    \SubsetR\DP

    \vspace{\infvspace}
    \Subsum\DP
      
  \end{center}
  \caption{Calcul de coercion par pr�dicats - version d�clarative}
  \label{fig:typing-decl-rules}
\end{figure}
}

%%% Local Variables: 
%%% mode: latex
%%% TeX-master: "subset-typing"
%%% LaTeX-command: "TEXINPUTS=\"style:$TEXINPUTS\" latex"
%%% End: 

\def\SubConv{
\UAX{Sub-Conv}
{$T \eqbi U$}
{$T \sub U$}
{} 
}

\def\SubRefl{
  \UAX{Sub-Refl}
  {}
  {$S \seq S$}
  {}
}

\def\SubTrans{
\BAX{Sub-Trans}
{$S \seq T$}
{$T \seq U$}
{$S \seq U$}
{}
} 

\def\SubProd{
\BAX{Sub-Prod}
{$U \seq T$} %"<|-|>"
{$V \seq W$}
{$\Pi x : T.V \seq \Pi x : U.W$}
{}
}

\def\SubSigma{
  \BAX{Sub-Sigma}
  {$T \seq U$}
  {$V \seq W$}
  {$\Sigma x : T. V \seq \Sigma y : U. W$}
  {}
}

\def\SubProof{
  \UAX{Sub-Proof}
  {$U \seq V$}
  {$U \seq \subset{x}{V}{P}$}
  {}
}

\def\SubSub{
  \UAX{Sub-Subset}
  {$U \seq V$}
  {$\subset{x}{U}{P} \seq V$}
  {}
}

\begin{figure}[h]
  \begin{center}
    \def\fCenter{\subd}
    
    \vspace{\infvspace}
    \SubConv\DP

    \vspace{\infvspace}
    \SubTrans\DP

    \vspace{\infvspace}
    \SubProd\DP

    \vspace{\infvspace}
    \SubSigma\DP
    
    \vspace{\infvspace}
    \SubProof\DP
    
    \vspace{\infvspace}
    \SubSub\DP
    
  \end{center}
  \label{subtyping-algo-rules}
  \caption{Sous-typage d�claratif}
\end{figure}

%%% Local Variables: 
%%% mode: latex
%%% TeX-master: "subset-typing"
%%% End: 

\def\AppA{
\TAX{App}
{$`G \seq f : T \quad \mu~T `= \Pi x : V. W$}
{$`G \seq u : U$}
{$U \sub V $}
{$`G \seq (f u) : W [ u / x ]$}
{}
}

\def\SumInfA{
  \TAX{SumInf}
  {$`G \seq t : T $}
  {$`G \seq u : U $}
  {$`G \seq \Sigma \_ : T.U : s$}
  {$`G \seq (t, u) : \Sigma \_ : T.U$}
  {}
}

\def\SumDepA{
  \TAX{SumDep}
  {$`G \seq t : T $}
  {$`G \seq u : U[t/x] $}
  {$`G \seq \Sigma x : T.U : s$}
  {$`G \seq (t, u : U) : \Sigma x : T.U$}
  {}
}

\def\typeaFig{
\begin{figure}[h]
  \begin{center}
    \def\fCenter{\wf}
    \def\type{\typea}
    \def\subt{\subta}
    \def\sub{\suba}
    
    \WfAtom\DP    
    \WfVar\DP
    
    \def\fCenter{\typea}
    \vspace{\infvspace}
    \PropSet\DP
    
    \vspace{\infvspace}
    \Var\DP
    
    \vspace{\infvspace}
    \Prod\DP
    
    \vspace{\infvspace}
    \Abs\DP

    \vspace{\infvspace}
    \AppA\DP

    \vspace{\infvspace}
    \LetIn\DP
    
    \vspace{\infvspace}
    \SigmaR\DP

    \vspace{\infvspace}
    \SumInfA\DP

    \vspace{\infvspace}
    \SumDepA\DP
    
    \vspace{\infvspace}
    \LetSum\DP
    
    \vspace{\infvspace}
    \Subset\DP
  \end{center}
  \label{typing-algo-rules}
  \caption{Calcul des constructions avec sous-typage (version algorithmique)}
\end{figure}
}

\def\typemuaFig{
\begin{figure}[h]
  \begin{eqnarray*}
    \subset{x}{U}{P} & "=>" & \mu~U \\
    x                & "=>" & x
  \end{eqnarray*}
  \label{mu-definition}
  \caption{D�finition de $\mu$}
\end{figure}
}

%%% Local Variables: 
%%% mode: latex
%%% TeX-master: "subset-typing"
%%% LaTeX-command: "TEXINPUTS=\"style:$TEXINPUTS\" latex"
%%% End: 

\def\SubConv{
\UAX{Sub-Conv}
                                %{$`G \judgetypei x : T$}
{$T \eqbi U$}
{$`G \seq x : T \impsub U$}
{} 
}

\def\SubProd{
  \BAX{SubProd}
  {$`G \seq x : U \impsub T$} %"<|-|>"
  {$`G, x : U \seq v : V \impsub W$}
  {$`G \seq \lambda x : T. v : \Pi x : T.V 
    \impsub \Pi x : U.W$}
  {}
}
\def\SubSigma{
  \BAX{Sub-Sigma}
  {$`G \seq t : T \impsub U$}
  {$`G \seq v : V[t/x] \impsub W[t/y]$}
  {$`G \seq (t, v) : \Sigma x : T. V
    \impsub \Sigma y : U. W$}
  {}
}

\def\SubLeft{
  \BAX{Sub-Left}
  {$`G \seq p : U \impsub V$}
  {$`G \typei P : \Pi x : U. \Prop$}
  {$`G \seq p : \subset{x}{U}{P} \impsub V$}
  {}
}
    
\def\SubRight{
  \UAX{Sub-Right}
  {$`G \seq p : T \impsub U$}
                                %{$`G \judgetype h : P~p$}
  {$`G \seq p : T \impsub \subset{x}{U}{P}$}
  {}
}

\begin{figure}[h]
  \begin{center}
    \def\fCenter{\judgesub}
    
    \SubConv\DP
    \vspace{\infvspace}

    \SubProd\DP

    \vspace{\infvspace}
    \SubSigma\DP
    
    \vspace{\infvspace}
    \SubLeft\DP
    
    \vspace{\infvspace}
    \SubRight\DP
    
  \end{center}
  \label{subtyping-algo-rules}
  \caption{Sous-typage}
\end{figure}

%%% Local Variables: 
%%% mode: latex
%%% TeX-master: "subset-typing"
%%% End: 

\newcommand{\timpl}[6]{#1 \typea #2 : #3 "~>" #4 \typec #5 : #6}
\newcommand{\subimpl}[4]{#1 \typec #2 : #3 \sub #4}

\def\WfAtomI{
  \UAX{Wf-Atom}
  {}
  {$\wf [] "~>"~\wf []$}
  {}
}  
  
\def\WfVarI{
  \UAX{Wf-Var}
  {$\timpl{`G}{A}{s}{`G'}{A'}{s}$}
  {$\wf `G, x : A "~>"~\wf `G', x : A'$}
  {$s `: \{ \Set, \Prop, \Type(i) \}$}
}

\def\PropSetI{
  \UAX{PropSet}
  {$\wf `G "~>" ~\wf `G'$}
  {$\timpl{`G}{s}{\Type(0)}{`G'}{s}{\Type(0)}$}
  {$s `: \{ \Prop, \Set \}$} 
}

\def\TypeTypeI{
  \UAX{Type}
  {$\wf `G "~>" \wf `G$}
  {$\timpl{`G}{\Type(i)}{\Type(i + 1)}{`G'}{\Type(i)}{\Type(i + 1)}$}
  {}
}

\def\VarI{
  \BAX{Var}
  {$\wf `G "~>"~\wf `G'$}
  {$x : A `: `G "~>" x : A' `: `G'$}
  {$\timpl{`G}{x}{A}{`G'}{x}{A'}$}
  {}
}

\def\ProdI{
  \BAX{Prod}
  {$\timpl{`G}{T}{s1}{`G'}{T'}{s1}$}
  {$\timpl{`G, x : T}{U}{s2}{`G', x : T'}{U'}{s2}$}
  {$\timpl{`G}{\Pi x : T.U}{s2}{`G'}{\Pi x : T'.U'}{s2}$}
  {$(s1, s2) `: \text{{\sc Sort}}$}
}

\def\AbsI{
  \BAX{Abs}
  {$\timpl{`G}{\Pi x : T. U}{s}{`G'}{\Pi x : T'. U'}{s}$}
  {$\timpl{`G, x : T}{M}{U}{`G', x : T'}{M'}{U'}$}
  {$\timpl{`G}{\lambda x : T. M}{\Pi x : T.U}
    {`G'}{\lambda x : T'. M'}{\Pi x : T'.U'}$}
  {}
}

\def\AppI{
  \QAX{App}
  {$\timpl{`G}{f}{T}{`G'}{f'}{T'}$}
  {$\mu~T' `= (\pi, \Pi x : V'. W')$}
  {$\timpl{`G}{u}{U}{`G'}{u'}{U'}$}
  {$\subimpl{`G'}{c}{U'}{V'}$}
  {$\timpl{`G}{f u}{W[u/x]}{`G'}{(\pi~f')~(c~u')}{W'[ c~u' / x ]}$}
  {}
}

\def\LetInI{
  \BAX{LetIn}
  {$\timpl{`G}{t}{T}{`G'}{t'}{T'}$}
  {$\timpl{`G, x : T}{v}{V}{`G', x : T'}{v'}{V'}$}
  {$\timpl{`G}{\letml x = t \inml v}{V[t / x]}
    {`G'}{\letml x = t' \inml v'}{V'[t' / x]}$}
  {}
}

\def\SigmaRI{
  \BAX{Sigma}
  {$\timpl{`G}{T}{s1}{`G'}{T'}{s1}$}
  {$\timpl{`G, x : T}{U}{s2}{`G', x : T'}{U'}{s2}$}
  {$\timpl{`G}{\Sigma x : T.U}{s2}{`G'}{\Sigma x : T'.U'}{s2} $}
  {$(s1, s2) `: \matht{Sort}$}
}


\def\SumInfI{
  \TAXWide{SumInf}
  {$\timpl{`G}{t}{T}{`G'}{t'}{T'}$}
  {$\timpl{`G}{u}{U}{`G'}{u'}{U'}$}
  {$\timpl{`G}{\Sigma \_ : T.U}{s}{`G'}{\Sigma \_ : T'.U'}{s}$}
  {$\timpl{`G}{(t, u)}{\Sigma \_ : T.U}{`G'}{(t', u')}{\Sigma \_ : T'.U'}$}
  {}
}

\def\SumDepI{
  \TAXWide{SumDep}
  {$\timpl{`G}{t}{T}{`G'}{t'}{T'}$}
  {$\timpl{`G}{u}{U[t/x]}{`G'}{u'}{U'[t'/x]}$}
  {$\timpl{`G}{\Sigma x : T.U}{s}{`G'}{\Sigma x : T'.U'}{s}$}
  {$\timpl{`G}{(t, u : U)}{\Sigma x : T.U}{`G'}{(t', u')}{\Sigma x : T'.U'}$}
  {}
}
 
\def\LetSumI{
  \BAX{LetSum}
  {$\timpl{`G}{t}{\Sigma x : T. U}{`G'}{t'}{\Sigma x : T'.U'}$}
  {$\timpl{`G, x : T, u : U}{v}{V}{`G, x : T', u : U'}{v'}{V'}$}
  {$\timpl{`G}{\letml (x, u) = t \inml v}{V}
    {`G'}{\letml (x, u) = t' \inml v'}{V'}$}
  {}
}

\def\SubsetI{
  \BAX{Subset}
  {$\timpl{`G}{U}{\Type}{`G'}{U'}{\Type}$}
  {$\timpl{`G, x : U}{P}{\Prop}{`G', x : U'}{P'}{\Prop} $}
  {$\timpl{`G}{\subset{x}{U}{P}}{\Type}{`G'}{\subset{x}{U'}{P'}}{\Type}$}
  {$$}
}

\def\marginleft{0em}

\def\typeiFig{
\begin{figure}[h]
  \begin{center}
    \def\fCenter{\wf}
    \def\type{\typec}
    
    \WfAtomI\DP

    \vspace{\infvspace}
    \WfVarI\DP
    
    \vspace{\infvspace}
    \PropSetI\DP
    
    \vspace{\infvspace}
    \VarI\DP
    
    \vspace{\infvspace}
    \ProdI\DP
    
    \vspace{\infvspace}
    \AbsI\DP
    
    \vspace{\infvspace}
    \AppI\DP

    \vspace{\infvspace}
    \LetInI\DP
    
    \vspace{\infvspace}
    \SigmaRI\DP

    \vspace{\infvspace}
    \SumInfI\DP

    \vspace{\infvspace}
    \SumDepI\DP
    
    \vspace{\infvspace}
    \LetSumI\DP
    
    \vspace{\infvspace}
    \SubsetI\DP
  \end{center}
  \label{fig:typing-impl-rules}
  \caption{R�ecriture du typage vers \CCI}
\end{figure}
}


\def\typemuiFig{
\begin{figure}[h]
  \begin{eqnarray*}
    (f, \subset{x}{U}{P}))   & "=>" & \letml (f, t) = \muimpl~(f, U) \inml (f `o
    \pi_1, t) \\
    x                        & "=>"  & x
  \end{eqnarray*}
  \label{fig:muimpl-definition}
  \caption{D�finition de $\muimpl$}
\end{figure}
}

%%% Local Variables: 
%%% mode: latex
%%% TeX-master: "subset-typing"
%%% LaTeX-command: "TEXINPUTS=\"style:$TEXINPUTS\" latex"
%%% End: 

\def\SubConvI{
\UAX{Sub-Conv}
{$T \eqbi U$}
{$\subimpl{`G}{T}{U}{`O}{(\lambda x. x)}{T'}{U'}$}
{} 
}

\def\SubProdI{
  \BAX{Sub-Prod}
  {$\subimpl{`G}{U}{T}{`G'}{c_1}{U'}{T'}$} %"<|-|>"
  {$\subimpl{`G}{V}{W}{`G'}{c_2}{V'}{W'}$}
  {$\subimpl{`G}{\Pi x : T.V}{\Pi x : U.W}{`G'}
    {\lambda f. c_2 `o f `o c_1}
    {\Pi x : T'.V'}{\Pi x : U'.W'}$}
  {}
}

\def\SubSigmaI{
  \BAX{Sub-Sigma}
  {$\subimpl{`G}{T}{U}{`G'}{c_1}{T'}{U'}$}
  {$\subimpl{`G}{V}{W}{`G'}{c_2}{V'}{W'}$}
  {$\subimpl{`G}{\Sigma x : T. V}{\Sigma x : U. W}{`G'}
    {\lambda (x, y). (c_1~x, c_2~y)}
    {\Sigma x : T'. V'}{\Sigma x : U'. W'}
    $}
  {}
}

\def\SubLeftI{
  \BAX{Sub-Left}
  {$\subimpl{`G}{U}{V}{`G'}{c}{U'}{V'}$}
  {$\timpl{`G, x : U}{P}{\Prop}{`G', x : U'}{P'}{\Prop}$}
  {$\subimpl{`G}{\subset{x}{U}{P}}{V}{`G'}
    {c `o \pi_1}
    {\subset{x}{U'}{P'}}{V'}$}
  {}
}
    
\def\SubRightI{
  \BAX{Sub-Right}
  {$\timpl{`G, x : U}{P}{\Prop}{`G', x : U'}{P'}{\Prop}$}
  {$\subimpl{`G}{T}{U}{`G'}{c}{T'}{U'}$}
  {$\subimpl{`G}{T}{\subset{x}{U}{P}}
    {`G'}{\lambda t : T'. \elt{U'}{P'}{(c~t)}{?(P'[c~t/x])}}
    {T'}{\subset{x}{U'}{P'}}$}
  {}
}

\begin{figure}[h]
  \begin{center}
    \def\fCenter{\subtd}
    
    \vspace{\infvspace}
    \SubConvI\DP

    \vspace{\infvspace}
    \SubProdI\DP

    \vspace{\infvspace}
    \SubSigmaI\DP
    
    \vspace{\infvspace}
    \SubLeftI\DP
    
    \vspace{\infvspace}
    \SubRightI\DP
    
  \end{center}
  \label{subtyping-impl-rules}
  \caption{R�ecriture du sous-typage vers \Coq}
\end{figure}

%%% Local Variables: 
%%% mode: latex
%%% TeX-master: "subset-typing"
%%% End: 


\newcommand{\src}[1]{\texttt{#1}}
\newcommand{\srcm}[1]{\text{\texttt{#1}}}

\def\thetitle{Coercion par pr�dicats en \Coq}

\pagestyle{plain}
\fancyhead[RO,L]{\thetitle}
\fancyfoot[C]{\thepage}

\title{\thetitle}

\author{Matthieu Sozeau \\ sous la direction de Christine Paulin \\
  \'Equipe {\sc Demons}, {\sc LRI}}

\date{\today}

%%% Local Variables: 
%%% mode: latex
%%% TeX-master: t
%%% End: 


\newboolean{showlog}
\setboolean{showlog}{false}

\title{\thetitle}

\author{Matthieu Sozeau}

\date{\today}

\begin{document}

\maketitle

\begin{abstract}
  \Coq{} est un assistant de preuve d'une grande expressivit� pour le
  d�veloppement de th�ories math�matiques et informatiques, ce qui permet
  de traiter un large �ventail de probl�mes. Le langage de \Coq{},
  constitu� d'un noyau fonctionnel de type \ML{} enrichi par des types
  d�pendants, permet de sp�cifier, v�rifier puis
  extraire des programmes corrects par construction. En contrepartie, les
  programmes sont plus difficiles � �crire et maintenir que dans un pur
  langage de programmation de type \ML, puisqu'ils m�langent les parties
  logiques et calculatoires. Pour rem�dier �
  ce probl�me, on propose un nouveau langage de description de programes,
  s'int�grant parfaitement � l'environnement de d�veloppement existant,
  qui permet de donner dans un premier temps le code et la sp�cification
  du programme et d'engendrer ensuite ses conditions de correction.
\end{abstract}

\newpage
\tableofcontents
\newpage

\chapter{Introduction}
Nous nous pla\c cons dans le cadre du syst�me d'aide � la preuve \Coq{},
auquel nous souhaitons int\'egrer un langage de programmation plus souple
que le langage actuellement utilis\'e.

\section{Pr\'esentation de \Coq}

\Coq~est un assistant de preuve dont la premi�re version date de 1985,
et qui est aujourd'hui d\'evelopp\'e dans le projet \PCRI{} \LogiCal{} (INRIA, LIX,
LRI, CNRS). Originellement bas\'e sur le Calcul des Constructions (\CoC),
il a \'et\'e \'etendu au \CCIfull~(\CCI) et contient aujourd'hui de
nombreuses am\'eliorations telles qu'un syst�me sophistiqu\'e d'extraction
de programmes ou encore des proc\'edures de d\'ecision pour automatiser la
preuve.

Le d\'eveloppement de \Coq~est intimement li\'e � l'isomorphisme de {\sc
Curry-Howard} qui montre le lien entre logique intuitionniste et calcul. Cet
isomorphisme \'etablit qu'\'elaborer une preuve du calcul propositionnel
intuitionniste est \'equivalent � \'ecrire un terme du
\lc~simplement typ\'e (\lcst). Par exemple, montrer que $A "=>" A$ pour un
certain $A$ revient � \'ecrire la fonction identit\'e $\lambda x : A. x$ qui
a bien pour type $A "->" A$. Chaque logique constructive est donc
associ\'ee � un \lc{} particulier. Dans \Coq{}, on utilise cet
isomorphisme pour v\'erifier les preuves. Le noyau est simplement un
typeur pour \CCI{}. Si on peut typer un terme $t$ de type $T$, alors on est
assur\'e d'avoir trouv\'e une preuve constructive $t$ de la formule $T$.
Cette dualit\'e se refl�te aussi � l'utilisation de \Coq{} o\`u l'on a 
les deux visions: logique (d\'eveloppement math\'ematique, preuve) et 
calcul (d\'eveloppement informatique, programme).

\subsection{Preuve}
\Coq{} est utilis\'e le plus souvent pour \'elaborer des th\'eories
math\'ematiques prouv\'ees m\'ecaniquement. Dans cette optique, l'utilisateur 
mod\'elise un probl�me par des structures math\'ematiques et veut prouver
certaines propri\'et\'es sur ce mod�le (par exemple la preuve du th\'eor�me
des quatres couleurs r\'ecemment termin\'ee \cite{Gonthier4col} utilisait 
des r\'esultats de g\'eom\'etrie alg\'ebrique).

Pour prouver un but sous certaines hypoth�ses, on utilise des
tactiques qui simulent un raisonnement d\'eductif pour l'utilisateur.
Celles-ci permettent par exemple d'introduire une hypoth�se: pour le but
$A "=>" A$ on peut introduire l'hypoth�se $H : A$ pour obtenir le but
$A$ ; ou bien d'en appliquer une (ou tout autre r\'esultat d\'ej� \'etabli): 
en appliquant l'hypoth�se $H$ on prouve le but directement. 
Ces tactiques peuvent \^etre d'une complexit\'e arbitraire (r\'e\'ecritures,
proc\'edures de d\'ecision pour l'arithm\'etique, etc \ldots).

Les tactiques utilis\'ees pour cr\'eer des preuves ne sont en fait
qu'une interface au-dessus du noyau de \Coq{} qui se r\'eduit � un typeur
pour \CCI. A la fin d'une preuve, on a en effet construit un terme 
($\lambda x : A. x$ dans notre exemple) que l'on va soumettre au typeur
dont le but est de v\'erifier qu'il est bien de type $A "->" A$. Les
tactiques peuvent cependant �tre arbitrairement complexes (r�solution
d'�quations de l'arithm�tique, r�ecritures, etc...).

\subsection{Programmes}
D'un point de vue preuve de programmes, on a donc un environnement qui
permet de v\'erifier qu'un programme (un terme du calcul) v\'erifie une
certaine sp\'ecification (son type). Les types d\'ependants permettent de
sp\'ecifier fortement les termes. Par exemple, la fonction $\sdef{div} :
\nat "->" \nat "->" \nat * \nat$ de \ML{} est plus fortement sp\'ecifi\'ee en 
\Coq{} par $\sdef{div} : \nat "->" \subset{x}{\nat}{x \neq 0} "->" \nat
* \nat$.
Seulement, on ne peut pas \'ecrire simplement un programme \ML{} et
donner sa sp\'ecification forte. Comme on a enrichi les types, on doit 
aussi enrichir les termes avec des termes de preuve, inutiles au 
calcul mais n\'ecessaires pour garantir la
correction logique du programme et le fait que la machine puisse
v\'erifier m\'ecaniquement la correction (annotations,...). Par exemple,
 si l'on veut appeler $\sref{div}$ sur $1$ et $n$ (pour $n : \nat$), il
faut construire un terme $\sref{div}~1~(\sref{elt}~(\lambda x : \nat
"->" x \neq 0)~x~p)$ o\`u $p$ est une preuve de $n \neq 0$. 
En effet, en \Coq~on a:
\begin{definition}[Type sous-ensemble]
  \label{subset-type-def}
  $\subset{x}{T}{P}$ est le type des termes de type $T$ v\'erifiant la
  propri\'et\'e $P$.
\end{definition}

Un objet de type $\subset{x}{T}{P}$ peut �tre vu comme une paire
$(x,p)$ ou $x$ est un objet de type $T$ (le t�moin) et $p$ une preuve
de $P[t/x]$. Seulement, le typeur a besoin de plus d'information,
l'annotation $\lambda x : \nat "->" x \neq 0$ est n�cessaire pour avoir
un syst�me d'inf�rence d�cidable (on ne peut
pas inf\'erer la propri�t� $P$ � partir de sa preuve $p$ puisqu'il est de type
$P[t/x]$). On voit donc ici que l'on doit rajouter de nombreuses
informations d'ordre logique � nos termes.

A l'inverse, on peut extraire un programme de toute preuve en \'eliminant les
parties logiques et en ne conservant que la partie calculatoire d'un terme.

\section{Motivation}

\Coq{} permet de d\'evelopper des programmes complexes,
de leur donner des sp\'ecifications fortes et de les v\'erifier
automatiquement. On peut m\^eme extraire de ces d\'eveloppements des
programmes corrects par construction. Il y a cependant certaines
difficult\'es � d\'evelopper en \Coq{} que nous allons \'etudier maintenant.

\subsection{Un langage trop expressif?} 

Le langage de \CCI{} permet de bien sp\'ecifier des fonctions non
triviales, par exemple, si l'on d\'efinit une fonction de division
euclidienne en \ML{} elle aurait le type: $  \valml~\sdef{div} : \nat "->" \nat "->"
\nat * \nat$. En \Coq, on peut d\'efinir:
\[\Definition~\sdef{div} : \forall a : \nat, \forall b : \nat,
b \neq 0 "->" \{~q : \nat~\&~\{~r : \nat `| r < b `^ a = b * q + r~\} \}\]

Les types d\'ependants permettent de bien relier les entr\'ees aux sorties
et donc de sp\'ecifier les programmes aussi fortement que l'on d\'esire, 
mais aussi de fa\c con concise. En revanche, le terme de preuve 
correspondant � \sref{div} est nettement plus long (de l'ordre de 60 lignes), 
et ne peut simplement pas \^etre \'ecrit d'une traite 
sans une expertise approfondie. Pour rem\'edier � ce
probl�me, on utilise des tactiques qui permettent d'\'ecrire la preuve/programme
incr\'ementalement (\note{figure ici?}voir figure \ref{fig:euclid-script}
page \pageref{fig:euclid-script}). L'inconv\'enient de cette
m\'ethode est que l'on n'obtient pas toujours le programme d\'esir\'e
au d\'epart, puisque les tactiques cachent profond\'ement leur effet sur le
terme de preuve. Certaines techniques de r\'e\'ecriture peuvent aussi
g\'en\'erer des termes de complexit\'e algorithmique bien moins optimale que
ce que l'utilisateur \'ecrirait. Cependant ce mode de fonctionnement est
utile et utilis\'e par la majorit\'e des utilisateurs de \Coq{} avec succ�s
(certification d'un compilateur \C, th\'eor�me des quatres couleurs
\cite{Gonthier4col}, \ldots).

\subsection{M\'elange logique et calcul}
Une difficult\'e essentielle lorsque l'on veut permettre � des
utilisateurs non experts de d\'evelopper dans \Coq{} est le ``m\'elange
des genres'' permanent entre logique et calcul. Pour appliquer une
fonction division qui attend un d\'enominateur diff\'erent de $0$ par
exemple, il faut passer � la fois l'argument lui m\^eme, mais aussi une
preuve de sa non-nullit\'e. Lorsque l'on a l'habitude de programmer, \c ca
n'est pas la chose la plus naturelle et l'on aimerait pouvoir d\'ecoupler
les parties codage et preuve pour simplement diviser le probl�me. Les
parties logiques pourront souvent \^etre r\'esolues automatiquement par des
tactiques.

\subsection{Objectif}
A long terme, on souhaite permettre � un utilisateur de programmer dans
un langage proche de \ML{} et de prouver ses programmes dans un deuxi�me
temps � l'aide de \Coq{} et ses tactiques. Une fois les preuves
termin\'ees, on peut extraire un programme correct par construction et
essentiellement \'equivalent � celui de d\'epart ou le r\'eutiliser facilement
dans l'environnement \Coq{}.

\section{Travaux Connexes}

La preuve de programmes fonctionnels est un domaine de recherche
actif. L'id\'ee d'\'etendre les langages \ML{} avec des types d\'ependants a
\'et\'e d\'evelopp\'ee dans \DML{} \cite{XiPfenning1999DTP}, \Cayenne{}
\cite{Augustsson99} et \Omegapdx{} \cite{Omega}. Il
s'agit dans ces travaux de faire un langage dont l'inf\'erence est d\'ecidable, donc
de restreindre les types d\'ependants � des domaines o\`u l'on peut faire de
la preuve automatique (\DML{}) ou bien d'accro�tre la puissance du langage
pour rendre l'utilisation des types d\'ependants plus ais\'ee (\Cayenne{}
a la r\'ecursivit\'e g\'en\'erale par exemple) mais en perdant l'id�e de
correction (et en perdant m�me la d�cidabilit� du typage pour
\Cayenne{}). 

Nous prenons le
contre-pied de ces travaux en acceptant de g\'en\'erer des obligations de
preuve et en essayant de trouver un langage le plus proche de \ML{}
possible tout en retenant la puissance de \Coq{} et des types
d\'ependants. Nous pr\'esentons maintenant des travaux directement li\'es �
notre contribution.


\subsection{La tactique \Program}
Il existe un travail r\'ealis\'e dans \Coq{} couvrant une partie de nos objectifs.
D\'evelopp\'ee par Catherine Parent \cite{conf/mpc/Parent95}, 
la tactique \Program{} permettait de synth\'etiser des preuves � partir de
programmes. L'id\'ee \'etait de trouver un langage de programmation
suffisamment restrictif pour r\'ealiser une inversion de l'extraction, 
c'est-�-dire, � partir d'un terme essentiellement calculatoire
(des annotations \'etaient n\'ecessaires), retrouver un terme de preuve
r\'ealisant la sp\'ecification donn\'ee. 
A partir de l�, on \'etait assur\'e que le programme extrait serait
identique � celui que l'on \'ecrivait pour sa partie informative. Cette
m\'ethode g\'en\'erale avait l'inconv\'enient d'\^etre peu intuitive et de ne pas
s'int\'egrer � l'environnement \Coq. En particulier, appeler une fonction
d�finie avec \Program{} est aussi difficile qu'avec n'importe quelle
d�finition \Coq{}. Il n'existe pas de m�canisme permettant de faire la
distinction de phase codage/preuve, qui permettrait de faire de simples
appels et de v�rifier ensuite que les arguments sont valides, ce qui est
beaucoup plus naturel lorsque l'on programme.
Li\'ee � l'extraction interne qui
a disparu dans les derni�res versions de \Coq{} (remplac\'ee par la
contribution de Pierre Letouzey \cite{LetouzeyPhD}), elle n'est plus 
maintenue aujourd'hui.

\subsection{Types sous-ensemble}
Plut\^ot que de continuer dans la m\^eme direction, nous avons 
cherch\'e � assouplir le syst�me. L'assistant de preuve \PVS{}
\cite{PVS-Semantics:TR} aux capacit\'es similaires � \Coq{}, int�gre un
m\'ecanisme d\'enomm\'e \ps{} que nous allons pr\'esenter maintenant.

Les types sous-ensembles (d�finition \ref{subset-type-def}) sont d'une
grande utilit\'e pour la sp\'ecification
de programmes, par exemple pour les pr\'e-conditions:
$\Definition~\sref{div} : \nat "->" \subset{x}{nat}{x \neq 0} "->" \nat
* \nat$.

L'id\'ee du \ps{} impl\'ement\'e dans \PVS{}
\cite{Shankar&Owre:WADT99,Rushby98:TSE} est de consid\'erer tout objet de
type $T$ comme un objet de type $\subset{x}{T}{P}$ pour $P$ vraie et
vice-versa. Comme tout objet $t$ de type $T$ ne v\'erifie pas forc\'ement la
propri\'et\'e $P$, on g\'en�re des ``\emph{Type-checking conditions}'' (\TCC), c'est
� dire que l'on demande � l'utilisateur de prouver $P[t/x]$ pour assurer
que le programme est correct.

\subsection{Coercions}
\PVS{} n'a pas la m\^eme architecture que \Coq{}, en particulier il n'y a
pas de termes de preuve et de noyau pour v\'erifier ces termes. Il faut
donc faire confiance � la quasi-totalit\'e du code pour croire en la
correction des programmes v\'erifi\'es. Le crit�re de {\sc De Bruijn},
qui dit en substance qu'un petit noyau est plus s\^ur n'est pas
respect\'e, alors que celui de \Coq~a m\^eme \'et\'e
formellement v\'erifi\'e \cite{Barras96a}.
 
Dans notre cas, il faut g\'en\'erer des termes de preuve et donc le code 
correspondant � ce ``sous-typage''. Une litt\'erature importante
\cite{conf/popl/Chen03,conf/csl/Luo96} existe
autour des syst�mes � coercions explicites dont nous nous sommes
inspir\'es pour r\'ealiser la g\'en\'eration des termes. Dans un syst�me �
coercions explicites, on peut faire des abus de notations tels que
utiliser un objet de type $T$ � la place d'un de type $U$, mais
on applique une coercion qui am�ne l'objet vers le type $U$ avant 
de retyper dans un syst�me sans coercions. G\'en\'eralement les coercions
sont tr�s similaires � des identit\'es, c'est-�-dire qu'elles sont calculatoirement
insignifiantes mais leur utilisation facilite le d\'eveloppement. Dans
\Coq{} par exemple le syst�me de coercions \cite{saibi97inheritance} a
permis de d\'evelopper des th\'eories alg\'ebriques r\'eutilisables sur
plusieurs structures instantan\'ement (un th\'eor�me sur les corps pouvant s'appliquer aux
anneaux).

Les extensions du Calcul des Constructions avec des notions de
sous-typage comme $\lambda C_\leq$ de Chen ne sont cependant pas
dans la m\^eme cat\'egorie que notre travail. En particulier, nous ne
nous int\'eressons pas aux propri\'et\'es de normalisation, pr\'eservation du
typage ou encore au fait d'avoir des sous-types minimaux dans notre
syst�me. On peut le voir plut\^ot comme un syst�me syntaxique intelligent
au-dessus du Calcul des Constructions.

%%% Local Variables: 
%%% mode: latex
%%% TeX-master: "subset-typing"
%%% LaTeX-command: "TEXINPUTS=\"style:$TEXINPUTS\" latex"
%%% End: 

\chapter{Le calcul de coercion par pr�dicats}
Nous avons d�velopp� un langage supportant le \ps{} utilisable dans
\Coq. L'utilisateur peut d�finir des programmes dans un langage plus
souple puis prouver certains buts pour obtenir finalement un terme de
\CCI{} complet v�rifiable par le noyau. On peut finalement utiliser les types d�pendants comme des types
simples et s'en occuper dans un deuxi�me temps (pour la preuve).
L'architecture est la suivante:
on type le programme dans notre langage, puis l'on r��crit le terme typ�
dans \CCI{} en laissant des ``trous'' dans les termes et enfin \Coq{} se
charge de g�n�rer les obligations correspondant � ces trous.


\setboolean{displayLabels}{true}
\def\setproptype{\{ \Set, \Prop, \Type(i) \}}
\def\nf{\downarrow~}

\section{Notations}
On utilise dans la suite les notations et abbr�viations suivantes:
\begin{list}{}{}
\item $`G \typed x : T : s `= `G \typed x : T `^ `G \typed T : s$.
\item $\_ \nf `= \text{forme normale}$
\item $`G \typei t : S \subi T `= `G \typei t : S `^{} S \subi T$
\end{list}


\section{Langage}
Le langage que nous voulons est tr�s proche de \ML, plus les annotations
n�cessaires pour avoir un typage pr�cis et d�cidable.

\subsection{Syntaxe}
La syntaxe (figure \ref{fig:syntax}) est directement inspir�e des langages fonctionnels.
On part du \lc{} (variables, abstraction et application) puis l'on
ajoute des constantes (pour les entiers, bool�ens, etc...) ainsi que les
couples. La syntaxe $(x := `a, t : `t)$ permet de
cr�er des paires d�pendantes, de type $\Sigma x : `t. `t$. 

Du c�t� des types, on a tout d'abord les types simples (constantes,
fl�che, produit cart�sien) qui sont des cas particuliers du produit ($\Pi$) et
de la somme ($\Sigma$) d�pendants. Les variables introduites par ces
types peuvent �tre utilis�es lors des applications de types. On
peut de plus abstraire sur les types avec le $\lambda$ (polymorphisme). 
Enfin on peut appliquer un type � un terme ($`t~`a$). Dans la suite, les r�gles sont donn�es avec l'id�e
qu'on fait les r�ductions de t�te n�cessaires pour obtenir un type
d'une certaine forme (produit, somme, sous-ensemble, etc \ldots).

\TODO{Finitude!, pbs de contextes/nommage (ajouter des conditions)}

%\vspace{-0.5cm}
\begin{figure}[ht]
  \begin{center}
    \subfigure[Termes]{\termgrammar}\quad
    \subfigure[Types]{\typegrammar}
  \end{center}
  \label{fig:syntax}
  \caption{Syntaxe}
\end{figure}
% \vspace{-1cm}

\section{Le langage \lng{}}
\label{section:russel}
Le langage que nous voulons est tr�s proche de \ML{}, plus les annotations
n�cessaires pour avoir un typage pr�cis et d�cidable. On �tudie ici une
restriction de \ML{}, purement fonctionnelle et sans filtrage, qu'on
�tendra dans la suite de notre travail. On n'a donc pas de types
inductifs mais on consid�re les types $\Sigma$, g�n�ralisation des
tuples de \ML{} form�s par l'op�rateur $*$.

\subsection{Syntaxe}
La syntaxe (figure \ref{fig:syntax}) est directement inspir�e des langages fonctionnels.
On part du \lc{} (variables, abstraction et application) puis l'on
ajoute des constantes (pour les entiers, bool�ens, etc...) ainsi que les
couples. La syntaxe $(x := `a, t : `t)$ permet de
cr�er des paires d�pendantes, de type $\Sigma x : `t. `t$. On peut aussi
appliquer un terme � un type pour instancier une fonction polymorphe par exemple.

Du c�t� des types, on a tout d'abord les types simples (constantes,
fl�che, produit cart�sien) qui sont des cas particuliers du produit ($\Pi$) et
de la somme ($\Sigma$) d�pendants. Les variables introduites par ces
types peuvent �tre utilis�es lors des applications de types. On
peut de plus abstraire sur les types avec le $\lambda$ (polymorphisme)
et les sortes.
Enfin on peut appliquer un type � un terme ($`t~`a$). 

%\vspace{-0.5cm}
\begin{figure}[ht]
  \begin{center}
    \subfigure[Termes]{\termgrammar}\quad
    \subfigure[Types]{\typegrammar}
  \end{center}
  \caption{Syntaxe}
  \label{fig:syntax}
\end{figure}
% \vspace{-1cm}

\subsection{S�mantique}
\typenvd

\typedFig
\subtdFig

La s�mantique du langage nous est donn�e par un syst�me de typage
(figure \ref{fig:typing-decl-rules} page \pageref{fig:typing-decl-rules}). Le
jugement de typage est d�fini inductivement par un ensemble de r�gles
d'inf�rence. 
Dans notre cas ce sont les r�gles du \CCfull{} (\CC{})
�tendu avec les $\Sigma$-types auxquelles on a ajout� une r�gle de
coercion (\irule{Coerce}, figure \ref{fig:typing-decl-rules}) que l'on trouve classiquement dans les syst�mes avec
sous-typage avec le nom de subsumption. 
Le jugement $\Gamma \typed t : T$ se lit: dans l'environnement $\Gamma$,
$t$ est de type $T$.

% \begin{remark}
%   En pratique, les types du Calcul des Constructions ne sont pas
%   toujours en forme normale et il peut donc �tre n�cessaire de les
%   r�duire (en t�te seulement) pour v�rifier des jugements du genre: 
%   $`G \seq t : \Pi x : T.V$.
% \end{remark}

La relation $\mathcal{R}$ d�finissant les produits formables
dans le syst�me est d�finie par les r�gles suivantes:
\[\begin{array}{cccll}
  s_1 & s_2 & s_3 & \text{Habitants} & \text{Exemple} \\
  \hline
  \Prop & \Prop & \Prop & \text{Implication logique} & x <= 0 "->" x = 0  \\
  \Set & \Set & \Set & \text{Fonctions} & \Pi x : \nat. \nat \\
  \Type & \Set & \Type & \text{Fonctions polymorphes} & \Pi A : \Set, A
  "->" A \\
  \Set & \Type & \Type & \text{Types d�pendants} & \sref{vector} : \nat "->" \Set : \Set "->" \Type \\
  \Set & \Prop & \Prop & \text{Termes dans les propositions} & 
  \Pi n : \nat. \Pi l : \text{list}~n. \text{length}~l = n \\
  \Type & \Prop & \Prop & \text{Impr�dicativit� de } \Prop & \Pi x : Prop. x `V `! x \\
%  \Type(i) & \Type(j) & \Type(max~i~j)  & \text{Connecteurs logiques,
%    \ldots} & \Pi A : \Prop. \Pi B : \Prop. A `^ B "->" B `^ A
\end{array}\]
On a un syst�me proche du \CCfull{} avec types $\Sigma$, mais
avec \Set{} pr�dicatif (comme dans \Coq{}).
On n'a pas $(\Prop,\Set,\Set)$ dans notre relation $\mathcal{R}$ pour
une bonne raison. Cela permet de cr�er des fonctions d�pendant de
propositions, par exemple $\Pi n : \nat, n > 0 "->" \Pi l :
\text{list}~A~n "->" A$. Or on veut � tout prix �viter d'introduire des
termes de preuve dans notre langage, et l'on voit que
cette fonction pourrait naturellement s'�crire $\Pi n : \mysubset{n}{\nat}{n > 0} "->" \Pi l :
\text{list}~A~n "->" A$. Encore une fois le type sous-ensemble nous
permet d'�viter d'avoir � passer des termes de preuve directement. 

Les sommes formables dans le syst�me sont r�duites au couples d'objets de
types de m�me sorte $s `: \{ \Prop, \Set \}$.
Dans le premier cas les habitants sont les couples de propositions
(codage du $`^$), dans le second ce sont les couples d'objets, soit les
paires de \ML.
Intuitivement, c'est le type sous-ensemble $\mysubset{x}{T}{P}$ qui permet
de faire des couples $\Set,\Prop$ habitant $\Set$. Les types $\Sigma x : U.V$ o� $U
: \Prop$ et $V : \Set$ n'ont pas d'int�r�t dans notre cas puisqu'ils
repr�sentent des objets de type $U `^ V$ mais on ne peut
pas utiliser $U$ dans notre syst�me. On pr�f�re coder ces objets par des
objets de type $\mysubset{x}{V}{U}$ (on n'est pas int�ress� par la preuve
de $U$ pour programmer).


La r�gle \irule{Coerce} formalise l'id�e que 
l'on peut utiliser un terme de type $T$ � la place d'un terme de type
$U$ si $T$ et $U$ sont dans une certaine relation. C'est l�
qu'interviendront les types sous-ensemble. \CC{} contient une r�gle de typage
similaire � \irule{Coerce}, la r�gle de conversion (\irule{Conv}), qui
dit essentiellement que deux types
$`b$-convertibles (on rappelle que l'on peut calculer dans les types
puisqu'on a l'abstraction, l'application, etc...) sont �quivalents.
On peut directement int�grer cette relation de $`b$-convertibilit� � notre
syst�me de coercion comme montr� figure \ref{fig:subtyping-decl-rules}
(\irule{SubConv}), � condition d'avoir l'inclusion
$`=_\beta~\subseteq~\subd + \text{\irule{SubConv}}$.
En fait notre notion de r�duction est un peu plus large que $\beta$
puisqu'on peut r�duire les $\sref{let}$:
$\letml~(x,y) = (u, v)~\inml~t$ se r�duit en $t[u/x][v/y]$. En
pratique cette constructions est du sucre syntaxique pratique au niveau
du typage (on peut inf�rer le type de $t$), mais elle est inessentielle au
niveau du calcul.
On peut ais�ment rajouter un $\letml~x=t~\inml~v$ � notre langage de
fa�on similaire: c'est �quivalent � $(\lambda x : T.v)~t$, mais $T$ peut
est inf�r� plut�t que donn� par l'utilisateur.

On consid�re les constantes comme des variables pr�d�finies 
dans nos contextes, par exemple on a la constante $\sref{list} : \Pi x :
nat. \Set$. 
L'ajout d'une constante � un contexte ne doit pas alt�rer sa
bonne formation comme pour le cas des variables, donc son type doit �tre
bien form� (en g�n�ral, toute d�finition de \Coq~donne lieu � une
constante dans notre syst�me si elle est bien typ�e).

\subsubsection{Jugement de coercion}
Notre syst�me de coercion par pr�dicats permet � l'utilisateur
d'utiliser une valeur de type $U$ l� o� l'on attend une valeur de type
$\mysubset{x}{V}{P}$ (\irule{SubProof}) si $U$ est lui-m�me coercible en $V$.
A l'inverse, on permet aussi d'utiliser une valeur de type
$\mysubset{x}{U}{P}$ (\irule{SubSub}) � la place d'une valeur de type
$V$ si $U$ est coercible vers $V$. Notre jugement de coercion est donc
sym�trique et laisse beaucoup de libert� � l'utilisateur au moment du
codage. Par exemple on peut d�river $u : \nat \type u : \mysubset{x}{\nat}{`_}$
Seulement, lors de la traduction de la d�rivation de coercion $\nat
\subd \mysubset{x}{\nat}{`_}$ (n�cessaire pour traduire l'abus de notation
$x : \mysubset{x}{\nat}{`_}$), l'utilisateur aura � r�soudre une obligation
de preuve de $`_$. On repose donc toujours sur la coh�rence du Calcul
des Constructions. 
Les r�gles \irule{SubProd} et \irule{SubSigma} permettent de faire des
coercions dans les types composites. Classiquement, la r�gle pour le 
produit fonctionnel est contravariante (une fonction sous-type d'une
autre accepte plus d'entr�es mais donne une sortie plus fine, voir
 \cite{journals/toplas/Castagna95}) et la r�gle pour le 
produit cart�sien covariante (une paire est coercible en une autre si 
leurs composantes sont coercibles deux-�-deux). Le sens des coercions
n'a pas d'importance dans le syst�me d�claratif puisqu'il est sym�trique
mais il est essentiel lors de la cr�ation des coercions que nous
d�crirons plus tard.

La r�gle \irule{SubTrans} assure que l'on a un syst�me compositionnel. Il y a ici une
analogie avec l'�limination des coupures dans les syst�mes logiques, o�
l'on montre que toute d�rivation utilisant la r�gle de \emph{modus ponens} ($A "=>" B$ et $B "=>" C$ implique
$A "=>" C$) peut se r��crire en une d�rivation ne l'utilisant
jamais. Dans les syst�mes � sous-typage, on montre de fa�on �quivalente
que l'on peut �liminer la r�gle de transitivit� ; premi�re �tape vers un
syst�me d�cidable.


Notre jugement de coercion identifie les types $U$ et $\mysubset{x}{U}{P}$
mais notre syst�me de typage ne permet pas d'�liminer (prendre la partie
preuve) ou d'introduire (cr�er un couple t�moin,preuve) des objets de
type sous-ensemble. Cela nous assure une certaine coh�rence, puisque
m�me si l'on ne v�rifie pas qu'un objet de type $U$ a bien la propri�t�
$P$, on ne peut pas raisonner sur le fait que $U$ a la propri�t� dans le
langage.


On ne fera pas la m�tath�orie du syst�me d�claratif ici, puisque
c'est une extension conservative du Calcul des Constructions et l'on
�tudiera en d�tail le syst�me algorithmique. Notre preuve de conservativit�
est simple: si l'on oublie les utilisations des types sous-ensemble de
notre syst�me de typage (\irule{Subset}) et de coercion
(\irule{SubProof} et \irule{SubSub}), alors le jugement de coercion est 
juste la $\beta$-convertibilit� et donc \irule{Coerce} et \irule{Conv} 
sont �quivalentes. Comme les autres r�gles de notre syst�me d�claratif
proviennent directement de \CC{}, on arrive � un syst�me strictement
�gal au syst�me du calcul des constructions. On peut donc s'appuyer sur
les r�sultats connus pour \CC{} pour une partie de notre syst�me.

Pour une �tude compl�te du \CCfull{}, se r�f�rer �
\cite{Barras99,Luo90}.
On va plut�t s'int�resser � la construction d'un algorithme de typage
correspondant � notre syst�me d�claratif.

% \section*{Propri�t�s �lementaires}

% \begin{fact}[Inversion du typage]
%   \label{inversion-typing-d}
%   On a les propri�t�s suivantes sur le jugement de typage:
%   \begin{enumerate}
%   \item Si $`G \type \lambda x : T.v : \Pi x : T.U : s$ alors $`G, x : T \type
%     v : U : s$.
%   \item Si $`G \type (t, v) : \Sigma x : T.U$ alors $`G, x : T \type U
%     : s1$ et $`G \type v : U[t/x]$.
%   \item Si $`G \type t : \mysubset{x}{U}{P}$ alors $`G \type t : U$ et
%     $`G, x : U \type P : \Prop$.
%   \end{enumerate}
% \end{fact}

% \begin{fact}[Convertibilit�]
%   \label{type-convertibility}
%   Si $`G \type T : s$ et $s `=_\beta s'$ alors $`G \type T : s'$.
% \end{fact}

% \begin{lemma}[Inversion pour le produit]
%   \label{inversion-prod-d}
%   Si $`G \type \Pi x : T.U : s$ alors $s `: \setproptype$, $`G \type T : t$ et 
%   $`G, x : T \type U : s$.
% \end{lemma}
% \begin{proof}
%   Par induction sur la taille de la d�rivation.
%   Les seules r�gles ayant pour conclusion possible un jugement de la
%   forme $\Pi x : T.U : s$ sont \irule{Prod} et \irule{Subsum}
%   (\irule{App} se termine par une application). Pour \irule{Prod} la
%   propri�t� est directe. Supposons que la derni�re r�gle appliqu�e fut
%   \irule{Subsum}. Alors il existe $s'$ tel que $`G \type \Pi x : T.U :
%   s'$ et par induction, $s' `: \setproptype$. Une
%   inspection des r�gles de sous-typage r�v�le que seules les
%   r�gles \irule{SubConv} et \irule{SubTrans} ont pu s'appliquer dans
%   la d�rivation $`G \type s' \sub s$. On en d�duit que $s `=_\beta s'$,
%   et il s'ensuit que $`G \type T : s'$ (par application de
%   \irule{Conv}). % \ref{type-convertibility}
%   On peut appliquer \irule{Subsum} � la fin de la d�rivation 
%   $`G, x : T \type U : s'$ pour obtenir le r�sultat $`G, x : T \type U :
%   s$.
  
  % La d�rivation est donc de la forme:
%   \begin{prooftree}
%     \AXC{$t `=_\beta s'$}
%     \UIC{$t \sub s'$}
%     \AXC{$\ldots$}
%     \AXC{$s `=_\beta t$}
%     \UIC{$s \sub t$}
%     \TIC{$s \sub s'$}
%   \end{prooftree}


%   Comme les classes de
%   $\beta$-�quivalence des �l�ments de $\setproptype$ sont r�duites � un
%   �l�ment,

% \end{proof}

% \TODO{Pas utilis�!}

% On montre tout d'abord qu'il existe des types principaux dans notre syst�me.

% \begin{lemma}[Principalit� du typage]
%   Pour tout $`G, t$, il existe $T$ tel que si $`G \typed t : T$ alors 
%   pour tout $S$ tel que $`G \typed t : S$ alors $T \subd S$.
% \end{lemma}

% \begin{proof}
%   Ce r�sultat d�coule du fait qu'il existe des sous-types principaux et
%   des sortes principales dans notre syst�me.
% \end{proof}

% \begin{lemma}[Unicit� du sortage]
%   Si $`G \typed T : s_1$ et $`G \typed T : s_2$ alors $s_1 = s_2$.
% \end{lemma}

% \begin{proof}
%   \begin{induction}{typing-decl}
%     \casetwo{PropSet}{Type}
%     Aucune autre r�gle ne peut typer $T$, donc $s_1 = s_2$
    
%     \case{Var} 
%     Les r�gles de conversion et de subsumption ne permettent pas de
%     typer $s1$...
    
%     \case{Prod}
%     \case{Abs}
%     \case{App}
%     \case{LetIn}
%     \case{Sigma}
%     \case{Sum}
%     \case{LetSum}
%     \case{Subset}
%     \case{Subsum}
%   \end{induction}
% \end{proof}


% \begin{lemma}[Sous-typage bien sort�]
%   \label{subtyping-sorts-d}
%   Si $S \subd T$, $`G \typed S : s_1$ et $`G \typed T : s_2$ alors $s_1 =
%   s_2$.
% \end{lemma}

% \begin{proof}
%   \begin{induction}[subtyping-decl]
%     \case{SubEq} Par unicit� du typage.

%     \case{SubTrans} Trivial.
    
%     \casetwo{SubProd}{SubSigma} Par induction les composantes
%     correspondantes sont dans la m�me sorte, donc les compos�s aussi.
    
%     \casetwo{SubProof}{SubSub} Comme le constructeur de types subset
%     est de type $\Set "->" \Prop "->" \Set$, on conserve bien les m�mes
%     sortes de part et d'autre.
%   \end{induction}
% \end{proof}

% \begin{lemma}[Bonne formation des contextes]
%   \label{wf-contexts-d}
%   Si $`G \type t : T$ alors $\typewf `G$.
% \end{lemma}
% \begin{proof}
%   \inductionon{typing-decl}
% \end{proof}

% \begin{fact}[Inversion du jugement de bonne formation]
%   \label{inversion-wf-d}
%   Si $\typewf `G, x : U$ alors $`G \type U : s$ et $s `: \{ \Set, \Prop, \Type(i) \}$.
% \end{fact}

% L'affaiblissement est n�cessaire pour montre le lemme de
% renforcement. Il �tablit que tout jugement peut �tre d�riv� dans un
% contexte �tendu par de nouvelles d�clarations.

% \begin{lemma}[Affaiblissement]
%   \label{weakening-d}
%   Si $`G, `D \type t : T$ alors pour tout $x : S `; `G, `D$ tel que
%   $\wf `G, x : S, `D$, $`G, x : S, `D \type t : T$
% \end{lemma}

% \begin{proof}
%   \begin{induction}[typing-decl]
%     \casetwo{PropSet}{Type} Trivial.

%     \case{Var}
%     On a $x : S `; `G, `D$, donc $`G, x : S, `D \type y : T$ est toujours d�rivable.
    
%     \case{Prod}
%     Par induction $`G, x : S, `D \type T : s1$ et $`G, x : S, `D,
%     y : T \seq U : s2$. On applique \irule{Prod} pour obtenir 
%     $`G, x : S, `D \type \Pi x : T.U : s2$. De m�me pour le reste des r�gles.

% %     \case{Abs}
% %     \case{App}
% %     \case{LetIn}
% %     \case{Sigma}
% %     \case{Sum}
% %     \case{LetSum}
% %     \case{Subset}
% %     \case{Subsum}
    

%   \end{induction}
% \end{proof}  

% Le renforcement montre que notre notion de sous-typage est correcte
% vis-�-vis du typage. On peut d�river les m�mes jugements dans des
% contextes o� les variables ont des types plus pr�cis.

% \begin{lemma}[Renforcement]
%   \label{narrowing-d}
%   \[ `G \seq S, S' : s, S' \sub S "=>" 
%   \left\{ \begin{array}{lcl}
%       \typewf `G, x : S, `D & "=>" & \typewf `G, x : S', `D \\
%       & `^{} & \\
%       `G, x : S, `D \seq t : T & "=>" & `G, x : S', `D \seq t : T
%     \end{array}
%   \right. \]
% \end{lemma}

% \begin{proof}
%   Par induction sur la taille de la d�rivation de typage ou de bonne formation.
    
%   \begin{induction}
%     \case{WfEmpty} Trivial.
    
%     \case{WfVar} 
%     La conclusion est $\typewf `G, x : S, `D$
    
%     \begin{induction}[text=Par induction sur la taille de $`D$]
%     \item[\protect{$`D = []$}]
%       La racine de la d�rivation est de la forme:
%       \begin{prooftree}
%         \UAX{WfVar}
%         {$`G \type S : s$}
%         {$\wf `G, x : S$}
%         {$s `: \{ \Set, \Prop, \Type(i) \}$}
%       \end{prooftree}
%       On a $`G \type S' : s$, donc par \irule{WfVar}, $\typewf `G, x : S'$.  

%     \item[\protect{$`D `= `D', y : U$}]
%       La racine de la d�rivation est de la forme:
%       \begin{prooftree}
%         \UAX{WfVar}
%         {$`G, x : S, `D' \type U : t$}
%         {$\wf `G, x : S, `D', y : U$}
%         {$s `: \{ \Set, \Prop, \Type(i) \}$}
%       \end{prooftree}
%       Par induction sur la d�rivation de typage $`G, x : S', `D' \seq U : t$,
%       on a donc bien $\typewf `G, x : S', `D', y : U$ par \irule{WfVar}.
%     \end{induction}
    
%     \casetwo{PropSet}{Type} 
%     Par induction, $\typewf `G, x : S', `D$, on applique simplement la r�gle.
    
%     \case{Var}
%     Par induction, $\typewf `G, x : S', `D$. La seule diff�rence avec le
%     contexte pr�cedent est le type associ� � $x$, donc si $t \not= x$, on
%     peut simplement r�appliquer \irule{Var}. Si $t `= x$ on construit la
%     d�rivation:

%     \begin{prooftree}
%       \BAX{Var}
%       {$\wf `G, x : S', `D$}
%       {$x : S' `: `G$}
%       {$`G, x : S', `D \seq x : S'$}
%       {}
%       \AXC{$`G, x : S', `D \type S : s$}
%       \AXC{$S' \sub S$} % `G \subt 
%       \TIC{$`G, x : S', `D \type x : S$}
%     \end{prooftree}
    
%     Par l'affaiblissement (lemme \ref{weakening-d}) et $`G \type S : s$,
%     on obtient la pr�misse $`G, x : S', `D \type S : s$.
    
%     \case{Prod} 
%     Par induction, $`G, x : S', `D \type T : s1$ et $`G, x : S', `D
%     y : T \seq U : s2$. On applique \irule{Prod} pour obtenir 
%     $`G, x : S' \type \Pi x : T.U : s2$. De m�me pour le reste des r�gles.
%   \end{induction}
% \end{proof}

% Maintenant que nous avons montr� que ce syst�me a bien les
% propri�t�s que l'ont veut pour la coercion par pr�dicats, on va le
% raffiner pour obtenir un algorithme de typage.


%$list nat \sub list {n:nat|n \neq 0}$ ?
%$list : Set -> Set$


%%% Local Variables: 
%%% mode: latex
%%% TeX-master: "subset-typing"
%%% LaTeX-command: "TEXINPUTS=\"style:$TEXINPUTS\" latex"
%%% End: 

\section{�laboration du syst�me algorithmique et propri�t�s}
\typenva

Pour pouvoir implanter le typeur, il nous faut un syst�me dirig� par la
syntaxe. Ce n'est pas le cas du syst�me d�claratif, aussi bien pour le
typage que pour la coercion. On va donc raffiner ces syst�mes dans
l'optique d'en extraire un algorithme. On note $\typea$ le jugement de
typage algorithmique d�fini figure \ref{fig:typing-algo-rules} page
\pageref{fig:typing-algo-rules} et
$\suba$ le jugement de coercion algorithmique pr�sent� figure
\ref{fig:subtyping-algo-rules}. Ces deux syst�mes sont quelque peu
�loign�s des originaux, n�anmoins nous allons montrer qu'ils sont
corrects et complets vis-�-vis de leurs g�niteurs. La correction (si l'on
a une d�rivation d'un jugement dans le syst�me algorithmique, alors
c'est un jugement valide du syst�me d�claratif) est le
sens le plus facile � montrer, nous allons donc commencer par l�. On
d�crira ensuite la m�thode de construction des syst�mes algorithmiques
pour aboutir d'une part au th�or�me de compl�tude qui montre qu'on peut d�river les
m�mes jugements (� coercion pr�s) dans le syst�me algorithmique que dans
le syt�me d�claratif et d'autre part � la d�cidabilit� du jugement de
typage algorithmique.

Il nous a fallu changer quelque peu les r�gles pour obtenir le syst�me
algorithmique. En particulier, on a utilis� la fonction $\mu_0$ de \PVS{}
\cite{PVS-Semantics:TR} renomm�e $\mualgo$ (figure \ref{fig:mualgo-definition})
ici pour op�rer des
\emph{d�compr�hensions}. Cette fonction efface les constructeurs de type
sous-ensemble en t�te d'un type, par exemple: $\mualgo(\mysubset{f}{\nat
"->" \nat}{f~0 = 0}) = \nat "->" \nat$. On verra son utilit� dans la suite.

\subsection{Notations}
On note $\hnf{x}$ la mise en forme normale de t�te de $x$ selon la
r�duction d�finie pr�c�demment. On note $\hat{=}$ l'�galit�
d�finitionelle.


\setboolean{displayLabels}{true}
\typenva
\typeaFig
\typemuaFig
\subtaFig

\subsection{Correction}
On montre tout d'abord la correction de la coercion algorithmique qui
sera n�cessaire pour la correction du typage:

\begin{theorem}[Correction de la coercion]
  \label{correct-coercion}
  Si $U \suba V$ alors $U \subd V$.
\end{theorem}

\begin{proof}
  Les r�gles du syst�me algorithmique sont un sous-ensemble des r�gles
  du syst�me d�cla\-ratif, except� pour la r�gle \irule{SubHnf}.
  On utilise \irule{SubConv} et \irule{SubTrans} pour montrer son
  admissibilit� dans dans le syst�me d�claratif.
  \begin{prooftree}
    \AXC{$U \eqbr \hnf{U}$}  
    \UIC{$U~\subd \hnf{U}$}
    \AXC{$\hnf{U}~\subd \hnf{T}$}
    \AXC{$\hnf{T} \eqbr T$}
    \UIC{$\hnf{T}~\subd T$}
    \BIC{$\hnf{U}~\subd T$}
    \BIC{$U \subd T$}
  \end{prooftree}
\end{proof}

On a besoin d'un lemme sur l'op�ration $\mualgo$ d�finie figure
\ref{fig:mualgo-definition}.

\begin{lemma}[$\mualgo$ et coercion]
  \label{mu-coercion}
  $T \suba \mualgo(T)$.
\end{lemma}

\begin{proof}
  Il suffit de suivre la d�finition de $\mualgo$. La mise en forme
  normale de t�te est �quivalente � l'utilisation de \irule{SubHnf} dans notre jugement de
  coercion. $\mualgo$ est en fait l'application r�p�t�e de la r�gle \irule{SubSub}.
\end{proof}

\begin{lemma}[Conservation des sortes par $\mu$]
  \label{mu-sorts}
  Si $`G \type S : s$ alors $`G \type \mu~S : s$
\end{lemma}

\begin{proof}
  Par le simple fait que si $S = \mysubset{x}{U}{P}$ alors $S : \Set$ et
  $U : \Set$ (par \irule{Subset}), sinon $S = \mu~S$.
\end{proof}

\begin{theorem}[Correction du typage]
  \label{correct-typing}
  Si $`G \typea t : T$ alors $`G \typed t : T$
\end{theorem}

\setboolean{displayLabels}{false}
\begin{proof}
  \begin{induction}[typing-algo]
  \item[- \irule{WfEmpty},\irule{WfVar},\irule{PropSet},\irule{Var},\irule{Prod},\irule{Abs},
    \irule{Sigma}, \irule{Sum}:] r�gles inchang�es.
    
    \case{LetSum}
    On a 
    \begin{prooftree}
      \LetSumA
    \end{prooftree}
    
    Par induction, $`G \typed t : S$, et par le lemme \ref{mu-coercion}
    et la correction de la coercion $S \subd \Sigma x : T. U$.
    On peut donc d�river $`G \typed t : \Sigma x : T.U$ � l'aide de \irule{Coerce}.
    On peut directement appliquer \irule{LetSum} � cette pr�misse et �
    l'hypoth�se d'induction $`G, x : T, y : U \typed v : V$.
    
    \case{App} On a:
    \def\fCenter{\typea}
    \begin{prooftree}
      \AppA
    \end{prooftree}
    
    Par induction, $`G \typed f : T$, et $T \subd \Pi x : V. W$ par le
    lemme \ref{mu-coercion} et la correction de la coercion.
    On peut donc d�river $`G \typed f : \Pi x : V.W$ � l'aide de la r�gle
    \irule{Coerce}.
    Par le lemme \ref{correct-coercion}, et l'hypoth�se $`G \typed u :
    U$, on obtient $`G \typed u : V$ par \irule{Coerce}.
    Donc, par \irule{App}, on a bien $`G \typed f u : W[u/x]$.
  \end{induction}  
\end{proof}

On a prouv� que notre syst�me algorithmique �tait correct, c'est-�-dire
que ses jugements valides sont bien inclus dans ceux du syst�me
d�claratif, il faut maintenant montrer qu'il les inclut tous (ou presque!).

\begin{subsubsection}{Notations}
On introduit la notation $`G \typea T, U : s$ pour $`G \typea T : s `^
`G \typea U : s$.
\end{subsubsection}

\subsection{Compl�tude et d�cidabilit�}
On va maintenant repartir des syst�mes d�claratifs pour montrer comment
l'on a construit les syst�mes algorithmiques. 


On s'int�resse tout d'abord au jugement de coercion.
Pour rendre le jugement de coercion d�cidable, il faut traiter les r�gles
\irule{SubConv} et \irule{SubHnf} qu'on peut appliquer � n'importe quel 
moment et la r�gle \irule{SubTrans} qui n'est pas dirig�e par la syntaxe (il faut
``deviner'' un type $T$). Le syst�me de coercion algorithmique (figure
\ref{fig:subtyping-algo-rules}) est le m�me que le syst�me
d�claratif (figure \ref{fig:subtyping-decl-rules}) mais o� l'on n'applique 
\irule{SubConv} seulement si aucune autre r�gle ne
s'applique apr�s avoir appliqu� \irule{SubHnf} 
et sans la r�gle \irule{SubTrans}.

\subsubsection{D�cidabilit� et compl�tude de la coercion}
On va montrer que les deux syst�mes de coercion sont �quivalents vis-�-vis de la conversion. On montrera plus tard pourquoi on peut �liminer la
r�gle de transitivit�.

On rappelle qu'on consid�re qu'on peut appliquer la normalisation de
t�te \irule{ConvHnf} avant toute
application d'une r�gle de typage ou coercion. Ainsi on peut tr�s bien
d�river $((\lambda x : \Set.x)~\nat) \suba \mysubset{x}{\nat}{P}$ puisque 
$\hnf{((\lambda x : \Set.x)~\nat)} = \nat$ et $\nat \suba
\mysubset{x}{\nat}{P}$ par \irule{SubProof} et \irule{SubConv}.

Il nous faut tout d'abord des lemmes d'inversion sur la conversion:
\begin{lemma}
  \label{conversion-pi}
  Si $\Pi x : T. U \eqbr S$ alors $\hnf{S} = \Pi x : T'. U'$ avec $T \eqbr T'$ et $U
  \eqbr U'$.
\end{lemma}
\begin{lemma}
  \label{conversion-sigma}
  Si $\Sigma x : T. U \eqbr S$ alors $\hnf{S} = \Sigma x : T'. U'$ avec $T \eqbr T'$ et $U
  \eqbr U'$.
\end{lemma}

On peut maintenant montrer:
\begin{lemma}[Conservation de la conversion par coercion]
  \label{conversion-coercion}
  Si $`G \typea T, U : s$ et $T \eqbr U$ alors $T \suba U$.
\end{lemma}
\begin{proof}
  Par induction sur la forme de $\hnf{T}$.
  
  \def\seq{\suba}.
  
  \begin{itemize}
  \item[$\hnf{T} = \Pi x : X.Y$:]
    Alors $\hnf{U} = \Pi y : V.W$ et $X \eqbr V$, $Y \eqbr W$
    d'apr�s le lemme \ref{conversion-pi}.
    Par induction $X \sub Y$ et $V \sub W$. 
    On applique alors \irule{SubProd} � ces deux pr�misses.
    
  \item[$\hnf{T} = \Sigma x : X.Y$:]
    Alors $\hnf{U} = \Sigma y : V.W$, avec $X \eqbr V$ et $Y \eqbr
    W$. Par induction et application de \irule{SubSigma}.
    
  \item[$\hnf{T} `= \mysubset{x}{X}{P}$:] 
    On a alors $\hnf{U} = \mysubset{x}{X'}{P'}$ avec $X \eqbr X'$, $P \eqbr
    P'$, et la propri�t� est vraie par \irule{SubLeft} et \irule{SubRight}:
    
    \begin{prooftree}
      \AXC{$X \sub X'$}
      \LeftLabel{\SubLeftRule}
      \UIC{$\mysubset{x}{X}{P} \sub X'$}
      \LeftLabel{\SubRightRule}
      \UIC{$\mysubset{x}{X}{P} \sub \mysubset{x}{X'}{P'}$}
    \end{prooftree}

  \item[Sinon:]
    On applique obligatoirement \irule{SubConv} et l'on a la pr�misse $T
    \eqbr U$, c'est donc direct.
  \end{itemize}
\end{proof}

Il n'y a pas de probl�me d'identification de sortes dans ce syst�me,
contrairement au syst�me $\lambda~C_\leq$ de Gang Chen \cite{ChenPhD},
puisqu'on n'a pas de cumulativit�. Le seul fait que les arguments sont
tous les deux sort�s avec la m�me sorte
avant de d�river le jugement de coercion nous assure que l'on ne fera
pas d'identification erron�e.

On va maintenant montrer que la r�gle \irule{SubTrans} est admissible
dans notre syst�me algorithmique. On montre ceci en l'�liminant de toute
d�rivation possible la faisant intervenir.

\typenva
Tout d'abord quelques lemmes n�cessaires pour la preuve:
\begin{lemma}[Coercion et $\mualgo$]
  \label{coercion-mu}
  \quad
  \begin{itemize}
  \item Si $\Pi x : X.Y \sub U$ alors $\mualgo(U) = \Pi x : X'.Y'$ et $X' \sub
    X$, $Y \sub Y'$.
  \item Si $\Sigma x : X.Y \sub U$ alors $\mualgo(U) = \Sigma x : X'.Y'$ et $X \sub
    X'$, $Y \sub Y'$.
  \item Pour tout $S,U$, $S \sub \mualgo(U)$ \ssi~$S \sub U$.
  \end{itemize}
\end{lemma}
\begin{proof}
  Par induction sur les d�rivations de $\suba$ et la d�finition de $\mualgo$.

  Dans le dernier cas, de gauche � droite on construit la d�rivation en
  ajoutant des applications de \irule{SubProof} et dans l'autre sens on
  est assur� de trouver la preuve dans la d�rivation m�me de $S \sub U$:
  si $U$ n'est pas de la forme sous-ensemble c'est direct. Sinon, on
  peut trouver dans la preuve (en partant de la racine) la premi�re
  utilisation de la r�gle \irule{SubProof}. A partir de l�, on cherche
  la premi�re utilisation d'une r�gle autre que \irule{SubProof} ou 
  \irule{SubSub}. On a une d�rivation de $S' \suba \mualgo(U)$, on
  peut r�appliquer les r�gles \irule{SubSub} oubli�es pr�c�demment 
  pour obtenir la preuve de $S \suba \mualgo(U)$.
\end{proof}

\begin{lemma}[Coercion et conversion]
  \label{coercion-conversion}
  Si $`G \type S,T,U : s$, $S \eqbr T$ et $T \sub U$ alors $S \sub U$
\end{lemma}

\begin{proof}
  Par simple inspection des r�gles on voit que le jugement ne peut
  distinguer deux termes $\beta$-�quivalents (ils ont forc�ment les
  m�mes formes normales de t�te).
\end{proof}

\begin{lemma}[Coercion et formes normales de t�te]
  \label{coercion-hnf}
  Si $T \suba U$ alors $\hnf{T}~\suba \hnf{U}$ est d�rivable par une
  d�rivation plus petite ou �gale.
\end{lemma}

\begin{proof}
  \begin{induction}[subtyping-algo]

    \case{SubConv} Trivial.
    \case{SubHnf} On prend la d�rivation en pr�misse.
    \casetwo{SubProd}{SubSigma} $T$ et $U$ sont �gaux � leurs formes normales
    de t�te, direct.

    \case{SubProof}
    Par induction $\hnf{T}~\suba \hnf{V}$, on applique \irule{SubProof}
    \case{SubSub} idem.
  \end{induction}  
\end{proof}

\begin{lemma}[Transitivit� de la coercion]
  \label{transitive-coercion}
  Pour tout $S, T, U$ tel que $`G \type S,T,U : s$ si
  $S \sub T$ et $T \sub U$ alors $S \sub U$.
\end{lemma}

\begin{proof} 
  %\TODO{Dans \cite{Pierce:TypeSystems}, voir p. 420}
  On proc�de par �limination de la r�gle \irule{SubTrans} dans toute
  d�rivation de $S \sub U$.
  Par induction sur l'ordre lexicographique $< depth(S \sub T) +
  depth(T \sub U), depth(S \sub U) >$.
  On peut supposer sans perte de g�n�ralit� qu'il n'y a pas
  d'applications successive de la r�gle \irule{SubHnf} dans nos
  d�rivations, par idempotence de la mise en forme normale de t�te.
  
  \begin{induction}

    \casetwo{SubConv}{*}\quad
    \begin{prooftree}
      \AXC{$S \eqbr T$}
      \UIC{$S \sub T$}
      \AXC{$T \sub U$}
      \BIC{$S \sub U$}
    \end{prooftree}
    
    Par le lemme \ref{coercion-conversion}, on �limine trivialement \irule{SubTrans}.

    \casetwo{SubHnf}{*}\quad
    \begin{prooftree}
      \AXC{$\hnf{S}~\sub \hnf{T}$}
      \UIC{$S \sub T$}
      \AXC{$T \sub U$}
      \BIC{$S \sub U$}
    \end{prooftree}

    Par le lemme \ref{coercion-hnf}, il existe une d�rivation de 
    $\hnf{T} \sub \hnf{U}$ de taille plus petite ou �gale � la
    d�rivation de $T \sub U$ on peut donc se ramener au cas:

    \begin{prooftree}
      \AXC{$\hnf{S}~\sub \hnf{T}$}
      \UIC{$S \sub T$}
      \AXC{$\hnf{T} \sub \hnf{U}$}
      \UIC{$T \sub U$}
      \BIC{$S \sub U$}
    \end{prooftree}

    
    Par cas sur la d�rivation de $\hnf{S} \sub \hnf{T}$
    \begin{induction}
      \case{SubConv}\quad
      \begin{prooftree}
        \AXC{$\hnf{S} \eqbr \hnf{T}$}
        \UIC{$\hnf{S}~\sub \hnf{T}$}
        \UIC{$S \sub T$}
        \AXC{$\hnf{T} \sub \hnf{U}$}
        \UIC{$T \sub U$}
        \BIC{$S \sub U$}
      \end{prooftree}
      
      Par le lemme \ref{coercion-conversion}.
      
      \case{SubProd}\quad
      \begin{prooftree}
        \AXC{$C \sub A$}
        \AXC{$B \sub D$}
        \BIC{$\hnf{S}=\Pi x : A.B~\sub \Pi x : C.D = \hnf{T}$}
        \UIC{$S \sub T$}
        \AXC{$\hnf{T} = \Pi x : C.D \sub \hnf{U}$}
        \UIC{$T \sub U$}
        \BIC{$S \sub U$}
      \end{prooftree}
    
      Par cas sur la d�rivation de $\Pi x : C.D \sub \hnf{U}$.
      \begin{itemize}
        \case{SubConv} Trivial.
        \case{SubProd} Alors on a
        \begin{prooftree}
          \AXC{$E \sub C$}
          \AXC{$D \sub F$}
          \BIC{$\Pi x : C.D \sub \Pi x : E.F$}
        \end{prooftree}
        
        On a donc la d�rivation:
        \begin{prooftree}
          \AXC{$E \sub C$}\AXC{$C \sub A$}
          \BIC{$E \sub A$}
          
          \AXC{$B \sub D$}\AXC{$D \sub F$}
          \BIC{$B \sub F$}
          \BIC{$\hnf{S} = \Pi x : A.B \sub \Pi x : E.F = \hnf{U}$}
          \UIC{$S \sub U$}
        \end{prooftree}
        
        La taille des d�rivations de $E \sub C$, $C \sub A$ et $B \sub
        D$, $D \sub F$ �tant plus
        petites que $S \sub T$ et $T \sub U$, on �limine bien la transitivit� dans ce cas.

        \case{SubProof} On a:
        \begin{prooftree}
          \AXC{$\Pi x : C.D \sub E$}
          \UIC{$\Pi x : C.D \sub \mysubset{y}{E}{P}$}
        \end{prooftree}

        Par induction, on a:

        \begin{prooftree}
          \AXC{$\Pi x : A.B \sub \Pi x : C.D$}
          \AXC{$\Pi x : C.D \sub E$}
          \BIC{$\Pi x : A.B \sub E$}          
          \UIC{$\hnf{S} = \Pi x : A.B \sub \mysubset{y}{E}{P} = \hnf{U}$}
          \UIC{$S \sub U$}
        \end{prooftree}
        
        Car $\Pi x : C.D \sub E$ est une d�rivation plus petite que $T
        \sub U$.
      \end{itemize}
            
      \case{SubSigma} De fa�on �quivalente au produit.
    
      \case{SubProof}\quad
      \begin{prooftree}
        \AXC{$\hnf{S} \sub C$}
        \UIC{$\hnf{S}~\sub \mysubset{y}{C}{P} = \hnf{T}$}
        \UIC{$S \sub T$}
        \AXC{$\mysubset{y}{C}{P} = \hnf{T} \sub \hnf{U}$}
        \UIC{$T \sub U$}
        \BIC{$S \sub U$}
      \end{prooftree}
      Encore une fois, par cas sur la d�rivation de $\mysubset{y}{C}{P}
      = \hnf{T} \sub \hnf{U}$:
      \begin{itemize}
        \case{SubConv} Trivial.
        \case{SubSub} On a:
        \begin{prooftree}
          \AXC{$C \sub \hnf{U}$}
          \UIC{$\mysubset{y}{C}{P} = \hnf{T} \sub \hnf{U}$}
        \end{prooftree}
        
        Par induction, on peut donc d�river:
        \begin{prooftree}
          \AXC{$\hnf{S} \sub C$}
          \AXC{$C \sub \hnf{U}$}
          \BIC{$\hnf{S} \sub \hnf{U}$}
          \UIC{$S \sub U$}
        \end{prooftree}
        
        Les d�rivations $\hnf{S} \sub C$ et $C \sub \hnf{U}$ �tant bien
        plus petites que $S \sub T$ et $T \sub U$.     
      \end{itemize}
      
      \case{SubSub} De m�me:
      \begin{prooftree}
        \AXC{$A \sub \hnf{T}$}
        \UIC{$\hnf{S} = \mysubset{y}{A}{P}~\sub \hnf{T}$}
        \UIC{$S \sub T$}
        \AXC{$\hnf{T} \sub \hnf{U}$}
        \UIC{$T \sub U$}
        \BIC{$S \sub U$}
      \end{prooftree}

      Se r�ecrit en:
      \begin{prooftree}
        \AXC{$A \sub \hnf{T}$}
        \AXC{$\hnf{T} \sub \hnf{U}$}
        \BIC{$A \sub \hnf{U}$}
        \UIC{$\hnf{S} = \mysubset{y}{A}{P}~\sub \hnf{U}$}
        \UIC{$S \sub U$}
      \end{prooftree}
    \end{induction}
  
    On peut faire le m�me raisonnement par cas sur la d�rivation de
    $\hnf{T} \sub \hnf{U}$. On peut donc se restreindre aux cas ou l'on
    n'utilise plus la r�gle \irule{SubHnf} dans les pr�misses de \irule{SubConv}.
        
    Les cas restants se montrent de fa�on similaire � leurs �quivalents
    dans la preuve pour le cas \irule{SubHnf}. Par exemple pour le
    produit:

    
    \case{SubProd}\quad
    \begin{prooftree}
      \AXC{$C \sub A$}
      \AXC{$B \sub D$}
      \BIC{$\Pi x : A.B \sub \Pi x : C.D$}
      \AXC{$\Pi x : C.D \sub U$}
      \BIC{$\Pi x : A.B \sub U$}
    \end{prooftree}

    On n'a seulement � traiter le cas ou $\Pi x : C.D \sub U$ est d�riv� par
    \irule{SubProd} ou \irule{SubProof}.
    \begin{itemize}
      \case{SubProd} Alors on a
        \begin{prooftree}
          \AXC{$E \sub C$}
          \AXC{$D \sub F$}
          \BIC{$\Pi x : C.D \sub \Pi x : E.F$}
        \end{prooftree}
        
        On a donc la d�rivation:
        \begin{prooftree}
          \AXC{$E \sub C$}\AXC{$C \sub A$}
          \BIC{$E \sub A$}
          
          \AXC{$B \sub D$}\AXC{$D \sub F$}
          \BIC{$B \sub F$}
          \BIC{$S = \Pi x : A.B \sub \Pi x : E.F = U$}
        \end{prooftree}
        
        La taille des d�rivations de $E \sub C$, $C \sub A$ et $B \sub
        D$, $D \sub F$ �tant plus
        petites que $\Pi x : A.B \sub \Pi x : C.D$ et $\Pi x : C.D \sub
        \Pi x : E.F$, on �limine bien la transitivit� dans ce cas.

        \case{SubProof} On a:
        \begin{prooftree}
          \AXC{$\Pi x : C.D \sub E$}
          \UIC{$\Pi x : C.D \sub \mysubset{y}{E}{P}$}
        \end{prooftree}

        Par induction, on a:

        \begin{prooftree}
          \AXC{$\Pi x : A.B \sub \Pi x : C.D$}
          \AXC{$\Pi x : C.D \sub E$}
          \BIC{$\Pi x : A.B \sub E$}          
          \UIC{$S = \Pi x : A.B \sub \mysubset{y}{E}{P} = U$}
        \end{prooftree}
        
        Car $\Pi x : C.D \sub E$ est une d�rivation plus petite que $T
        \sub U$.
      \end{itemize}



    
  \end{induction}
\end{proof}

\begin{corrolary}[Compl�tude de la coercion]
  \label{complete-coercion}
  Si $U \subd V$ alors $U \suba V$.
\end{corrolary}

\begin{proof}
  Les r�gles des deux syst�mes sont les m�mes except� \irule{SubTrans}
  qui est admissible dans le syst�me algorithmique. De plus
  l'application restreinte de la conversion ne change pas les jugements
  d�rivables (lemme \ref{conversion-coercion}).
\end{proof}

En cons�quence $\subd$ et $\suba$ sont �quivalentes. Le syst�me
d'inf�rence de $\suba$ donne donc un algorithme pour d�cider de la relation
de coercion. L'ind�terminisme entre les r�gles \irule{SubProof} et
\irule{SubSub} ne pose pas de probl�me: on peut laisser le choix � 
l'implantation puisque le syst�me est confluent. \irule{SubHnf}
formalise le fait qu'on peut avoir � r�duire en t�te avant d'appliquer
les autres r�gles (pour obtenir un produit, une somme ou un sous-ensemble).

\subsubsection{D�cidabilit� et compl�tude du typage}
Le syst�me algorithmique correspond au syst�me d�claratif o� l'on a enlev� la r�gle
de coercion \irule{Coerce} et chang� certaines r�gles pour obtenir un syst�me d�cidable
(voir figure \ref{fig:typing-algo-rules}).
On va proc�der de fa�on similaire � l'�limination de la transitivit�
pour montrer que la r�gle \irule{Coerce} n'est plus n�cessaire dans le
syst�me algorithmique. On va montrer en fait que
toute d�rivation de typage utilisant \irule{Coerce} peut se r��crire en
une d�rivation n'utilisant cette r�gle qu'� sa racine.

% \paragraph{Sommes d�pendantes}
% On veut pouvoir faire le plus d'inf�rence possible dans notre syt�me
% algorithmique, on a donc introduit une r�gle \irule{SumInf} qui ne force
% pas � annoter les paires. Dans le cas ou le terme n'est pas annot�, on
% consid�re donc que la somme n'est pas d�pendante. En effet on remarque
%  qu'il n'est pas possible d'inf�rer le type $U$ � partir du seul terme $(t, u)$. Cela
% n�cessiterait de r�soudre un probl�me d'unification d'ordre sup�rieur
% auquel il n'y a pas de solution la plus g�n�rale. 
% On a donc dans le syst�me algorithmique deux r�gles pour les sommes, 
% dont une (\irule{SumDep}) permettant d'annoter le terme avec le type $U$ recherch�. 

\paragraph{\'Elimination de la coercion}
\typenva
On veut maintenant montrer la compl�tude de notre syst�me. Dans un
syst�me � sous-typage, le th�or�me correspondant est parfois nomm�
typage minimal ``\emph{minimal typing}'' puisque son �nonc� revient �
dire que tout terme a un type minimal dans les deux syst�mes. En effet
notre th�or�me est le suivant:
$`G \typed t : T "=>" `G \typea t : U \suba T$. Le typage algorithmique
assigne bien un seul type � un terme $t$ mais comme on a des coercions, le
type inf�r� $U$ peut �tre un peu diff�rent du type $T$. Dans notre cas
particulier $U$ sera certainement un type moins riche que $T$ (par
exemple $\nat$ par rapport � $\mysubset{x}{\nat}{P}$). Lorsque l'on
d�veloppera des programmes, on donnera une sp�cification forte et l'on
fera une coercion entre le type inf�r� et la sp�cification pour obtenir
au final (apr�s r��criture dans \Coq) un terme du type $T$ le plus riche.
On a besoin de quelques lemmes pour montrer que notre syst�me o�
\irule{Coerce} a �t� �limin� est complet:

\begin{lemma}[$\beta$-equivalence et $\mualgo$]
  \label{beta-mu}
  Si $X \sub Y$ et $\mualgo(Y) = \Sigma x : T.U$ alors $\mualgo(X) = \Sigma x : T'.U'$
  et $T' \sub T$, $U' \sub U$.
  Si $X \sub Y$ et $\mualgo(Y) = \Pi x : T.U$ alors $\mualgo(X) = \Pi x :
  T'.U'$ et $T \sub T'$, $U' \sub U$.
\end{lemma}
\begin{proof}
  Par induction sur la d�rivation de coercion, on fait le cas pour $\Sigma$.
  \begin{induction}  
    \case{SubConv} Trivial, puisqu'on aura $\mualgo(X) = \mualgo(Y)$.

    \case{SubHnf} Trivial puisque pour tout $x$, $\mualgo(x) =
    \mualgo(\hnf{x})$.
    
    \case{SubProd} Impossible, $\mualgo$ ne traversant pas les produits.

    \case{SubSigma} Direct, on a une d�rivation de $\Sigma x : T'. U'
    \sub \Sigma x : T.U$.
    
    \case{SubLeft} Ici, $Y `= \mysubset{x}{V}{P}$, on peut donc d�duire que
    $\mualgo(Y) = \mualgo(V) = \Sigma x : T.U$. On 
    applique l'hypoth�se de r�currence avec $X \sub V$ et on obtient:
    $\mualgo(X) = \Sigma x : T'.U' `^ T' \sub T `^ U' \sub U$.

    \case{SubRight} Ici, $X `= \mysubset{x}{V}{P}$. Par induction, 
    $\mualgo(V) = \mualgo(X) = \Sigma x : T'.U' `^ T' \sub T `^ U' \sub U$.
  \end{induction}
\end{proof}

\begin{lemma}[Bonne formation des contextes]
  \label{wf-contexts-a}
  Si $`G \type t : T$ alors $\typewf `G$.
\end{lemma}
\begin{proof}
  \inductionon{typing-decl}
\end{proof}

\begin{fact}[Inversion du jugement de bonne formation]
  \label{inversion-wf-a}
  Si $\typewf `G, x : U$ alors il existe $s$, $`G \type U : s$ et $s `: \setproptype$.
\end{fact}

\begin{lemma}[Affaiblissement]
  \label{weakening-a}
  Si $`G, `D \type t : T$ alors pour tout $x : S `; `G, `D$ tel que
  $\wf `G, x : S, `D$, $`G, x : S, `D \type t : T$
\end{lemma}

\begin{proof}
  \begin{induction}[typing-decl]
    \case{PropSet} Trivial.

    \case{Var}
    On a $x : S `; `G, `D$, donc $`G, x : S, `D \type y : T$ est toujours d�rivable.
    
    \case{Prod}
    Par induction $`G, x : S, `D \type T : s1$ et $`G, x : S, `D,
    y : T \type U : s2$. On applique \irule{Prod} pour obtenir 
    $`G, x : S, `D \type \Pi x : T.U : s2$. De m�me pour le reste des r�gles.
  \end{induction}
\end{proof}  

La restriction montre que notre notion de coercion est correcte
vis-�-vis du typage. On peut d�river les m�mes jugements dans des
contextes o� les variables ont des types �quivalents. Ici la taille des
d�rivations ne change pas.

\begin{lemma}[Restriction]
  \label{narrowing-a}
  \[ `G \seq S, S' : s, S' \sub S "=>" 
  \left\{ \begin{array}{lcl}
      \typewf `G, x : S, `D & "=>" & \typewf `G, x : S', `D \\
      & `^{} & \\
      `G, x : S, `D \seq t : T & "=>" & `G, x : S', `D \seq t : T' \suba T 
    \end{array}
  \right. \]
\end{lemma}

\begin{proof}
  On peut remarquer que si $T `: \setproptype$ alors $T'$ doit �tre �gal
  � $T$ par d�finition de la coercion.

  Par induction sur la taille de la d�rivation de typage ou de bonne formation.
    
  \begin{induction}
    \case{WfEmpty} Trivial.
    
    \case{WfVar} 
    La conclusion est $\typewf `G, x : S, `D$
    
    \begin{induction}[text=Par induction sur la taille de $`D$]
    \item[\protect{$`D = []$}]
      La racine de la d�rivation est de la forme:
      \begin{prooftree}
        \UAX{WfVar}
        {$`G \type S : s$}
        {$\wf `G, x : S$}
        {$s `: \setproptype$}
      \end{prooftree}
      On a donc $`G \type S' : s$, et par \irule{WfVar}, $\typewf `G, x : S'$.  
      
    \item[\protect{$`D `= `D', y : U$}]
      La racine de la d�rivation est de la forme:
      \begin{prooftree}
        \UAX{WfVar}
        {$`G, x : S, `D' \type U : s$}
        {$\wf `G, x : S, `D', y : U$}
        {$s `: \setproptype$}
      \end{prooftree}
      Par induction sur la d�rivation de typage $`G, x : S', `D' \seq U : s$,
      on a donc bien $\typewf `G, x : S', `D', y : U$ par \irule{WfVar}.
    \end{induction}
    
    \case{PropSet}
    Par induction, $\typewf `G, x : S', `D$, on applique simplement la r�gle.
    
    \case{Var}
    Par induction, $\typewf `G, x : S', `D$. La seule diff�rence avec le
    contexte pr�c�dent est le type associ� � $x$, donc si $t \not= x$, on
    peut simplement r�appliquer \irule{Var}. Si $t `= x$ on a la
    d�rivation:

    \begin{prooftree}
      \BAX{Var}
      {$\wf `G, x : S', `D$}
      {$x : S' `: `G$}
      {$`G, x : S', `D \seq x : S'$}
      {}
    \end{prooftree}
    
    On a bien $S' \suba S$, la propri�t� est donc bien v�rifi�e.
    
    \case{Prod} 
    Par induction, $`G, x : S', `D \type T : s1$ et $`G, x : S', `D
    y : T \seq U : s2$. On applique \irule{Prod} pour obtenir 
    $`G, x : S' \type \Pi x : T.U : s2$. De m�me pour le reste des r�gles.
  \end{induction}
\end{proof}

Il nous faut montrer des lemmes faisant intervenir la substitution pour
pouvoir prouver la compl�tude.
\begin{lemma}[Substitutivit� de $\mualgo$]
  \label{substitutive-mu}
  Si $\mualgo(T) = \Pi y : U.V$ alors $\mualgo(T[u/x]) = \Pi y :
  U[u/x].V[u/x]$.
  Si $\mualgo(T) = \Sigma y : U.V$ alors $\mualgo(T[u/x]) = \Sigma y : U[u/x].V[u/x]$.
\end{lemma}

\begin{proof}
  On montre la propri�t� pour les produits, la preuve est similaire pour
  les sommes. Par induction sur la taille de $T$.

  Il suffit de suivre la d�finition de $\mualgo$.
  Si $T$ est de la forme $\mysubset{y}{V}{P}$ alors
  on a $\mualgo(V) = \Pi y : U.V$ et par induction
  $\mualgo(V[u/x]) = \Pi y : U[u/x].V[u/x]$.
  Il s'ensuit directement que $\mualgo(T[u/x]) = \Pi y :
  U[u/x].V[u/x]$.

  Si $T$ est diff�rent d'un type sous-ensemble alors $\mualgo(T) = T$.
  Donc $T \eqbr \Pi y : U.V$. Il s'ensuit que $T$ est de la forme $\Pi y
  : U'.W'$ et donc $T[u/x] = \Pi y : U'[u/x].V'[u/x] =
  \mualgo(T[u/x])$. Par substitutivit� de la conversion, il s'ensuit que 
  $\Pi y : U'[u/x].V'[u/x] \eqbr \Pi y : U[u/x].V[u/x]$.
\end{proof}

\begin{lemma}[Substitutivit� de la coercion]
  \label{substitutive-coercion}
  Si $U \suba T$ alors pour tout $u$, $U[u/x] \suba T[u/x]$.
\end{lemma}

\begin{proof}
  \begin{induction}[subtyping-algo]
    \case{SubConv}
    Direct par pr�servation de l'�quivalence $\eqbr$ par substitution.
    
    \case{SubHnf}
    Par induction, $(\hnf{U})[u/x] \suba (\hnf{T})[u/x]$. 
    Par le lemme \ref{coercion-hnf}, $\hnf{((\hnf{U})[u/x])} \suba
    \hnf{((\hnf{T})[u/x])}$. 
    Donc $\hnf{U[u/x]}~\suba \hnf{T[u/x]}$ et par \irule{SubHnf}, 
    $U[u/x] \suba T[u/x]$.

    \case{SubProd}
    Par induction $U[u/x] \suba T[u/x]$ et $V[u/x] \suba W[u/x]$, donc
    $\Pi y : T[u/x].V[u/x] \suba \Pi y : U[u/x].W[u/x]$. La propri�t�
    est donc bien v�rifi�e.
    
    \case{SubSigma} Direct par induction.
    
    \case{SubSub} Par induction, $U'[u/x] \suba V[u/x]$. On applique
    \irule{SubLeft} pour obtenir $\mysubset{y}{U'[u/x]}{P} \suba V[u/x]$. 
    
    \case{SubRight} Direct par induction.
  \end{induction}
\end{proof}

\begin{lemma}[Substitutivit� du typage]
  \label{substitutive-typing}
  Si $`G \typea u : U$ alors
  \[ \left\{ \begin{array}{lcl}
      `G, x : U, `D \typea t : T & "=>" & `G, `D[u/x] \typea t[u/x] : T[u/x] \\
      \wf `G, x : U, `D & "=>" & \wf `G, `D[u/x]
    \end{array}\right. \]
\end{lemma}

\begin{proof}
  \typenva
  Par induction mutuelle sur la d�rivation de typage $`G, x : U, `D
  \typea t : T$ ou $\wf `G, x : U, `D$.
  
  \begin{induction}
    \case{WfEmpty} Trivial.

    \case{WfVar}
    Par induction sur $`D$.
    \begin{itemize}
    \item[\protect{$`D = []$}]
      On a alors $`G \typea U : s$ donc $\wf `G$ et trivialement, $\wf
      `G, `D[u/x]$.

    \item[\protect{$`D = `D', y : T$}]
      On a alors $`G, x : U, `D' \typea T : s$ et par induction
      $`G, `D'[u/x] \typea T[u/x] : s[u/x] = s$. Donc on peut appliquer
      \irule{WfVar} pour obtenir $\wf `G, `D'[u/x], y : T[u/x]$ soit
      $\wf `G, `D[u/x]$
    \end{itemize}
    
    \case{PropSet}
    La substitution n'a aucun effet et $`G, `D[u/x]$ est bien
    form� par induction.
    
    \case{Var}
    Par induction, $\wf `G, `D[u/x]$.
    Si $t `= x$ alors on a $T = U$ et $T[u/x] = U$ puisque $x$
    n'appara�t pas dans $U$. On a donc $`G, `D[u/x] \typea t[u/x] = u :
    T[u/x] = U$, qui peut s'obtenir par affaiblissement de $`G \typea u
    : U$. 
    Si $y : T `: `G$ alors on applique simplement \irule{Var}.
    Si $y : T `: `D$ alors $y : T[u/x] `: `D[u/x]$ et on obtient
    $`G, `D[u/x] \typea y[u/x] :  T[u/x]$ par \irule{Var}.
    
    \case{Prod}
    Par induction  $`G, `D[u/x] \typea T[u/x] : s_1[u/x]$ et
    $`G, `D[u/x], y : T[u/x] \typea M[u/x] : s_2[u/x]$. 
    On peut appliquer \irule{Prod} pour obtenir $`G, `D[u/x] \typea \Pi
    y : T[u/x].M[u/x] : s_2[u/x]$ soit $`G, `D[u/x] \typea (\Pi y :
    T.M)[u/x] : s_2[u/x]$.
    De fa�on similaire pour les autres cas.

    \case{App}
    On �tudie le cas de l'application qui requiert un lemme suppl�mentaire.
    Par induction, on a $`G, `D[u/x] \typea f[u/x] : T[u/x]$ et
    $`G, `D[u/x] \typea a[u/x] : A[u/x]$. Si $\mualgo(T) = \Pi y :
    V.W$ alors $\mualgo(T[u/x]) = \Pi y : V[u/x].W[u/x]$ (lemme
    \ref{substitutive-mu}). Par induction, on a aussi $`G, `D[u/x] \typea
    A[u/x],V[u/x] : s$. Enfin, par substitutivit� de la coercion on a $A[u/x]
    \suba V[u/x]$. On peut donc appliquer \irule{App} pour obtenir 
    $`G, `D[u/x] \typea (f[u/x]~a[u/x]) : W[u/x][a[u/x]/y]$. Or
    $W[u/x][a[u/x]/y] = W[a/y][u/x]$ ($y `; \freevars{u}$). On a donc bien $`G, `D[u/x] \typea (f~a)[u/x] :
    (W[a/y])[u/x]$.
    On a un raisonnement similaire pour \irule{LetSum}.
  \end{induction}
  
\end{proof}

\begin{lemma}[Substitutivit� du typage avec coercion]
  \label{substitutive-typing-coercion}
  Si $`G, x : V \typea t : T \sub U$ et $`G \typea u : V$
  alors $`G \typed t[u/x] : T[u/x] \sub U[u/x]$.
\end{lemma}

\begin{proof}
  Par substitutivit� du typage (\ref{substitutive-typing}) on a $`G \typed t[u/x] : T[u/x]$.
  Par le lemme pr�c�dent $T[u/x] \suba U[u/x]$.
\end{proof}

On a maintenant tout les ingr�dients pour montrer la compl�tude de notre
syst�me de typage vis-�-vis du syst�me d�claratif.

\setboolean{displayLabels}{true}
\begin{theorem}[Compl�tude du typage]
  \label{complete-typing}
  Si $`G \typed t : T$ alors $`E U, `G \typea t : U \sub T$.
  Si $\typewf `G$ dans le syst�me d�claratif alors $\typewf `G$ dans le
  syst�me algorithmique.
\end{theorem}

\begin{proof}
  \typenva
  Par induction mutuelle sur les d�rivations de typage et bonne formation.
  \begin{induction}
    \case{WfEmpty} Trivial.
    
    \case{WfVar}\quad
    \typenvd
    \begin{prooftree}
      \WfVar
    \end{prooftree}
    Par induction $`E s', `G \typea A : s' \sub s$. On a forc�ment $s' =
    s$ puisque les sortes ne sont en relation qu'avec elles-m�mes.
    On applique \irule{WfVar} pour obtenir $\typewf `G, x : A$.
    
    \case{PropSet} Trivial.

    \case{Var}\quad
    \typenvd
    \begin{prooftree}
      \Var
    \end{prooftree}
    Par induction $\typewf `G$ et $x : A `: `G$ , direct par
    \irule{Var}.
    
    \case{Prod}\quad
    \typenvd
    \begin{prooftree}
      \Prod
    \end{prooftree}
    
    Direct par induction et le fait qu'une sorte ne peut �tre en relation
    qu'avec elle m�me.
        
    \case{Abs} \quad
    \typenvd
    \begin{prooftree}
      \Abs
    \end{prooftree}
    Par induction $`E U', `G, x : T \typea M : U' \suba U$.
    Or cela implique $`G, x : T \typea U' : s$, donc on peut d�river
    $`G \typea \Pi x : T.U' : s$. On a donc bien $`G \typea \lambda x :
    T.M : \Pi x : T.U' \suba \Pi x : T.U$.
    
    \case{App} On a 
    \typenvd
    \begin{prooftree}
      \App
    \end{prooftree}
    
    \typenva
    Par induction, $`E T, `G \typea f : T \suba \Pi x : V. W$ et
    $`E U, `G \subta u : U \sub V$.
    
    Si $T \suba \Pi x : V.W$ alors $\mualgo(T) = \Pi x : V'.W'$ avec
    $V \suba V'$ et $W' \suba W$ (lemme \ref{beta-mu}).

    Par transitivit� de la coercion: $U \suba V'$, on peut donc d�river 
    \begin{prooftree}
      \TAX{App}
      {$`G \seq f : T \quad \mualgo(T) = \Pi x : V'. W'$}
      {$`G \seq u : U \quad `G \seq U, V' : s$}
      {$U \suba V'$}
      {$`G \seq (f u) : W' [ u / x ]$}
      {}
    \end{prooftree}
    
    Par substitutivit� de la coercion (lemme
    \ref{substitutive-coercion}), $W'[u/x] \suba W[u/x]$, la propri�t�
    est donc bien v�rifi�e.
    
    \case{Sigma}\quad
    \typenvd
    \begin{prooftree}
      \SigmaR
    \end{prooftree}

    Par induction $`E s', `G \typea T : s' \sub s$ et $`E s'', `G, x : T 
    \typea U : s' \suba s$ o� $s `: \{ \Prop, \Set \}$. Encore une fois  
    les sortes $s$, $s'$ et $s''$ doivent �tre �gales. C'est direct
    par \irule{Sigma}.

    \case{Sum}\quad
    \typenvd
    \begin{prooftree}
      \Sum
    \end{prooftree}
    \typenva
    
    Ici, l'annotation nous force � utiliser le jugement de coercion.
    Par induction, $`E s', \Sigma x : T.U : s' \suba s$, $`E T', `G
    \type t : T' \suba T$ et $`E U', `G \typea u : U' \suba U[t/x]$.
    On peut montrer $`G \type \Sigma x : T'.U : s$.
    En effet, par inversion de $`G \seq \Sigma x : T.U : s$ on a
    $`G, x : T \seq U : s$ et par restriction ($T' \sub T$), $`G, x : T' \seq
    U : s$.
    Comme $T' \suba T$ on obtient $\Sigma x : T'.U \suba \Sigma x : T.U$.
    On peut donc d�river:
    \begin{prooftree}
      \QAX{SumDep}
      {$`G \seq t : T'$}
      {$`G \seq u : U'$}
      {$`G \seq U[t/x], U' : s \quad U' \suba U[t/x]$}
      {$`G \seq \Sigma x : T'.U : s$}
      {$`G \seq (x \coloneqq~t, u : U) : \Sigma x : T'.U$}
      {}
    \end{prooftree} 
        
    \case{LetSum} On a
    \typenvd
    \begin{prooftree}
      \LetSum
    \end{prooftree}
    
    \typenva
    Par induction, $`E S, `G \typea t : S \suba \Sigma x : T.U$ et 
    $`E V', `G, x: T, y : U \typea v : V' \suba V$.
    On a $\mualgo(S) = \Sigma x : T'.U'$ avec $T' \suba T$ et $U'
    \suba U$. Par restriction on peut donc d�river $`G, x : T', y : U'
    \seq v : V'' \suba V'$.
    
    On a donc la d�rivation suivante dans le syst�me algorithmique:
    \begin{prooftree}
      \TAX{LetSum}
      {$`G \seq t : S$}
      {$\mualgo(S) = `S x : T'. U'$}
      {$`G, x : T', y : U' \seq v : V''$}
      {$`G \seq \letml~(x, u) = t~\inml~v : V''$}
      {}
    \end{prooftree}
    
    Comme $V'' \suba V' \suba V$, la propri�t� est vraie par
    transitivit� de la coercion.

    \casetwo{Conv}{Coerce}
    Dans les deux cas on a inductivement $`E T', `G \typea t : T'
    \suba T$. Avec \irule{Conv} on a $T \eqbr S$, donc $T' \suba S$ par
    le lemme \ref{coercion-conversion}. Pour \irule{Coerce} on a $T \subd S$.
    Par compl�tude de la coercion, $T \suba S$ et par transitivit� de la
    coercion, $T' \suba S$. La propri�t� est donc bien v�rifi�e dans les
    deux cas.
  \end{induction}
\end{proof}

On combine les th�or�mes de correction et de compl�tude pour obtenir
 la propri�t� suivante entre les deux syst�mes:
\begin{corrolary}[�quivalence des syst�mes d�claratif et algorithmique]
  $`G \typed t : T$ \ssi{} il existe $U$ tel que $`G \typea t : U$ et $U \suba T$.
\end{corrolary}

On a maintenant un syst�me raffin� d�rivant les m�mes jugements (�
coercion pr�s) que le syst�me d�claratif. On veut en extraire un
algorithme de typage. Pour cela on doit pouvoir r�soudre deux probl�mes:
\begin{itemize}
\item\textbf{V�rification de type.} On donne $`G$,$t$ et $T$ et l'on doit
  d�cider si $`G \typea t : T$ ;
\item\textbf{Inf�rence de type.} On donne $`G$,$t$ et l'on doit trouver $T$ tel
  que $`G \typea t : T$ si c'est d�rivable, sinon on �choue.
\end{itemize}
En pratique, la v�rification a besoin de l'inf�rence puisque lorsqu'on
v�rifie une application $f u : T$ on doit inf�rer le type de $f$.
On montre donc les th�or�mes suivants:

\begin{theorem}[D�cidabilit� de l'inf�rence dans le syst�me algorithmique]
  Le probl�me d'inf�rence $`G \typea t :~?$ est d�cidable.
\end{theorem}

\begin{proof}
  Il suffit d'observer que les r�gles de typage sont dirig�es par la
  syntaxe du deuxi�me argument et permettent donc d'inf�rer un type pour
  tout terme. En lisant les pr�misses de chaque r�gle de gauche �
  droite, on voit que l'inf�rence est d�cidable.
\end{proof}

\begin{theorem}[D�cidabilit� de $\typea$]
  La relation de typage $`G \typea t : T$ est d�cidable.
\end{theorem}
\begin{proof}
  Direct. On utilise le th�or�me pr�c�dent pour le cas de l'application.
\end{proof}

On a d�sormais un algorithme de typage pour notre syst�me avec
coercions. Ce syst�me est tr�s lib�ral puisqu'il permet de consid�rer
des objets comme v�rifiant des propri�t�s arbitraires sans les
montrer. Il nous faut maintenant remettre de la logique dans nos termes
pour s'assurer qu'ils sont corrects.

%%% Local Variables: 
%%% mode: latex
%%% TeX-master: "subset-typing"
%%% LaTeX-command: "TEXINPUTS=\"style:$TEXINPUTS\" latex"
%%% End: 

\section{G�n�ration des obligations de preuve}
On a maintenant un syst�me de typage d�cidable et l'on veut d�sormais
traduire ses d�rivations dans \CCI{} dont le jugement de typage est $\typec$. 

\subsection{D�finition de la r��criture vers \Coq}
\typenvi
La traduction transforme une d�rivation dans notre syst�me algorithmique
vers une d�rivation de \CCI{} valide. Le jugement 
$\timpl{`G}{t}{T}{`G'}{t'}{T'}$ se lit: on transforme le s�quent
$`G \typea t : T$ (syst�me algorithmique) en $`G' \typec t' : T'$
(\Coq). Le jugement $\subimpl{`G}{c}{T}{U}$ se lit: la coercion de $T$ �
$U$ est $c$ et on construit le s�quent $`G \typec c : T "->" U$.
La traduction est un homomorphisme (elle conserve la structure de la
d�rivation et se rappelle r�cursivement) except� pour l'application, ce qui
est normal puisque nous avons un syst�me tr�s proche de \CCI{}. Le fait
de traduire aussi les environnements $\Gamma$ est d� au fait que nous
faisons la coercion dans les types,  donc les environnements (listes
de couples $(\text{nom}, \text{type})$) doivent aussi �tre r��crits. Cela assure aussi la
coh�rence avec l'environnement g�n�ral de \Coq, c'est-�-dire
l'int�gration transparente de notre tactique dans les d�veloppements
\Coq~et la r�utilisabilit� des programmes g�n�r�s. En cons�quence, les
types sp�cifi�s ne sont donc pas toujours pr�serv�s (on veut pouvoir y
introduire des coercions).

\typeiFig
\typemuiFig

Ici la fonction $\muimpl$ renvoie � la fois un type (qu'on demande
�quivalent � un produit) et une fonction de coercion qui va faire les
projections n�cessaires sur l'objet \Coq~$f'$. En effet dans \Coq~les
objets de type sous-ensemble $\subset{x}{T}{P}$ sont cod�s par un terme 
de la forme $\sref{elt}~t~p$ dont on peut extraire les parties objet 
(un certain $t$ de type $T$, par la projection $\Pi_1$) et preuve
 (de type $P[t/x]$). Il faut donc faire exactement
une projection pour atteindre par exemple la fonction d'un objet de type
 $\subset{f}{\nat "->" \nat}{f~0 \neq 0}$.

Le jugement de coercion $U \suba V$ nous assure qu'il est possible de
d�river le jugement $`G' \typec U' \subi V'$ et donc de cr�er une coercion de $U'$
� $V'$ soit une fonction de type $U' "->" V'$ dans \CCI.
On trouve ici l'essence du m�canisme de coercion par pr�dicats. 

\subtiFig

\vspace{1.4em}
\begin{itemize}
\item[ \SubConvI\DP:] \quad\\

  Cr�e une coercion identit� puisque \CCI{}~a la r�gle de conversion. 
  \vspace{1em}

\item[\SubLeftI\DP:] \quad\\
  
  Engendre une projection,
  c'est le cas o� l'on ne s'int�resse pas � la preuve accompagnant
  l'objet. 
  \vspace{1em}

\item[\SubRightI\DP:] \quad\\

  Correspond � la g�n�ration d'une
  obligation de preuve dans \PVS. On utilise le m�canisme des variables
  existentielles (not�es $?:\text{type}$) d�crit plus loin pour donner 
  l'information au syst�me qu'il faut compl�ter le terme � un endroit
  donn� avec un nouveau terme de type appropri�. On peut ais�ment cr�er
  des obligations qui ne seront pas prouvables mais cela rel�ve de la
  responsabilit� de l'utilisateur.
  \vspace{1em}

\item[\SubProdI\DP,] \quad\\
  

\item[\SubSigmaI\DP:] \quad\\

  R�alisent respectivement
  les coercions pour les produits fonctionnels et cart�siens.

\end{itemize}
  
\subsection{Propri�t�s}
On veut montrer que si l'on a une d�rivation dans notre syst�me
algorithmique, alors son image par la r��criture est une d�rivation
valide de \CCI{} (par induction sur la d�rivation dans le syst�me
algorithmique). 


% Ce travail est en cours � ce jour, nous nous sommes
% plut�t pench�s sur l'impl�mentation du typeur et de la fonction de
% r��criture avant de commencer cette derni�re preuve.

%%% Local Variables: 
%%% mode: latex
%%% TeX-master: "subset-typing"
%%% LaTeX-command: "TEXINPUTS=\"style:$TEXINPUTS\" latex"
%%% End: 


\section{La tactique \Subtac}
Nous avons d�velopp� la tactique \Subtac{} disponible dans la version
\CVS{}~courante de \Coq{} (\url{http://coq.inria.fr}). Elle permet de
cr�er un programme, le typer et g�n�rer un terme incomplet
correspondant (voir annexe \ref{fig:euclid-subtac}). 

\subsection{Existentielles}
La g�n�ration des buts correspondant aux variables existentielles et la
formation du terme final sont laiss�es � la tactique \Refine~et au
syst�me de gestion des existentielles de \Coq. Certaines limitations 
dans l'impl�mentation du raffinement (le m�canisme permettant de manipuler
des termes ``� trous'') devront �tre d�pass�es pour obtenir
une contribution totalement fonctionnelle. On peut cependant esp�rer
r�soudre ces probl�mes de fa�on satisfaisante dans un futur proche.

\subsection{Traitement de la r�cursion}
Lorsque l'on d�veloppe un programme r�cursif dans un syst�me tel que
\Coq, on est forc� de fournir une preuve de terminaison de son
algorithme. Pour cela, on montre g�n�ralement qu'on a un ordre bien
fond� sur le type de l'argument de r�cursion et que chaque appel respecte
cet ordre. Nous avons ajout� des facilit�s d'�criture de fonctions
r�cursives � notre langage ; on ajoute les existentielles
correspondant aux preuves que l'ordre est bien fond� ou qu'il est bien
respect� par les termes. Ainsi lors du raffinement on obtient naturellement
les buts correspondants � prouver.

\subsection{Traitement des inductifs}
Notre langage ne prend pas encore en compte les d�finitions inductives g�n�rales.
Au-del� du traitement des types sous-ensemble, on a un support minimal
pour les inductifs � deux constructeurs qui correspondent � des bool�ens
annot�s par des propri�t�s logiques (voir traitement de la
conditionnelle figure \ref{fig:euclid-subtac}). 

%%% Local Variables: 
%%% mode: latex
%%% TeX-master: "subset-typing"
%%% LaTeX-command: "TEXINPUTS=\"style:$TEXINPUTS\" latex"
%%% End: 


\chapter{Conclusion}
Nous avons d�velopp� un langage de programmation plus souple que le
langage de \Coq{} mais conservant sa richesse d'expression (types
d�pendants). Il permet de d�coupler la description algorithmique de la
v�rification. La correction des termes engendr�s est
garantie par le syst�me sous-jacent qui offre ensuite la possibilit�
d'extraire un programme correct par construction dans un langage de type
\ML. D'autre part, cette m�thode s'int�gre bien dans l'environnement
\Coq{} et ouvre la voie � la r�alisation de travaux plus complexes par
des utilisateurs non-experts. Cela constitue la premi�re �tape vers un
environnement de programmation s�re utilisable dans \Coq. 

\ifthenelse{\boolean{showlog}}{
\clearpage
\section*{Journal}

\subsection*{8 mars}
Nouvelle r�gle de produit fonctionel avec contravariance bien typ�e,
produit d�pendant (\SubSigmaRule) covariant.
Un exemple de produit avec contravariance se trouve dans \cite{cal00coherence}, p. 6.
Exemple d'utilisation int�ressante:

\BAX{}
{$`G \judgetypei f : \{ \phi : \even "->" `N `| `A x : `N, \phi~x ``<= x \}$}
{$`G \judgetypei g : `N "->" `N := \matht{pred}$}
{$`G \judgesubi g ``<= f$}
{}
\DP

Le sous-typage avec coercions: Luo, Callaghan, Sa�bi \cite{saibi97inheritance}... 
\begin{itemize}
\item Uniformit� du sous-typage: ne d�pend pas du contexte.
\item Coercions d�clar�es dans l'environement (ex: Coq).
\end{itemize}

Dans HOL, Joe Hurd simule le \emph{predicate subtyping} � la PVS avec
des \emph{predicate sets} \cite{hurd2001a}. Technique adaptable � Coq ?

%%% Local Variables: 
%%% mode: latex
%%% TeX-master: "~/research/coq/papers/subset-typing"
%%% End: 

\subsection*{9 mars}
Le \ps{} dans HOL n'est pas correct, il peut �tre subverti ais�ment �
cause d'une sorte de covariance des domaines de fonctions et de la
fonction de suppression des predicats (similaire � $\mu$):
$inv : `R^{\neq 0} "->" `R `: `R "->" `R$. Le d�veloppement dans \HOL{}
est fait directement dans le language, et il est argu� qu'il n'est pas
possible de le faire correctement � cause de l'exemple
pr�c�dent. L'algorithme de sous-typage est simplement une g�n�ration de
tout les sous-types possibles � partir d'un ensemble de r�gles de
sous-typage pour les constantes et constructions logiques ou fonctionelles.

Trouv� un article \cite{stumpsubset} sur les types sous-ensembles dans PF, logique d'ordre sup�rieur
avec fonctions partielles... Permet de traiter le cas
$\ifml 1 / i > 0 \thenml i \neq 0 \elseml `_$, qui g�n�re une obligation de
preuve $i \neq 0$ dans \PVS{}. Evidemment on ne risque pas de pouvoir
typer ce code en \Coq{} lorsque $`/ : `Z "->" \{ x : `Z `| x \neq 0 \} "->" `Z$ mais
certaines id�es peuvent �tre int�ressantes. 

%%% Local Variables: 
%%% mode: latex
%%% TeX-master: "~/research/coq/papers/subset-typing"
%%% End: 

\subsection*{11 mars}
Typage de la division euclidienne:
$\matht{div} : `A a : `N, `A b : \{ x : `N `| x \neq 0 \}, `S q : `N, `S r : \{ n : `N
`| n < b \}, a = bq + r := \funml a~b "=>" \ifml a < b \thenml (0, a) \elseml \letml (q, r) =
\matht{div}~(a - b)~b \inml (q + 1, r)$.

Soit $`t_{div} = `A a : `N, `A b : \{ x : `N `| x \neq 0 \}, `E q : `N, `E r : \{ n : `N
`| n < b \}, a = bq + r$ et $`G = \matht{div} : `t_{div}$:



\AXC{$1$}
\AXC{$2$}
\BIC{$`G, a : `N, b : `N^{*} \seq \ifml a < b \thenml (0, a) \elseml \letml (q, r) =
  \matht{div}~(a - b)~b \inml (q + 1, r) : `E q \dots$}
\doubleLine
\UIC{$`G \seq \funml a~b "=>" \ifml a < b \thenml (0, a) \elseml \letml (q, r) =
  \matht{div}~(a - b)~b \inml (q + 1, r) : `t_{div}$}
\DisplayProof

Soit $`G_{if} = `G, a : `N, b : `N^{*}, a ``/< b$.
\begin{prooftree}
  \AXC{$`G_{if} \seq b : `N^{*}$}
  \RightLabel{$`b = `N^{*}$}
  \UIC{$`G_{if} \seq b : `b$}

  \AXC{$`G_{if} \seq b : `g$}
  \AXC{$`G_{if} \seq (- a) : `g "->" `b''$}
  \AXC{$`G_{if} \judgetypea $}
  \TIC{$`G_{if} \seq (- a) b : `b''$}
  \AXC{$`G_{if} \seq \matht{div} : `b''' "->" `b' "->" `a$}
  \AXC{$`G_{if} \judgesubd a - b : `b'' ``<= `b''' "~>" t$}
  \UIC{$`G_{if} \seq \matht{div}~(a - b) "~>" \matht{div}~t : `b' "->" `a$}

  \TIC{$`G_{if} \seq \matht{div}~(a - b) : `b' "->" `a$}

  \AXC{$`G_{if} \judgesubd b : `b ``<= `b' "~>" p $}
  \TIC{$`G_{if} \seq \matht{div}~(a - b)~b "~>" \matht{div}~(a - b)~p : `a$}
  
  
  \AXC{$`G_{if}, (q, r) : `a \seq (q + 1, r) : `E q \dots$}
  \RightLabel{1}
  \BIC{$`G_{if} \seq \letml (q, r) = \matht{div}~(a - b)~b \inml (q + 1, r) : `E q \dots$}
\end{prooftree}  

\RightLabel{2}
\AXC{$`G, a : `N, b : `N^{*}, a < b \seq (0, a) : `E q \dots$}
\DisplayProof


%%% Local Variables: 
%%% mode: latex
%%% TeX-master: "~/research/coq/papers/subset-typing"
%%% End: 

\subsection*{14 mars}
R�gles d'introduction, d'�limination et de formation pour $\Pi$, $\Sigma$, sous-types
pr�dicats. Nouveau jugement de typage par r�ecriture. Nombreuses
questions � discuter avec Christine. Beaucoup de bruit dans le buro !


%%% Local Variables: 
%%% mode: latex
%%% TeX-master: "~/research/coq/papers/subset-typing"
%%% End: 

\subsection*{15 mars}
Distinction inf\'erence et typage. 
\begin{description}
\item[Inf\'erence ($"~>"$)] \`a la ML, on v\'erifie: $`G \typea p "~>" T "=>"
  `G \typed p : T$. 
\item[Typage ($:$)]. On a $`G \typed p : T$, on veut $`E U, `G \typea p
  "~>" U `^ `G \typea U "~>" T$. On utilise le sous-typage g\'en\`erant les
  obligations de preuve.
\end{description}

Eliminer \rname{LetSub}, inutilisable en pratique.

Quelques points \`a m\'editer:
\begin{itemize}
\item $`O \type 3 : `N "~>" `O \type 3 : `N^*$ ? D\'ependance envers le
  terme pour le sous-typage. De m\^eme, $2 : `N ``<= `N^*$, on devrait
  parler de renforcement.
\item On peut restreindre le sous-typage aux projections de types
  subsets avec l'\'egalit\'e syntaxique.
\item Je peux garder mes r\^egles de sous-typages, si elles sont syntax-directed!
\end{itemize}

Sous-typage \`a l'application et variable suffisante pour l'ad\'equation ?
On distingue les deux phases, pas de sous-typage \`a l'application.

V\'erifier Sub-{Left, Right}, l'application du sous-typage.

%%% Local Variables: 
%%% mode: latex
%%% TeX-master: "~/research/coq/papers/subset-typing"
%%% LaTeX-command: "TEXINPUTS=\"style:$TEXINPUTS\" latex"
%%% End: 

\subsection*{16 mars}
Il faut faire du sous-typage dans la sp�cification aussi:
$f : x : \subset{n}{`N}{0 \neq} "->" \subset{n}{`N}{x <}$.

Les deux phases:
\begin{description}
\item[Inf�rence] on donne les types impr�cis, ie: dans $x > n$, 
  $n "~>" `N$.
\item[Typage] on traverse la premi�re d�rivation de typage en ajoutant
  les coercions appropri�es, par exemple:
  $`G \type n : \subset{n}{`N}{0 \neq}$,  $\type_{inf} n "~>" `N$ est
  r�ecrit en: 
  $`G \type \pi_{1}~n : `N$.
\end{description}

\subsubsection*{Soir!}

Formalisation des trois jugements:
\begin{description}
\item[$\typed$] Typage d�claratif, syst�me ind�cidable, repr�sentant
  exactement ce qu'on veut ajouter comme fonctionnalit�.
\item[$\typei$] Version algorithmique, utilisant le dernier jugement
  pour r�aliser l'ad�quation avec la pr�sentation d�clarative.
\item[$\judgesubi$] ``Sous-typage'', sans obligations de preuves,
  d�cidable et � peu pr�s d�terministe.
\end{description}

Il faudra ensuite faire la traduction dans \Coq, avec de nouveaux
jugements r�ecrivant les d�rivations.

Propri�t�s � montrer:
\begin{itemize}
\item $`G \typed t : T "=>" `E U, `G \typei t : U `^ `G \judgesubi t : U \sub T$
\item $`G \typei t : T "=>" `G \typed t : T$
\end{itemize}

%%% Local Variables: 
%%% mode: latex
%%% TeX-master: "~/research/coq/papers/subset-typing"
%%% End: 

\subsection*{17 mars}

\begin{lemma}[Substitutivit� du sous-typage]
  \label{substitutive-subtyping}
  $`G \judgesub t : T \impsub U "=>" `G \typed t : T[u/x] \impsub U[u/x]$
\end{lemma}
\begin{proof}
  Par induction sur la d�rivation de sous-typage:

  \begin{itemize}
  \item[\SubConvRule:] On a $T \eqbi U$ et l'on suppose 
    $x \typed T$. On a donc bien $`G \typed x : U$.
    
  \item[\SubProdRule:] On a
    \begin{prooftree}
      \SubProd
    \end{prooftree}
    
    Par induction, $`G \typed x : U$, $`G \typed x : T$, 
    $`G, x : U \typed v : V$ et $`G, x : U \typed v : W$.
    Par \AbsRule, $`G \typed \lambda x : U.v : \Pi x : U.W$.
    
  \item[\SubSigmaRule:] On a 
    \begin{prooftree}
      \SubSigma
    \end{prooftree}
    
    Par hypoth�se d'induction, $`G \typed t : T, U$, $`G \typed v : V[t/x], W[t/y]$.
    Par \SumRule, $(t, v) : \Sigma y : U, W$.

  \item[\SubLeftRule:] On a
    \begin{prooftree}
      \SubLeft
    \end{prooftree}
    
    Par induction, $`G \typed p : V$.

  \item[\SubRightRule:] On a
    \begin{prooftree}
      \SubRight
    \end{prooftree}
    
    Par induction, $`G \typed p : U$. Par \SubsetRule, $`G \typed p :
    \subset{x}{U}{P}$.
    
  \end{itemize}
\end{proof}

\begin{lemma}[Inversion du sous-typage]
  \label{inversion-subtyping}
\end{lemma}

\begin{lemma}[D�rivabilit� de la r�flexivit�, transitivit� du sous-typage]
  \label{refl-trans-subtyping}
\end{lemma}

\begin{lemma}[Correction du sous-typage]
  \label{correct-subtyping}
  Si $`G \judgesub t : U \impsub V$ alors $`G \typed t : U "=>" `G
  \typed t : V$.
\end{lemma}

\setboolean{displayLabels}{false}

\begin{proof}
  Par induction sur la d�rivation de sous-typage:

  \begin{itemize}
  \item[\SubConvRule:] On a $T \eqbi U$ et l'on suppose 
    $x \typed T$. On a donc bien $`G \typed x : U$.
    
  \item[\SubProdRule:] On a
    \begin{prooftree}
      \SubProd
    \end{prooftree}
    
    Par induction, $`G \typed x : U$, $`G \typed x : T$, 
    $`G, x : U \typed v : V$ et $`G, x : U \typed v : W$.
    Par \AbsRule, $`G \typed \lambda x : U.v : \Pi x : U.W$.
    
  \item[\SubSigmaRule:] On a 
    \begin{prooftree}
      \SubSigma
    \end{prooftree}
    
    Par hypoth�se d'induction, $`G \typed t : T, U$, $`G \typed v : V[t/x], W[t/y]$.
    Par \SumRule, $(t, v) : \Sigma y : U, W$.

  \item[\SubLeftRule:] On a
    \begin{prooftree}
      \SubLeft
    \end{prooftree}
    
    Par induction, $`G \typed p : V$.

  \item[\SubRightRule:] On a
    \begin{prooftree}
      \SubRight
    \end{prooftree}
    
    Par induction, $`G \typed p : U$. Par \SubsetRule, $`G \typed p :
    \subset{x}{U}{P}$.
    
  \end{itemize}
  
\end{proof}


\begin{lemma}[Inversion du typage]
\label{inversion-typing}
\end{lemma}

\begin{lemma}[Correction du typage]
  \label{correct-typing}
  $`G \typei t : T "=>" `G \typed t : T$
\end{lemma}

\begin{proof}
  Par induction sur la d�rivation dans le syst�me algorithmique:

  \begin{description}
  \item[\WfAtomRule,\WfVarRule,\PropSetRule,\VarRule,\ProdRule,\AbsRule,
    \LetInRule, \SigmaRule, \SumRule, \LetSumRule:] r�gles inchang�es.

  \item[\AppRule:] On a
    \def\fCenter{\typei}
    \begin{prooftree}
      \AppI
    \end{prooftree}
    
    Par induction, $`G \typed f : \Pi x : V. W $.
    Par le lemme \ref{correct-sub}, et l'hypoth�se $`G \typed u : U$, 
    $`G \typed u : V$. Donc, par \AppRule, on a bien $`G \typed f u :
    W[u/x]$.
  \end{description}
  
\end{proof}

\begin{lemma}[Compl�tude du typage]
  \label{complete-typing}
  $`G \typed t : T "=>" `E U, `G \typei t : U `^ `G \judgesub t : U \impsub T$
\end{lemma}

\begin{proof}
  Par induction sur la d�rivation dans le syst�me d�claratif:

  \begin{description}
  \item[\WfAtomRule,\WfVarRule,\PropSetRule,\VarRule,\ProdRule,\AbsRule,
    \LetInRule, \SigmaRule, \SumRule, \LetSumRule:] r�gles inchang�es.
    
  \item[\AppRule:] On a 
    \begin{prooftree}
      \App
    \end{prooftree}
    
    Par induction, $`E X, Y, `G \typei f : \Pi x : X. Y `^ 
    `G \judgesub f : \Pi x : X. Y \impsub \Pi x : V. W$ et
    $`E U, `G \typei u : U `^ `G \judgesub u : U \impsub V$.
    
    Si $`G \judgesub f : \Pi X.Y \impsub \Pi x : V.W$, alors 
    $`G \judgesub x : V \impsub X$ (lemme \ref{inversion-subtyping}). 
    Par transitivit� du sous-typage \ref{refl-trans-subtyping}, 
    $`G \judgesub U \impsub X$. On peut donc appliquer \AppRule{} pour
    obtenir $`G \typei f u : Y[u/x]$. Par la covariance du
    produit en son codomaine et la substitivit� du sous-typage,
    on a $`G \judgesub f u : Y[u/x] \impsub W[u/x]$, la propri�t� est
    donc bien v�rifi�e.
    
  \item[\SubsetRule:] On a
    \begin{prooftree}
      \Subset
    \end{prooftree}
    
    Par induction, $`E T, `G \typei x : T `^ `G \judgesub x : T \impsub U$.
    

  \item[\LetSubRule:]
    
  \item[\ConvRule:]

  \end{description}
  
\end{proof}


%%% Local Variables: 
%%% mode: latex
%%% TeX-master: "~/research/coq/papers/subset-typing"
%%% End: 

\subsection*{22 mars}
Continue les preuves...
Trouver les bons lemmes de substitution!

Enlever la condition $A atomique$ dans \SubRightRule. Ca donne une
strat�gie d�terministe mais n'est pas indispensable dans cette pr�sentation.

%%% Local Variables: 
%%% mode: latex
%%% TeX-master: "~/research/coq/papers/subset-typing"
%%% End: 

\subsection*{23 mars}
Lu \cite{Chen:POPL-2003} en d�tail ainsi que
\cite{DBLP:journals/tcs/LuoS99} ou Zhaohui laisse en suspens la question
de la pertinence d'avoir des r�gles pour les produits reliant 
$\Pi x: A. \matht{list}~ `N$ et $\Pi x : A. \{ \matht{list}~ `N `| \ldots \}$ dans notre syst�me.
Une partie de \cite{DBLP:conf/csl/Luo96}. Plus int�ressant est
peut-�tre l'article sur les combinaisons incoh�rentes de coercions pour
les types sommes \cite{DBLP:conf/types/LuoL03}. Les objectifs de
coh�rence et d'�limination de la transitivit� ne sont pas loin des
notres: on veut que 'computationellement' les coercions soit
inessentielles (au contraire on accepte tout dans la partie logique)
et avoir un sous-typage avec de bonnes propri�t�s.

Merci Jean-Christophe pour
Mechanical Metatheory for the masses - The {\sc PoplMark} Challenge !

%%% Local Variables: 
%%% mode: latex
%%% TeX-master: "~/research/coq/papers/subset-typing"
%%% End: 
        
\subsection*{24 mars}
J'ai enlev� les termes du jugement de sous-typage d�claratif qui n'en a
pas besoin.

Preuves d'�quivalence entre syst�me algorithmique et d�claratif.
Plusieurs probl�mes:
\begin{itemize}
\item Besoin d'annotations de typage au let, ou alors restriction du
  type on a pas le droit d'utiliser le sous-typage au let. Ou alors on
  peut r�ecrire la d�rivation $x : S \sub t : T$ en $x : S' \sub t : T$
  lorsque $S' \sub S$ (narrowing) mais de fa�on � ce que l'on ajoute pas
  d'utilisation de $\sub$.
\item Application d'une fonction dans un type subset... on ne peut pas
  r�ecrire la d�rivation: soit on ajoute une fonction de
  'd�compr�hension' qu'on applique � gauche ou on autorise le
  sous-typage complet avec une r�gle du style:
  \begin{prooftree}
    \QAX{App}
    {$`G \seq f : T$}
    {$`G \subtd T \sub \Pi x : V. W$}
    {$`G \seq u : V' $}
    {$`G \subtd V' \sub V$}
    {$`G \seq (f u) : W [ u / x ]$}
    {$$}
  \end{prooftree}

  Mais on en revient � trouver une fonction pour d�cider de la deuxi�me
  pr�misse. On doit donc reprendre la fonction $\mu_0$ de \PVS{}.

  D�finition de $\mu_0$:
  \begin{eqnarray*}
    \subset{x}{U}{P} & "=>" & \mu_0~U \\
    x                & "=>" & x
  \end{eqnarray*}
  
  et l'on obtient la r�gle:
  \begin{prooftree}
    \QAX{App}
    {$`G \subtd f : T$}
    {$\mu_0~T = \Pi x : V. W$}
    {$`G \seq u : V' $}
    {$`G \subtd V' \sub V$}
    {$`G \seq (f u) : W [ u / x ]$}
    {$$}
  \end{prooftree}
\item ce n'est qu'une preuve informelle ;)
\end{itemize}

Comment traiter la conversion ? Voir la th�se de Chen.


%%% Local Variables: 
%%% mode: latex
%%% TeX-master: "~/research/coq/papers/subset-typing"
%%% End: 
        
\subsection*{25 mars}
Question de la transitivit� de notre "sous-typage".
On peut avoir/demander une forme restreinte de transitivit� du genre: 
$`G \subta t : A \sub B "~>" t' `^ `G \subta t' : B \sub C "~>" t'' "=>"
`G \subta t : A \sub C "~>" t''$. Mais dans un sens puisque notre
sous-typage d�pend des termes, il est clair que l'on a
$`G \subta t : A \sub B "~>" t' `^ `G \subta x : B \sub C "~>" t'' "=>"
`G \subta t : A \sub C "~>" t''$ si $x `; `G$ mais pas plus. Il faut
r�flechir � quelle est la solution la plus utile/souhaitable.
Lecture de la 3�me partie de \cite{ChenPhD} sur le sous-typage dans
$\lambda CC_{\leq}$.

Lecture d'articles sur la syntaxe abstraite pour le challenge {\sc
  PoplMark}\ldots


%%% Local Variables: 
%%% mode: latex
%%% TeX-master: "~/research/coq/papers/subset-typing"
%%% End: 
        
\subsection*{30 mars}
Quelques questions:
\begin{itemize} 
\item Unification pour \SumRule{} probl�matique ? Impl�mentation dans
  \Coq.
\item M�canisme d'annotations de type � ajouter ? Seulement aux lets ou
  partout ?
\end{itemize}

%%% Local Variables: 
%%% mode: latex
%%% TeX-master: "~/research/coq/papers/subset-typing"
%%% End: 
        
\subsection*{31 mars}
Corrections d'hier...
Inf�rence des sommes, plus de contextes au sous-typage.
V�rification des sortes. Quelques explications sur la conversion.
Probl�me d'enlever les termes pour \SubSigmaRule{}, on perd les
d�pendances.

D�finition de $\muterm$, exemple:
$\muterm~(\lambda x. x)~\subset{x}{\subset{y}{`N}{y > 0}}{x \neq 0} = (id `o \pi_1 `o \pi_1, `N)$

%%% Local Variables: 
%%% mode: latex
%%% TeX-master: "~/research/coq/papers/subset-typing"
%%% End: 
\subsection*{1er avril}
On enl�ve les contextes au sous-typage algorithmique d�finitivement.
Pour faciliter les preuves on consid�re que le sous-typage contient la
$\eqbi$ d�s le d�part.
Ecriture de la r�ecriture du sous-typage pour Coq. 
On fait du sous-typage sur les traductions et l'on devra montrer que $U
\suba V "=>" `E t, t : U' \subi V'$.
R�ecriture: $\subimpl{`G}{U}{V}{`G'}{t}{U'}{V'}$. On connait $`G, U, V,
`G', U', V'$ et l'on cherche la coercion $t$.
        
%%% Local Variables: 
%%% mode: latex
%%% TeX-master: "~/research/coq/papers/subset-typing"
%%% End: 
\subsection*{5 avril}
Correction r��criture du sous-typage.
Ajout des r�gles de formation de subset manquantes.
On change le jugement de typage de $\subset{x}{U}{P}$ pour $P$ ayant une
variable libre $x$.

Id�e de terme � trous instanci� ? (remplacerait $c u'$ dans la
conclusion de l'application). 

%%% Local Variables: 
%%% mode: latex
%%% TeX-master: "~/research/coq/papers/subset-typing"
%%% End: 
\subsection*{1\textsuperscript{er} Juin}
Fin du codage: il reste des probl�mes avec les existentielles mais �a
n'est pas ma partie.

Corrections du papier, �criture de la grammaire du langage.
Mon sujet de th�se est pr�t! Mon CV aussi.

%%% Local Variables: 
%%% mode: latex
%%% TeX-master: "~/research/coq/papers/subset-typing"
%%% End: 
\subsection*{6 Juin}
Fini la fiche SIREDO, fini le curriculum!



%%% Local Variables: 
%%% mode: latex
%%% TeX-master: "~/research/coq/papers/subset-typing"
%%% End: 
\subsection*{10 Juin}
Probl�mes du shadowing/masquage des variables dans l'env. (n�cessaire
pour renforcement...)

Ne devrait ont pas avoir \Set{} plut�t que \Type{} pour le type de la
donn�e des subsets ?? (utile pour inversion du produit).

Utilisation des formes normales pour les sortes (inversion du produit?).

Pour quoi ai-je prouv� l'inversion du produit ???

%%% Local Variables: 
%%% mode: latex
%%% TeX-master: "~/research/coq/papers/subset-typing"
%%% End: 
}{}

\bibliography{../bib/bib-joehurd,../bib/barras,../bib/pvs-bib,../bib/bcp,../bib/Luo,subset-typing,../bib/cparent/cparent}
\bibliographystyle{plain}

\renewcommand{\thefootnote}{}
\footnotetext{Ce rapport a �t� pr�par� sous \LaTeX~avec la fonte 
  \texttt{Computer Modern Bright}}

\newpage
\appendix
\appendix
\chapter{Exemples}
\begin{figure}[ht]
\begin{verbatim}
Definition div : forall a : nat, forall b : nat, 
  b <> 0 -> { q : nat & { r : nat | r < b /\ a = b * q + r } }.
Proof.
intros a ; pattern a ; apply lt_wf_rec ; intros. (* R�cursion *)
elim (lt_ge_dec n b). (* If then else *)
intros. (* Premi�re branche *)
(* Structure du terme *)
refine (existS _ 0 _) ; refine (exist _ n _) ; refine (conj _ _) ;
[ assumption | rewrite mult_0_r ; rewrite plus_0_l ; reflexivity ]. (* Preuve *)
(* Seconde branche *)
intros ; assert (n - b < n). (* Preuve pour l'appel *)
apply lt_minus ; [ apply (ge_le _ _ b0) | apply (nat_neq_0_gt_0 b H0) ].
induction (H (n - b) H1 b H0). (* Appel r�cursif *)
induction p ; induction p. (* Destruction du r�sultat *)
refine (existS _ (S x) _) ; refine (exist _ x0 _). (* Structure du terme *)
(* Preuve *)
split.
assumption.
pose (eq_plus_eq _ _ H3 b).
assert (n - b + b = n) ; try omega.
rewrite <- H4 ; rewrite e ; rewrite plus_comm ; rewrite plus_assoc.
replace (b + b * x) with (b * S x).
reflexivity.
rewrite mult_comm ; simpl ; pattern (x * b) ; rewrite mult_comm.
reflexivity.
Qed.
\end{verbatim}
  \caption{Script de preuve de la division euclidienne}
  \label{fig:euclid-script}
\end{figure}

\begin{figure}[ht]
\begin{verbatim}
(* Subtac ne g�re pas encore les notations de Coq *)
Definition neq (A : Type) (x y : A) : Prop := x <> y.
Definition div_prop (a b q r : nat) := a = (b * q) + r /\ r < b. 
Definition lt_ge_dec (x y : nat) : { x < y } + { x >= y }.
Proof.
  intros ; elim (le_lt_dec y x) ; intros ; auto with arith.
Defined.

Recursive program mydiv (a : nat) { well_founded lt a lt_wf } : 
  { b : nat | neq nat b O } ->
  [ q : nat ] { r : nat | div_prop a b q r } :=
  fun { y : nat | neq nat y O } =>
    if lt_ge_dec a y
    then (q := O, a : { r : nat | div_prop a y q r })
    else let (q', r) = mydiv (minus a y) y in 
        (q := S q', r : { r : nat | div_prop a y q r }).

(* Dans Coq, mydiv aura le type:
forall a : nat, forall b : { b : nat | b <> 0 },
{ q : nat & { r : nat | div_prop a (proj1_sig b) q r } } *)

(* Obligations de preuves engendr�es *)
(* Hypoth�ses communes: *)
a : nat
mydiv : (n : nat) n < a -> forall b : { b : nat | b <> 0 },
 { q : nat & { r : nat | div_prop n (proj1_sig b) q r } }
y : { b : nat | b <> 0 }

(* (q := 0, a ...)
[ H : a < proj1_sig y, |- div_prop a (proj1_sig y) 0 a]

(* Argument de r�cursion *)
[H : a >= proj1_sig y |- a - proj1_sig y < a]

(* (q := S q', r) *)
[ H : a >= proj1_sig y, q' : nat,
  r : { r : nat | div_prop (a - proj1_sig y) (proj1_sig y) q' r }
|- div_prop a (proj1_sig y) (S q')  (proj1_sig r)]
\end{verbatim}
  \caption{La division euclidienne avec \Subtac}
  \label{fig:euclid-subtac}
\end{figure}

\begin{figure}[ht]
\begin{verbatim}
Recursive program mydiv (a : nat) using lt proof lt_wf  :
  { b : nat | neq nat b O } -> [ q : nat ] { r : nat | div_prop a b q r } :=
  fun { b : nat | neq nat b O } =>
    if lt_ge_dec a b
      then (q := O, a : { r : nat | div_prop a b q r })
      else let (q', r) = mydiv (minus a b) b in
        (q := S q', r : { r : nat | div_prop a b q r }).
unfold neq ; simpl ; intros.
induction b ; simpl ; simpl in H ; omega.

unfold neq, div_prop ; simpl ; intros.
induction b ; induction r ; simpl ; simpl in H, p0 ; intuition.
rewrite mult_comm ; simpl.
rewrite mult_comm ; simpl.
omega.

unfold neq, div_prop ; simpl ; induction b ; simpl ; intros.
intuition.
Qed.
\end{verbatim}
  \caption{La division euclidienne avec \Subtac: script}
  \label{fig:euclid-subtac-script}
\end{figure}


%%% Local Variables: 
%%% mode: latex
%%% TeX-master: "subset-typing"
%%% LaTeX-command: "TEXINPUTS=\"..:style:$TEXINPUTS\" latex"
%%% End: 


\chapter{Liste des figures}
\listoffigures

\end{document}

%%% Local Variables: 
%%% mode: latex
%%% TeX-master: "subset-typing"
%%% LaTeX-command: "TEXINPUTS=\"style:$TEXINPUTS\" latex"
%%% End: 
