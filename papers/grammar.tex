
\section*{Grammaire du langage}

\begin{figure}[h]
  \begin{displaymath}
    \begin{array}{lcl}
      `a, `b & \Coloneqq & x \\
      & | & \funml{}~x~:~`t "=>" `a \\
      & | & `a~`b \\ 
      & | & (`a,~`b) \\
      & | & (x \coloneqq `a,~`b~: `a) \\
      & | & 0 \ldots n \\
      & | & \letml~x = `a ~\inml~`b \\
      & | & \letml~(x_1, \ldots, x_n) = `a ~\inml~ `b     
    \end{array}
  \end{displaymath}
  \caption{Termes}
  \label{fig:gram-terms}
  
\end{figure}

\begin{figure}[h]
  \begin{displaymath}
    \begin{array}{lcl}
      `t, `s & \Coloneqq & x \\
      & | & \verb|Inductive| \\
      & | & `t~`s \\
      & | & `t~`a \\
      & | & \Pi x : `t. `s \\
      & | & \Sigma x : `t. `s \\
      & | & `t * `s \\
      & | & \subset{x}{`t}{`s} \\
      & & \\
      \verb|Inductive| & \Coloneqq & ident
    \end{array}
  \end{displaymath}
  \caption{Types}
  \label{fig:gram-types}
\end{figure}

On peut mettre des termes dans les types seulement lors de l'application
d'un type (bonne mod�lisation du ``positionement interne'' des termes?). 

%%% Local Variables: 
%%% mode: latex
%%% TeX-master: "subset-typing"
%%% End: 
