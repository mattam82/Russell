\def\GD{`G, x : V, `D}
\def\Gr{`G, x : X, `D[e~x/x]}

\begin{lemma}[Coercion de sortes]
  \label{subti-coercion-sorts}
  Si $`G \typec e : s \subi T$ ou $`G \typec e : T \subi s$ alors $T
  \eqbi s$ et $e \eqbi \lambda x : s.x$.
\end{lemma}
\begin{proof}  
  Clairement on ne peut pas d�river $s \subihnf \hnf{T}$ quelque soit
  $T$ v�rifiant $`G \typec T : s'$. En effet seule la r�gle
  \irule{SubProof} pourrait s'appliquer, mais cela impliquerait que
  $\hnf{T} = \mysubset{x}{U}{P}$ avec $s \subi U$ et ainsi de suite. La
  seule possibilit� est de d�river $s \eqbi T$ ou $s \eqbi U$, auquel cas $U$ est une
  sorte ce qui contredit le fait que $\mysubset{x}{U}{P} : s$ dans le
  cas pr�c�dent. On a d�rive donc $s \subi T$ si et seulement si $s
  \eqbi T$.
\end{proof}

\begin{lemma}[Substitutivit� de la coercion]
  \label{subti-coercion-subst}
  Si $`G \typec X, V, T, U : s$, $`G \typec e : X \subi V$ et
  $`G, x : V, `D \typec c : T \subi U$ alors
  $`G, x : X, `D[e~x/x] \typec c[e~x/x] : T[e~x/x] \subi U[e~x/x]$.   
\end{lemma}

\begin{proof}
  Soit $T' = T[e~x/x]$, $U' = U[e~x/x]$.

  \begin{induction}[subtyping-algo]
  \case{SubHnf} 
  On a: 
  \begin{prooftree}
    \AXC{$\subimplhnf{\GD}{c}{\hnf{T}}{\hnf{U}}$}
    \UIC{$\subimpl{\GD}{c}{T}{U}$}
  \end{prooftree}
  
  On a plusieurs possibilit�s:
  \begin{itemize}
  \item Si $\hnf{T} = x$. Alors la seule r�gle appliquable est
    \irule{SubProof}, on a donc:
    \begin{prooftree}
      \AXC{$\subimpl{\GD}{c'}{x}{W}$}
      \UIC{$\subimplhnf{\GD}{c = (\lambda t : x.\ldots(c'~t))}{\hnf{T} = x}{\hnf{U} = \mysubset{y}{W}{P}}$}
    \end{prooftree}
    
    Par induction: \[\subimpl{\Gr}{c'[e~x/x]}{e~x}{W[e~x/x]}\]
    Or on peut appliquer \irule{SubHnf} et \irule{SubProof} 
    pour d�river le jugement $\subimpl{\GD}{?}{T'}{U'}$:

    \begin{prooftree}
      \AXC{$\subimpl{\Gr}{c'[e~x/x]}{e~x}{W[e~x/x]}$}
      \UIC{$\subimplhnf{\GD}{\lambda t :
          x[e~x/x].\ldots(c'~t))}{\hnf{T'} = e~x}{\hnf{U'} = \mysubset{y}{W[e~x/x]}{P[e~x/x]}}$}
      \UIC{$\subimpl{\Gr}{?}{T'}{U'}$}
    \end{prooftree}
    
    La coercion inf�r�e est bien $c[e~x/x]$.
    On ne peut appliquer d'autre r�gle que \irule{SubProof}, en effet,
    comme $T : s$ on a $\hnf{T} = x : s$ et donc $X \eqbi s$. Par le
    lemme \ref{subti-coercion-sorts} on en d�duit que $\hnf{T'} = x$
    donc aucune autre r�gle ne peut s'appliquer.
    
  \item Si $\hnf{U} = x$ alors la seule r�gle applicable est
    \irule{SubSub} et l'on a un r�sultat similaire au cas pr�c�dent:
    
    \begin{prooftree}
      \AXC{$\subimpl{\Gr}{c'[e~x/x]}{W[e~x/x]}{x}$}
      \UIC{$\subimplhnf{\GD}{c'[e~x/x] `o \pi_1}{\hnf{T'} =
          \hnf{T}[e~x/x] = \mysubset{y}{W[e~x/x]}{P[e~x/x]}}{\hnf{U'}}$}
      \UIC{$\subimpl{\Gr}{?}{T'}{U'}$}
    \end{prooftree}
   
  \item Sinon, $\hnf{T} "/=" x$ et $\hnf{U} "/=" x$ donc $\hnf{T'} =
    \hnf{T}[e~x/x]$ et $\hnf{U'} = \hnf{U}[e~x/x]$.

    Dans ce cas on va <<rejouer>> la d�rivation $\subimplhnf{\GD}{c}{\hnf{T}}{\hnf{U}}$.
    
    Par cas sur cette d�rivation:
    \begin{induction}
      \case{SubProd}\quad
      On a:
      \begin{prooftree}
        \BAX{}
        {$\subimpl{\GD}{c_1}{C}{A}$}
        {$\subimpl{\GD, y : C}{c_2}{B[c_1~y/y]}{D}$}
        {$\subimplhnf{\GD}{\lambda f : \Pi x : A.B.~\lambda x :
            C.~c_2~(f~(c_1~x))}
          {\Pi x : A.B}{\Pi x : C.D}$}
        {}
      \end{prooftree}

      Par induction: 
      \[\subimpl{\Gr}{c_1[e~x/x]}{C[e~x/x]}{A[e~x/x]}\]
      \[\subimpl{\Gr, y : C[e~x/x]}{c_2[e~x/x]}{B[c_1~y/y][e~x/x] = B[e~x/x][c_1[e~x/x]/y]}{D[e~x/x]}\]
      
      On en d�duit par \irule{SubProd}:
      \[\subimplhnf{\Gr}{c[e~x/x]}{(\Pi y : A.B)[e~x/x]}{(\Pi y : C.D)[e~x/x]}\]
      On peut v�rifier:
      $c[e~x/x] : \hnf{T'} \subihnf \hnf{U'} = \lambda f : (\Pi x : A.B)[e~x/x].~\lambda x :
      C[e~x/x].~c_2[e~x/x]~(f~(c_1[e~x/x]~x))$.
      
      \case{SubSigma} idem.
      
      \case{SubSub}\quad
      \begin{prooftree}
        \AXC{$\subimpl{\GD}{c}{T}{\hnf{U}}$}
        \UIC{$\subimplhnf{\GD}{\lambda t : \mysubset{y}{T}{P}.~c~(\pi_1~t)}
          {\mysubset{y}{T}{P}}{\hnf{U}}$}       
      \end{prooftree}
      
      donne:
      \begin{prooftree}
        \AXC{$\subimpl{\Gr}{c[e~x/x]}{T[e~x/x]}{\hnf{U}[e~x/x]}$}
        \UIC{$\subimplhnf{\Gr}{\lambda t : (\mysubset{y}{T}{P})[e~x/x].~c[e~x/x]~(\pi_1~t)}
          {\mysubset{y}{T[e~x/x]}{P[e~x/x]}}{\hnf{U}[e~x/x]}$}
      \end{prooftree}
      
      \case{SubProof} idem.
    \end{induction}
  \end{itemize}

  \case{SubConv}
  Dans ce cas, c'est direct par pr�servation de la $\beta$-�quivalence par
  substitution. Pour montrer que la nouvelle coercion est �gale �
  $c[e~x/x]$ il faut raisonner par cas sur la forme de $T$ et $U$. 
  \begin{itemize} 
  \item Si la forme de $T$ (respectivement $U$) est �gale � $x$ 
    alors $\hnf{U} = x$ (reps. $\hnf{T} = x$). Or $\GD \typec T,
    U : s$, donc $\GD \typec x : s$, soit $X \eqbi s$. Par le lemme
    \ref{subti-coercion-sorts}, on d�duit que $e = \lambda x : s.x$.
    On a donc: $\hnf{T'} = \hnf{U'} = \hnf{e~x} = x$. Or $x \subhnf x$
    n'est pas d�rivable donc on a la d�rivation:
    \begin{prooftree}
      \AXC{$T' \eqbi U'$}
      \UIC{$\subimpl{\Gr}{\lambda y : T'.y}{T'}{U'}$}
    \end{prooftree}
    
    On a bien $c[e~x/x] = \lambda y : T[e~x/x].y$.
  \item Sinon, on a $\hnf{T'} = \hnf{T}[e~x/x]$ et $\hnf{U'} =
    \hnf{U}[e~x/x]$. Or on sait que $\hnf{T} \subhnf \hnf{U}$ n'est pas
    d�rivable, donc la subsitution ne peut pas l'�tre non plus (on
    substitue sous les symboles de t�te dans ce cas). Cependant, on a toujours $T'
    \eqbi U'$ donc on va d�river la coercion attendue.
    
  \end{itemize}
  


\end{induction}
\end{proof}


%%% Local Variables: 
%%% mode: latex
%%% TeX-master: "subset-typing"
%%% LaTeX-command: "TEXINPUTS=\"style:$TEXINPUTS\" latex"
%%% End: 
