\documentclass[preprint,10pt]{sigplanconf}
\usepackage{xspace}
\usepackage{amsmath}
\usepackage{abbrevs}

% \def\Program{{\sc Program}\xspace}
% \def\Coq{{\sc Coq}\xspace}
% \def\Russell{{\sc Russell}\xspace}

\begin{document}

\conferenceinfo{ICFP'07}{1-3 October 2007, Freiburg, Germany.} 
\copyrightyear{2007}
%\copyrightdata{[to be supplied]} 

\titlebanner{Draft}        % These are ignored unless
\preprintfooter{Description of the Program extension of Coq}   % 'preprint' option specified.

\title{\Program{}ming in \Coq}
%\subtitle{}

\authorinfo{Matthieu Sozeau}
{Universit\'e Paris Sud, CNRS, Laboratoire LRI, UMR 8623, Orsay, F-91405\\
INRIA Futurs, ProVal, Parc Orsay Universit\'e, F-91893}
{sozeau@lri.fr}

\maketitle

\begin{abstract}
  \Program is an environment for programming with dependent types in the
  \Coq proof assistant. It provides commands for building programs in a
  high-level language \Russell and proving their properties using \Coq.
  Its essential feature is to permit writing
  apparently partial functionnal programs and providing support for proving their
  totality. This is accomplished by an ingenuous handling of subset
  types in \Russell. 
  This paper gives a walk-through the definition of programs using this
  method and shows how it solves the problem of partiality
  satisfactorily for a number of its instances.
\end{abstract}

%\category{CR-number}{subcategory}{third-level}

% \terms
% term1, term2

\keywords
Functionnal Programming, Coq, partiality, subsets, coercions

\section{Introduction}
Writing programs with rich specifications in Type Theory is a nightmare.
This stems from the fact that programs need to carry their own
correctness certificates if given non-trivial types. For example, the
function of type $\Pi x : X.\{ x' : X | x = x'\}$, which
computationally behaves like the identity must be written $\lambda x :
X. (x, refl x)$. When dealing with such simple specifications it may be
acceptable to write the proof terms directly, but it soon becomes
intractable. Basically, you don't want to manipulate proof terms by hand
in you program, even though you assign it a complex type. Another
problem in regular programming is partiality in the definition domain of
a function, e.g. for head and tail. You may not try to apply these
functions to empty lists, otherwise you get an error. Using rich
specifications you can force the argument to be non-nil but then you
don't want to manipulate the proof in the definition or at the call
sites. The \Program method handles this problem by generalizing the use of
subset types and letting the \Russell type-checker generate proof
obligations where proof manipulation is inevitable. After obligations
are proved the system builds a legitimate \Coq term.

The paper is organized as follows: in the next section we introduce the
\Program environment, its foundations and technical underpinnings. Then
in section \ref{sec:Practice} we explore how it handles the current (mal)practices of functionnal
programming from three angles: partial functions and rich specifications
in subsection \ref{sec:Practice:PartialDomains}, pattern matching
and rich datatypes in subsection \ref{sec:Practice:Data} and finally
termination in subsection \ref{sec:Practice:Computations}. Next we will discuss issues of the 
implementation in section \ref{sec:Implementation} and finally
conclude in section \ref{sec:Conclusion}.

\section{What is \Program ?}
\label{sec:Program}
\subsection{From \PVS to \Russell}
\label{sec:Program:PVS-Russell}
Subset typing, PVS success, metatheory of russell

\subsection{From \Russell to \Coq}
\label{sec:Program:Russell-Coq}
Elaboration, correctness proof, obligations system, extraction.

\section{(Mal-)Practice}
\label{sec:Practice}
Some implicit reasoning in fun prog to formalize.
``Total functional programming''
Bridging the gap between fun prog and proof assistants.
Use examples developped along with Program.

\subsection{Domain partiality \& rich specifications}
\label{sec:Practice:PartialDomains}
Restrict the domain, forget about the proof, do not lose it.
Simple structure, complex type.
head, tail, tip

\subsection{Pattern matching  \& rich datatypes}
\label{sec:Practice:Data}
Prove exhaustiveness, under eqs and ineqs.
Dependent pattern matching to equational reasonning.
nth, fingertree, ...

\subsection{Termination \& rich computations}
\label{sec:Practice:Computations}
Use measures and well founded orders, pb when reasonning, computing.
div, hnf?, reduce

\section{Implementation issues}
\label{sec:Implementation}
in \Coq, no eta/proof irrelevance. 


\subsection{Proof Irrelevance}

\section{Conclusion}
\label{sec:Conclusion}

% \appendix
% \section{Appendix Title}

% This is the text of the appendix, if you need one.

\acks

Acknowledgments, if needed.

\begin{thebibliography}{}

\bibitem{smith02}
Smith, P. Q. reference text

\end{thebibliography}

\end{document}
1-59593-090-6/05/0007

%%% Local Variables: 
%%% mode: latex
%%% TeX-master: t
%%% LaTeX-command: "BIBINPUTS=\"bib\" TEXINPUTS=\"~/research/publication/styles:style:.:\" latex"
%%% End: 
