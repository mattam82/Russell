\chapter{Introduction}
Nous nous pla\c cons dans le cadre du syst\`eme d'aide \`a la preuve \Coq{},
auquel nous souhaitons int\'egrer un langage de programmation plus souple
que le langage actuellement utilis\'e.

\section{Pr\'esentation de \Coq}

\Coq~est un assistant de preuve dont la premi\`ere version date de 1985,
et qui est aujourd'hui d\'evelopp\'e dans le projet \PCRI{} \LogiCal{} (INRIA, LIX,
LRI, CNRS). Originellement bas\'e sur le Calcul des Constructions (\CoC),
il a \'et\'e \'etendu au \CCIfull~(\CCI) et contient aujourd'hui de
nombreuses am\'eliorations telles qu'un syst\`eme sophistiqu\'e d'extraction
de programmes ou encore des proc\'edures de d\'ecision pour automatiser la
preuve.

Le d\'eveloppement de \Coq~est intimement li\'e \`a l'isomorphisme de {\sc
Curry-Howard} qui montre le lien entre logique intuitionniste et calcul. Cet
isomorphisme \'etablit qu'\'elaborer une preuve du calcul propositionnel
intuitionniste est \'equivalent \`a \'ecrire un terme du
\lc~simplement typ\'e (\lcst). Par exemple, montrer que $A "=>" A$ pour un
certain $A$ revient \`a \'ecrire la fonction identit\'e $\lambda x : A. x$ qui
a bien pour type $A "->" A$. Chaque logique constructive est donc
associ\'ee \`a un \lc{} particulier. Dans \Coq{}, on utilise cet
isomorphisme pour v\'erifier les preuves. Le noyau est simplement un
typeur pour \CCI{}. Si on peut typer un terme $t$ de type $T$, alors on est
assur\'e d'avoir trouv\'e une preuve constructive $t$ de la formule $T$.
Cette dualit\'e se refl\`ete aussi \`a l'utilisation de \Coq{} o\`u l'on a 
les deux visions: logique (d\'eveloppement math\'ematique, preuve) et 
calcul (d\'eveloppement informatique, programme).

\subsection{Preuve}
\Coq{} est utilis\'e le plus souvent pour \'elaborer des th\'eories
math\'ematiques prouv\'ees m\'ecaniquement. Dans cette optique, l'utilisateur 
mod\'elise un probl\`eme par des structures math\'ematiques et veut prouver
certaines propri\'et\'es sur ce mod\`ele (par exemple la preuve du th\'eor\`eme
des quatres couleurs r\'ecemment termin\'ee \cite{Gonthier4col} utilisait 
des r\'esultats de g\'eom\'etrie alg\'ebrique).

Pour prouver un but sous certaines hypoth\`eses, on utilise des
tactiques qui simulent un raisonnement d\'eductif pour l'utilisateur.
Celles-ci permettent par exemple d'introduire une hypoth\`ese: pour le but
$A "=>" A$ on peut introduire l'hypoth\`ese $H : A$ pour obtenir le but
$A$ ; ou bien d'en appliquer une (ou tout autre r\'esultat d\'ej\`a \'etabli): 
en appliquant l'hypoth\`ese $H$ on prouve le but directement. 
Ces tactiques peuvent \^etre d'une complexit\'e arbitraire (r\'e\'ecritures,
proc\'edures de d\'ecision pour l'arithm\'etique, etc \ldots).

Les tactiques utilis\'ees pour cr\'eer des preuves ne sont en fait
qu'une interface au-dessus du noyau de \Coq{} qui se r\'eduit \`a un typeur
pour \CCI. A la fin d'une preuve, on a en effet construit un terme 
($\lambda x : A. x$ dans notre exemple) que l'on va soumettre au typeur
dont le but est de v\'erifier qu'il est bien de type $A "->" A$. Les
tactiques peuvent cependant �tre arbitrairement complexes (r�solution
d'�quations de l'arithm�tique, r�ecritures, etc...).

\subsection{Programmes}
D'un point de vue preuve de programmes, on a donc un environnement qui
permet de v\'erifier qu'un programme (un terme du calcul) v\'erifie une
certaine sp\'ecification (son type). Les types d\'ependants permettent de
sp\'ecifier fortement les termes. Par exemple, la fonction $\sdef{div} :
\nat "->" \nat "->" \nat$ de \ML{} est plus fortement sp\'ecifi\'ee en 
\Coq{} par $\sdef{div} : \nat "->" \subset{x}{\nat}{x \neq 0} "->" \nat$.
Seulement, on ne peut pas \'ecrire simplement un programme \ML{} et
donner sa sp\'ecification forte. Comme on a enrichi les types, on doit 
aussi enrichir les termes avec des termes de preuve, inutiles au 
calcul mais n\'ecessaires pour garantir la
correction logique du programme et le fait que la machine puisse
v\'erifier m\'ecaniquement la correction (annotations,...). Par exemple,
 si l'on veut appeler $\sref{div}$ sur $1$ et $n$ (pour $n : \nat$), il
faut construire un terme $\sref{div}~1~(\sref{elt}~(\lambda x : \nat
"->" x \neq 0)~x~p)$ o\`u $p$ est une preuve de $n \neq 0$. La
fonctionnelle est n\'ecessaire pour la d\'ecidabilit\'e du syst\`eme (on ne peut
pas inf\'erer le type de $P$ \`a partir de $p$ puisqu'il est de type
$P[n/x]$), et la preuve $p$ pour la correction logique.

A l'inverse, on peut extraire un programme de toute preuve en \'eliminant les
parties logiques et en ne conservant que la partie calculatoire d'un terme.

\section{Motivation}

\Coq{} permet de d\'evelopper des programmes complexes,
de leur donner des sp\'ecifications fortes et de les v\'erifier
automatiquement. On peut m\^eme extraire de ces d\'eveloppements des
programmes corrects par construction. Il y a cependant certaines
difficult\'es \`a d\'evelopper en \Coq{} que nous allons \'etudier maintenant.

\subsection{Un langage trop expressif?} 

Le langage de \CCI{} permet de bien sp\'ecifier des fonctions non
triviales, par exemple, si l'on d\'efinit une fonction de division
euclidienne en \ML{} elle aurait le type: $  \valml~\sdef{div} : \nat "->" \nat "->"
\nat * \nat$. En \Coq, on peut d\'efinir:
\[\Definition~\sdef{div} : \forall a : \nat, \forall b : \nat,
b \neq 0 "->" \{~q : \nat~\&~\{~r : \nat `| r < b `^ a = b * q + r~\} \}\]

Les types d\'ependants permettent de bien relier les entr\'ees aux sorties
et donc de sp\'ecifier les programmes aussi fortement que l'on d\'esire, 
mais aussi de fa\c con concise. En revanche, le terme de preuve 
correspondant \`a \sref{div} est nettement plus long (de l'ordre de 60 lignes), 
et ne peut simplement pas \^etre \'ecrit d'une traite 
sans une expertise approfondie. Pour rem\'edier \`a ce
probl\`eme, on utilise des tactiques qui permettent d'\'ecrire la preuve/programme
incr\'ementalement (\note{figure ici?}voir figure \ref{fig:euclid-script}
page \pageref{fig:euclid-script}). L'inconv\'enient de cette
m\'ethode est que l'on n'obtient pas toujours le programme d\'esir\'e
au d\'epart, puisque les tactiques cachent profond\'ement leur effet sur le
terme de preuve. Certaines techniques de r\'e\'ecriture peuvent aussi
g\'en\'erer des termes de complexit\'e algorithmique bien moins optimale que
ce que l'utilisateur \'ecrirait. Cependant ce mode de fonctionnement est
utile et utilis\'e par la majorit\'e des utilisateurs de \Coq{} avec succ\`es
(certification d'un compilateur \C, th\'eor\`eme des quatres couleurs
\cite{Gonthier4col}, \ldots).

\subsection{M\'elange logique et calcul}
Une difficult\'e essentielle lorsque l'on veut permettre \`a des
utilisateurs non experts de d\'evelopper dans \Coq{} est le ``m\'elange
des genres'' permanent entre logique et calcul. Pour appliquer une
fonction division qui attend un d\'enominateur diff\'erent de $0$ par
exemple, il faut passer \`a la fois l'argument lui m\^eme, mais aussi une
preuve de sa non-nullit\'e. Lorsque l'on a l'habitude de programmer, \c ca
n'est pas la chose la plus naturelle et l'on aimerait pouvoir d\'ecoupler
les parties codage et preuve pour simplement diviser le probl\`eme. Les
parties logiques pourront souvent \^etre r\'esolues automatiquement par des
tactiques.

\subsection{Objectif}
A long terme, on souhaite permettre \`a un utilisateur de programmer dans
un langage proche de \ML{} et de prouver ses programmes dans un deuxi\`eme
temps \`a l'aide de \Coq{} et ses tactiques. Une fois les preuves
termin\'ees, on peut extraire un programme correct par construction et
essentiellement \'equivalent \`a celui de d\'epart ou le r\'eutiliser facilement
dans l'environnement \Coq{}.

\section{Travaux Connexes}

La preuve de programmes fonctionnels est un domaine de recherche
actif. L'id\'ee d'\'etendre les langages \ML{} avec des types d\'ependants a
\'et\'e d\'evelopp\'ee dans \DML{} \cite{XiPfenning1999DTP}, \Cayenne{}
\cite{Augustsson99} et \Omegapdx{} \cite{Omega}. Il
s'agit dans ces travaux de faire un langage dont l'inf\'erence est d\'ecidable, donc
de restreindre les types d\'ependants \`a des domaines o\`u l'on peut faire de
la preuve automatique (\DML{}) ou bien d'affaiblir la puissance du langage
pour rendre l'utilisation des types d\'ependants plus ais\'ee (\Cayenne{}
n'a pas la r\'ecursivit\'e g\'en\'erale par exemple). Nous prenons le
contre-pied de ces travaux en acceptant de g\'en\'erer des obligations de
preuve et en essayant de trouver un langage le plus proche de \ML{}
possible tout en retenant la puissance de \Coq{} et des types
d\'ependants. Nous pr\'esentons maintenant des travaux directement li\'es \`a
notre contribution.


\subsection{La tactique \Program}
Il existe un travail r\'ealis\'e dans \Coq{} couvrant une partie de nos objectifs.
D\'evelopp\'ee par Catherine Parent \cite{conf/mpc/Parent95}, 
la tactique \Program{} permettait de synth\'etiser des preuves \`a partir de
programmes. L'id\'ee \'etait de trouver un langage de programmation
suffisamment restrictif pour r\'ealiser une inversion de l'extraction, 
c'est-\`a-dire, \`a partir d'un terme essentiellement calculatoire
(des annotations \'etaient n\'ecessaires), retrouver un terme de preuve
r\'ealisant la sp\'ecification donn\'ee. 
A partir de l\`a, on \'etait assur\'e que le programme extrait serait
identique \`a celui que l'on \'ecrivait pour sa partie informative. Cette
m\'ethode g\'en\'erale avait l'inconv\'enient d'\^etre peu intuitive et de ne pas
s'int\'egrer \`a l'environnement \Coq. Li\'ee \`a l'extraction interne qui
a disparu dans les derni\`eres versions de \Coq{} (remplac\'ee par la
contribution de Pierre Letouzey \cite{LetouzeyPhD}), elle n'est plus 
maintenue aujourd'hui.

\subsection{Types sous-ensemble}
Plut\^ot que de continuer dans la m\^eme direction, nous avons 
cherch\'e \`a assouplir le syst\`eme. L'assistant de preuve \PVS{}
\cite{PVS-Semantics:TR} aux capacit\'es similaires \`a \Coq{}, int\`egre un
m\'ecanisme d\'enomm\'e \ps{} que nous allons pr\'esenter maintenant.
Tout d'abord une d\'efinition:
\begin{definition}[Type sous-ensemble]
  $\subset{x}{T}{P}$ est le type des termes de type $T$ v\'erifiant la
  propri\'et\'e $P$.
\end{definition}

Les types sous-ensembles sont d'une grande utilit\'e pour la sp\'ecification
de programmes, par exemple pour les pr\'e-conditions:
$\Definition~\sref{div} : \nat "->" \subset{x}{nat}{x \neq 0} "->" \nat
`* \nat$.

L'id\'ee du \ps{} impl\'ement\'e dans \PVS{}
\cite{Shankar&Owre:WADT99,Rushby98:TSE} est de consid\'erer tout objet de
type $T$ comme un objet de type $\subset{x}{T}{P}$ pour $P$ vraie et
vice-versa. Comme tout objet $t$ de type $T$ ne v\'erifie pas forc\'ement la
propri\'et\'e $P$, on g\'en\`ere des ``\emph{Type-checking conditions}'' (\TCC), c'est
\`a dire que l'on demande \`a l'utilisateur de prouver $P[t/x]$ pour assurer
que le programme est correct.

\subsection{Coercions}
\PVS{} n'a pas la m\^eme architecture que \Coq{}, en particulier il n'y a
pas de termes de preuve et de noyau pour v\'erifier ces termes. Il faut
donc faire confiance \`a la quasi-totalit\'e du code pour croire en la
correction des programmes v\'erifi\'es. Le crit\`ere de {\sc De Bruijn},
qui dit en substance qu'un petit noyau (et celui de \Coq~a m\^eme \'et\'e
formellement v\'erifi\'e \cite{Barras96a}) est plus s\^ur n'est pas respect\'e.
 
Dans notre cas, il faut g\'en\'erer des termes de preuve et donc le code 
correspondant \`a ce ``sous-typage''. Une litt\'erature importante
\cite{conf/popl/Chen03,conf/csl/Luo96} existe
autour des syst\`emes \`a coercions explicites dont nous nous sommes
inspir\'es pour r\'ealiser la g\'en\'eration des termes. Dans un syst\`eme \`a
coercions explicites, on peut faire des abus de notations tels que
utiliser un objet de type $T$ \`a la place d'un de type $U$, mais
on applique une coercion qui am\`ene l'objet vers le type $U$ avant 
de retyper dans un syst\`eme sans coercions. G\'en\'eralement les coercions
sont tr\`es similaires \`a des identit\'es, c'est-\`a-dire qu'elles sont calculatoirement
insignifiantes mais leur utilisation facilite le d\'eveloppement. Dans
\Coq{} par exemple le syst\`eme de coercions \cite{saibi97inheritance} a
permis de d\'evelopper des th\'eories alg\'ebriques r\'eutilisables sur
plusieurs structures instantan\'ement (un th\'eor\`eme sur les corps pouvant s'appliquer aux
anneaux).

Les extensions du Calcul des Constructions avec des notions de
sous-typage comme $\lambda C_\leq$ de Chen ne sont cependant pas
dans la m\^eme cat\'egorie que notre travail. En particulier, nous ne
nous int\'eressons pas aux propri\'et\'es de normalisation, pr\'eservation du
typage ou encore au fait d'avoir des sous-types minimaux dans notre
syst\`eme. On peut le voir plut\^ot comme un syst\`eme syntaxique intelligent
au-dessus du Calcul des Constructions.

%%% Local Variables: 
%%% mode: latex
%%% TeX-master: "subset-typing"
%%% LaTeX-command: "TEXINPUTS=\"style:$TEXINPUTS\" latex"
%%% End: 
