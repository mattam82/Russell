\def\iG{\ipG{`G}}

\begin{lemma}[Substitutivit� de l'interpr�tation]
  Si:
  \begin{itemize}
  \item $\iG \typec \ipt{u}{\iG} : \ipt{U}{\iG}$, 
  \item $\ipG{`G, x : V} \typec \ipt{W}{\ipG{`G, x : V}} : s$ 
  \item $c : \ipt{U}{\iG} \subi \ipt{V}{\iG}$ 
  \end{itemize}
  
  alors $\ipt{W}{\ipG{`G, x : V}}[c~u/x] = \ipt{W[u/x]}{\iG}$.
\end{lemma}
\begin{proof}
  Par induction structurelle sur $W$.
  
  \begin{induction}
  \item[$W `= y$]
    Si $y = x$, alors on doit montrer:
    $c~u = \ipt{u}{\iG}$
    


\end{proof}  



\setboolean{displayLabels}{false}
\begin{theorem}[Correction de l'interpr�tation]
  \label{correct-interp}
  Si $`G \typea t : T$ alors on a $\iG \typec \ipt{t}{\iG} :
  \ipt{T}{\iG}$ avec $\ipT{t}{`G} = \ipt{T}{\iG}$.
  Si $\wf `G$ alors $\wf \iG$.
\end{theorem}


\begin{proof}
  Par induction mutuelle sur la d�rivation de typage ou de bonne
  formation.

  \begin{induction}
    \case{WfEmpty} Trivial.

    \case{WfVar}
    Par induction $\iG \typec \ipt{A}{\iG} :
    \ipt{s}{\iG}$. Par inversion du jugement de bonne formation
    dans \CCI{}, $\typewf \iG$.
    Or, $\ipt{s}{\iG} = s$ ($s `: \{ \Prop, \Set, \Type \}$), donc 
    on peut appliquer \irule{WfVar} dans \CCI{} pour obtenir:
    $\wf \iG, x : \ipt{A}{\iG}$, soit $\wf \ipG{`G, x : A}$.

    \case{PropSet}
    Direct par induction, $\iG \typec \ipt{s}{\iG} = s :
    \ipt{\Type}{\iG} = \Type$. La deuxi�me condition est directe
    par d�finition de l'interpr�tation.
    
    \case{Var} On a:
    \begin{prooftree}
      \Var
    \end{prooftree}
    Par induction, $\wf \iG$ et par simple inspection de la
    d�finition de l'interpr�tation des contextes, si $x : A `: `G$ alors
    $x : \ipt{A}{`D} `: \iG$ pour $`D ``( `G$. Par affaiblissement
    dans \CCI{}, on obtient ais�ment $\iG \typec x : \ipt{A}{`D}$ 
    � partir de  $\ipG{`D} \typec x : \ipt{A}{\ipG{`D}}$. Or il est clair par
    la d�finition de l'interpr�tation que $\ipt{A}{`D} = \ipt{A}{`G}$  
    si $`G$ est une extension de $`D$ puisqu'on utilise l'environnement
    uniquement au moment de typer les variables et
    $\iG(x) = \ipG{`D}(x)$ pour tout $x$ utilis� dans $A$.
    On a donc $\iG \typec x : \ipt{A}{\iG}$.
    La deuxi�me condition est montr�e par $\ipT{x}{`G} = \iG(x) = 
    \ipt{A}{\ipG{`D}} = \ipt{A}{\iG}$.
      
    \case{Prod} On a:
    \begin{prooftree}
      \Prod
    \end{prooftree}
    
    Par induction $\iG \typec \ipt{T}{\iG} : \ipt{s_1}{\iG}
    = s_1$ et $\ipG{`G, x : T} \typec \ipt{U}{\ipG{`G, x : T}} :
    \ipt{s_2}{\ipG{`G, x : T}} = s_2$.
    On d�plie l'interpr�tation pour obtenir:
    $\iG, x : \ipt{T}{\iG} \typec \ipt{U}{\iG{}, x : \ipt{T}{\iG}} :
    s_2$.
    
    Par \irule{Prod} dans \CCI, on obtient:
    $\iG{} \typec \Pi x : \ipt{T}{\iG}. \ipt{U}{\iG{}, x : \ipt{T}{\iG}}
    : s_3$.
    Or $\ipt{\Pi x : T.U}{\iG} = \Pi x : \ipt{T}{\iG}. \ipt{U}{\iG{}, x
      : \ipt{T}{\iG}}$, donc on a bien:
    $\iG \typec \ipt{\Pi x : T.U}{\iG} : s_3 = \ipt{s_3}{\iG}$.
    La seconde condition est directe d'apr�s la d�finition de $\mathcal{R}$ qui
    est une fonction.

    \case{Abs} On a:
    \begin{prooftree}
      \Abs
    \end{prooftree}
    
    Par induction $\iG \typec \ipt{\Pi x : T.U}{\iG} : \ipt{s}{\iG}$
    et $\ipG{`G, x : T} \typec \ipt{M}{\ipG{`G, x : T}} :
    \ipt{U}{\ipG{`G, x : T}}$.
    On d�plie l'interpr�tation pour obtenir:
    $\iG, x : \ipt{T}{\iG} \typec \ipt{M}{\iG{}, x : \ipt{T}{\iG}} :
    \ipt{U}{\ipG{`G, x : T}}$.
    
    Par \irule{Abs} dans \CCI, on obtient:
    $\iG{} \typec \lambda x : \ipt{T}{\iG}. \ipt{M}{\iG{}, x : \ipt{T}{\iG}}
    : \ipt{\Pi x : T.U}{\iG}$.
    C'est �quivalent �:
    $\iG \typec \ipt{\lambda x : T.U}{\iG} : \ipt{\Pi x : T.U}{\iG}$.
    La seconde condition se montre ainsi:
    $\ipT{\lambda x : T.M}{\iG} = \Pi x : \ipt{T}{\iG}. \ipT{M}{\iG, x :
      \ipt{T}{\iG}}$. Or par induction, 
    $\ipT{M}{\iG, x : \ipt{T}{\iG}} = \ipt{U}{\iG, x : \ipt{T}{\iG}}$, donc 
    $\ipT{\lambda x : T.M}{\iG} = \Pi x : \ipt{T}{\iG}. \ipt{U}{\iG, x :
      \ipt{T}{\iG}} = \ipt{\Pi x : T.U}{\iG}$.
    
    \case{App} On a:
    \begin{prooftree}
      \AppA
    \end{prooftree}

    C'est le cas int�ressant puisqu'on ajoute ici des coercions.
    Par induction, $\iG{} \typec \ipt{f}{\iG} : \ipt{T}{\iG}$,
    $\iG{} \typec \ipt{u}{\iG} : \ipt{U}{\iG}$,
    $\iG{} \typec \ipt{U}{\iG}, \ipt{V}{\iG} : \ipt{s}{\iG}$.
    
    On va proc�der par �tapes: d'abord la construction d'une fonction
    $\pi \ipt{f}{\iG} : \ipt{\Pi x : V.W}{\iG}$ puis la construction de
    l'argument $c \ipt{u}{\iG} : \ipt{V}{\iG}$. On aura plus qu'� les
    appliquer pour obtenir le jugement:
    $\iG{} \typec \ipt{f~u}{\iG} : \ipt{W}{\iG, x :
      \ipt{V}{\iG}}[c~u/x]$.
    On montrera enfin que $\ipt{W}{\iG, x : \ipt{V}{\iG}}[c~u/x] = 
    \ipt{W[u/x]}{\iG}$ ce qui nous permettra de d�duire:
    $\iG{} \typec \ipt{f~u}{\iG} : \ipt{W[u/x]}{\iG}$.




    


    
    



    

    \case{Sigma}
    \case{Sum}
    \case{LetSum}
    \case{Subset}

  \end{induction}
\end{proof}


%%% Local Variables: 
%%% mode: latex
%%% TeX-master: "subset-typing"
%%% LaTeX-command: "TEXINPUTS=\"style:$TEXINPUTS\" latex"
%%% End: 
