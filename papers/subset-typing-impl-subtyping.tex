\begin{lemma}[Unicit� de la coercion]
  \label{coercion-unicity}
  Si $`G \typea T, U : s$ alors si $\subimpl{`G}{c}{T}{U}$ et
  $\subimpl{`G}{c'}{T}{U}$ alors $c = c'$.
\end{lemma}

\begin{proof}
  Par induction lexicographique sur la paire de d�rivations de $c$ et $c'$.
  Seule la r�gle \irule{SubHnf} peut s'appliquer n'importe quand. 
  Supposons que la d�rivation de $c$ se termine par une application de \irule{SubHnf}.
  Quelque soit la derni�re r�gle appliqu�e dans la d�rivation de $c'$,
  on a $T = \hnf{T}$ et $U = \hnf{U}$. Par induction on a donc $c = c'$.
\end{proof}

On peut donc raisonner comme ceci: si l'on parvient � construire une
d�rivation de $\subimpl{`G}{c}{T}{U}$ alors c'est l'unique d�rivation
existante.

\begin{lemma}[Coercion et formes normales de t�te]
  \label{substi-coercion-hnf}
  Si $`G \typec c : T \suba U$ alors $`G \typec c' : \hnf{T}~\suba
  \hnf{U}$ avec $c = c'$ est d�rivable par une d�rivation plus
  petite ou �gale.
\end{lemma}

\begin{proof}
  Par idempotence de la mise en forme normale de t�te, on a la m�me
  d�rivation dans le cas o� la derni�re r�gle appliqu�e �tait
  \irule{SubHnf}, sinon c'est trivial.
\end{proof}

\begin{lemma}[Coercion de termes convertibles]
  \label{subti-eqb-coercion-eqbres-id}
  Si $`G \typea T, U : s$ et $T \eqbr U$ alors il existe $c$,
  $\subimpl{`G}{c}{T}{U}$ avec $c \eqbres \ctxdot$.
\end{lemma}

\begin{proof}
  Par induction sur le nombre de constructeurs $\Pi, \Sigma, \{|\}$ dans
  les formes normales de $T$ et $U$ (on note $\nf{T}$ la forme normale
  de $T$).

  \begin{itemize}  
  \item Si $\nf{T}$ n'a pas pour symbole de t�te, $\Pi, \Sigma$ ou
    $\{|\}$, alors \irule{SubConv} est la seule r�gle applicable et on
    a $c = \ctxdot$. On a donc trait� les cas ou le nombre de symboles
    $\Pi, \Sigma, \{|\}$ dans $T$ et $U$ est sup�rieur ou �gal � z�ro, seulement
    s'il n'appara�t pas en t�te. On traite maintenant le seul autre cas possible.

  \item Si $\hnf{T} = \Pi y : A.B$, alors $\hnf{U} = \Pi y : A'.B'$ avec $A
    \eqbr A'$, $B \eqbr B'$. Par induction, $\subimpl{`G}{c_1 \eqbres \ctxdot}{A'}{A}$
    et $\subimpl{`G, x : A'}{c_2 \eqbres \ctxdot}{B}{B'}$ ($\nf{A}, \nf{A'}, \nf{B}, \nf{B'}$ sont des
    sous-termes stricts de $\nf{T}$ et $\nf{U}$). On peut donc d�river: 
    $\subimpl{`G}{\lambda x : \ip{A'}{`G}. c_2[\ctxdot~c_1[x]]}
    {\Pi y : A.B}{\Pi y : A'.B'}$. Puis par application de
    \irule{SubHnf}, la coercion de $T$ � $U$.
        
    On a donc:
    \begin{eqnarray*}
      c & = & \lambda x : \ip{A'}{`G}. c_2[\ctxdot~(c_1[x]) \\
      & \eqbres & \lambda x : \ip{A'}{`G}. \ctxdot~x \\
      & \eqbres & \ctxdot
    \end{eqnarray*}
    
  \item Si $\hnf{T} = \Sigma y : A.B$ alors $\hnf{U} = \Sigma y : A'.B'$ avec $A
    \eqbr A'$, $B \eqbr B'$. Par induction, $\subimpl{`G}{c_1 \eqbres \ctxdot}{A}{A'}$
    et $\subimpl{`G, x : A}{c_2 \eqbres \ctxdot}{B}{B'}$ ($\nf{A},
    \nf{A'}, \nf{B}, \nf{B'}$ sont des
    sous-termes stricts de $\nf{T}$ et $\nf{U}$). On peut donc d�river: 
    $\subimpl{`G}{(c_1[\pi_1~\ctxdot], c_2[\pi_1~\ctxdot/x][\pi_2~\ctxdot])}
    {\Pi y : A.B}{\Pi y : A'.B'}$. Puis par application de
    \irule{SubHnf}, la coercion de $T$ � $U$.

    On a donc:
    \begin{eqnarray*}
      c & = & (c_1[\pi_1~\ctxdot], c_2[\pi_1~\ctxdot/x][\pi_2~\ctxdot]) \\
      & "->>"_\beta & (\pi_1~\ctxdot, \pi_2~\ctxdot) \\
      & =_\SP & \ctxdot
    \end{eqnarray*}
    
  \item Le cas des sous-ensembles est un peu diff�rent.
    Si $\hnf{T} = \mysubset{x}{T'}{P}$ alors $\hnf{U} =
    \mysubset{x}{U'}{P'}$ avec $T' \eqbr U'$ et $P \eqbr P'$.
    La d�rivation va avoir la forme suivante:
    \begin{prooftree}
      \AXC{$\subimpl{`G}{c' \eqbr \ctxdot}{T'}{U'}$}
      \UIC{$\subimpl{`G}{d = c'[\pi_1~\ctxdot]}{T'}{U'}$}
      \UIC{$\subimpl{`G}{c = \elt{\ip{U'}{`G}}{\ip{\lambda x : U'.P''}{`G}}{d}
            {\ex{\ipG{`G}}{\ip{P''}{`G}[d/x]}}}{\mysubset{x}{T'}{P'}}{U'}$}
      \UIC{$\subimpl{`G}{c}{\mysubset{x}{T'}{P'}}{\mysubset{x}{U'}{P''}}$}
      \UIC{$\subimpl{`G}{c}{T}{U}$}
    \end{prooftree}
    
    On a donc $d = c'[\pi_1~\ctxdot] \eqbr \pi_1~\ctxdot$.
    Comme $P' \eqbr P''$, la preuve de l'existentielle
    $\ex{\ipG{`G}}{\ip{P''}{`G}[\pi_1~\ctxdot/x]}$ peut �tre consid�r�e
    comme �quivalente � $\pi_2~\ctxdot$. 
    \TODO{Probl�me de l'interpr�tation, on n'a pas $P' \eqbr P'' "=>" \ip{P'}{`G} =
      \ip{P''}{`G}$}

    On peut v�rifier:
    \begin{eqnarray*}
      c & = & \elt{\ip{U'}{`G}}{\ip{\lambda x : U'.P''}{`G}}{d}
      {\ex{\ipG{`G}}{\ip{P''}{`G}[d/x]}} \\
      & = & \elt{\ip{U'}{`G}}{\ip{\lambda x : U'.P''}{`G}}{(\pi_1~\ctxdot)}
      {\ex{\ipG{`G}}{\ip{P''}{`G}[\pi_1~\ctxdot/x]}} \\
      & = & \elt{\ip{U'}{`G}}{\ip{\lambda x : U'.P''}{`G}}{(\pi_1~\ctxdot)}
      {(\pi_2~\ctxdot)} \\
      & =_\SP & \ctxdot
    \end{eqnarray*}
    
  \end{itemize}
\end{proof}

\def\GD{`G, x : U, `D}
\def\Gr{`G, `D[u/x]}
\def\iGD{\ipG{\GD}}
\def\iGr{\ipG{\Gr}}


\begin{lemma}[Coercion de sortes]
  \label{subti-coercion-sorts}
  Si $`G \typec e : s \subi T$ ou $`G \typec e : T \subi s$ alors $T
  \eqbi s$ et $e \eqbi \lambda x : s.x$.
\end{lemma}
\begin{proof}  
  Clairement on ne peut d�river $s \suba T$ que par \irule{SubConv}
  (�ventuellement pr�c�d� de \irule{SubHnf}). En effet seule la r�gle
  \irule{SubProof} pourrait s'appliquer, mais cela impliquerait que
  $T \eqbi \mysubset{x}{U}{P}$ avec $s \suba U$ et ainsi de suite. La
  seule possibilit� est de d�river $s \eqbi T$ ou $s \eqbi U$, auquel cas $U$ est une
  sorte ce qui contredit le fait que $\mysubset{x}{U}{P} : s$ dans le
  cas pr�c�dent. On a d�rive donc $s \subi T$ si et seulement si $s
  \eqbi T$.
\end{proof}

\begin{lemma}[Stabilit� de la coercion par substitution]
  \label{subti-coercion-subst}
  Si $\GD \typea T, T' : s$, $`G \typec u :
  U$, alors
  $\begin{array}{lcl}
    \GD \typea t : T & "=>" & \ip{t[u/x]}{\Gr} \eqbe
    \ip{t}{\GD}[\ip{u}{`G}/x] \\
    \subimpl{\GD}{c}{T}{T'} & "=>" & \subimpl{\Gr}{c'}{T[u/x]}{T'[u/x]}
    `^ c' \eqbei c[\ip{u}{`G}/x]
  \end{array}$
  
\end{lemma}

\begin{proof}
  Par induction mutuelle sur les d�rivation de $c$ et $t$.

  \begin{induction}
    \case{SubHnf} On a: 
    \begin{prooftree}
      \AXC{$\subimpl{\GD}{c}{\hnf{T}}{\hnf{T'}}$}
      \UIC{$\subimpl{\GD}{c}{T}{T'}$}
    \end{prooftree}
    
    \def\as{\overrightarrow{a}}    
    \def\bs{\overrightarrow{b}}    
    \def\asux{\overrightarrow{a[u/x]}}
    \def\bsux{\overrightarrow{b[u/x]}}
    \def\ipux{[\ip{u}{`G}/x]}

    Par induction on a
    $\subimpl{\Gr}{c'}{\hnf{T}[u/x]}{\hnf{T'}[u/x]}$ avec $c' \eqbe
    c\ipux$.
    Si $\hnf{T[u/x]} = \hnf{T}[u/x]$ et $\hnf{T'[u/x]} = \hnf{T'}[u/x]$
    c'est direct par induction. Sinon, on a $\hnf{T} = x~\as$ ou $\hnf{T'} =
    x~\as$. Les deux cas sont similaires, on traite le cas ou $\hnf{T} =
    x~\as$. Le jugement $x~\as \suba \hnf{T'}$
    ne peut �tre d�riv� que par \irule{SubProof} ou
    \irule{SubConv}. Dans le premier cas, cela implique qu'on a une
    d�rivation de $d : x~\as \suba U'$ o� $\hnf{T'} = \mysubset{y}{U'}{P}$. Par
    induction on a donc une d�rivation de
    $\subimpl{\Gr}{d'}{u~\asux}{U'[u/x]}$ avec $d' \eqbei
    d\ipux$. 
    Par le lemme \ref{substi-coercion-hnf} on a donc une d�rivation de 
    $\subimpl{\Gr}{d'}{\hnf{(u~\asux)}}{\hnf{(U'[u/x])}}$ avec $d' \eqbei
    d\ipux$. 
    
    \begin{prooftree}
      \AXC{$\subimpl{\Gr}{d'}{\hnf{(u~\asux)}}{\hnf{(U'[u/x])}}$}
      \UIC{$\subimpl{\Gr}{d'}{\hnf{(u~\asux)}}{U'[u/x]}$}
      \UIC{$\subimpl{\Gr}{c'}{\hnf{T[u/x]}}{\hnf{(T'[u/x])} = \mysubset{y}{U'[u/x]}{P[u/x]}}$}
      \UIC{$\subimpl{\Gr}{c'}{T[u/x]}{T'[u/x]}$}
    \end{prooftree}
      
      On a $\ip{\hnf{T}}{\GD}\ipux \eqbi
      \ip{\hnf{T}[u/x]}{\Gr}$ par induction. Or $\hnf{(\hnf{T}[u/x])} =
      \hnf{(u~\asux)} = \hnf{(T[u/x])}$, donc, du fait qu'on met en forme
      normale de t�te avant interpr�tation \TODO{Pas explicite dans la
        def de l'interpr�tation}:
      \begin{eqnarray}
        \ip{\hnf{(T[u/x])}}{\Gr} & = & \ip{\hnf{T}[u/x]}{\Gr} \\
        & \eqbi & \ip{\hnf{T}}{\GD}\ipux \label{eq:hnftux}
      \end{eqnarray}
      
      Par induction, on a $\ip{U'[u/x]}{\Gr} \eqbi \ip{U'}{\GD}\ipux$ et
      $\ip{P[u/x]}{\Gr} = \ip{P}{\GD}\ipux$. 
      On en d�duit:
      \begin{eqnarray*}
        \ip{P[u/x]}{\Gr}[d'~t/y] & \eqbi & \ip{P}{\GD}\ipux[d'~t/y] \\
        & \eqbi & \ip{P}{\GD}\ipux[d\ipux~t/y] \\
        & = & \ip{P}{\GD}[d~t/y]\ipux
      \end{eqnarray*}
      
      Soit $A = \ipG{\Gr, t : \hnf{T[u/x]}} \typec \ip{P[u/x]}{}[d'~t/y]$ et
      $B = \ipG{\GD, t : \hnf{T}} \typec \ip{P}{}[d~t/y]$. On a
      $A \eqbi B\ipux$ car $\ipG{\Gr, t : \hnf{T[u/x]}} \eqbi \ipG{\GD, t :
        \hnf{T}}\ipux$ par \ref{eq:hnftux} et les conclusion des
      existentielles sont convertibles par le r�sultat pr�c�dent.
      
      On a donc:
      \begin{eqnarray*}
        c' & = & \lambda t : \ip{\hnf{(T[u/x])}}{\Gr}. \\
        & &
        \elt{\ip{U'[u/x]}{\Gr}}
        {\ip{(\lambda y : (U'[u/x]).P[u/x])}{\Gr}}
        {(d'~t)}
        {?_A} \\
        & \eqbe & \lambda t : \ip{\hnf{(T[u/x])}}{\Gr}. \\
        & &
        \elt{\ip{U'[u/x]}{\Gr}}
        {\ip{(\lambda y : (U'[u/x]).P[u/x])}{\Gr}}
        {(d\ipux~t)}
        {?_A} \\
        & \eqbe & 
        \lambda t : \ip{\hnf{T}}{\GD}\ipux. \\
        & &
        \elt{\ip{U'}{\GD}\ipux}
        {\ip{(\lambda y : U'.P)}{\GD}\ipux}
        {(d\ipux~t)}
        {?_A} \\
        & \eqbe &
        (\lambda t : \ip{\hnf{T}}{\GD}.
        \elt{\ip{U'}{\GD}}{\ip{\lambda y : U'.P}{\GD}}
        {(d~t)}
        {?_B})\ipux \\
        & = & c\ipux
      \end{eqnarray*}
          
      Dans le cas ou l'on applique directement \irule{SubConv} cela
      implique que $\hnf{T} = x = \hnf{T'}$ donc par reflexivit� de la
      coercion \ref{subti-reflexive}, $\subimpl{\Gr}{c'}{u}{u}$ est d�rivable et $c' \eqbei
      \lambda y : \ip{u}{`G}.y \eqbei (\lambda y : x.y)\ipux$.
      
    \case{SubConv}
    On a $T \eqbi T'$, donc $T[u/x] \eqbi T'[u/x]$ par substitutivit� de
    la $\beta$-�quivalence. Par le lemme \ref{subti-eqb-coercion-eqbe-id}, on
    sait qu'il existe $c'$, $\Gr \typec c' : T[u/x] \sub T'[u/x]$ telle
    que $c' \eqbe \lambda x : \ip{T[u/x]}{\Gr}.x$. On a bien $c' \eqbe
    c\ipux$ car $\ip{T[u/x]}{\Gr} \eqb \ip{T}{\GD}\ipux$ par induction.
    
    \case{SubProd}
    On a:
    \begin{prooftree}
      \AXC{$\subimplhyp{\GD}{c_1}{A'}{A}$}
      \AXC{$\subimplhyp{\GD, y : A'}{c_2}{B}{B'}$}
      \BIC{$\subimplconcl{\GD}{f}
        {\lambda x : \ip{A'}{\GD}.~c_2~(f~(c_1~x))}
        {\Pi y : A.B}{\Pi y : A'.B'}$}
    \end{prooftree}
    
    Par induction et application de \irule{SubProd}:
    \begin{prooftree}
      \AXC{$\subimplhyp{\Gr}{c_1'}{A'[u/x]}{A[u/x]}$}
      \AXC{$\subimplhyp{\Gr, y : A'[u/x]}{c_2'}{B[u/x]}{B'[u/x]}$}
      \BIC{$\subimpl{\Gr}{c'}
        {\Pi y : A[u/x].B[u/x]}{\Pi y : A'[u/x].B'[u/x]}$}
    \end{prooftree}
    
    o� \[c' = \lambda f : \ip{\Pi y : A[u/x].B[u/x]}{\Gr}.\lambda x :
    \ip{A'[u/x]}{\Gr}.~c_2'~(f~(c_1'~x))\]
    Par induction, $\ip{A'[u/x]}{\Gr} \eqbe \ip{A'}{\GD}\ipux$ et 
    $\ip{\Pi y : A[u/x].B[u/x]}{\Gr} \eqbe \ip{\Pi y : A.B}{\GD}\ipux$.
    On a donc bien $c' \eqbe c\ipux$.
   
    \casethree{SubSigma}{SubProof}{SubSub} Idem, direct par induction.
  
    \case{PropSet} Trivial.
    
    \case{Var} 
    On a:
    \typenva
    \begin{prooftree}
      \BAX{Var}
      {$\wf \GD$}
      {$y : T `: \GD$}
      {$\GD \seq y : T$}
      {}
    \end{prooftree}    
    \typenvi

    \begin{itemize}
    \item Si $x "/=" y$, alors par d�finition de l'interpr�tation,
      on doit montrer
      $\ip{y[u/x]}{\Gr} = y = \ip{y}{\GD}\ipux$.
      
    \item Sinon, $t[u/x] = u$ et $\ip{u}{\Gr} = \ip{x}{\GD}\ipux =
      \ip{u}{\GD}$. On utilise ici le fait que $u$ est bien typ� dans
      l'environnement $`G$, donc dans toute extension bien form�e de cet
      environnement par affaiblissement.      
    \end{itemize}
    
    \case{App}\quad
    \typenva
    \begin{prooftree}
      \TAX{App}
      {$\GD \seq f : F \quad \mualgo(F) = \Pi y : A. B : s$}
      {$\GD \seq e : E \quad `G \seq E, A : s$}
      {$E \sub A$}
      {$\GD \seq (f~e) : B [ e / y ]$}
      {}
    \end{prooftree}
    \typenvi

    Par induction:
    \[\ip{f[u/x]}{\Gr} \eqbi \ip{f}{\GD}\ipux\]
    \[\ip{e[u/x]}{\Gr} \eqbi \ip{e}{\GD}\ipux\]
    
    \def\afe{`a}    
    Par d�finition de la substitution et de l'interpr�tation, 
    \begin{eqnarray*}
      \ip{(f~e)}{\GD}
      & = & (((\pi_F~\ip{f}{\GD})~(c_e~\ip{e}{\GD})) \\
      \text{ o� } & & \\
      \pi_F & = & \sref{coerce}~F~(\Pi y : A.B) \\
      c_e & = & \sref{coerce}~E~A.
    \end{eqnarray*}
    
    Clairement, $\pi_F$ et $c_e$ sont d�rivables, puisqu'on part de
    jugements d�rivables par la coercion algorithmique.

    On a la substitutivit� du typage \ref{substitutive-typing} donc le
    jugement substitu� est:
    \typenva
    \begin{prooftree}
      \TAX{App}
      {$\Gr \seq f[u/x] : F[u/x] \quad \mualgo(F[u/x]) = \Pi y : A[u/x]. B[u/x] : s$}
      {$\Gr \seq e[u/x] : E[u/x] \quad `G \seq E[u/x], A[u/x] : s$}
      {$E[u/x] \sub A[u/x]$}
      {$\Gr \seq (f~e)[u/x] : B' [ e[u/x] / y ]$}
      {}
    \end{prooftree}
       
    Soit $e' = e[u/x]$ et $f' = f[u/x]$, on a donc d'autre part:
    \begin{eqnarray*}
      \ip{(f~e)[u/x]}{\Gr}
      & = & \ip{f'~e'}{\Gr} \\
      & = & (\pi_{F[u/x]}~\ip{f'}{\Gr})~(c_{e'}~\ip{e'}{\Gr}) \\
      \text{ o� } & & \\
      \pi_{F[u/x]} & = & \sref{coerce}~F[u/x]~(\Pi y : A[u/x].B[u/x]) \\
      c_{e'} & = & \sref{coerce}~E[u/x]~A[u/x]
    \end{eqnarray*}
    
    Par induction, il existe $d$, $e$:
    $\Gr \typec d \eqbe \pi_F\ipux : F[u/x] \suba (\Pi y : A.B)[u/x]$
    et 
    $\Gr \typec e \eqbe c_e\ipux : E[u/x] \suba A[u/x]$.
    
    \[\begin{array}{ll}
      \firsteq{\ip{f~e}{\GD}\ipux}
      
      \step{D�finition de l'interpr�tation}
      {=}(\pi_F~\ip{f}{\GD})~(c_e~\ip{e}{\GD})\ipux 

      \step{D�finition de la substitution}
      {=}{(\pi_F\ipux~\ip{f}{\GD}\ipux)~(c_e\ipux~\ip{e}{\GD}\ipux)}

      \step{Application de l'hypoth�se d'induction pour les termes}
      {\eqbe}{(\pi_F\ipux~\ip{f'}{\Gr})~(c_e\ipux~\ip{e'}{\Gr})}

      \step{Application de l'hypoth�se d'induction pour les coercions}
      {\eqbe}{(d~\ip{f'}{\Gr})~(e~\ip{e'}{\Gr})}

      \step{Unicit� des coercions: $d \eqbe \pi_{F'}$ et $e \eqb
        c_{e'}$}
      {\eqbe}{(\pi_{F'}~\ip{f'}{\Gr})~(c_{e'}~\ip{e'}{\Gr})}
      
      \step{D�finition de l'interpr�tation}
      {\eqbe}{\ip{(f~e)[u/x]}{\Gr}}
    \end{array}\]
    
    \casethree{Prod}{Sigma}{Subset} Par induction.

    \case{Abs}
    \typenva
    On a:
    \begin{prooftree}
      \AXC{$\GD \seq \Pi y : T. U : s $}
      \AXC{$\GD, y : T \seq M : U $}
      \BIC{$\GD \seq \lambda y : T. M : \Pi y : T.U$}
    \end{prooftree}
    
    On a bien:
    \[\begin{array}{ll}
      \firsteq{\ip{\lambda y : T.M}{\GD}\ipux}
      
      \step{D�finition de l'interpr�tation}
      {=}{\lambda y : \ip{T}{\GD}\ipux.\ip{M}{\GD, y : T}\ipux}

      \step{Application de l'hypoth�se de r�currence}
      {\eqbe}{\lambda y : \ip{T[u/x]}{\Gr}.\ip{M[u/x]}{\Gr, y : T[u/x]}}

      \step{D�finition de l'interpr�tation}
      {=}{\ip{\lambda y : T[u/x].M[u/x]}}
    \end{array}\]
      
    \casetwo{LetSum}{SumDep} \TODO{Par induction}


  \end{induction}
\end{proof}

On va maintenant �tendre la relation de coercion aux contextes de
mani�re canonique.
\typenva
\begin{definition}[Coercion de contextes]
  \label{coercion-ctx}
  On d�finit inductivement la coercion de deux contextes de coercions
  algorithmiques par les r�gles suivantes:
  \begin{itemize}
  \item $[] \sub []$
  \item $`G, x : T \sub `G', x : T'$ si $`G \sub `G'$ et $T \sub T'$.
  \end{itemize}
\end{definition}

De m�me pour les coercions explicites d�riv�es par le jugement $`G
\typec c : T \suba S$.
\begin{definition}[Coercion explicites de contextes]
  \label{coercion-ctx-i}
  On d�finit inductivement la coercion de deux contextes de coercions
  explicites par les r�gles suivantes:
  \begin{itemize}
  \item $[] \sub []$
  \item $`r, c : `G, x : T \sub `G', x : T'$ si $`r : `G \sub `G'$ et 
    $`G \typec c : T \sub T'$.
  \end{itemize}
\end{definition}

Clairement toute coercion de contexte algorithmique correspond � une
coercion de contexte explicite et vice-versa.

\begin{definition}[Extension de la substitution aux coercions de
  contextes]
  \label{subst-coercion-ctx}
  On d�finit la substitution d'une coercion de contexte inductivement:
  \begin{itemize}
  \item $t[[]] = t$
  \item $t[`r, c : `G, x : T \sub T'] = t[`r : `G][c~x/x]$
  \end{itemize}
\end{definition}
\begin{lemma}[Stabilit� par affaiblissement]
  Si $`r : `D \sub `G$, $`G \typec c : T \subi T'$ et $\ip{T'}{`D} \eqbe
  \ip{T}{`G}[`r]$, alors $`D \typec c' : T \sub T'$ et $c' \eqb c[`r]$.
\end{lemma}

\begin{proof}
  Par induction sur la d�rivation de coercion:

  \begin{induction}
    \case{SubHnf} 
    On a $\subimpl{`G}{c}{\hnf{T}}{\hnf{T'}}$. Par induction, 
    $`D \typec c[`r] : \hnf{T} \sub \hnf{T'}$. 
    On peut donc d�river $`D \typec c[`r] : T \sub T'$ par \irule{SubHnf}.

    \case{SubConv}
    On a $T \eqb T'$ et $c = \lambda x : \ip{T}{`G}.x$. 
    On peut donc d�river $\subimpl{`D}{c' = \lambda x : \ip{T}{`D}.x}{T}{T'}$.
    Or $\ip{T}{`D} \eqb \ip{T'}{`G}[`r]$, on a donc bien $c' = c[`r]$.

    \case{SubProd}
    On a:
    \begin{prooftree}
      \BAX{}
      {$\subimpl{`G}{c_1}{C}{A}$}
      {$\subimpl{`G, y : C}{c_2}{B}{D}$}
      {$\subimplhnf{`G}{\lambda f : \ip{\Pi x : A.B}{`G}.~\lambda x :
          \ip{C}{`G}.~c_2~(f~(c_1~x))}
        {\Pi x : A.B}{\Pi x : C.D}$}
      {}
    \end{prooftree}
    Par induction on a {$\subimpl{`D}{c_1[`r]}{C}{A}$}. On peut d�finir
    la coercion $`s = `r, (\lambda x : \ip{C}{`D}.x) : `D, x : C \sub `G, x : C$ et obtenir par
    induction:
    $\subimpl{`G, y : C}{c_2[`s]}{B}{D}$
    
    On peut alors appliquer \irule{SubProd} pour obtenir:
    \[\subimplhnf{`G}{c' = \lambda f : \ip{\Pi x : A.B}{`G}.~\lambda x :
      \ip{C}{`G}.~c_2[`s]~(f~(c_1[`r]~x))}{\Pi x : A.B}{\Pi x : C.D}\]
    
    Comme $\ip{\Pi x : A.B}{`G} = \Pi x : \ip{A}{`G}.\ip{B}{`G, x : A}$
    on a:
    \[\begin{array}{ll}
      \firsteq{(\lambda f : \ip{\Pi x : A.B}{`G}.~\lambda x :
          \ip{C}{`G}.~c_2~(f~(c_1~x)))[`r]}

        \step{D�finition de la substitution}
        {=}{\lambda f : \ip{\Pi x : A.B}{`G}[`r].~\lambda x :
          \ip{C}{`G}[`r].c_2[`r]~(f~(c_1[`r]~x))}

        \step{Application de l'hypoth�se de r�currence}
        {\eqbe}{\lambda f : \ip{\Pi x : A.B}{`D}.~\lambda x :
          \ip{C}{`D}.c_2[`r]~(f~(c_1[`r]~x))}
        
        \step{Coercion identit� dans $`s$}
        {=}{\lambda f : \ip{\Pi x : A.B}{`D}.~\lambda x :
          \ip{C}{`D}.c_2[`s]~(f~(c_1[`r]~x))}      
      \end{array}\]
      
    \case{SubSigma}
    \TODO{todo!}
    On a:
    \begin{prooftree}
      \BAX{}
      {$\subimpl{`G}{c_1}{C}{A}$}
      {$\subimpl{`G, y : C}{c_2}{B}{D}$}
      {$\subimplhnf{`G}{\lambda f : \ip{\Pi x : A.B}{`G}.~\lambda x :
          \ip{C}{`G}.~c_2~(f~(c_1~x))}
        {\Pi x : A.B}{\Pi x : C.D}$}
      {}
    \end{prooftree}
    Par induction on a {$\subimpl{`D}{c_1[`r]}{C}{A}$}. On peut d�finir
    la coercion $`s = `r, (\lambda x : \ip{C}{`D}.x) : `D, x : C \sub `G, x : C$ et obtenir par
    induction:
    $\subimpl{`G, y : C}{c_2[`s]}{B}{D}$
    
    On peut alors appliquer \irule{SubProd} pour obtenir:
    \[\subimplhnf{`G}{c' = \lambda f : \ip{\Pi x : A.B}{`G}.~\lambda x :
      \ip{C}{`G}.~c_2[`s]~(f~(c_1[`r]~x))}{\Pi x : A.B}{\Pi x : C.D}\]
    
    Comme $\ip{\Pi x : A.B}{`G} = \Pi x : \ip{A}{`G}.\ip{B}{`G, x : A}$
    on a:
    \[\begin{array}{ll}
      \firsteq{(\lambda f : \ip{\Pi x : A.B}{`G}.~\lambda x :
          \ip{C}{`G}.~c_2~(f~(c_1~x)))[`r]}

        \step{D�finition de la substitution}
        {=}{\lambda f : \ip{\Pi x : A.B}{`G}[`r].~\lambda x :
          \ip{C}{`G}[`r].c_2[`r]~(f~(c_1[`r]~x))}

        \step{Application de l'hypoth�se de r�currence}
        {\eqbe}{\lambda f : \ip{\Pi x : A.B}{`D}.~\lambda x :
          \ip{C}{`D}.c_2[`r]~(f~(c_1[`r]~x))}
        
        \step{Coercion identit� dans $`s$}
        {=}{\lambda f : \ip{\Pi x : A.B}{`D}.~\lambda x :
          \ip{C}{`D}.c_2[`s]~(f~(c_1[`r]~x))}      
      \end{array}\]
      

  \end{induction}
\end{proof}


\begin{lemma}[Substitutivit� de la coercion]
  \label{subti-coercion-subst}
  Si $`G \typec X, V, T, U : s$, $`G \typec e : X \subi V$ et
  $`G, x : V, `D \typec c : T \subi U$ alors
  $`G, x : X, `D[e~x/x] \typec c[e~x/x] : T[e~x/x] \subi U[e~x/x]$.   
\end{lemma}

\begin{proof}
  Soit $T' = T[e~x/x]$, $U' = U[e~x/x]$.

  \begin{induction}[subtyping-algo]
  \case{SubHnf} 
  On a: 
  \begin{prooftree}
    \AXC{$\subimplhnf{\GD}{c}{\hnf{T}}{\hnf{U}}$}
    \UIC{$\subimpl{\GD}{c}{T}{U}$}
  \end{prooftree}
  
  On a plusieurs possibilit�s:
  \begin{itemize}
  \item Si $\hnf{T} = x$. Alors la seule r�gle appliquable est
    \irule{SubProof}, on a donc:
    \begin{prooftree}
      \AXC{$\subimpl{\GD}{c'}{x}{W}$}
      \UIC{$\subimplhnf{\GD}{c = (\lambda t : x.\ldots(c'~t))}{\hnf{T} = x}{\hnf{U} = \mysubset{y}{W}{P}}$}
    \end{prooftree}
    
    Par induction: \[\subimpl{\Gr}{c'[e~x/x]}{e~x}{W[e~x/x]}\]
    Or on peut appliquer \irule{SubHnf} et \irule{SubProof} 
    pour d�river le jugement $\subimpl{\GD}{?}{T'}{U'}$:

    \begin{prooftree}
      \AXC{$\subimpl{\Gr}{c'[e~x/x]}{e~x}{W[e~x/x]}$}
      \UIC{$\subimplhnf{\GD}{\lambda t :
          x[e~x/x].\ldots(c'~t))}{\hnf{T'} = e~x}{\hnf{U'} = \mysubset{y}{W[e~x/x]}{P[e~x/x]}}$}
      \UIC{$\subimpl{\Gr}{?}{T'}{U'}$}
    \end{prooftree}
    
    La coercion inf�r�e est bien $c[e~x/x]$.
    On ne peut appliquer d'autre r�gle que \irule{SubProof}, en effet,
    comme $T : s$ on a $\hnf{T} = x : s$ et donc $X \eqbi s$. Par le
    lemme \ref{subti-coercion-sorts} on en d�duit que $\hnf{T'} = x$
    donc aucune autre r�gle ne peut s'appliquer.
    
  \item Si $\hnf{U} = x$ alors la seule r�gle applicable est
    \irule{SubSub} et l'on a un r�sultat similaire au cas pr�c�dent:
    
    \begin{prooftree}
      \AXC{$\subimpl{\Gr}{c'[e~x/x]}{W[e~x/x]}{x}$}
      \UIC{$\subimplhnf{\GD}{c'[e~x/x] `o \pi_1}{\hnf{T'} =
          \hnf{T}[e~x/x] = \mysubset{y}{W[e~x/x]}{P[e~x/x]}}{\hnf{U'}}$}
      \UIC{$\subimpl{\Gr}{?}{T'}{U'}$}
    \end{prooftree}
   
  \item Sinon, $\hnf{T} "/=" x$ et $\hnf{U} "/=" x$ donc $\hnf{T'} =
    \hnf{T}[e~x/x]$ et $\hnf{U'} = \hnf{U}[e~x/x]$.

    Dans ce cas on va <<rejouer>> la d�rivation $\subimplhnf{\GD}{c}{\hnf{T}}{\hnf{U}}$.
    
    Par cas sur cette d�rivation:
    \begin{induction}
      \case{SubProd}\quad
      On a:
      \begin{prooftree}
        \BAX{}
        {$\subimpl{\GD}{c_1}{C}{A}$}
        {$\subimpl{\GD, y : C}{c_2}{B[c_1~y/y]}{D}$}
        {$\subimplhnf{\GD}{\lambda f : \Pi x : A.B.~\lambda x :
            C.~c_2~(f~(c_1~x))}
          {\Pi x : A.B}{\Pi x : C.D}$}
        {}
      \end{prooftree}

      Par induction: 
      \[\subimpl{\Gr}{c_1[e~x/x]}{C[e~x/x]}{A[e~x/x]}\]
      \[\subimpl{\Gr, y : C[e~x/x]}{c_2[e~x/x]}{B[c_1~y/y][e~x/x] = B[e~x/x][c_1[e~x/x]/y]}{D[e~x/x]}\]
      
      On en d�duit par \irule{SubProd}:
      \[\subimplhnf{\Gr}{c[e~x/x]}{(\Pi y : A.B)[e~x/x]}{(\Pi y : C.D)[e~x/x]}\]
      On peut v�rifier:
      $c[e~x/x] : \hnf{T'} \subihnf \hnf{U'} = \lambda f : (\Pi x : A.B)[e~x/x].~\lambda x :
      C[e~x/x].~c_2[e~x/x]~(f~(c_1[e~x/x]~x))$.
      
      \case{SubSigma} idem.
      
      \case{SubSub}\quad
      \begin{prooftree}
        \AXC{$\subimpl{\GD}{c}{T}{\hnf{U}}$}
        \UIC{$\subimplhnf{\GD}{\lambda t : \mysubset{y}{T}{P}.~c~(\pi_1~t)}
          {\mysubset{y}{T}{P}}{\hnf{U}}$}       
      \end{prooftree}
      
      donne:
      \begin{prooftree}
        \AXC{$\subimpl{\Gr}{c[e~x/x]}{T[e~x/x]}{\hnf{U}[e~x/x]}$}
        \UIC{$\subimplhnf{\Gr}{\lambda t : (\mysubset{y}{T}{P})[e~x/x].~c[e~x/x]~(\pi_1~t)}
          {\mysubset{y}{T[e~x/x]}{P[e~x/x]}}{\hnf{U}[e~x/x]}$}
      \end{prooftree}
      
      \case{SubProof} idem.
    \end{induction}
  \end{itemize}

  \case{SubConv}
  Dans ce cas, c'est direct par pr�servation de la $\beta$-�quivalence par
  substitution. Pour montrer que la nouvelle coercion est �gale �
  $c[e~x/x]$ il faut raisonner par cas sur la forme de $T$ et $U$. 
  \begin{itemize} 
  \item Si la forme de $T$ (respectivement $U$) est �gale � $x$ 
    alors $\hnf{U} = x$ (reps. $\hnf{T} = x$). Or $\GD \typec T,
    U : s$, donc $\GD \typec x : s$, soit $X \eqbi s$. Par le lemme
    \ref{subti-coercion-sorts}, on d�duit que $e = \lambda x : s.x$.
    On a donc: $\hnf{T'} = \hnf{U'} = \hnf{e~x} = x$. Or $x \subhnf x$
    n'est pas d�rivable donc on a la d�rivation:
    \begin{prooftree}
      \AXC{$T' \eqbi U'$}
      \UIC{$\subimpl{\Gr}{\lambda y : T'.y}{T'}{U'}$}
    \end{prooftree}
    
    On a bien $c[e~x/x] = \lambda y : T[e~x/x].y$.
  \item Sinon, on a $\hnf{T'} = \hnf{T}[e~x/x]$ et $\hnf{U'} =
    \hnf{U}[e~x/x]$. Or on sait que $\hnf{T} \subhnf \hnf{U}$ n'est pas
    d�rivable, donc la subsitution ne peut pas l'�tre non plus (on
    substitue sous les symboles de t�te dans ce cas). Cependant, on a toujours $T'
    \eqbi U'$ donc on va d�river la coercion attendue.
    
  \end{itemize}
  


\end{induction}
\end{proof}


%%% Local Variables: 
%%% mode: latex
%%% TeX-master: "subset-typing"
%%% LaTeX-command: "TEXINPUTS=\"style:$TEXINPUTS\" latex"
%%% End: 


\begin{lemma}[Transitivit� de la coercion]
  \label{subi-trans}
  S'il existe $c_1, c_2$ tels que $`G \typec c_1 : S \sub T$ et $`D
  \typec c_2 : T \sub U$ avec $`r : `D \sub `G$ et $\ip{T}{`G}[`r] \eqbr
  \ip{T}{`D}$,
  alors $`E!c, `D \typec c : S \sub U$ et $c \eqbr c_2 `o c_1[`r]$.
\end{lemma}
\begin{proof}
  Par induction lexicographique sur la paire de d�rivations de $c_1$ et $c_2$.

  \begin{induction}
       
    \case{SubConv} On va traiter les cas o� cette r�gle est utilis�e en
    racine d'une des deux d�rivations, d'abord � gauche puis � droite.

    \begin{prooftree}
      \AXC{$S \eqbr T$}
      \UIC{$\subimpl{`G}{c_1 = \ctxdot}{S}{T}$}
      \AXC{$\subimpl{`D}{c_2}{T}{U}$}
      \noLine\BIC{}
    \end{prooftree}
    
    Les conditions de bord de \irule{SubConv} nous donnent comme
    hypoth�ses que $S$ et $T$ sont en forme normale de t�te et que 
    $S "/=" \Pi, \Sigma, \{|\}$ et $T "/=" \{|\}$.
    Par inversion de la $\beta\rho$-�quivalence $S \eqbr T$, on a aussi,
    $T "/=" \Pi, Sigma$.
    Les seules r�gles pouvant s'appliquer a la fin de la d�rivation de
    $c_2$ sont donc \irule{SubConv}, \irule{SubHnf} et \irule{SubProof}.

    \begin{itemize}
      \case{SubConv} Alors on a $U = \hnf{U} "/=" \Pi, \Sigma, \{|\}$.
      La coercion compos�e est alors $\ctxdot `o \ctxdot$. C'est bien la coercion
      d�riv�e pour le jugement $\subimpl{`D}{\ctxdot}{S}{U}$ par \irule{SubConv}.
      
      \case{SubHnf} On a alors $\subimpl{`D}{c_2}{\hnf{T}}{\hnf{U}}$.
      Comme $T = \hnf{T}$, on peut appliquer l'hypoth�se d'induction pour
      obtenir une coercion $c$ telle que $\subimpl{`D}{c}{S}{\hnf{U}}$ et
      $c \eqbr c_2 `o c_1[`r]$.
      Une application de \irule{SubHnf} suffit pour obtenir une
      d�rivation de $\subimpl{`D}{c}{S}{U}$ avec $c \eqbr c_2 `o c_1[`r]$.
      
      \case{SubProof} Ici on a:
      \begin{prooftree}
        \AXC{$\subimpl{`D}{d}{T}{U'}$}
        \UIC{$\subimpl{`D}{c_2 = {\elt{\ip{U'}{`D}}{\ip{\lambda x : U'.P}{`D}}{d}
              {\ex{\ipG{`G}}{\ip{P}{`D}[d/x]}}}}{T}{U = \mysubset{x}{U'}{P}}$}
      \end{prooftree}

      Par induction, il existe une coercion $d' \eqbr d `o c_1[`r] = d$ telle que
      $\subimpl{`D}{d'}{S}{U'}$. On applique \irule{SubProof} pour obtenir
      la coercion $c$ de $S$ � $U$. Clairement $c \eqbr c_2 `o c_1[`r] =
      c_2 `o [] = c_2$.
      
    \end{itemize}
    
    Supposons maintenant que la d�rivation de $c_2$ termine par une
    application de \irule{SubConv}. Alors $T$ et $U$ sont en forme
    normale de t�te et $T, U "/=" \Pi, \Sigma, \{|\}$.
    Les seules r�gles pouvant appara�tre en racine de la d�rivation de
    $c_1$ sont donc \irule{SubConv}, \irule{SubHnf} et \irule{SubSub}.

    \begin{itemize}
      \case{SubConv} Alors on a $S = \hnf{S} "/=" \Pi, \Sigma, \{|\}$.
      La coercion compos�e est alors $\ctxdot `o \ctxdot$. C'est bien la coercion
      d�riv�e pour le jugement $\subimpl{`D}{\ctxdot}{S}{U}$ par \irule{SubConv}.
      
      \case{SubHnf} On a alors $\subimpl{`G}{c_1}{\hnf{S}}{\hnf{T}}$.
      Comme $T = \hnf{T}$, on peut appliquer l'hypoth�se d'induction pour
      obtenir une coercion $c$ telle que $\subimpl{`D}{c}{\hnf{S}}{U}$ et
      $c \eqbr c_2 `o c_1[`r]$.
      Une application de \irule{SubHnf} suffit pour obtenir une
      d�rivation de $\subimpl{`D}{c}{S}{U}$ avec $c = c_2 `o c_1[`r]$.
      
      \case{SubSub} Ici on a:
      \begin{prooftree}
        \AXC{$\subimpl{`G}{d}{S'}{T}$}
        \UIC{$\subimpl{`G}{c_1 = d[\pi_1~\ctxdot]}{S = \mysubset{x}{S'}{P}}{T}$}
      \end{prooftree}

      Par induction, il existe une coercion $d' \eqbr c_2 `o d[`r] = d[`r]$ telle que
      $\subimpl{`D}{d'}{S'}{U}$. On applique \irule{SubSub} pour obtenir
      la coercion $c = d[`r][\pi_1~\ctxdot]$ de $S$ � $U$. On a \[c =
      d[`r][\pi_1~\ctxdot] = d[\pi_1~\ctxdot][`r] \eqbr c_2 `o c_1[`r] \]

    \end{itemize}

    \case{SubHnf}\quad
    \begin{prooftree}
      \AXC{$\subimpl{`G}{c}{\hnf{S}}{\hnf{T}}$}
      \UIC{$\subimpl{`G}{c}{S}{T}$}
      \AXC{$\subimpl{`D}{d}{T}{U}$}
      \noLine\BIC{} % $\subimpl{`G}{d `o c}{S}{U}$
    \end{prooftree}
    
    Si $T = \hnf{T}$ alors c'est trivial par induction et application de
    \irule{SubHnf}.
    Sinon, la seule r�gles permettant de d�river $\subimpl{`D}{d}{T}{U}$
    est \irule{SubHnf}. Il suffit alors d'appliquer l'hypoth�se
    d'induction pour obtenir une d�rivation de
    $\subimpl{`D}{c}{\hnf{S}}{\hnf{U}}$ avec $c \eqbr c_2 `o c_1[`r]$ puis
    \irule{SubHnf} nous permet de construire le jugement
    $\subimpl{`G}{c}{S}{U}$. 

    De m�me si l'on a une application de \irule{SubHnf} � la racine de
    la d�rivation de droite.
    
    On peut donc se ramener au cas o� \irule{SubConv} et
    \irule{SubHnf} ne sont appliqu�es � la racine d'aucune des 
    deux d�rivations.

    
    \case{SubProd}\quad
    \begin{prooftree}
      \AXC{$\subimpl{`G}{c_1}{X'}{X}$}
      \AXC{$\subimpl{`G, x : X'}{c_2}{Y}{Y'}$}
      \BIC{$\subimpl{`G}{c = \lambda x : \ip{X'}{`G}.c_2[\ctxdot~c_1[x]]}{\Pi x : X.Y}{\Pi x : X'.Y'}$}
      \AXC{$\subimpl{`D}{d}{\Pi x : X'.Y'}{U}$}
      \noLine\BIC{}
    \end{prooftree}
    
    Par induction sur la d�rivation $\subimpl{`D}{d}{\Pi x :
      X'.Y'}{U}$. Seules deux r�gles peuvent s'appliquer � la racine:
    \begin{induction}
      \case{SubProd}\quad
      On a $U = \Pi x : S.T$ et la d�rivation a la forme:
      \begin{prooftree}
        \AXC{$\subimpl{`D}{d_1}{S}{X'}$}
        \AXC{$\subimpl{`D, x : S}{d_2}{Y'}{T}$}
        \BIC{$\subimpl{`D}{d = (\lambda x : \ip{S}{`D}.d_2[\ctxdot~d_1[x]])}
          {\Pi x : X'.Y'}{\Pi x : S.T}$}
      \end{prooftree}

      On peut construire une coercion de contextes de $`q = `r, d_1 : (`D, x : S) \sub `G, x : X'$. 

      Par le lemme \ref{narrowing-i-coercion} on obtient:
      \[\subimpl{`D, x : S}{c_2'}{Y}{Y'}\] de m�me
      taille que la d�rivation de $c_2$ tel que $c_2' \eqbr c_2[`q]$
      
      On obtient de m�me $\subimpl{`D}{c_1'}{X'}{X}$ avec $c_1' \eqbr c_1[`r]$.

      En utilisant une coercion de contextes partout l'identit�, on
      obtient par induction avec les d�rivations de $c_1[`r]$ et $d_1$
      d'une part et $c_2'$ et $d_2$ d'autre part, deux coercions
      $e_1, e_2$ telles que:
      \[\subimpl{`D}{e_1 \eqbr c_1' `o d_1}{S}{X}\] et 
      \[\subimpl{`D, x : S}{e_2 \eqbr d_2 `o c_2'}{Y}{T}\]
      
      On en d�duit:
      \begin{prooftree}
        \AXC{$\subimpl{`D}{e_1}{S}{X}$}
        \AXC{$\subimpl{`D, x : S}{e_2}{Y}{T}$}
        \BIC{$\subimpl{`D}{e = \lambda x : \ip{S}{`D}.e_2[\ctxdot~e_1[x]]}
          {\Pi x : X.Y}{\Pi x : S.T}$}
      \end{prooftree}

      On a bien $e \eqbr d `o c[`r]$:
      \[\begin{array}{ll}
        \firsteq{d `o c[`r]}

        \step{D�finition de $c$ et $d$}
        {=}{(\lambda x : \ip{S}{`D}.d_2[\ctxdot~d_1[x]]) `o (\lambda
          y : \ip{X'}{`G}.c_2[\ctxdot~c_1[y]])[`r]}
        
        \step{Composition des contextes}
        {=}{\lambda x : \ip{S}{`D}.d_2[(\lambda y :
          \ip{X'}{`G}.c_2[\ctxdot~c_1[y]])[`r]~d_1[x]]}

        \step{Substitution dans l'abstraction}
        {=}{\lambda x : \ip{S}{`D}.d_2[(\lambda y :
          \ip{X'}{`G}[`r].c_2[\ctxdot~c_1[y]][`r])~d_1[x]]}
        
        \step{R�duction}
        {"->"_{\beta}}
        {\lambda x :
          \ip{S}{`D}.d_2[(c_2[\ctxdot~c_1[y]][`r])[d_1[x]/y]]}
        
        \step{$y `; `r$}
        {=}
        {\lambda x :
          \ip{S}{`D}.d_2[(c_2[`r][\ctxdot~c_1[`r][y]])[d_1[x]/y]]}
        
        \step{D�finition de $`q$}
        {=}{\lambda x : \ip{S}{`D}.d_2[c_2[`q][\ctxdot~c_1[`q][d_1[x]]]]}

        \step{$d_2 `o c_2[`q] \eqbr e_2$}
        {\eqbr}{\lambda x : \ip{S}{`D}.e_2[\ctxdot~c_1[`q][d_1[x]]]}

        \step{$c_1[`q] = c_1[`r]$}
        {\eqbr}{\lambda x : \ip{S}{`D}.e_2[\ctxdot~e_1[x]]}
        
        \step{D�finition}
        {=}{e}
      \end{array}\]
      
      \case{SubProof}
      Ici $U = \mysubset{y}{U'}{P}$ et la d�rivation commence
      par:
      \begin{prooftree}
        \AXC{$\subimpl{`D}{e}{\Pi x : X'.Y'}{U'}$}
        \UIC{$\subimpl{`D}{d = \elt{\ip{U'}{`D}}
            {\ip{\lambda x : U'.P}{`D}}{e}
            {\ex{\ipG{`D}}{\ip{P}{`D}[e/x]}}}
          {\Pi x : X'.Y'}{\mysubset{y}{U'}{P}}$}
      \end{prooftree}

      Par induction on a $\subimpl{`D}{f \eqbr e `o c[`r]}{\Pi x : X.Y}{U'}$.
      On peut donc d�river:
      \begin{prooftree}
        \AXC{$\subimpl{`G}{f}{\Pi x : X.Y}{U'}$}
        \UIC{$\subimpl{`G}{d' = \elt{\ip{U'}{`G}}
            {\ip{\lambda x : U'.P}{`G}}{f}
            {\ex{\iG}{\ip{P}{`G}[f/x]}}}
          {\Pi x : X.Y}{U}$}
      \end{prooftree}
      
      On peut v�rifier que $d' \eqbr d `o c[`r]$:

      \begin{eqnarray*}
        d `o c[`r] & = & (\elt{\ip{U'}{`G}}
        {\ip{\lambda x : U'.P}{`G}}{e}
        {\ex{\iG}{\ip{P}{`G}[e/x]}})[c[`r]] \\
        & = & \elt{\ip{U'}{`G}}
        {\ip{\lambda x : U'.P}{`G}}{e[c[`r]]}
        {\ex{\iG}{\ip{P}{`G}[e[c[`r]]/x]}} \\
        & \eqbr & \elt{\ip{U'}{`G}}
        {\ip{\lambda x : U'.P}{`G}}{f}
        {\ex{\iG}{\ip{P}{`G}[f/x]}} \\
        & = & d'
      \end{eqnarray*}
      
    \end{induction}
    
    \case{SubSigma}\quad
    De fa�on �quivalente � \irule{SubProd}, on fait le cas si
    \irule{SubSigma} est utilis�e � la pr�misse droite.

    \begin{prooftree}
      \AXC{$\subimpl{`G}{c}{T}{\Sigma x : X'.Y'}$}
      \AXC{$\subimpl{`D}{d_1}{X'}{X}$}
      \AXC{$\subimpl{`D, x : X'}{d_2}{Y'}{Y}$}
      \BIC{$\subimpl{`D}{d = (d_1[\pi_1~\ctxdot], d_2[\pi_1~\ctxdot/x][\pi_2~\ctxdot])}{\Sigma x : X'.Y'}{\Sigma x : X.Y}$}
      \noLine\BIC{}
    \end{prooftree}

    Par induction sur la d�rivation de $\subimpl{`G}{c}{T}{\Sigma x
      : X'.Y'}$:
    \begin{induction}
      \case{SubSigma}
      On a:
      \begin{prooftree}
        \AXC{$\subimpl{`G}{c_1}{S}{X'}$}
        \AXC{$\subimpl{`G, x : S}{c_2}{T}{Y'}$}
        \BIC{$\subimpl{`G}{c = (c_1[\pi_1~\ctxdot], c_2[\pi_1~\ctxdot/x][\pi_2~\ctxdot])}{\Sigma x
            : S.T}{\Sigma x : X'.Y'}$}
      \end{prooftree}
      
      Par affaiblissement pour les coercions (lemme
      \ref{narrowing-i-coercion}) on a $\subimpl{`D}{c_1'}{S}{X'}$ avec $c_1' \eqbr c_1[`r]$.
      On peut aussi construire la coercion de contexte
      $`q = `r, \ctxdot : (`D, x : S) \sub (`G, x : S)$. Par le m�me lemme on obtient
      $\subimpl{`D, x : S}{c_2'}{T}{Y'}$ avec $c_2' \eqbr c_2[`q] =
      c_2[`r]$. On peut enfin construire le jugement
      $\subimpl{`D, x : S}{d_2'}{Y'}{Y}$ avec $d_2' \eqbr d_2[c_1'[x]/x]$.
      
      Par induction en utilisant une coercion de contexte partout
      l'identit� on a donc:
      \begin{prooftree}
        \AXC{$\subimpl{`D}{e_1 \eqbr d_1 `o c_1[`r]}{S}{X}$}
        \AXC{$\subimpl{`D, x : S}{e_2 \eqbr d_2' `o c_2'}{T}{Y}$}
        \BIC{$\subimpl{`D}{e = (e_1[\pi_1~\ctxdot],
            e_2[\pi_1~\ctxdot/x][\pi_2~\ctxdot])}
          {\Sigma x : S.T}{\Sigma x : X.Y}$}
      \end{prooftree}
      

      On peut v�rifier:
      \[\begin{array}{ll}
        \firsteq{d `o c[`r]}

        \step{D�finition de $c$}
        {=}{(d_1[\pi_1~c[`r]], d_2[\pi_1~c[`r]/x][\pi_2~c[`r]])}
        
        \step{R�duction}
        {"->"_\rho}{(d_1[c_1[`r][\pi_1~\ctxdot]], d_2[(c_1[\pi_1~\ctxdot])[`r]/x]
          [c_2[`r][\pi_1~\ctxdot/x][\pi_2~\ctxdot]])}

        \step{D�finition de $e_1$}
        {\eqbr}
        {(e_1[\pi_1~\ctxdot], d_2[c_1[`r][\pi_1~\ctxdot]/x][c_2[`r][\pi_1~\ctxdot/x][\pi_2~\ctxdot]])}
        
        \step{D�finition de $c_1'$}
        {\eqbr}
        {(e_1[\pi_1~\ctxdot], d_2[c_1'[\pi_1~\ctxdot]/x][c_2[`r][\pi_1~\ctxdot/x][\pi_2~\ctxdot]])}
        

        \step{D�finition de $d_2'$}
        {\eqbr}
        {(e_1[\pi_1~\ctxdot],
          d_2'[\pi_1~\ctxdot/x][c_2[`r][\pi_1~\ctxdot/x][\pi_2~\ctxdot]])}

        \step{$x `; c_2[\pi_1~\ctxdot/x]$}
        {\eqbr}
        {(e_1[\pi_1~\ctxdot], d_2'[c_2[`r][\pi_1~\ctxdot/x][\pi_2~\ctxdot]][\pi_1~\ctxdot/x])}
        
        \step{D�finition de $c_2'$}
        {\eqbr}
        {(e_1[\pi_1~\ctxdot], d_2'[c_2'[\pi_1~\ctxdot/x][\pi_2~\ctxdot]][\pi_1~\ctxdot/x])}
        
        \step{D�finition de $e_2$}
        {\eqbr}
        {(e_1[\pi_1~\ctxdot],
          e_2'[\pi_1~\ctxdot/x][\pi_2~\ctxdot][\pi_1~\ctxdot/x])}

        \step{Deuxi�me substitution inutile}
        {=}{(e_1[\pi_1~\ctxdot], e_2[\pi_1~\ctxdot/x][\pi_2~\ctxdot])}

        \step{D�finition de $e$}
        {=}{e}
      \end{eqnarray*}
      
      \case{SubSub}
      On a: 
      \begin{prooftree}
        \AXC{$\subimpl{`G}{c'}{T}{\Sigma x : X'.Y'}$}
        \UIC{$\subimpl{`G}{c = c' `o \pi_1}{\mysubset{y}{T}{P}}{\Sigma x : X'.Y'}$}
      \end{prooftree}
      
      Par induction, $\subimpl{`G}{f \eqbr d `o c'}{T}{\Sigma x :
        X.Y}$, on a donc:
      \begin{prooftree}
        \AXC{$\subimpl{`G}{f \eqbr d `o c'}{T}{\Sigma x : X.Y}$}
        \UIC{$\subimpl{`G}{g = f `o \pi_1}{\mysubset{y}{T}{P}}{\Sigma x : X.Y}$}
      \end{prooftree}
      
      On a bien: $g = f `o \pi_1 \eqbr d `o c' `o \pi_1 \eqbr d `o c$.

    \end{induction}
    
    
    \case{SubProof}\quad
    \begin{prooftree}
      \AXC{$\subimpl{`G}{e}{S}{T}$}
      \UIC{$\subimpl{`G}{c = (\lambda t : S.~\elt{T}{(\lambda x :
            T.P)}{(e~t)}{?_{P[e~t/x]}})}
        {\hnf{S}}{\mysubset{x}{T}{P}}$}
      \AXC{$\subimpl{`G}{d}{\mysubset{x}{T}{P}}{\hnf{U}}$}
      \BIC{$$} 
    \end{prooftree}
    
    Si $\subimpl{`G}{d}{\mysubset{x}{T}{P}}{\hnf{U}}$ alors
    $\subimpl{`G}{e'}{T}{\hnf{U}}$ est
    d�rivable par une d�rivation plus petite et $d = e' `o \pi_1$.
    Par induction, avec l'hypoth\`ese $\subimpl{`G}{e}{\hnf{S}}{T}$, il existe $f$ tel que
    $\subimpl{`G}{f}{\hnf{S}}{\hnf{U}}$ est d�rivable par une d�rivation
    n'utilisant pas \irule{SubTrans} et $f \eqbr e' `o e$. 
    On a bien $f \eqbr e' `o e \eqbr d `o c \eqbr e' `o \pi_1 `o c$ car
    \begin{eqnarray*}
      e' `o \pi_1 `o c & = & e' `o \pi_1 `o (\lambda t : S.~\elt{T}{(\lambda x :
        T.P)}{(e~t)}{?_{P[e~t/x]}}) \\
      & = &
      \lambda x.e'~(\pi_1~((\lambda t : S.~\elt{T}{(\lambda x :
        T.P)}{(e~t)}{?_{P[e~t/x]}})~x)) \\
      & "->"_{\beta} &
      \lambda x.e'~(\pi_1~(\elt{T}{(\lambda x :
        T.P)}{(e~x)}{?_{P[e~x/x]}})) \\
      & "->"_{\beta} & \lambda x.e'~(e~x) \\
      & = & e' `o e
    \end{eqnarray*}      
    
    \case{SubSub}\quad
    \begin{prooftree}
      \AXC{$\subimpl{`G}{c}{S}{\hnf{T}}$}
      \UIC{$\subimpl{`G}{d = c `o \pi_1}
        {\mysubset{x}{S}{P}}{\hnf{T}}$}
      \AXC{$\subimpl{`G}{e}{\hnf{T}}{\hnf{U}}$}
      \BIC{$$}
    \end{prooftree}
    
    Par induction il existe une d�rivation de $\subimpl{`G}{f \eqbr e `o c}
    {S}{U}$ n'utilisant pas \irule{SubTrans}. On peut donc d�river:

    \begin{prooftree}
      \AXC{$\subimpl{`G}{f}{S}{U}$}
      \UIC{$\subimpl{`G}{f `o \pi_1}
        {\mysubset{x}{S}{P}}{U}$}
    \end{prooftree}
    
    On a bien $f `o \pi_1 \eqbr e `o c `o \pi_1$.
    
  \end{induction}
\end{proof}


On peut maintenant �noncer le lemme de symm�trie:

\begin{lemma}[Sym�trie de la coercion]
  \label{subi-sym}
  S'il existe $c$ tel que $`G \typec c : A \subi B$ alors $`E!c^{-1}, `G
  \typec c^{-1} : B \subi A$ et $c^{-1} `o c \eqbre \sref{id}~A$ et $c
  `o c^{-1} \eqbre \sref{id}~B$, o� $\sref{id} \coloneqq \lambda X :
  Set.\lambda x : X. x$.
\end{lemma}
\begin{proof}
  On sait que le jugement $\subi$ est symm�trique, c'est � dire qu'on a
  l'existence des inverses. On utilise la transitivit� pour montrer le
  reste du lemme. Si $\subimpl{`G}{c}{A}{B}$ et
  $\subimpl{`G}{c^{-1}}{B}{A}$ alors il existe $f$ et $f^{-1}$ tel que
  $\subimpl{`G}{f \eqbr c^{-1} `o c}{A}{A}$ et $\subimpl{`G}{f^{-1}
    \eqbr c `o c^{-1}}{B}{B}$. Par unicit� des coercions, $f \eqbre
  \lambda x : A.x$ et $f^{-1} \eqbre \lambda x : B.x$. En g�n�ral $f$ et
  $f'$ sont en forme $\eta$-longue et pas l'identit�.
\end{proof}

%%% Local Variables: 
%%% mode: latex
%%% TeX-master: "subset-typing"
%%% LaTeX-command: "TEXINPUTS=\"style:$TEXINPUTS\" latex"
%%% End: 


%%% Local Variables: 
%%% mode: latex
%%% TeX-master: "subset-typing"
%%% LaTeX-command: "TEXINPUTS=\"style:$TEXINPUTS\" latex"
%%% End: 
