
\begin{lemma}[Unicit� de la coercion]
  \label{coercion-unicity}
  Si $`G \typea T, U : s$ alors si $\subimpl{`G}{c}{T}{U}$ et
  $\subimpl{`G}{c'}{T}{U}$ alors $c = c'$.
\end{lemma}

\begin{proof}
  Par induction lexicographique sur la paire de d�rivations de $c$ et $c'$.
  Seule la r�gle \irule{SubHnf} peut s'appliquer n'importe quand. 
  Supposons que la d�rivation de $c$ se termine par une application de \irule{SubHnf}.
  Quelque soit la derni�re r�gle appliqu�e dans la d�rivation de $c'$,
  on a $T = \hnf{T}$ et $U = \hnf{U}$. Par induction on a donc $c = c'$.
\end{proof}

On peut donc raisonner comme ceci: si l'on parvient � construire une
d�rivation de $\subimpl{`G}{c}{T}{U}$ alors c'est l'unique d�rivation
existante.

\begin{lemma}[Coercion et formes normales de t�te]
  \label{substi-coercion-hnf}
  Si $`G \typec c : T \suba U$ alors $`G \typec c' : \hnf{T}~\suba
  \hnf{U}$ avec $c \eqbr c'$ est d�rivable par une d�rivation plus
  petite ou �gale.
\end{lemma}

\begin{proof}
  Par idempotence de la mise en forme normale de t�te, on a la m�me
  d�rivation dans le cas ou la derni�re r�gle appliqu�e �tait
  \irule{SubHnf}, sinon c'est trivial.
\end{proof}

\begin{lemma}[Coercion de termes convertibles]
  \label{subti-eqb-coercion-eqbe-id}
  Si $`G \typea T, U : s$ et $T \eqbr U$ alors il existe $c$, $`G \typec
  c : T \suba U$ avec $c \eqbre \ctxdot$.
\end{lemma}

\begin{proof}
  Par induction sur la forme normale de $T$ not�e $\nf{T}$.

  \begin{itemize}
  \item $\nf{T} = \Pi y : A.B$ Alors, $\nf{U} = \Pi y : A'.B'$ avec $A
    \eqbr A'$, $B \eqbr B'$. Par induction, $c_1 \eqbre \ctxdot: A' \sub A$
    et $c_2 \eqbre \ctxdot : B \sub B'$. On a donc:
    \begin{eqnarray*}
      c & = & \lambda x : \ip{A'}{`G}. c_2[\ctxdot~(c_1[x]) \\
      & \eqbre & \lambda x : \ip{A'}{`G}. \ctxdot~x \\
      & \eqbre & \ctxdot
    \end{eqnarray*}
    De fa�on �quivalente pour $\Sigma, \{|\}$.
    
  \item Si $\nf{T}$ n'a pas pour symbole de t�te, $\Pi, \Sigma$ ou
    $\{|\}$, alors il est possible que $U$ est pour symbole de t�te un
    type sous-ensemble auquel cas le m�canisme est similaire.  Dans tout
    les autres cas, \irule{SubConv} est la seule r�gle applicable et on
    a $c = \ctxdot$.
  \end{itemize}
\end{proof}

\def\GD{`G, x : U, `D}
\def\Gr{`G, `D[u/x]}
\def\iGD{\ipG{\GD}}
\def\iGr{\ipG{\Gr}}


\begin{lemma}[Coercion de sortes]
  \label{subti-coercion-sorts}
  Si $`G \typec e : s \subi T$ ou $`G \typec e : T \subi s$ alors $T
  \eqbr s$ et $e = []$.
\end{lemma}
\begin{proof}  
  Clairement on ne peut d�river $s \suba T$ que par \irule{SubConv}
  (�ventuellement pr�c�d� de \irule{SubHnf}). En effet seule la r�gle
  \irule{SubProof} pourrait s'appliquer, mais cela impliquerait que
  $T \eqbr \mysubset{x}{U}{P}$ avec $s \suba U$ et ainsi de suite. La
  seule possibilit� est de d�river $s \eqbr T$ ou $s \eqbr U$, auquel cas $U$ est une
  sorte ce qui contredit le fait que $\mysubset{x}{U}{P} : s$ dans le
  cas pr�c�dent. On d�rive donc $s \subi T$ si et seulement si $s
  \eqbr T$.
\end{proof}

\begin{lemma}[Stabilit� de la coercion par substitution]
  \label{subti-coercion-subst}
  Si $\GD \typea T, T' : s$, $`G \typec u :
  U$, alors
  $\begin{array}{lcl}
    \GD \typea t : T & "=>" & \ip{t[u/x]}{\Gr} \eqbre
    \ip{t}{\GD}[\ip{u}{`G}/x] \\
    \subimpl{\GD}{c}{T}{T'} & "=>" & \subimpl{\Gr}{c'}{T[u/x]}{T'[u/x]}
    `^ c' \eqbre c[\ip{u}{`G}/x]
  \end{array}$
  
\end{lemma}

\begin{proof}
  Par induction mutuelle sur les d�rivation de $c$ et $t$.

  \begin{induction}
    \case{SubHnf} On a: 
    \begin{prooftree}
      \AXC{$\subimpl{\GD}{c}{\hnf{T}}{\hnf{T'}}$}
      \UIC{$\subimpl{\GD}{c}{T}{T'}$}
    \end{prooftree}
    
    \def\as{\overrightarrow{a}}    
    \def\bs{\overrightarrow{b}}    
    \def\asux{\overrightarrow{a[u/x]}}
    \def\bsux{\overrightarrow{b[u/x]}}
    \def\ipux{[\ip{u}{`G}/x]}

    Par induction on a
    $\subimpl{\Gr}{c'}{\hnf{T}[u/x]}{\hnf{T'}[u/x]}$ avec $c' \eqbre
    c\ipux$.
    Si $\hnf{T[u/x]} = \hnf{T}[u/x]$ et $\hnf{T'[u/x]} = \hnf{T'}[u/x]$
    c'est direct par induction. Sinon, on a $\hnf{T} = x~\as$ ou $\hnf{T'} =
    x~\as$. Les deux cas sont similaires, on traite le cas ou $\hnf{T} =
    x~\as$. Le jugement $x~\as \suba \hnf{T'}$
    ne peut �tre d�riv� que par \irule{SubProof} ou
    \irule{SubConv}. Dans le premier cas, cela implique qu'on a une
    d�rivation de $d : x~\as \suba U'$ o� $\hnf{T'} = \mysubset{y}{U'}{P}$. Par
    induction on a donc une d�rivation de
    $\subimpl{\Gr}{d'}{u~\asux}{U'[u/x]}$ avec $d ' \eqbre
    d\ipux$. 
    Par le lemme \ref{substi-coercion-hnf} on a donc une d�rivation de 
    $\subimpl{\Gr}{d'}{\hnf{(u~\asux)}}{\hnf{(U'[u/x])}}$ avec $d' \eqbre
    d\ipux$. 
    
    \begin{prooftree}
      \AXC{$\subimpl{\Gr}{d'}{\hnf{(u~\asux)}}{\hnf{(U'[u/x])}}$}
      \UIC{$\subimpl{\Gr}{d'}{\hnf{(u~\asux)}}{U'[u/x]}$}
      \UIC{$\subimpl{\Gr}{c'}{\hnf{T[u/x]}}{\hnf{(T'[u/x])} = \mysubset{y}{U'[u/x]}{P[u/x]}}$}
      \UIC{$\subimpl{\Gr}{c'}{T[u/x]}{T'[u/x]}$}
    \end{prooftree}
    
    Par induction, on a $\ip{U'[u/x]}{\Gr} \eqbr \ip{U'}{\GD}\ipux$ et
    $\ip{P[u/x]}{\Gr} = \ip{P}{\GD}\ipux$. 
    On en d�duit:
    \begin{eqnarray*}
      \ip{P[u/x]}{\Gr}[d'/y] & 
      \eqbre & \ip{P}{\GD}\ipux[d'/y] \\
      & \eqbre & \ip{P}{\GD}\ipux[d\ipux~t/y] \\
      & = & \ip{P}{\GD}[d/y]\ipux
    \end{eqnarray*}
    
    Soit $A = \ipG{\Gr} \typec \ip{P[u/x]}{\Gr}[d'/y]$ et
    $B = \ipG{\GD} \typec \ip{P}{\GD}[d/y]$. On a
    $A \eqbr B\ipux$ car $\ipG{\Gr} \eqbr \ipG{\GD}\ipux = \ipG{`G}, \ipG{`D}\ipux$ 
    par application de l'hypoth�se d'induction sur chaque �l�ment de
    $`D$ et les conclusion des existentielles sont convertibles par le r�sultat pr�c�dent.
    
    On a donc:
    \begin{eqnarray*}
      c' & = &
      \elt{\ip{U'[u/x]}{\Gr}}
      {\ip{(\lambda y : (U'[u/x]).P[u/x])}{\Gr}}
      {d'}
      {?_A} \\
      & \eqbre &
      \elt{\ip{U'[u/x]}{\Gr}}
      {\ip{(\lambda y : (U'[u/x]).P[u/x])}{\Gr}}
      {d\ipux}
      {?_A} \\
      & \eqbre &
      \elt{\ip{U'}{\GD}\ipux}
      {\ip{(\lambda y : U'.P)}{\GD}\ipux}
      {d\ipux}
      {?_A} \\
      & \eqbre &
      (\elt{\ip{U'}{\GD}}{\ip{\lambda y : U'.P}{\GD}}{d}
      {?_B})\ipux \\
      & = & c\ipux
    \end{eqnarray*}
    
    Dans le cas ou l'on applique directement \irule{SubConv} cela
    implique que $\hnf{T} = x = \hnf{T'}$ donc par reflexivit� de la
    coercion \ref{subti-reflexive}, $\subimpl{\Gr}{c'}{u}{u}$ est d�rivable et $c' \eqbre
    [] \eqbre []\ipux$.
      
    \case{SubConv}
    On a $T \eqbr T'$, donc $c \eqbre []$. D'autre part, $T[u/x] \eqbr T'[u/x]$ par substitutivit� de
    la $\beta\rho$-�quivalence. Par le lemme \ref{subti-eqb-coercion-eqbe-id}, on
    sait qu'il existe une coercion $c'$ telle que le jugement $\Gr
    \typec c' : T[u/x] \sub T'[u/x]$ est d�rivable et $c' \eqbre []$.
    On a bien $c' \eqbre c\ipux$.
    
    \case{SubProd}
    On a:
    \begin{prooftree}
      \AXC{$\subimpl{\GD}{c_1}{A'}{A}$}
      \AXC{$\subimpl{\GD, y : A'}{c_2}{B}{B'}$}
      \BIC{$\subimpl{\GD}
        {\lambda x : \ip{A'}{\GD}.~c_2[[]~c_1[x]]}
        {\Pi y : A.B}{\Pi y : A'.B'}$}
    \end{prooftree}
    
    Par induction et application de \irule{SubProd}:
    \begin{prooftree}
      \AXC{$\subimpl{\Gr}{c_1'}{A'[u/x]}{A[u/x]}$}
      \AXC{$\subimpl{\Gr, y : A'[u/x]}{c_2'}{B[u/x]}{B'[u/x]}$}
      \BIC{$\subimpl{\Gr}{c'}
        {\Pi y : A[u/x].B[u/x]}{\Pi y : A'[u/x].B'[u/x]}$}
    \end{prooftree}
    
    o� \[c' = \lambda x : \ip{A'[u/x]}{\Gr}.~c_2'[[]~c_1'[x] `^ c_1'
    \eqbre c_1\ipux `^ c_2' \eqbre c_2\ipux\]
    Par induction, $\ip{A'[u/x]}{\Gr} \eqbre \ip{A'}{\GD}\ipux$,
    on a donc bien $c' \eqbre c\ipux$.
   
    \casethree{SubSigma}{SubProof}{SubSub} Idem, direct par induction.
  
    \case{PropSet} Trivial.
    
    \case{Var} 
    On a:
    \typenva
    \begin{prooftree}
      \BAX{}
      {$\wf \GD$}
      {$y : T `: \GD$}
      {$\GD \seq y : T$}
      {}
    \end{prooftree}    
    \typenvi

    \begin{itemize}
    \item Si $x "/=" y$, alors par d�finition de l'interpr�tation,
      on doit montrer
      $\ip{y[u/x]}{\Gr} = y = \ip{y}{\GD}\ipux$.
      
    \item Sinon, $t[u/x] = u$ et $\ip{u}{\Gr} = \ip{x}{\GD}\ipux =
      \ip{u}{\GD}$. On utilise ici le fait que $u$ est bien typ� dans
      l'environnement $`G$, donc dans toute extension bien form�e de cet
      environnement par affaiblissement.      
    \end{itemize}
    
    \case{App}\quad
    \typenva
    \begin{prooftree}
      \TAX{}
      {$\GD \seq f : F \quad \mualgo(F) = \Pi y : A. B : s$}
      {$\GD \seq e : E \quad `G \seq E, A : s$}
      {$E \sub A$}
      {$\GD \seq (f~e) : B [ e / y ]$}
      {}
    \end{prooftree}
    \typenvi

    Par induction:
    \[\ip{f[u/x]}{\Gr} \eqbr \ip{f}{\GD}\ipux\]
    \[\ip{e[u/x]}{\Gr} \eqbr \ip{e}{\GD}\ipux\]
    
    \def\afe{`a}    
    Par d�finition de la substitution et de l'interpr�tation, 
    \begin{eqnarray*}
      \ip{(f~e)}{\GD}
      & = & (((\pi_F~\ip{f}{\GD})~(c_e~\ip{e}{\GD})) \\
      \text{ o� } & & \\
      \pi_F & = & \coerce{\GD}{F}{(\Pi y : A.B)} \\
      c_e & = & \coerce{\GD}{E}{A}.
    \end{eqnarray*}
    
    Clairement, $\pi_F$ et $c_e$ sont d�rivables, puisqu'on part de
    jugements d�rivables par la coercion algorithmique.

    On a la substitutivit� du typage \ref{substitutive-typing} donc le
    jugement substitu� est:
    \typenva
    \begin{prooftree}
      \TAX{App}
      {$\Gr \seq f[u/x] : F[u/x] \quad \mualgo(F[u/x]) = \Pi y : A[u/x]. B[u/x] : s$}
      {$\Gr \seq e[u/x] : E[u/x] \quad `G \seq E[u/x], A[u/x] : s$}
      {$E[u/x] \sub A[u/x]$}
      {$\Gr \seq (f~e)[u/x] : B' [ e[u/x] / y ]$}
      {}
    \end{prooftree}
       
    Soit $e' = e[u/x]$ et $f' = f[u/x]$, on a donc d'autre part:
    \begin{eqnarray*}
      \ip{(f~e)[u/x]}{\Gr}
      & = & \ip{f'~e'}{\Gr} \\
      & = & \pi_{F[u/x]}[\ip{f'}{\Gr}]~c_{e'}[\ip{e'}{\Gr}] \\
      \text{ o� } & & \\
      \pi_{F[u/x]} & = & \coerce{\Gr}{F[u/x]}{(\Pi y : A[u/x].B[u/x])} \\
      c_{e'} & = & \coerce{\Gr}{E[u/x]}{A[u/x]}
    \end{eqnarray*}
    
    Par induction, il existe des coercions $d$, $e$ telles que:
    $\subimpl{\Gr}{d}{F[u/x]}{(\Pi y : A.B)[u/x]}$
    et $\subimpl{\Gr}{e}{E[u/x]}{A[u/x]}$ avec
    $d \eqbre \pi_F\ipux$ et $e \eqbre c_e\ipux$. Par unicit� des
    coercions on en d�duit que $d = \pi_{F[u/x]}$ et $e = c_{e'}$.

    \[\begin{array}{ll}
      \firsteq{\ip{f~e}{\GD}\ipux}
      
      \step{D�finition de l'interpr�tation}
      {=}(\pi_F[\ip{f}{\GD}])~(c_e[\ip{e}{\GD}])\ipux 

      \step{D�finition de la substitution}
      {=}{(\pi_F\ipux[\ip{f}{\GD}\ipux])~(c_e\ipux[\ip{e}{\GD}\ipux])}

      \step{Application de l'hypoth�se d'induction pour les termes}
      {\eqbre}{(\pi_F\ipux[\ip{f'}{\Gr}])~(c_e\ipux[\ip{e'}{\Gr}])}

      \step{Application de l'hypoth�se d'induction pour les coercions}
      {\eqbre}{(d[\ip{f'}{\Gr}])~e[\ip{e'}{\Gr}]}

      \step{Unicit� des coercions}
      {\eqbre}{\pi_{F[u/x]}[\ip{f'}{\Gr}]~c_{e'}[\ip{e'}{\Gr}]}
      
      \step{D�finition de l'interpr�tation}
      {\eqbre}{\ip{(f~e)[u/x]}{\Gr}}
    \end{array}\]
    
    \casethree{Prod}{Sigma}{Subset} Par induction.

    \case{Abs}
    \typenva
    On a:
    \begin{prooftree}
      \AXC{$\GD \seq \Pi y : T. U : s $}
      \AXC{$\GD, y : T \seq M : U $}
      \BIC{$\GD \seq \lambda y : T. M : \Pi y : T.U$}
    \end{prooftree}
    
    On a bien:
    \[\begin{array}{ll}
      \firsteq{\ip{\lambda y : T.M}{\GD}\ipux}
      
      \step{D�finition de l'interpr�tation}
      {=}{\lambda y : \ip{T}{\GD}\ipux.\ip{M}{\GD, y : T}\ipux}

      \step{Application de l'hypoth�se de r�currence}
      {\eqbre}{\lambda y : \ip{T[u/x]}{\Gr}.\ip{M[u/x]}{\Gr, y : T[u/x]}}

      \step{D�finition de l'interpr�tation}
      {=}{\ip{\lambda y : T[u/x].M[u/x]}{\Gr}}
    \end{array}\]
      
    \casetwo{LetSum}{SumDep} \TODO{Par induction}


  \end{induction}
\end{proof}

On va maintenant �tendre la relation de coercion aux contextes de
mani�re canonique.
\typenva
\begin{definition}[Coercion de contextes]
  \label{coercion-ctx}
  On d�finit inductivement la coercion de deux contextes de coercions
  algorithmiques par les r�gles suivantes:
  \begin{itemize}
  \item $[] \sub []$
  \item $`G, x : T \sub `G', x : T'$ si $`G \sub `G'$ et $T \sub T'$.
  \end{itemize}
\end{definition}

De m�me pour les coercions explicites d�riv�es par le jugement $`G
\typec c : T \suba S$.
\begin{definition}[Coercion explicites de contextes]
  \label{coercion-ctx-i}
  On d�finit inductivement la coercion de deux contextes de coercions
  explicites par les r�gles suivantes:
  \begin{itemize}
  \item $[] \sub []$
  \item $`r, c : `G, x : T \sub `G', x : T'$ si $`r : `G \sub `G'$ et 
    $`G \typec c : T \sub T'$.
  \end{itemize}
\end{definition}

Clairement toute coercion de contexte algorithmique correspond � une
coercion de contexte explicite et vice-versa.

\begin{definition}[Extension de la substitution aux coercions de
  contextes]
  \label{subst-coercion-ctx}
  On d�finit la substitution d'une coercion de contexte inductivement:
  \begin{itemize}
  \item $t[[]] = t$
  \item $t[`r, c : `G, x : T \sub T'] = t[`r : `G][c[x]/x]$
  \end{itemize}
\end{definition}

\begin{lemma}[Stabilit� par affaiblissement]
  \label{narrowing-i-coercion}
  Si $`r : `D \sub `G$, $`G \typec c : T \subi T'$ et $\ip{T'}{`D} \eqbre
  \ip{T'}{`G}[`r]$, alors $`D \typec c' : T \sub T'$ et $c' \eqbre c[`r]$.
\end{lemma}

\begin{proof}
  Par induction sur la d�rivation de coercion:

  \begin{induction}
    \case{SubHnf} 
    On a $\subimpl{`G}{c}{\hnf{T}}{\hnf{T'}}$. Par induction, 
    $\subimpl{`D}{c[`r]}{\hnf{T}}{\hnf{T'}}$. 
    On peut donc d�river $`D \typec c[`r] : T \sub T'$ par \irule{SubHnf}.

    \case{SubConv}
    On a $T \eqbre T'$ et $c = []$. 
    On peut donc d�river $\subimpl{`D}{c' = []}{T}{T'}$,
    on a bien $c' = c[`r]$.

    \case{SubProd}
    On a:
    \begin{prooftree}
      \BAX{}
      {$\subimpl{`G}{c_1}{C}{A}$}
      {$\subimpl{`G, x : C}{c_2}{B}{D}$}
      {$\subimpl{`G}{\lambda x : \ip{C}{`G}.~c_2[[]~c_1[x]]}
        {\Pi x : A.B}{\Pi x : C.D}$}
      {}
    \end{prooftree}

    Par induction on a $\subimpl{`D}{c_1[`r]}{C}{A}$. On peut d�finir
    la coercion $`s = `r, [] : `D, x : C \sub `G, x : C$ et obtenir par
    induction: $\subimpl{`D, x : C}{c_2'}{B}{D}$ avec $c_2' \eqbre c_2[`s]$.
    
    On peut alors appliquer \irule{SubProd} pour obtenir:
    \[\subimpl{`D}
    {c' = \lambda x : \ip{C}{`D}.~c_2[`s][[]~c_1[`r][x]]}{\Pi x : A.B}{\Pi x : C.D}\]
    
    On a:
    \[\begin{array}{ll}
      \firsteq{(\lambda x : \ip{C}{`G}.~c_2[[]~c_1[x]])[`r]}

        \step{D�finition de la substitution}
        {=}{\lambda x : \ip{C}{`G}[`r].(c_2[`r])[[]~(c_1[`r])[x]]}

        \step{Application de l'hypoth�se de r�currence}
        {\eqbre}{\lambda x : \ip{C}{`D}.(c_2[`r])[[]~(c_1[`r])[x]]}
        
        \step{Coercion identit� dans $`s$}
        {=}{\lambda x : \ip{C}{`D}.(c_2[`s])[[]~(c_1[`r])[x]]}
      \end{array}
      \]
      
    \case{SubSigma}\quad
    \begin{prooftree}
      \BAX{}
      {$\subimpl{`G}{c_1}{A}{C}$}
      {$\subimpl{`G, x : A}{c_2}{B}{D}$}
      {$\subimpl{`G}{t}{(c_1[\pi_1~[]], c_2~[\pi_2~[]])}
        {\Sigma x : A.B}{\Sigma x : C.D}$}
      {}
    \end{prooftree}
    Par induction on a {$\subimpl{`D}{c_1[`r]}{A}{C}$}. On peut d�finir
    la coercion $`s = `r, [] : `D, x : A \sub `G, x : A$ et obtenir par
    induction:
    $\subimpl{`D, x : A}{c_2[`s]}{B}{D}$
    
    On peut alors appliquer \irule{SubSigma} pour obtenir:
    \[\subimpl{`G}
    {c' = ((c_1[`r])[\pi_1~[]], (c_2[`s])[\pi_2~[]])}
    {\Sigma x : A.B}{\Sigma x : C.D}\]
    
    On a:
    \[\begin{array}{ll}
      \firsteq{
        ((c_1[\pi_1~[]], c_2[\pi_2~[]]))[`r]}
      
        \step{D�finition de la substitution}
        {=}{((c_1[`r])[\pi_1~[]], (c_2[`r])[\pi_2~[]])}

        \step{Coercion identit� dans $`s$}
        {=}{((c_1[`r])[\pi_1~[]], (c_2[`s])[\pi_2~[]])}
      \end{array} 
      \]
      
   \case{SubProof}\quad
   \begin{prooftree}
     \UAX{}
     {$\subimpl{`G}{d}{T}{U}$}
     {$\subimpl{`G}
       {c = \elt{\ip{U}{`G}}
         {\ip{(\lambda x : U.P)}{`G}}
         {d}
         {\tex{\ipG{`G}}{\ip{P}{`G, x : U}[d/x]}}}
       {T}{\mysubset{x}{U}{P}}$}
     {}
   \end{prooftree}

   Par induction, on a $\subimpl{`D}{d[`r]}{T}{U}$. On peut donc d�river:
   \[\subimpl{`D}{c' = \elt{\ip{U}{`D}}
     {\ip{(\lambda x : U.P)}{`D}}
     {d[`r]}
     {\tex{\ipG{`D}}{\ip{P}{`D, x : U}[d[`r]/x]}}}
   {T}{\mysubset{x}{U}{P}}\]
   
   On a bien $c[`r] \eqbr c'$, car:
   \begin{equations}
     \firsteq{(\ipG{`G} \vdash (\ip{P}{`G, x : U})[d/x])[`r]}
     
     \step{D�finition de la substitution, $x `; `G$}
     {=}{\ipG{`G}[`r] \vdash (\ip{P}{`G, x : U}[`r])[d[`r]/x]}

     \step{Hypoth�se de r�currence}
     {\eqbr}{\ipG{`D} \vdash (\ip{P}{`G, x : U}[`r])[d[`r]/x]}

     \step{$\ip{P}{`G, x : U}[`r] = \ip{P}{`D, x : U}$}
     {\eqbr}{\ipG{`D} \vdash \ip{P}{`D, x : U}[d[`r]/x]}
     
   \end{equations}
   
   \case{SubSub} Direct par induction.
  \end{induction}
\end{proof}

\begin{lemma}[Stabilit� par affaiblissement]
  \label{narrowing-i-term}
  Si $`r : `D \sub `G$, $`G \typec t : T$ alors 
  $`D \typec t : T'$ et il existe $`a$, $\subimpl{`D}{`a}{T'}{T}$ avec
  $\ip{t}{`G}[`r] \eqbre `a[\ip{t}{`D}]$.
\end{lemma}

\begin{proof}
  La premi�re partie du lemme se d�duit par applications r�p�t�es du
  lemme de restriction \ref{narrowing-a}.
  Par induction sur la d�rivation de typage:

  \begin{induction}
    \case{Var} 
    On a:
    \typenva
    \begin{prooftree}
      \BAX{Var}
      {$\wf `G$}
      {$y : T `: `G$}
      {$`G \seq y : T$}
      {}
    \end{prooftree}    
    
    Par d�finition de la coercion de contextes, il existe $c : y : T' \sub T
    `: `r$. Soit $`a = c$, on a bien: $\ip{y}{`G}[`r] = c[y] =
    `a[\ip{y}{`D}]$.
    
    \case{App}
    \typenva
    On a: 
    \begin{prooftree}
      \TAX{App}
      {$`G \seq f : F \quad \mualgo(F) = \Pi y : A. B$}
      {$`G \seq e : E \quad `G \seq E, A : s$}
      {$E \sub A $}
      {$`G \seq (f~e) : B [ e / y ]$}
      {}
    \end{prooftree}
    
    et donc par induction, $`D \seq f : F'$ avec
    $\subimpl{`D}{`a_f}{F'}{F}$ et
    $`D \seq e : E'$ avec $\subimpl{`D}{`a_e}{E'}{E}$.

    Par d�finition de l'interpr�tation, 
    \begin{eqnarray*}
      \ip{(f~e)}{`G}
      & = & (((\pi_F[\ip{f}{`G}])~(c_e[\ip{e}{`G}])) \\
      \text{ o� } & & \\
      \pi_F & = & \coerce{`G}{F}{(\Pi y : A.B)} \\
      c_e & = & \coerce{`G}{E}{A}.
    \end{eqnarray*}
    
    Clairement, $\pi_F$ et $c_e$ sont d�rivables, puisqu'on part de
    jugements d�rivables par la coercion algorithmique.

    De m�me:
    \begin{eqnarray*}
      \ip{f~e}{`D}
      & = & (\pi_{F'}[\ip{f'}{`D}])~(c_{e'}[\ip{e'}{`D}]) \\
      \text{ o� } & & \\
      \pi_{F'} & = & \coerce{`D}{F'}{(\Pi y : A'.B')} \\
      c_{e'} & = & \coerce{`D}{E'}{A'}
    \end{eqnarray*}

    On a la coercion $\subimpl{`G}{\pi_F}{F}{\Pi y : A.B}$ ; 
    par le lemme \ref{subti-coercion-subst} on obtient
    $\subimpl{`D}{\pi_F[`r]}{F}{\Pi y : A.B}$.
    On a donc les coercions suivantes:
    \[
    \xymatrix{
      F\ar[rr]_{\pi_F[`r]} & & 
      \Pi y : A.B \\
      F'\ar[u]^{`a_f}\ar[rr]^{\pi_{F'}} & &
      {\Pi y : A'.B'}\ar@{-->}[u]^{c_f}}
    \]

    Par symm�trie et transitivit� de la coercion, on en d�duit qu'il
    existe $c_f$, $\subimpl{`D}{c_f}{\Pi y : A'.B'}{\Pi y : A.B}$.
    La coercion $c_f$ est n�cessairement de la forme:
    $\lambda y : \ip{A}{`D}.c_2[[]~c_1[y]]$ o� 
    $\subimpl{`D}{c_1}{A}{A'}$ et $\subimpl{`D, x : A}{c_2}{B'}{B}$.

    Par le lemme \ref{subti-coercion-subst} on a
    $\subimpl{`D}{c_e[`r]}{E}{A}$.

    On a donc les coercions suivantes pour l'argument $e$:   
    \[
    \xymatrix{
      E\ar[rr]_{c_e[`r]} & & 
      A\ar[d]^{c_1} \\
      E'\ar[u]^{`a_e}
      \ar@{-->}[rr]^{c_{e'}} & & A'}
    \]
       

    Ce diagramme commute, il existe donc une coercion $c_{E',A} \eqbre
    c_e[`r] `o `a_e : E' \sub A$.    
    Par le lemme \ref{narrowing-i-coercion} on a donc
    $\subimpl{`D, x : E'}{c_2'}{B'}{B}$ avec $c_2' \eqbre
    c_2[(c_e[`r])[`a_e[x]]/x]$.

    Par le lemme \ref{subti-coercion-subst} avec $`D \typec e : E'$ on obtient:
    $\subimpl{`D}{c_2''}{B'[e/x]}{B[e/x]}$ avec \[c_2'' \eqbre
    c_2'[\ip{e}{`D}/x] \eqbre c_2[(c_e[`r])[`a_e[e]]/x]\]. 

  
    Soit $`a = c_2''$, on peut v�rifier:
    \[\begin{array}{ll}
      \firsteq{`a[\ip{f~e}{`D}]}
      
      \step{D�finition de l'interpr�tation}
      {=}{`a[(\pi_{F'}[\ip{f}{`D}])[c_{e'}[\ip{e}{`D}]]]}
      
      \step{D�finitions de $`a$ et $c_{e'}$}
      {=}{c_2[c_e[`r]~[(`a_e[\ip{e}{`D}])]/x]~[
        ((\pi_{F'}[\ip{f}{`D}])~[(c_1 `o c_e[`r] `o `a_e)[\ip{e}{`D}]])]}
      
      \step{D�finition de la composition}
      {=}{c_2[c_e[`r]~[(`a_e[\ip{e}{`D}])]/x]~[
        ((\pi_{F'}[\ip{f}{`D}])~[c_1[c_e[`r][`a_e[\ip{e}{`D}]]]])]}

      \step{D�finition de $c_f$ ($= \lambda x.c_2[[]~c_1[x]]$)}
      {=}{c_f[(\pi_{F'}[\ip{f}{`D}])~[c_e[`r]~[`a_e[\ip{e}{`D}]]]]}
      
      \step{Commutation du diagramme 1}
      {\eqbre}
      {(\pi_F[`r][`a_f[\ip{f}{`D}]]~[c_e[`r][`a_e~\ip{e}{`D}]])}
      
      \step{D�finitions de $`a_f$ et $`a_e$}{=}
      {(\pi_F[`r][\ip{f}{`G}[`r]])~[c_e[`r][\ip{e}{`G}[`r]]]}

      \step{D�finition de l'interpr�tation}{=}{\ip{f~e}{`G}[`r]}
      \vspace{0.2em}
    \end{array}\]
    
    \vspace{1cm}
    
    \TODO{Autres cas}
  \end{induction}   
\end{proof}

%%% Local Variables: 
%%% mode: latex
%%% TeX-master: "subset-typing"
%%% LaTeX-command: "TEXINPUTS=\"style:$TEXINPUTS\" latex"
%%% End: 


\begin{lemma}[Transitivit� de la coercion]
  \label{subi-trans}
  S'il existe $c_1, c_2$ tels que $`G \typec c_1 : S \sub T$ et $`G
  \typec c_2 : T \sub U$
  alors $`E!c, `G \typec c : S \sub U$ et $c \eqbi c_2 `o c_1$.
\end{lemma}
\begin{proof}
  
  \begin{induction}[subtyping-decl]
    
    \case{SubConv}\quad
    \begin{prooftree}
      \AXC{$S \eqbi T$}
      \UIC{$`G \typec c_1 = \lambda x.x : S \sub T$}
      \AXC{$`G \typec c_2 : T \sub U$}
      \BIC{$`G \typec c_2 `o c_1 : S \sub U$}
    \end{prooftree}
    
    Par le lemme \ref{coercion-conversion-impl}, on �limine trivialement
    \irule{SubTrans}, des deux c\^ot\'es. On considere donc dans le cas suivant uniquement
    les derivations avec \irule{SubHnf} pour les deux premisses.
    
    \case{SubHnf}\quad
    \begin{prooftree}
      \AXC{$`G \typec c : \hnf{S}~\subhnf \hnf{T}$}
      \UIC{$`G \typec c : S \sub T$}
      \AXC{$`G \typec d : \hnf{T}~\subhnf \hnf{U}$}
      \UIC{$`G \typec d : T \sub U$}
      \BIC{$`G \typec d `o c: S \sub U$}
    \end{prooftree}

    On va proceder par induction sur les derivations de typage dans
    le systeme $\subhnf$ pour montrer que $\subimplhnf{`G}{a}{\hnf{S}}{\hnf{T}}$ et
    $\subimplhnf{`G}{b}{\hnf{T}}{\hnf{U}}$ implique qu'il existe $c \eqbi b `o a :
    \hnf{S} \subhnf \hnf{U}$.
    \begin{induction}
      
      \case{SubProd}\quad
      \begin{prooftree}
        \AXC{$`G \typec c_1 : X' \sub X$}
        \AXC{$`G, x : X' \typec c_2 : Y[c_1~x/x] \sub Y'$}
        \BIC{$`G \typec c = \lambda f.\lambda x.c_2~(f~(c_1~x)) : \Pi x : X.Y \subhnf \Pi x : X'.Y'$}
        \AXC{$`G \typec d : \Pi x : X'.Y' \subhnf \hnf{U}$}
        \BIC{$$} %$`G \typec c = c' `o (\lambda f.\lambda x.~c_2~(f~(c_1~x))): \Pi x : X.Y \sub U$}
      \end{prooftree}
      
      Par induction sur la d�rivation $\subimplhnf{`G}{d}{\Pi x :
        X'.Y'}{\hnf{U}}$. Seules deux r�gles peuvent s'appliquer � la racine:
      \begin{induction}
        \case{SubProd}\quad
        On a $\hnf{U} = \Pi x : S.T$ et la d�rivation a la forme:
        \begin{prooftree}
          \AXC{$`G \typec d_1 : S \sub X'$}
          \AXC{$`G, x : S \typec d_2 : Y'[d_1~x/x] \sub T$}
          \BIC{$\subimplhnf{`G}{d = (\lambda f.\lambda x.d_2~(f~(d_1~x)))}
            {\Pi x : X'.Y'}{\Pi x : S.T}$}
        \end{prooftree}

        Par substitutivit� (lemme \ref{subti-coercion-subst}) on a:
        \[\subimpl{`G, x :
          S}{c_2[d_1~x/x]}{Y[c_1~x/x][d_1~x/x]}{Y'[d_1~x/x]}\] de m�me
        taille que la d�rivation de $c_2$.
        
        Par induction il existe $e_1, e_2$ tels que:
        \[\subimpl{`G}{e_1 \eqbi c_1 `o d_1}{S}{X}\] et 
        \[\subimpl{`G, x : S}{e_2 \eqbi d_2 `o
          c_2[d_1~x/x]}{Y[c_1~x/x][d_1~x/x]}{T}\]
        
        Or $Y[c_1~x/x][d_1~x/x] = Y[c_1~(d_1~x)/x] = Y[e_1~x/x]$.
        
        Donc
        \[\subimplhnf{`G}{e = \lambda f.\lambda x.~e_2~f~(e_1~x)}{\Pi x : X.Y}{\Pi x
          : S.T}\] est d�rivable sans \irule{SubTrans}:
        
        \begin{prooftree}
          \AXC{$`G \typec e_1 : S \sub X$}
          \AXC{$\subimpl{`G, x : S}{e_2}{Y[e_1~x/x]}{T}$}
          \BIC{$\subimplhnf{`G}{\lambda f.\lambda x.~e_2~f~(e_1~x)}
            {\Pi x : X.Y}{\Pi x : S.T}$}
        \end{prooftree}

        On a bien $e \eqbi d `o c$:
        \begin{eqnarray*}
          d `o c & = & 
          \lambda f : \Pi x : X.Y. d~(c~f) \\
          & "->"_{\beta} &
          \lambda f : \Pi x : X.Y. d~(\lambda x : X'.~c_2~(f~(c_1~x))) \\
          & "->"_{\beta} & 
          \lambda f : \Pi x : X.Y. \lambda x : S.~d_2~((\lambda x : X'.c_2~(f~(c_1~x)))~(d_1~x)) \\
          & "->"_{\beta} & 
          \lambda f : \Pi x : X.Y. \lambda x : S.~d_2~(c_2[d_1~x/x]~(f~(c_1~(d_1~x)))) \\
          & = & 
          \lambda f : \Pi x : X.Y. \lambda x : S.~e_2~(f~(e_1~x)) \\
          & = & e
        \end{eqnarray*}
        
        \case{SubProof}
        Ici $\hnf{U} = \mysubset{y}{U'}{P}$ et la d�rivation commence
        par:
        \begin{prooftree}
          \AXC{$`G \typec e : \Pi x : X'.Y' \sub U'$}
          \UIC{$\subimplhnf{`G}{d = (\lambda t : (\Pi x : X'.Y'). \elt{U'}{\lambda x :
                U'.P}{e~t}{?_{P[e~t/x]}})}
            {\Pi x : X'.Y'}{\mysubset{y}{U'}{P}}$}
        \end{prooftree}

        Par induction on a $`G \typec f \eqbi e `o c : \Pi x : X.Y \sub
        U'$.
        On peut donc d�river:
        \begin{prooftree}
          \AXC{$`G \typec f : \Pi x : X.Y \sub U'$}
          \UIC{$\subimplhnf{`G}{d' = (\lambda t : (\Pi x : X.Y). \elt{U'}{\lambda x :
                U'.P}{f~t}{?_{P[f~t/x]}})}
            {\Pi x : X.Y}{\mysubset{y}{U'}{P}}$}
        \end{prooftree}
        
        On peut v�rifier que $d' \eqbi d `o c$:

        \begin{eqnarray*}
          d `o c & = & (\lambda t : (\Pi x : X'.Y'). \elt{U'}{(\lambda x :
            U'.P)}{(e t)}{?_{P[e~t/x]}}) `o c \\
          & = &  \lambda t : (\Pi x : X.Y).
          (\lambda t : (\Pi x : X'.Y').  \elt{U'}{(\lambda x : U'.P)}{(e
            t)}{?_{P[e~t/x]}})~(c~t) \\          
          & "->"_{\beta} & 
          \lambda t : (\Pi x : X.Y).
          \elt{U'}{(\lambda x :U'.P)}{(e~(c~t))}{?_{P[e~(c~t)/x]}} \\
          & \eqbi & 
          \lambda t : (\Pi x : X.Y).
          \elt{U'}{(\lambda x :U'.P)}{(f~t)}{?_{P[f~t)/x]}} \\
          & = & d'
        \end{eqnarray*}
      
      \end{induction}
      
      \case{SubSigma}\quad
      De fa�on �quivalente � \irule{SubProd}, on fait le cas si
      \irule{SubSigma} est utilis�e � la pr�misse droite.

      \begin{prooftree}
        \AXC{$\subimplhnf{`G}{c}{T}{\Sigma x : X'.Y'}$}
        \AXC{$\subimpl{`G}{d_1}{Y'}{Y}$}
        \AXC{$\subimpl{`G, x : X'}{d_2}{Y'}{Y[d_1~x/x]}$}
        \BIC{$\subimplhnf{`G}{d = \lambda (x,y).(d_1~x, d_2~y)}{\Sigma x : X'.Y'}{\Sigma x : X.Y}$}
        \BIC{$$} %$`G \typec c = c' `o (\lambda f.\lambda x.~c_2~(f~(c_1~x))): \Pi x : X.Y \sub U$}
      \end{prooftree}

      Par induction sur la d�rivation de $\subimplhnf{`G}{c}{T}{\Sigma x
        : X'.Y'}$:
      \begin{induction}
        \case{SubSigma}
        On a:
        \begin{prooftree}
          \AXC{$\subimpl{`G}{c_1}{S}{Y'}$}
          \AXC{$\subimpl{`G, x : X'}{c_2}{T}{Y'[c_1~x/x]}$}
          \BIC{$\subimplhnf{`G}{c = \lambda (x,y).(c_1~x, c_2~y)}{\Sigma x
              : S.T}{\Sigma x : X'.Y'}$}
        \end{prooftree}
        
        Par substitutivit� on a:
        \[\subimpl{`G, x :
          S}{d_2[c_1~x/x]}{Y'[c_1~x/x]}{Y[d_1~x/x][c_1~x/x] = Y[d_1~(c_1~x)/x]}\]
        
        Par induction on a donc:
         \begin{prooftree}
          \AXC{$\subimpl{`G}{e_1 \eqbi d_1 `o c_1}{S}{Y}$}
          \AXC{$\subimpl{`G, x : S}{e_2 \eqbi d_2[c_1~x/x] `o c_2}{T}{Y[e_1~x/x]}$}
          \BIC{$\subimplhnf{`G}{e = \lambda (x,y).(e_1~x, e_2~y)}{\Sigma x
              : S.T}{\Sigma x : X.Y}$}
        \end{prooftree}
        
        On peut v�rifier:
        \begin{eqnarray*}
          d `o c & = & \lambda (x,y).d~(c~(x,y)) \\
          & "->"_\beta & \lambda (x,y).d~(c_1~x, c_2~y) \\
          & = & \lambda (x,y).(\lambda (x,y).(d_1~x, d_2~y))~(c_1~x,
          c_2~y) \\
          & "->"_\beta & 
          \lambda (x,y). (d_1~(c_1~x), d_2[c_1~x/x] (c_2~y)) \\
          & \eqbi & \lambda (x,y).(e_1~x, e_2~y) \\
          & = & e
        \end{eqnarray*}
        
        \case{SubSub}
        On a: 
        \begin{prooftree}
          \AXC{$\subimpl{`G}{c'}{T}{\Sigma x : X'.Y'}$}
          \UIC{$\subimplhnf{`G}{c = c' `o \pi_1}{\mysubset{y}{T}{P}}{\Sigma x : X'.Y'}$}
        \end{prooftree}
        
        Par induction, $\subimpl{`G}{f \eqbi d `o c'}{T}{\Sigma x :
          X.Y}$, on a donc:
        \begin{prooftree}
          \AXC{$\subimpl{`G}{f \eqbi d `o c'}{T}{\Sigma x : X.Y}$}
          \UIC{$\subimplhnf{`G}{g = f `o \pi_1}{\mysubset{y}{T}{P}}{\Sigma x : X.Y}$}
        \end{prooftree}
        
        On a bien: $g = f `o \pi_1 \eqbi d `o c' `o \pi_1 \eqbi d `o c$.

      \end{induction}
      
      
      \case{SubProof}\quad
      \begin{prooftree}
        \AXC{$\subimpl{`G}{e}{S}{T}$}
        \UIC{$\subimplhnf{`G}{c = (\lambda t : S.~\elt{T}{(\lambda x :
              T.P)}{(e~t)}{?_{P[e~t/x]}})}
          {\hnf{S}}{\mysubset{x}{T}{P}}$}
        \AXC{$\subimplhnf{`G}{d}{\mysubset{x}{T}{P}}{\hnf{U}}$}
        \BIC{$$} 
      \end{prooftree}
      
      Si $\subimplhnf{`G}{d}{\mysubset{x}{T}{P}}{\hnf{U}}$ alors
      $\subimpl{`G}{e'}{T}{\hnf{U}}$ est
      d�rivable par une d�rivation plus petite et $d = e' `o \pi_1$.
      Par induction, avec l'hypoth\`ese $\subimpl{`G}{e}{\hnf{S}}{T}$, il existe $f$ tel que
      $\subimpl{`G}{f}{\hnf{S}}{\hnf{U}}$ est d�rivable par une d�rivation
      n'utilisant pas \irule{SubTrans} et $f \eqbi e' `o e$. 
      On a bien $f \eqbi e' `o e \eqbi d `o c \eqbi e' `o \pi_1 `o c$ car
      \begin{eqnarray*}
        e' `o \pi_1 `o c & = & e' `o \pi_1 `o (\lambda t : S.~\elt{T}{(\lambda x :
          T.P)}{(e~t)}{?_{P[e~t/x]}}) \\
        & = &
        \lambda x.e'~(\pi_1~((\lambda t : S.~\elt{T}{(\lambda x :
          T.P)}{(e~t)}{?_{P[e~t/x]}})~x)) \\
        & "->"_{\beta} &
        \lambda x.e'~(\pi_1~(\elt{T}{(\lambda x :
          T.P)}{(e~x)}{?_{P[e~x/x]}})) \\
        & "->"_{\beta} & \lambda x.e'~(e~x) \\
        & = & e' `o e
      \end{eqnarray*}      
      
      \case{SubSub}\quad
      \begin{prooftree}
        \AXC{$\subimpl{`G}{c}{S}{\hnf{T}}$}
        \UIC{$\subimplhnf{`G}{d = c `o \pi_1}
          {\mysubset{x}{S}{P}}{\hnf{T}}$}
        % \UIC{$\subimpl{`G}{d}
        %{\mysubset{x}{S}{P}}{T}$}        
        \AXC{$\subimplhnf{`G}{e}{\hnf{T}}{\hnf{U}}$}
        %\UIC{$\subimpl{`G}{e}{T}{U}$}
        \BIC{$$} %$\subimpl{`G}{e `o d}{\mysubset{x}{S}{P}}{U}$}
      \end{prooftree}
      
      Par induction il existe une d�rivation de $\subimpl{`G}{f \eqbi e `o c}
      {S}{U}$ n'utilisant pas \irule{SubTrans}. On peut donc d�river:

      \begin{prooftree}
        \AXC{$\subimpl{`G}{f}{S}{U}$}
        \UIC{$\subimpl{`G}{f `o \pi_1}
          {\mysubset{x}{S}{P}}{U}$}
      \end{prooftree}
      
      On a bien $f `o \pi_1 \eqbi e `o c `o \pi_1$.
    
    \end{induction}
  \end{induction}
\end{proof}

%%% Local Variables: 
%%% mode: latex
%%% TeX-master: "subset-typing"
%%% LaTeX-command: "TEXINPUTS=\"style:$TEXINPUTS\" latex"
%%% End: 


%%% Local Variables: 
%%% mode: latex
%%% TeX-master: "subset-typing"
%%% LaTeX-command: "TEXINPUTS=\"style:$TEXINPUTS\" latex"
%%% End: 
