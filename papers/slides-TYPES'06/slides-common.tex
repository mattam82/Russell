\usepackage[latin1]{inputenc}
\usepackage{xspace} % To get the right spacings in front of : and so on
\usepackage[french]{babel}
\usepackage{abbrevs}
\usepackage{subfigure}
\usepackage{ifthen}
\usepackage{coqdoc}
\usepackage{bussproofs}
\usepackage{pgf,pgfarrows,pgfnodes}

\EnableBpAbbreviations
\def\fCenter{\vdash}
\def\seq{\fCenter}

\hypersetup{
  pdftitle = Subset coercions in Coq,
  pdfauthor = Matthieu Sozeau,
  pdfsubject = Theoretical Computer Science
 } 

\mode<all>
{
\setbeamerfont{block title}{size={}}
\usefonttheme{professionalfonts}
\setbeamercovered{invisible}
\useoutertheme{default}
\useinnertheme{rounded}
\usecolortheme{rose}
\usecolortheme{seahorse}
\setbeamertemplate{navigation symbols}{}
}

\usepackage{concmath}
\usepackage[T1]{fontenc}

\usepackage{txfonts}
\newcommand{\termgrammar}
{$\begin{array}{lcl}
    `a & \Coloneqq & x \\
    & | & \funml{}~x~:~`t "=>" `a \\
    & | & `a~`a \\ 
    & | & `a~`t \\
    & | & \text{\emph{constante}} \\
% & | & (`a,~`a) \\
    & | & (x \coloneqq `a,~`a~: `t) \\
    & | & \letml~x = `a ~\inml~`a \\
    & | & \letml~(x, y) = `a ~\inml~ `a

%    & | & \ifml~`a~\thenml~`a~\elseml~`a
  \end{array}$}

\newcommand{\typegrammarOrig}
{$\begin{array}{lcl}
    `t & \Coloneqq & x \\
    & | & `t "->" `t \\
    & | & `t~`t \\
    & | & \text{\emph{constante}} \\
    & | & `t * `t \\
    & | & \Sigma x : `t. `t \\
    & | & \lambda{}~x~:~s "=>" `t \\
    & | & `t~`a \\
    & | & \Pi x : `t. `t \\
    & | & \subset{x}{`t}{`t}
  \end{array}$}

\newcommand{\typegrammar}
{$\begin{array}{lcl}
    `t & \Coloneqq & x \\
    & | & `t~`t \\
    & | & `t~`a \\
    & | & \lambda{}~x~:~`t "=>" `t \\
    & | & \Pi x : `t. `t \\
    & | & \Sigma x : `t. `t \\
%    & | & `t * `t \\
    & | & \subset{x}{`t}{`t} \\
    & | & \Set \\
    & | & \Prop \\
    & | & \Type \\
    & | & \text{\emph{constante}} 
%    & & \\
%    \text{{\tt Inductive}} & \Coloneqq & ident
  \end{array}$}

\newboolean{defineTheoremEn}
\newboolean{defineTheoremFr}
\setboolean{defineTheoremEn}{true}
\setboolean{defineTheoremFr}{false}
\newtheorem{theorem}{Th�or�me}[section]
\newtheorem{lemma}[theorem]{Lemme}
\newtheorem{fact}[theorem]{Fait}
\newtheorem{proposition}[theorem]{Proposition}
\newtheorem{definition}[theorem]{D�finition}
\newtheorem{example}[theorem]{Exemple}
\newtheorem{remark}[theorem]{Remarque}
\newtheorem{corrolary}[theorem]{Corrolaire}

\makeatletter

\newcommand{\UR}[2]{\RightLabel{\rulelabel{#1}}\UIC{#2}}
\newcommand{\URL}[2]{\LeftLabel{\rulelabel{#1}}\UIC{#2}}

\newboolean{displayLabels}
\setboolean{displayLabels}{true}

\newcommand{\LeftRuleLabel}[1]{
  \@ifnotmtarg{#1}
  {\ifthenelse{\boolean{displayLabels}}{\LeftLabel{\rulelabel{#1}}}{}}
}

\newcommand{\UAX}[4]{\AXC{#2}
  \LeftRuleLabel{#1}
  \@ifnotmtarg{#4}{\RightLabel{#4}}
  \UIC{#3}}

\newcommand{\BAX}[5]{\AXC{#2}\AXC{#3}
  \LeftRuleLabel{#1}
  \@ifnotmtarg{#5}{\RightLabel{#5}}
  \BIC{#4}}

\newcommand{\TAX}[6]{\AXC{#2}\AXC{#3}\AXC{#4}
  \LeftRuleLabel{#1}
  \@ifnotmtarg{#6}{\RightLabel{#6}}
  \TIC{#5}}

\newcommand{\QAX}[7]{\AXC{#2}\noLine\UIC{#3}\AXC{#4}\AXC{#5}
  \LeftRuleLabel{#1}
  \@ifnotmtarg{#7}{\RightLabel{#7}}
  \TIC{#6}}

\newcommand{\BR}[2]{\RightLabel{\rulelabel{#1}}\BIC{#2}}
\newcommand{\BRL}[2]{\LeftLabel{\rulelabel{#1}}\BIC{#2}}

\makeatother

%%% Local Variables: 
%%% mode: latex
%%% TeX-master: t
%%% End: 

\setboolean{displayLabels}{true}

%\def\ifstreq#1#2{\ifEqString{#1}{#2}}

\def\impsub{\rightslice}

\def\judgewf{\vdash_{wf}}
\def\judgetyped{\vdash}
\def\judgetypea{\vdash_{\bullet}}
\def\judgetypei{\vdash_{\Box}}
\def\typed{\judgetyped}
\def\typedwf{\judgewf}
\def\typea{\judgetypea}
\def\typeawf{\judgetypea_{wf}}
\def\typei{\judgetypei}
\def\typeiwf{\judgetypei_{wf}}
\def\type{\typed}
\def\typewf{\typedwf}
\def\judgesubd{\vdash}
\def\judgesuba{\vdash_{\bullet}}
\def\judgesubi{\vdash_{\Box}}
\def\subtd{\judgesubd}
\def\subta{\judgesuba}
\def\subti{\judgesubi}
\def\subt{\subtd}
\def\subd{\rightslice}
\def\suba{\rightslice_{\bullet}}
\def\subi{\rightslice_{\Box}}
\def\sub{\subd}

\def\judgerw{~{\mbox{$"~>"$}}~}
\def\wf{\judgewf}

\newcommand{\matht}[1]{\text{{\tt #1}}}

\def\even{\matht{even}}
\def\odd{\matht{odd}}
\def\Set{\matht{Set}}
\def\Prop{\matht{Prop}}
\def\Type{\matht{Type}}
\def\ps{\emph{predicate subtyping}}

\def\WfAtomRule{Wf-Atom}
\def\WfVarRule{Wf-Var}
\def\PropSetRule{PropSet}
\def\TypeRule{Type}
\def\VarRule{Var}
\def\ProdRule{Prod}
\def\AbsRule{Abs}
\def\AppRule{App}
\def\LetInRule{Let-In}
\def\SigmaRule{Sigma}
\def\SumRule{Sum}
\def\LetSumRule{Let-Sum}
\def\SumInfRule{Sum-Inf}
\def\SumDepRule{Sum-Dep}
\def\SubsetRule{Subset}
\def\LetSubRule{Let-Sub}
\def\SubsumRule{Subsumption}
\def\ConvRule{Conv}

\def\SubReflRule{Sub-Refl}
\def\SubTransRule{Sub-Trans} 
\def\SubConvRule{Sub-Conv}
\def\SubProdRule{Sub-Prod}
\def\SubSigmaRule{Sub-Sigma}
\def\SubLeftRule{Sub-Left}
\def\SubRightRule{Sub-Right}
\def\SubProofRule{Sub-Proof}
\def\SubSubRule{Sub-Subset}
\def\SubTransRule{Sub-Trans}

\def\ifstreq#1#2{\def\testa{#1}\def\testb{#2}\ifx\testa\testb }

\def\inductionon#1{
  \ifstreq{#1}{typing-decl}{Par induction sur la d�rivation de typage.}
  \else\ifstreq{#1}{typing-algo}{Par induction sur la d�rivation de
    typage dans le syst�me algorithmique.}
  \else\ifstreq{#1}{typing-impl}{Par induction sur la d�rivation de
    typage.}
  \else\ifstreq{#1}{subtyping-decl}{Par induction sur la d�rivation de
    sous-typage.}
  \else\ifstreq{#1}{subtyping-algo}{Par induction sur la d�rivation de
    sous-typage dans le syst�me algorithmique.}
  \else\ifstreq{#1}{subtyping-impl}{Par induction sur la d�rivation de
    sous-typage.}
  \fi
}

\newenvironment{induction}[1][text=\empty]{
  \if#1\empty\else\inductionon{#1}
  \begin{list}{Unset default item}{}}
  {\end{list}}

%% Should be able to work with \else...
\newcommand{\inductionrule}[1]
{\ifstreq{#1}{WfAtom}{\WfAtomRule}
\fi\ifstreq{#1}{WfVar}\WfVarRule
\fi\ifstreq{#1}{PropSet}\PropSetRule
\fi\ifstreq{#1}{Type}\TypeRule
\fi\ifstreq{#1}{Var}\VarRule
\fi\ifstreq{#1}{Prod}\ProdRule
\fi\ifstreq{#1}{Abs}\AbsRule
\fi\ifstreq{#1}{LetIn}\LetInRule
\fi\ifstreq{#1}{Sigma}\SigmaRule
\fi\ifstreq{#1}{Sum}\SumRule
\fi\ifstreq{#1}{LetSum}\LetSumRule
\fi\ifstreq{#1}{App}\AppRule
\fi\ifstreq{#1}{SumInf}\SumInfRule
\fi\ifstreq{#1}{SumDep}\SumDepRule
\fi\ifstreq{#1}{LetSub}\LetSubRule
\fi\ifstreq{#1}{Subsum}\SubsumRule
\fi\ifstreq{#1}{Conv}\ConvRule
\fi\ifstreq{#1}{Subset}\SubsetRule

\fi\ifstreq{#1}{SubRefl}\SubReflRule
\fi\ifstreq{#1}{SubTrans}\SubTransRule
\fi\ifstreq{#1}{SubConv}\SubConvRule
\fi\ifstreq{#1}{SubProd}\SubProdRule
\fi\ifstreq{#1}{SubSigma}\SubSigmaRule
\fi\ifstreq{#1}{SubLeft}\SubLeftRule
\fi\ifstreq{#1}{SubRight}\SubRightRule
\fi\ifstreq{#1}{SubProof}\SubProofRule
\fi\ifstreq{#1}{SubSub}\SubSubRule
\fi}

\newcommand{\rulename}[1]{{\bf \inductionrule{#1}}}

\def\indrule{\rulename}

\def\case#1{\item[- \rulename{#1} :]}
\newcommand{\casetwo}[2]{\item[- \rulename{#1}, \rulename{#2} :]}
\def\casethree#1#2#3{\item[- \rulename{#1}, \rulename{#2},
  \rulename{#3} :]}
\def\casefour#1#2#3#4{\item[- \rulename{#1}, \rulename{#2},
  \rulename{#3}, \rulename{#4} :]}

\newcommand{\rname}[1]{{\bf \rulename{#1}}}
\newcommand{\rulelabel}[1]{{\bf (\rulename[#1])}}

%%% Local Variables: 
%%% mode: latex
%%% TeX-master: "subset-typing"
%%% LaTeX-command: "TEXINPUTS=\"style:$TEXINPUTS\" latex"
%%% End: 

\def\infvspace{0.5cm}

\newcommand{\WfEmptyFull}[1]{
  \UAX{WfEmpty}
  {}
  {$\seq []$}
  {}
}  
\newcommand{\WfEmpty}[1][\Gamma]{\WfEmptyFull{#1}}
  
\newcommand{\WfVarFull}[4]{
  \UAX{WfVar}
  {$\tgen{#1}{#2}{#3}$}
  {$\wf #1, #4 : #2$}
  {$#3 `: \setproptype{} `^{} #4 `; #1$}
}
\newcommand{\WfVar}[1][\Gamma]{\WfVarFull{#1}{A}{s}{x}}

\newcommand{\PropSetFull}[2]{
  \UAX{PropSet}
  {$\wf #1$}
  {$\tgen{#1}{#2}{\Type}$}
  {$#2 `: \setprop$} 
}
\newcommand{\PropSet}[1][\Gamma]{\PropSetFull{#1}{s}}


\newcommand{\TypeTypeFull}[1]{
  \UAX{Type}
  {$\wf #1$}
  {$\tgen{#1}{\Type(i)}{\Type(i + 1)}$}
  {$i `: `N$}
}
\newcommand{\TypeType}[1][\Gamma]{\TypeTypeFull{#1}}

\newcommand{\VarFull}[3]{
  \BAX{Var}
  {$\wf #1$}
  {$#2 : #3 `: #1$}
  {$\tgen{#1}{#2}{#3}$}
  {}
}
\newcommand{\Var}[1][\Gamma]{\VarFull{#1}{x}{A}} 

\newcommand{\ProdFull}[7]{
  \BAX{Prod}
  {$\tgen{#1}{#2}{#3}$}
  {$\tgen{#1, #4 : #2}{#5}{#6}$}
  {$\tgen{#1}{\Pi #4 : #2.#5}{#7}$}
  {$(#3, #6, #7) `: \mathcal{R} `^{} #4 `; #1$}
}
\newcommand{\Prod}[1][\Gamma]{\ProdFull{#1}{T}{s_1}{x}{U}{s_2}{s_3}}

\newcommand{\AbsFull}[6]{
  \BAX{Abs}
  {$\tgen{#1}{\Pi #2 : #3.#4}{#5}$}
  {$\tgen{#1, #2 : #3}{#6}{#4}$}
  {$\tgen{#1}{\lambda #2 : #3. #6}{\Pi #2 : #3.#4}$}
  {$#2 `; #1$}
}

\newcommand{\Abs}[1][\Gamma]{\AbsFull{#1}{x}{T}{U}{s}{M}}

 \newcommand{\AppFull}[6]{
  \BAX{App}
  {$\tgen{#1}{#2}{\Pi #3 : #4. #5}$}
  {$\tgen{#1}{#6}{#4}$}
  {$\tgen{#1}{(#2 #6)}{#5 [ #6 / #3 ]}$}
  {$$}
}

\newcommand{\App}[1][\Gamma]{\AppFull{#1}{f}{x}{V}{W}{u}}

\newcommand{\SigmaRFull}[5]{
  \BAX{Sigma}
  {$\tgen{#1}{#2}{#3}$}
  {$\tgen{#1, #4 : #2}{#5}{#3}$}
  {$\tgen{#1}{\Sigma #4 : #2.#5}{#3}$}
  {$#3 `: \{ \Prop, \Set \} `^{} #4 `; #1$} 
}
\newcommand{\SigmaR}[1][\Gamma]{\SigmaRFull{#1}{T}{s}{x}{U}}

\newcommand{\SumDepFull}[7][\Gamma]{
  \TAX{SumDep}
  {$\tgen{#1}{\Sigma #2 : #5. #6}{#7}$}
  {$\tgen{#1}{#3}{#5}$}
  {$\tgen{#1}{#4}{#6[#3 / #2]}$}
  {$\tgen{#1}{\pair{\Sigma #2 : #5.#6}{#3}{#4}}{\Sigma #2 : #5.#6}$}
  {}
}

\newcommand{\SumDep}[1][\Gamma]{\SumDepFull[#1]{x}{t}{u}{T}{U}{s}}
 
\newcommand{\PiLeftFull}[5][\Gamma]{
  \UAX{PiLeft}
  {$\tgen{#1}{#2}{`S #3 : #4.#5}$}
  {$\tgen{#1}{\pi_1~#2}{#4}$}
  {}
}
\newcommand{\PiLeft}[1][\Gamma]{\PiLeftFull[#1]{t}{x}{T}{U}}

\newcommand{\PiRightFull}[5][\Gamma]{
  \UAX{PiRight}
  {$\tgen{#1}{#2}{`S #3 : #4.#5}$}
  {$\tgen{#1}{\pi_2~#2}{#5[\pi_1~#2/#3]}$}
  {}
}
\newcommand{\PiRight}[1][\Gamma]{\PiRightFull[#1]{t}{x}{T}{U}}


\newcommand{\SubsetFull}[4]{
  \BAX{Subset}
  {$\tgen{#1}{#3}{\Set}$}
  {$\tgen{#1, #2 : #3}{#4}{\Prop}$}
  {$\tgen{#1}{\mysubset{#2}{#3}{#4}}{\Set}$}
  {$#2 `; #1$}
}
\newcommand{\SubsetR}[1][\Gamma]{\SubsetFull{#1}{x}{U}{P}}


\newcommand{\SubsumFull}[5]{
  \TAX{Subsum}
  {$\tgen{#1}{#4}{#5}$}
  {$\tgen{#1}{#2}{#3}$}
  {$#5 \sub #2$} % #1 \subt 
  {$\tgen{#1}{#4}{#2}$}
  {}
}
\newcommand{\Subsum}[1][\Gamma]{\SubsumFull{#1}{T}{s}{t}{U}} 

\def\Coerce{\Subsum}

\newcommand{\ConvFull}[5]{
  \TAX{Conv}
  {$\tgen{#1}{#2}{#3}$}
  {$\tgen{#1}{#4}{#5}$}
  {$#5 \eqbr #2$}
  {$\tgen{#1}{#4}{#2}$}
  {}
}
\newcommand{\Conv}[1][\Gamma]{\ConvFull{#1}{T}{s}{t}{U}}

\newcommand{\typedRules}{
  \begin{center}
    \def\seq{\typed}
    \def\fCenter{\wf}
    \WfEmpty \DP\quad
    \WfVar \DP
    
    \def\fCenter{\typed}
    \vspace{\infvspace}
    \PropSet\DP

    \vspace{\infvspace}
    \Var\DP
    
    \vspace{\infvspace}
    \Prod\DP
    
    \vspace{\infvspace}
    \Abs\DP

    \vspace{\infvspace}
    \App\DP

    \vspace{\infvspace}
    \SigmaR\DP

    \vspace{\infvspace}
    \SumDep\DP
    
    \vspace{\infvspace}
    \PiLeft\DP
    \quad
    \PiRight\DP

    \vspace{\infvspace}
    \SubsetR\DP

    \vspace{\infvspace}
    \Subsum\DP
      
  \end{center}
}

\def\typedFig
{
\begin{figure}[tb]
  \typedRules
  \caption{Calcul de coercion par pr�dicats - version d�clarative}
  \label{fig:typing-decl-rules}
\end{figure}
}

%%% Local Variables: 
%%% mode: latex
%%% TeX-master: "subset-typing"
%%% LaTeX-command: "TEXINPUTS=\"style:$TEXINPUTS\" latex"
%%% End: 

\def\SubConv{
  \UAX{SubConv}
  {$T \eqbi U$}
  {$T \sub U$}
  {} 
}

\def\SubRefl{
  \UAX{SubRefl}
  {}
  {$S \seq S$}
  {}
}

\def\SubTrans{
\BAX{SubTrans}
{$S \seq T$}
{$T \seq U$}
{$S \seq U$}
{}
} 

\def\SubProd{
\BAX{SubProd}
{$U \seq T$} %"<|-|>"
{$V \seq W$}
{$\Pi x : T.V \seq \Pi x : U.W$}
{}
}

\def\SubSigma{
  \BAX{SubSigma}
  {$T \seq U$}
  {$V \seq W$}
  {$\Sigma x : T. V \seq \Sigma y : U. W$}
  {}
}

\def\SubProof{
  \UAX{SubProof}
  {$U \seq V$}
  {$U \seq \subset{x}{V}{P}$}
  {}
}

\def\SubSub{
  \UAX{SubSub}
  {$U \seq V$}
  {$\subset{x}{U}{P} \seq V$}
  {}
}

\def\subtdFig{
\begin{figure}[ht]
  \begin{center}
    \def\fCenter{\subd}
    
    \vspace{\infvspace}
    \SubConv\DP

    \vspace{\infvspace}
    \SubTrans\DP

    \vspace{\infvspace}
    \SubProd\DP

    \vspace{\infvspace}
    \SubSigma\DP
    
    \vspace{\infvspace}
    \SubProof\DP
    
    \vspace{\infvspace}
    \SubSub\DP
    
  \end{center}
  \caption{Coercion par pr�dicats - version d�clarative}
  \label{fig:subtyping-decl-rules}
\end{figure}
}

\def\subtdShort{
\begin{figure}[ht]
  \begin{center}
    \def\fCenter{\subd}
    \SubConv\DP
    \SubProof\DP
    \SubSub\DP

    \vspace{1cm}
    \SubProd\DP
    \SubSigma\DP
    %\SubTrans\DP
  \end{center}
  \caption{Coercion par pr�dicats - version d�clarative}
  \label{fig:subtyping-decl-rules-short}
\end{figure}
}

%%% Local Variables: 
%%% mode: latex
%%% TeX-master: "subset-typing"
%%% LaTeX-command: "TEXINPUTS=\"style:$TEXINPUTS\" latex"
%%% End: 



\def\AppA{
\TAX{App}
{$\talgo{`G}{f}{T} \quad \mualgo(T) = \Pi x : V. W : s$}
{$\talgo{`G}{u}{U} \quad \talgo{`G}{U, V}{s'}$}
{$\subalgo{`G}{U}{V} $}
{$\talgo{`G}{(f u)}{W [ u / x ]}$}
{}
}

\def\LetSumA{
  \TAX{LetSum}
  {$`G \seq t : S$}
  {$\mualgo(S) = `S x : T. U $}
  {$`G, x : T, u : U \seq v : V $}
  {$`G \seq \letml~(x, u) = t~\inml~v : V$}
  {$x,y `; `G$}
}

\def\PiLeftA{
  \BAX{PiLeft}
  {$\tgen{`G}{t}{S}$}
  {$\mualgo(S) = `S x : T.U$}
  {$\tgen{`G}{\pi_1~t}{T}$}
  {}
}

\def\PiRightA{
  \BAX{PiRight}
  {$\tgen{`G}{t}{S}$}
  {$\mualgo(S) = `S x : T.U$}
  {$\tgen{`G}{\pi_2~t}{U[\pi_1~t/x]}$}
  {}
}


\def\SumInfA{
  \TAX{SumInf}
  {$`G \seq t : T $}
  {$`G \seq u : U $}
  {$`G \seq \Sigma \_ : T.U : s$}
  {$`G \seq (t, u) : \Sigma \_ : T.U$}
  {}
}

\def\SumDepAold{
  \QAX{SumDep}
  {$`G \seq t : T$}
  {$`G \seq u : U'$}
  {$`G \seq \Sigma x : T.U : s$}
  {$`G \seq U' : s \quad U' \suba U[t/x]$}
  {$`G \seq (x \coloneqq~t, u : U) : \Sigma x : T.U$}
  {}
}

\def\SumDepA{
  \QAX{SumDep}
  {$`G \seq t : T$}
  {$`G \seq u : U'$}
  {$`G \seq \Sigma x : T.U : s$}
  {$`G \seq U' : s \quad U' \suba U[t/x]$}
  {$`G \seq \pair{\Sigma x : T.U}{t}{u} : \Sigma x : T.U$}
  {}
}

\def\typeaFig{
\begin{figure}[t]
  \begin{center}
    \def\fCenter{\wf}
    \def\type{\typea}
    \def\subt{\subta}
    \def\sub{\suba}
    
    \WfEmpty\DP
    \quad
    \WfVar\DP
    
    \def\fCenter{\typea}
    \vspace{\infvspace}
    \PropSet\DP
%    \quad
%    \TypeType\DP
    
    \vspace{\infvspace}
    \Var\DP
    
    \vspace{\infvspace}
    \Prod\DP
    
    \vspace{\infvspace}
    \Abs\DP

    \vspace{\infvspace}
    \AppA\DP

%    \vspace{\infvspace}
%    \LetIn\DP
    
    \vspace{\infvspace}
    \SigmaR\DP

%    \vspace{\infvspace}
%    \SumInfA\DP

    \vspace{\infvspace}
    \SumDepA\DP
    
    \vspace{\infvspace}
    \PiLeftA\DP
    \quad
    \PiRightA\DP
    %\LetSumA\DP
    
    \vspace{\infvspace}
    \Subset\DP
  \end{center}
  \caption{Calcul de coercion par pr�dicats - version algorithmique}
  \label{fig:typing-algo-rules}
\end{figure}
}

\def\typemuaFig{
  \begin{figure}[ht]
    \begin{eqnarray*}
      \mualgo'(\mysubset{x}{U}{P}) & "=>" & \mualgo'(\downarrow{U}) \\
      \mualgo'(x)                & "=>" & x \\
      \\
      \mualgo(x) & "=>" & \mualgo' (\downarrow{x})
    \end{eqnarray*}
    \caption{D�finition de $\mualgo$}
    \label{fig:mualgo-definition}
\end{figure}
}

%%% Local Variables: 
%%% mode: latex
%%% TeX-master: "subset-typing"
%%% LaTeX-command: "TEXINPUTS=\"style:$TEXINPUTS\" latex"
%%% End: 

%% \def\SubProd{
%%   \BAX{Sub-Prod}
%%   {$`G \seq U \sub T$} %"<|-|>"
%%   {$`G, x : U \seq V \sub W$}
%%   {$`G \seq \Pi x : T.V \sub \Pi x : U.W$}
%%   {}
%% }

%% \def\SubSigma{
%%   \BAX{Sub-Sigma}
%%   {$`G \seq T \sub U$}
%%   {$`G \seq V \sub W$}
%%   {$`G \seq \Sigma x : T. V \sub \Sigma x : U. W$}
%%   {}
%% }

\def\SubHnf{
  \UAX{SubHnf}
  {$\hnf{T}~\sub \hnf{U}$}
  {$T \sub U$}
  {} 
}

\def\SubLeft{
  \BAX{Sub-Left}
  {$`G \seq U \sub V$}
  {$`G \type P : \Pi x : U. \Prop$}
  {$`G \seq \subset{x}{U}{P} \sub V$}
  {}
}
    
\def\SubRight{
  \UAX{Sub-Right}
  {$`G \seq T \sub U$}
                                %{$`G \judgetype h : P~p$}
  {$`G \seq T \sub \subset{x}{U}{P}$}
  {}
}

\def\subtaFig{
\begin{figure}[ht]
  \begin{center}
    \def\fCenter{\suba}
    \def\type{\typea}
    \def\sub{\suba}

    \SubConv\DP
    \quad
    \SubHnf\DP
    \vspace{\infvspace}

    \SubProd\DP

    \vspace{\infvspace}
    \SubSigma\DP
    
    \vspace{\infvspace}
    \SubProof\DP

    \vspace{\infvspace}    
    \SubSub\DP
  \end{center}
  \caption{Coercion par pr�dicats - version algorithmique}
  \label{fig:subtyping-algo-rules}
\end{figure}
}

%%% Local Variables: 
%%% mode: latex
%%% TeX-master: "subset-typing"
%%% LaTeX-command: "TEXINPUTS=\"style:$TEXINPUTS\" latex"
%%% End: 


\def\typec{\vdash_{CCI}}

\newcommand{\timpl}[6]{#1 \typea #2 : #3 "~>" #4 \typec #5 : #6}
\newcommand{\subimpl}[4]{#1 \typec #2 : #3 \sub #4}

\def\muterm{\mu_{\mbox{term}}}

\def\WfAtomI{
  \UAX{Wf-Atom}
  {}
  {$\wf [] "~>"~\wf []$}
  {}
}  
  
\def\WfVarI{
  \UAX{Wf-Var}
  {$\timpl{`G}{A}{s}{`G'}{A'}{s}$}
  {$\wf `G, x : A "~>"~\wf `G', x : A'$}
  {$s `: \{ \Set, \Prop, \Type(i) \}$}
}

\def\PropSetI{
  \UAX{PropSet}
  {$\wf `G "~>" ~\wf `G'$}
  {$\timpl{`G}{s}{\Type(0)}{`G'}{s}{\Type(0)}$}
  {$s `: \{ \Prop, \Set \}$} 
}

\def\TypeTypeI{
  \UAX{Type}
  {$\wf `G "~>" \wf `G$}
  {$\timpl{`G}{\Type(i)}{\Type(i + 1)}{`G'}{\Type(i)}{\Type(i + 1)}$}
  {}
}

\def\VarI{
  \BAX{Var}
  {$\wf `G "~>"~\wf `G'$}
  {$x : A `: `G "~>" x : A' `: `G'$}
  {$\timpl{`G}{x}{A}{`G'}{x}{A'}$}
  {}
}

\def\ProdI{
  \BAX{Prod}
  {$\timpl{`G}{T}{s1}{`G'}{T'}{s1}$}
  {$\timpl{`G, x : T}{U}{s2}{`G', x : T'}{U'}{s2}$}
  {$\timpl{`G}{\Pi x : T.U}{s2}{`G'}{\Pi x : T'.U'}{s2}$}
  {$(s1, s2) `: \text{{\sc Sort}}$}
}

\def\AbsI{
  \BAX{Abs}
  {$\timpl{`G}{\Pi x : T. U}{s}{`G'}{\Pi x : T'. U'}{s}$}
  {$\timpl{`G, x : T}{M}{U}{`G', x : T'}{M'}{U'}$}
  {$\timpl{`G}{\lambda x : T. M}{\Pi x : T.U}
    {`G'}{\lambda x : T'. M'}{\Pi x : T'.U'}$}
  {}
}

\def\AppI{
  \QAX{App}
  {$\timpl{`G}{f}{T}{`G'}{f'}{T'}$}
  {$\mu~T' `= (\pi, \Pi x : V'. W')$}
  {$\timpl{`G}{u}{U}{`G'}{u'}{U'}$}
  {$\subimpl{`G'}{c}{U'}{V'}$}
  {$\timpl{`G}{f u}{W[u/x]}{`G'}{(\pi~f')~(c~u')}{W'[ c~u' / x ]}$}
  {}
}

\def\LetInI{
  \BAX{LetIn}
  {$\timpl{`G}{t}{T}{`G'}{t'}{T'}$}
  {$\timpl{`G, x : T}{v}{V}{`G', x : T'}{v'}{V'}$}
  {$\timpl{`G}{\letml x = t \inml v}{V[t / x]}
    {`G'}{\letml x = t' \inml v'}{V'[t' / x]}$}
  {}
}

\def\SigmaRI{
  \BAX{Sigma}
  {$\timpl{`G}{T}{s1}{`G'}{T'}{s1}$}
  {$\timpl{`G, x : T}{U}{s2}{`G', x : T'}{U'}{s2}$}
  {$\timpl{`G}{\Sigma x : T.U}{s2}{`G'}{\Sigma x : T'.U'}{s2} $}
  {$(s1, s2) `: \matht{Sort}$}
}


\def\SumInfI{
  \TAXWide{SumInf}
  {$\timpl{`G}{t}{T}{`G'}{t'}{T'}$}
  {$\timpl{`G}{u}{U}{`G'}{u'}{U'}$}
  {$\timpl{`G}{\Sigma \_ : T.U}{s}{`G'}{\Sigma \_ : T'.U'}{s}$}
  {$\timpl{`G}{(t, u)}{\Sigma \_ : T.U}{`G'}{(t', u')}{\Sigma \_ : T'.U'}$}
  {}
}

\def\SumDepI{
  \TAXWide{SumDep}
  {$\timpl{`G}{t}{T}{`G'}{t'}{T'}$}
  {$\timpl{`G}{u}{U[t/x]}{`G'}{u'}{U'[t'/x]}$}
  {$\timpl{`G}{\Sigma x : T.U}{s}{`G'}{\Sigma x : T'.U'}{s}$}
  {$\timpl{`G}{(t, u : U)}{\Sigma x : T.U}{`G'}{(t', u')}{\Sigma x : T'.U'}$}
  {}
}
 
\def\LetSumI{
  \BAX{LetSum}
  {$\timpl{`G}{t}{\Sigma x : T. U}{`G'}{t'}{\Sigma x : T'.U'}$}
  {$\timpl{`G, x : T, u : U}{v}{V}{`G, x : T', u : U'}{v'}{V'}$}
  {$\timpl{`G}{\letml (x, u) = t \inml v}{V}
    {`G'}{\letml (x, u) = t' \inml v'}{V'}$}
  {}
}

\def\SubsetI{
  \BAX{Subset}
  {$\timpl{`G}{U}{\Type}{`G'}{U'}{\Type}$}
  {$\timpl{`G, x : U}{P}{\Prop}{`G', x : U'}{P'}{\Prop} $}
  {$\timpl{`G}{\subset{x}{U}{P}}{\Type}{`G'}{\subset{x}{U'}{P'}}{\Type}$}
  {$$}
}

\def\marginleft{0em}

\def\typeiFig{
\begin{figure}[ht]
  \begin{center}
    \def\fCenter{\wf}
    \def\type{\typec}
    
    \WfAtomI\DP

    \vspace{\infvspace}
    \WfVarI\DP
    
    \vspace{\infvspace}
    \PropSetI\DP
    
    \vspace{\infvspace}
    \VarI\DP
    
    \vspace{\infvspace}
    \ProdI\DP
    
    \vspace{\infvspace}
    \AbsI\DP
    
    \vspace{\infvspace}
    \AppI\DP

    \vspace{\infvspace}
    \LetInI\DP
    
    \vspace{\infvspace}
    \SigmaRI\DP

    \vspace{\infvspace}
    \SumInfI\DP

    \vspace{\infvspace}
    \SumDepI\DP
    
    \vspace{\infvspace}
    \LetSumI\DP
    
    \vspace{\infvspace}
    \SubsetI\DP
  \end{center}
  \label{typing-impl-rules}
  \caption{R�ecriture du typage vers \Coq}
\end{figure}
}

\def\typemuiFig{
\begin{figure}
  \begin{eqnarray*}
    (f, \subset{x}{U}{P}))   & "=>" & \letml (f, t) = \muterm~(f, U) \inml (f `o
    \pi_1, t) \\
    x                        & "=>"  & x
  \end{eqnarray*}
  \label{muimpl-definition}
  \caption{D�finition de $\muterm$}
\end{figure}
}

%%% Local Variables: 
%%% mode: latex
%%% TeX-master: "subset-typing"
%%% LaTeX-command: "TEXINPUTS=\"style:$TEXINPUTS\" latex"
%%% End: 

\def\ctxdot{\text{\textbullet}}

\newcommand{\SubConvI}[1][\Gamma]{
\UAX{SubConv}
{$T \eqbr U$}
{$\subimpl{#1}{\ctxdot}{T}{U}$}
{$T = \hnf{T} `^ T "/=" \Pi, \Sigma, \{|\} `^ U = \hnf{U}$}
}

\newcommand{\SubConvIs}[1][\Gamma]{
\UAX{SubConv}
{$T \eqbr U$}
{$\subimpl{#1}{\ctxdot}{T}{U}$}
{} 
}

\newcommand{\SubHnfI}[1][\Gamma]{
\UAX{SubHnf}
{$\subimpl{#1}{c}{\hnf{T}}{\hnf{U}}$}
{$\subimpl{#1}{c}{T}{U}$}
{$T "/=" \hnf{T} `V U "/=" \hnf{U}$} 
}

\newcommand{\SubProdI}[1][\Gamma]{%
  \BAX{SubProd}
  {$\subimpl{#1}{c_1}{U}{T} : s$}
  {$\subimpl{#1, x : U}{c_2}{V}{W}$}  
  {$\subimpl{#1}{\lambda x : \ip{U}{#1}. c_2[\ctxdot~(c_1[x])]}
    {\Pi x : T.V}{\Pi x : U.W}$}
  {}
}

\newcommand{\SubSigmaI}[1][\Gamma]{%
\BAX{SubSigma}
{$\subimpl{#1}{c_1}{T}{U}$}
{$\subimpl{#1, x : T}{c_2}{V}{W}$}
{$\subimpl{#1}{\pair{\ip{\Sigma x : U.W}{`G}}{c_1[\pi_1~\ctxdot]}{c_2[\pi_2~\ctxdot][\pi_1~\ctxdot/x]}}
  {\Sigma x : T. V}{\Sigma x : U. W}$}
{}
}

\newcommand{\SubSubI}[1][\Gamma]{%
\UAX{SubSub}
{$\subimpl{#1}{c}{U}{T}$}
{$\subimpl{#1}{c[\eltpit~\ctxdot]}{\mysubset{x}{U}{P}}{T}$}
{$T = \hnf{T} `^ \hnf{T} "/=" \{|\}$}
}

\newcommand{\SubSubIs}[1][\Gamma]{%
\UAX{SubSub}
{$\subimpl{#1}{c}{U}{T}$}
{$\subimpl{#1}{c[\eltpit~\ctxdot]}{\mysubset{x}{U}{P}}{T}$}
{}
}
   
\newcommand{\SubProofI}[1][\Gamma]{%
  \BAX{SubProof}
  {$\subimpl{#1}{c}{T}{U}$}
  {$\timpl{#1}{\mysubset{x}{U}{P}}{\Set}$}
  {$\subimpl{#1}
    {\elt{\ip{U}{#1}}{\ip{\lambda x : U.P}{#1}}{c}
      {\ex{\ip{P}{#1, x : U}[c/x]}}}
  {T}{\mysubset{x}{U}{P}}$}
{$T = \hnf{T}$}
}

\def\subtiFig{
\begin{figure}[ht]
  \begin{center}
    \def\fCenter{\subtd}

    \vspace{\infvspace}
    \SubConvI\DP
    
    \vspace{\infvspace}
    \SubHnfI\DP

    \vspace{\infvspace}
    \SubProdI\DP

    \vspace{\infvspace}
    \SubSigmaI\DP
    
    \vspace{\infvspace}
    \SubSubI\DP
    
    \vspace{\infvspace}
    \SubProofI\DP
    
  \end{center}
  \caption{R��criture de la coercion vers \CCI}
  \label{fig:coerce-impl-rules}
\end{figure}
}

\def\subtisc{
  \begin{center}
    \def\fCenter{\subti}
    \SubConvIs\DP
    \quad
    \SubHnfI\DP

    \vspace{\infvspace}
    \SubProdI\DP
    
    \vspace{\infvspace}
    \SubSigmaI\DP

    \vspace{\infvspace}
    \SubSubIs\DP

    \vspace{\infvspace}
    \SubProofI\DP
  \end{center}}

\def\subtis{
\begin{figure}[ht]
  \subtisc
  \caption{R��criture de la coercion vers \CCI}
  \label{fig:coerce-impl-rules-short}
\end{figure}
}

%%% Local Variables: 
%%% mode: latex
%%% TeX-master: "subset-typing"
%%% LaTeX-command: "TEXINPUTS=\"style:.:\" latex"
%%% End: 


\title %[Short Paper Title] (optional, use only with long paper titles)
{Subset coercions in \Coq}

\author[Matthieu Sozeau]
{{\sc Matthieu Sozeau} \\ under direction of {\sc Christine Paulin-Mohring}}


\def\LRI{\name{LRI}}
\def\INRIAFuturs{\name{INRIA Futurs}}
\def\Proval{\name{Proval}}

\institute[]
{
  Paris-Sud 11 University \\
  \LRI{} - \Demons{} Team \& \INRIAFuturs{} - \Proval{} Project
}

\date[TYPES'06] % (optional, should be abbreviation of conference name)
{TYPES'06 Workshop \\
18-21 April 2006}

\subject{Theoretical Computer Science}

\pgfdeclareimage[height=0.5cm]{ups-logo}{../figures/ups-logo}
\pgfdeclareimage[height=0.5cm]{lri-logo}{../figures/lri-logo}
\pgfdeclareimage[height=0.5cm]{inria-logo}{../figures/inria-logo}

\logo{\pgfuseimage{ups-logo}
  \pgfuseimage{lri-logo}
  \pgfuseimage{inria-logo}}

\def\imgheight{7.5cm}

\pgfdeclareimage[height=\imgheight]{bigpic1}{bigpic1}
\pgfdeclareimage[height=\imgheight]{bigpic2}{bigpic2}
\pgfdeclareimage[height=\imgheight]{bigpic3}{bigpic3}
\pgfdeclareimage[height=\imgheight]{bigpic4}{bigpic4}
\pgfdeclareimage[height=\imgheight]{bigpic5}{bigpic5}

\pgfdeclareimage[height=3cm]{lambda-cube}{../figures/lambda-cube}

\pgfdeclareimage[interpolate=true,height=8cm]{subtac-syntax}{capturesyntax}
\pgfdeclareimage[interpolate=true,height=8cm]{myhd-proof}{capturemyhd-proof}
\pgfdeclareimage[interpolate=true,height=8cm]{myhd-extract}{capturemyhd-extract}
\pgfdeclareimage[interpolate=true,height=8cm]{mytail-extract}{capturemytail-extract}
\pgfdeclareimage[interpolate=true,height=8cm]{mytest-good}{capturemytest-ok}
\pgfdeclareimage[interpolate=true,height=8cm]{mytest-good-extract}{capturemytest-ok-extract}
\pgfdeclareimage[interpolate=true,height=8cm]{mytest-bad}{capturemytest-bad}

\pgfdeclareimage[interpolate=true,height=8cm]{append-proof}{captureappend-proof}
\pgfdeclareimage[interpolate=true,height=8cm]{append-extract}{captureappend-extract}
\pgfdeclareimage[interpolate=true,height=8cm]{append-app}{captureappend-app}

\def\typea{\vdash}
\def\suba{\rightslice}

\AtBeginSection[]
{
  \begin{frame}<beamer>
    \frametitle{Outline}
    \tableofcontents[currentsection]
  \end{frame}
}

\begin{document}

\begin{frame}
  \titlepage
\end{frame}

\logo{}

\begin{frame}
  \frametitle{The Big Picture}
  \only<1>{\pgfuseimage{bigpic1}}\only<2>{\pgfuseimage{bigpic2}}\only<3>{\pgfuseimage{bigpic3}}\only<4>{\pgfuseimage{bigpic4}}\only<5>{\pgfuseimage{bigpic5}}
\end{frame}

\section{The idea}

\begin{frame}
  \frametitle{A simple idea}
  
  \begin{block}{Definition}
    $\mysubset{x}{T}{P}$ is the set of objects of set $T$ verifying property $P$. 
  \end{block}
  
  \begin{itemize}
  \item Useful for specifying, widely used in mathematics ;    
  \item Separates object and property.
  \end{itemize}
  \pause
  \begin{block}{Adapting the idea} 
    \begin{center}
      \BAX{}{$t : T$}
      {$P[t]$}
      {$t : \mysubset{x}{T}{P}$}
      {}\DP\quad      
      \UAX{}{$t : \mysubset{x}{T}{P}$}
      {$t : T$}
      {}
      \DP
    \end{center}
  \end{block}  
\end{frame}

\begin{frame}[t]
  \frametitle{From ``\ps''\ldots}
  
  \begin{block}{\PVS{}}
    \vspace{0.7em}
    \begin{itemize}
    \item Specialized typing algorithm for subset types, generating
      \emph{Type-checking conditions}.    
      \begin{tabular}{lcll}
        $t : \mysubset{x}{T}{P}$ & used as & $t : T$
        & ok \\
        $t : T$ & used as & $t : \mysubset{x}{T}{P}$ 
        & if $P[t/x]$
      \end{tabular}
      \pause
      
    \item[{\bf +}] Practical success ; \pause
    \item[{\bf --}] No strong safety guarantee in \PVS{}.
    \end{itemize}  
  \end{block}
%   \pause 
%   \begin{block}{\Coq{}}
%     Use coercions to explicit the equivalence:
%     \begin{tabular}{lclcl}
%       $t : \mysubset{x}{T}{P}$ & \only<4>{$"<=>"$} \only<5->{$"->"$} & $t : T$
%       & \only<5->{$"~>"$} & \only<5->{$\pi_1~t$ \\
%         $t : T$ & $"->"$ & $t : \mysubset{x}{T}{P}$ 
%         & $"~>"$ & $\sref{elt}~{t}~(\alert<3->{? : P[t/x]})$}
%     \end{tabular}
%   \end{block}
\end{frame}

\begin{frame}[t]
  \frametitle{\ldots to Subset coercions}
  
  \begin{enumerate}
  \item<1-> A property-irrelevant language (\lng{}) with \alert{decidable} typing ;
  \item<2-> A total traduction to \Coq{} terms with holes ;
  \item<3-> A method to turn the holes into proof obligations.
  \end{enumerate}
  
  \only<1-2>{
  \begin{center}
    \UAX{}{$`G \typea t : \mysubset{x}{T}{P}$}
    {$`G \typea \uncover<2>{\eltpit~}t : T$}
    {}\DP

    \vspace{0.5cm}
    \BAX{}{$`G \typea t : T$}
    {$`G, x : T \type P : \Prop$}
    {$\uncover<2>{(}t\uncover<2>{, ?)} : \mysubset{x}{T}{P}$}
    {\uncover<2>{$`G \type ? : P[t]$}}\DP
  \end{center}
  }

%   \begin{center}
%   \begin{tabular}{lclcl}
%     $t : \mysubset{x}{T}{P}$ & \only<1>{$"<=>"$} \only<2->{$"->"$} & $t : T$
%     & \only<2->{$"~>"$} & \only<2->{$\pi_1~t$ \\
%       $t : T$ & $"->"$ & $t : \mysubset{x}{T}{P}$ 
%       & $"~>"$ & $\sref{elt}~{t}~(\alert<3->{? : P[t/x]})$}
%   \end{tabular}
% \end{center}

\end{frame}

\section{Theoretical development}

\subsection{\lng{}}

\begin{frame}[t,label=typingdecl]
  \frametitle{\lng{}'s typing $\typed$ and coercion $\subd$}
  
  Calculus of Constructions - \irule{Conv} +
  \typenvd
  \begin{center}
    \vspace{0.5cm}
    \Coerce\DP

    \pause
    \def\fCenter{\subd}
    
    \SubConv\DP
    \SubTrans\DP

    \vspace{0.5cm}
    \only<3->{\SubSub\DP
    \SubProof\DP}

    \vspace{0.5cm}
    \only<6->{\SubProd\DP}
%      \SubSigma\DP}
  \end{center}

  \only<4-5>{
    \begin{center}
      \AXC{$0 : \nat$}
      \AXC{$\nat~\sub \mysubset{x}{\nat}{x \neq 0}$}
      \BIC{$0 : \mysubset{x}{\nat}{x \neq 0}$}
      \only<5>{\UIC{\alert{$? : 0 \neq 0$}}}
      \DP
    \end{center}
  }
\end{frame}

\begin{frame}[t]
  \frametitle{Results}

  \begin{theorem}[Decidability of type checking]
    $`G \type t : T$ is decidable.
  \end{theorem}  
  
  \uncover<2->{
  \begin{lemma}[Elimination of transitivity]
    If $T \sub U `^ U \sub V$ then $T \sub V$.
  \end{lemma}}
  
  \only<1-2>{
    \begin{prooftree}
      \TAX{App}
      {$\tgen{`G}{f}{T} \quad T \suba \Pi x : A.B$}
      {$\tgen{`G}{e}{E}$}
      {$E \suba A$}
      {$\tgen{`G}{(f~e)}{B [ e / x ]}$}
      {}
    \end{prooftree}
  }
  
  \uncover<3->{
    \begin{definition}[Forgetful inclusion]
      If $`G \typecci t : T$ then $f(`G) \typea f(t) : f(T)$ if $f$ is
      defined on $`G, t$ and $T$. 
    \end{definition}

    \[\begin{array}{lcl}
      f~(\eltpit~t) & = & f~t \\
      f~(\eltpip~t) & = & `_ \\
      f~(\elt{T}{P}{t}{p}) & = & f~t
    \end{array}\]
  }
    
\end{frame}

\def\typec{\vdash_{?}}
\renewcommand{\subimpl}[4]{#1 \typec #2 : #3 \sub #4}

\newcommand{\blue}[1]{{\color{blue}#1}}

\subsection{Traduction in \Coq{}}
\setboolean{displayLabels}{false}
\begin{frame}[t]
  \frametitle{Traduction: the target}

  \[`G \type t : T "->" \ipG{`G} \typec \ip{t}{`G} : \ip{T}{`G}\]
  
  \begin{center}
    \UAX{}
    {$`G \typec P : \Prop$}
    {$`G \typec ?_{P} : P$}
    {}\DP
    
    \vspace{0.25cm}
    \BAX{}{$`G \typec t : T$}
    {$`G \typec p : P[t/x]$}
    {$\elt{T}{P}{t}{p} : \mysubset{x}{T}{P}$}
    {}\DP\quad
    \UAX{}{$`G \typec t : \mysubset{x}{T}{P}$}
    {$`G \typec \eltpit~t : T$}
    {}
    \DP
  \end{center}
  
\end{frame}

\begin{frame}[t]
  \frametitle{Traduction: deriving explicit coercions}
  
  \begin{block}{Goal}
    \[T \sub U "->" `G \typec \blue{c}[\ctxdot] : \ip{T}{`G} \sub
    \ip{U}{`G}\]
  \end{block}
  
  \uncover<2->{
    \begin{definition}
      \begin{center}
        \def\fCenter{\sub}    
        \UAX{SubConv}
        {$T \eqbr U$}
        {$\subimpl{`G}{\uncover<3->{\blue{\ctxdot}}}{T}{U}$}
        {}\DP
        
        \vspace{0.25cm}
        \uncover<4->{$\subimpl{`G}{\blue{\eltpit~\ctxdot}}{\mysubset{x}{T}{P}}{T}$}
        
        \vspace{0.25cm}
        \uncover<5->{
          {$\subimpl{`G}
            {\blue{\elt{}{}{\ctxdot}
                {\ex{\ip{P}{`G, x : T}[\ctxdot/x]}}}}
            {T}{\mysubset{x}{T}{P}}$}}
      \end{center}
    \end{definition}
  
    \uncover<6->{
      \begin{example}      
        \begin{prooftree}
          \AXC{$`G \typec 0 : \nat$}
          \AXC{$`G \typec c : \nat \sub \mysubset{x}{\nat}{x \neq 0}$}
          \BIC{$`G \typec (\elt{}{}{\ctxdot}{?_{(x \neq 0)[\ctxdot/x]}})[0]
            = \elt{}{}{0}{?_{\alert{0 \neq 0}}} : \mysubset{x}{\nat}{x \neq 0}$}
        \end{prooftree}
      \end{example}}
  }
\end{frame}

\begin{frame}[t]
  \frametitle{Traduction: interpretation of terms $\ip{}{`G}$}
    
  \begin{theorem}[Soundness]
    If $`G \typea t : T$ then $\ipG{`G} \typec \ip{t}{`G} : \ip{T}{`G}$.
  \end{theorem}

  \begin{example}[Application]
    \setboolean{displayLabels}{false}
    \typenva
    \begin{center}
      \TAX{App}
      {$\tgen{`G}{f}{T} \quad T \suba \Pi x : V. W : s$}
      {$\tgen{`G}{u}{U}$} % \quad \tgen{#1}{#9, #5}{#7'}$}
      {$U \suba V$}
      {$\tgen{`G}{(f~u)}{W[u/x]}$}
      {}
      \DP
    \end{center}
    
    \typenvi
    \[\begin{array}{lcll}
      \ip{f~u}{`G} 
      & = & \letml~\pi = \coerce{`G}{T}{(\Pi x : V.W)}~\inml & \\
      & & \letml~c = \coerce{`G}{U}{V}~\inml & \\
      & & (\pi[\ip{f}{`G}])~(c[\ip{u}{`G}]) & \\
    \end{array}\]    
  \end{example}

\end{frame}


\begin{frame}[t]
  \frametitle{Theoretical matters}
  
  $\typec$'s equational theory:
  \[\begin{array}{llcll}
    (\beta) & (\lambda x : X.e)~v & `= & e[v/x] & \\
%    (\pi_i) & \pi_i~\pair{T}{e_1}{e_2} & `= & e_i & \\
    (\sigma_i) & \sigma_i~(\elt{E}{P}{e_1}{e_2}) & `= & e_i & \\
    (\eta) & (\lambda x : X.e~x) & `= & e & \text{if $x `; FV(e)$} \\ % et $e : \Pi x : X.Y$} \\
    (\text{\SP}) % & \pair{\Sigma x : X.Y}{\pi_1~e}{\pi_2~e} & `= & e & \\ % \text{if $e : \Sigma x : X. Y$} \\
    & \elt{E}{P}{(\eltpit~e)}{(\eltpip~e)} & `= & e & \\%\text{if $e : \mysubset{x}{E}{P}$} \\
    \uncover<2->{\alert{(\sigma)} & \elt{E}{P}{t}{p} & `= & \elt{E}{P}{t'}{p'} & \text{if $t
        `= t'$}}
  \end{array}\]

  \only<2->{$"=>"$ Proof Irrelevance}

\end{frame}

\section{The implementation}

\begin{frame}[t]
  \frametitle{The \Program vernacular}
  
  \begin{block}{Architecture}
    Wrap around \Coq{}'s vernacular commands (\texttt{Definition},
    \texttt{Fixpoint}, \ldots).
    
    \begin{enumerate}
    \item Use the \Coq{} parser ;
    \item<2-> Typecheck $`G \type t : T$ and generate $\ip{t}{`G} :
      \ip{T}{`G}$ ;
    \item<3-> Interactive proving of obligations ;
    \item<4-> Final definition.
    \end{enumerate}
  \end{block}

  \[ \uncover<1-3>{\texttt{Program}~}\Definition~f~:
  \uncover<2->{\llbracket} T \uncover<2->{\rrbracket_{`G}} := 
  \uncover<2->{\llbracket} t \uncover<2->{\rrbracket_{`G}}\uncover<3->{ + \text{ obligations}}. \]
  
  \uncover<5->{
  \begin{remark}[Restriction]
    We assume $`G \typecci \ip{T}{`G} : s$.
  \end{remark}}
  
\end{frame}

\begin{frame}[t]
  \frametitle{\Program: The list example}
  \vspace{-0.75em}
  \begin{center}%
    \only<1>{\pgfuseimage{subtac-syntax}}%
    \only<2>{\pgfuseimage{myhd-proof}}%
    \only<3>{\pgfuseimage{myhd-extract}}%
    \only<4>{\pgfuseimage{mytail-extract}}%
    \only<5>{\pgfuseimage{mytest-good}}%
    \only<6>{\pgfuseimage{mytest-bad}}%
    \only<7>{\pgfuseimage{append-proof}}%
    \only<8>{\pgfuseimage{append-extract}}%
    \only<9>{\pgfuseimage{append-app}}%
  \end{center}
\end{frame}

\begin{frame}
  \frametitle{Conclusion}
  
  \begin{block}{Our contribution}
    A more \alert{flexible} programming language, \alert{conservative} over
    \CCI, \alert{integrated} with the existing environment and a formal justification
    of ``\ps{}''.
  \end{block}
  \pause
  \begin{block}{Future work}
    \vspace{0.2em}
    \begin{itemize}
    \item Application to more constructs ((co-)inductive types) and commands.
    \item Improvements of \Coq{} (existential variables, type inference,
      proof irrelevance).
    \item Complete and useful interpretation of \ML{} languages.
    \end{itemize}
  \end{block}
  
\end{frame}

\begin{frame}[t]
  \frametitle{Addendum: some practical enhancements}
  
  \begin{itemize}
  \item Handling of dependent existential variables (WIP).
  \item<2-> Pattern-matching and equalities.
  \item<3-> Well-founded recursion.
  \end{itemize}

  \only<2>{
    \[\matchml~v~\returnml~T~\withml~p_1 "=>" t_1 \cdots{} p_n "=>" t_n \]
    \[\begin{array}{lcl}
      (\matchml~v~\asml~t' & \returnml & (t' = t "->" T)~\withml \\
      p_1 "=>" \funml{}~H & "=>" & t_1 \\
      & \vdots{} & \\
      p_n "=>" \funml{}~H & "=>" & t_n) \\
      (\sref{refl\_equal}~\mu{(t)}) & &
    \end{array}
    \]
  }

  \only<3->{
    \[ \texttt{Program}~\Fixpoint~f~(a : nat)~\{ \mlkw{wf}~lt~a \} : \nat := t \]
  }
  \only<4->{
    \[\begin{array}{lcl}
      a & : & \nat \\
      f & : & \{ x : \nat `| x < a \} "->" \nat \\
      \hline
      t & : & \nat
    \end{array}\]
  }

\end{frame}


\end{document}



% \begin{frame}[t]
%   \frametitle{Technically $\ldots$}
  
%   \begin{block}{Coercions for \PVS's \ps}
%     \vspace{0.7em}
%     \begin{itemize}
%     \item<1-> A type system based on \CC + $\Sigma$ types
%       ($\typea, \eqbr$) with a \alert{larger} conversion ($\suba$). 
%     \item<2-> An interpretation in \CCI ($\typec, `=$) with metavariables.
%     \item<5-> Needs eta-rules and proof-irrelevance ; $`=$ is really $\eqbpers$.
%     \item<6-> An improved apparatus for handling \alert{dependent} metavariables in \Coq{} (WIP).
%     \end{itemize}
%   \end{block}
%   \only<1>{
%     \begin{center}
%       $\mysubset{x}{A}{P} \eqdef \Sigma x : A.P$ where $A : \Set$ and 
%       $x : A \typea P : \Prop{}$.
%       $A \suba \mysubset{x}{A}{P}$ and $\mysubset{x}{A}{P} \suba A$
%     \end{center}
%   }

%   \only<3-4>{
%     Derivation of coercions:
%     \setboolean{displayLabels}{false}
%     \begin{center}
%       \SubSubI\DP
      
%       \vspace{\infvspace}
%       \only<4>{\SubProofI\DP}
%     \end{center}
%   }
  
%   \only<5>{
%     \begin{theorem}[Preservation of the equivalence]
%       If $\talgo{`G}{T, T'}{s}$, $T \eqbr T'$ then 
%       $\ip{T}{`G} `= \ip{T'}{`G}$.
%     \end{theorem}
    
%     \begin{center}
%       $(\sigma)\quad\elt{E}{P}{t}{p} = \elt{E}{P}{t'}{p'} \text{ if } t `=
%       t'$
%     \end{center}
%   }
  
% \end{frame}


%\end{document}

% \begin{frame}
%   \frametitle{What's missing ?}
  
%   A phase distinction between programming and proving, using
%   \emph{Type Checking Conditions}.
%   \pause
%   \begin{block}{Technical requirements}
%     \begin{itemize}
%     \item Metavariables ;
%     \item Proof shell support.
%     \end{itemize}
%   \end{block}
    
% \end{frame}

% \begin{frame}[t]
%   \frametitle{Contribution}
  
%   An integration of \PVS{} ``Predicate Subtyping'' in \Coq.
% \end{frame} 

% \subsection{Related work}

% \begin{frame}
%   \frametitle{Inverting extraction}
%   \begin{block}{The \Program{} tactic (\name{Catherine Parent}), 94}
%     \begin{itemize}
%     \item[{\bf +}] General method for proof synthesis from 
%       programs ;
      
%     \item[{\bf --}] Complex technic, generating relatively complicated
%       terms and not well integrated with the assistant but completely
%       coupled with the extraction mechanism.
      
%     \item Unmaintained (incompatible with the current extraction mechanism). , \Refine.
%     \end{itemize}
%   \end{block}
% \end{frame}


% \begin{frame}[t]
%   \frametitle{Subtyping \& coercions}

%   \begin{itemize}
%   \item Zhaohui Luo: ``\emph{coercive subtyping}'', theoretical aspects
%     (coherence, transitivity of coercions).
%   \item Gang Chen: subtyping and coercions in dependent type systems
%     ($\lambda P, \lambda C$)
%   \item Amokrane Sa�bi: coercion system for \Coq, with inheritance.
%   \end{itemize}

% \end{frame}

% \section{Contribution}

% \subsection{A language for programming in \Coq} 
% \begin{frame}[t]
%   \frametitle{Architecture}
  
%   \begin{itemize}
%   \item<1-> Writing programs in a language (\lng{}) with an enriched equivalence;
%   \item<2-> Interpretation to \CCI, inserting coercions ; 
%   \item<3-> Proof obligations solving.
%   \end{itemize}

%   \begin{tabular}{lclcl}
%     $t : \mysubset{x}{T}{P}$ & \only<1>{$"<=>"$} \only<2->{$"->"$} & $t : T$
%     & \only<2->{$"~>"$} & \only<2->{$\pi_1~t$ \\
%       $t : T$ & $"->"$ & $t : \mysubset{x}{T}{P}$ 
%       & $"~>"$ & $\sref{elt}~{t}~(\alert<3->{? : P[t/x]})$}
%   \end{tabular}
    
% \end{frame}

% \begin{frame}
%   \frametitle{\lng{} syntax}
%   \begin{columns}
%     \begin{column}{0.5\textwidth}
%       \begin{block}{Termes}
%         \termgrammar
%       \end{block}
%     \end{column}
%     \begin{column}{0.5\textwidth}
%       \begin{block}{Types}
%         \typegrammar
%       \end{block}
%     \end{column}
%   \end{columns}  
% \end{frame}


% \begin{frame}
%   \frametitle{Some facts}
 
%   \begin{itemize}
%   \item $\eqbr ``(= \subd$.
%   \item $\subd$ is an equivalence relation.
%   \item Conservativity over \CCI{}.
%   \end{itemize}

% \end{frame}

% \typenva

% \begin{frame}[t,label=sysalgo]
%   \frametitle{Algorithmic system}

%   \begin{itemize}
%   \item<1-> Restrict coercion rule to application premise.
%     \hyperlink{typingdecl<3>}{\beamergotobutton{}} ;
    
%   \item<2-> Deterministic coercion algorithm (transitivity elimination).
%     \hyperlink{subtypingdecl<3>}{\beamergotobutton{}} ;
%   \end{itemize}

%   \only<1>{\TAX{App}
%     {$`G \seq f : T \quad \mu(T) `= \Pi x : V. W$}
%     {$`G \seq u : U$}
%     {$U \subd V $}
%     {$`G \seq (f u) : W [ u / x ]$}
%     {}
%     \DP
    
%     \begin{eqnarray*}
%       \mu~\mysubset{x}{U}{P} & "=>" & \mu~U \\
%       \mu~x                & "=>" & x
%     \end{eqnarray*}
%   }
  
%   \only<3>{
%     \begin{theorem}[Soundness \& Completness]
%       $`G \typed t : T$ iff there exist $U$ so that $`G \typea t : U$
%       where $U \suba T$ and $U \suba T$ iff $U \subd T$.
%     \end{theorem}
%     \begin{fact}[D�cidabilit� de $\typea$]
%       Pour tout $`G$, $t$, $T$, $`G \typea t : T$ est d�cidable.
%     \end{fact}
%   }
  
% \end{frame}

% \subsection{Interpretation in \CCI}
% \typenvi
% \begin{frame}<beamer>
%   \frametitle{Plan}
%   \tableofcontents[currentsection,currentsubsection]
% \end{frame}


% \renewcommand{\subimpl}[4]{#1 \typec {\color{blue}#2} : #3 \sub #4} %{{\color{blue}#3} \sub {\color{blue}#4}}}

% \setboolean{displayLabels}{false}
% \begin{frame}[t]
%   \frametitle{Explicit coercions}
%   % Remettre de la logique, v�ritable s�mantique 
%   \pause
%   \begin{center}
%       \def\fCenter{\subti}          
%       \SubConvI\DP

%       \vspace{0.25cm}     
%       \only<3->{
%         \SubProdI\DP
        
%         \vspace{0.25cm}
%         \SubSigmaI\DP
%       }
      
%       \vspace{0.25cm}
%       \only<4->{\SubSubI\DP}

%       \vspace{0.25cm}
%       \only<5->{\UAX{SubProof}
%       % {$`G, x : U \typec P : \Prop$}
%       {$\subimpl{`G}{c}{T}{U}$}
%       {$\subimpl{`G}
%         {\elt{\ip{U}{`G}}{\ip{\lambda x : U.P}{`G}}{c}{\ex{P}[c/x]}}
%         {T}{\mysubset{x}{U}{P}}$}
%       {}\DP
%       }
%     \end{center}

%     \only<4>{$0 : \nat$ and $\nat~\sub \mysubset{x}{\nat}{x \neq 0}$
%       but \[(\lambda t : T. \sref{elt}~(c~t)~(?_{P[c~t/x]}))[0]
%       \eqbr \sref{elt}~0~?_{\alert{0 \neq 0}}\]}

% \end{frame}

% \setboolean{displayLabels}{true}
% \renewcommand{\subimpl}[4]{#2 : #3 \sub #4}

% \renewcommand{\timpl}[6]{#1 \typea #2 : #3\only<2->{"~>" #4 \typec #5 : #6}}

% \newcommand{\QAXII}[7]{\AXC{\only<2->{#3}}\noLine\UIC{#2}
%   \AXC{\only<2->{#5}}\noLine\UIC{#4}
%   \LeftRuleLabel{#1}\BIC{#6}}

% \begin{frame}[t]
%   \frametitle{Interpretation of terms $\ip{}{`G}$}
    
%   \begin{itemize}
%   \item Follows algorithmic typing.
%   \item Straightforward extension to contexts.
%   \end{itemize}
  
%   \only<2->{
%     \begin{example}[Application]
%       \setboolean{displayLabels}{false}
%       \begin{prooftree}
%         \typenva
%         \AppA
%       \end{prooftree}
      
%       \typenvi
%       \[\begin{array}{lcll}
%         \ip{f~u}{`G} 
%         & = & \letml~F~=\typeafn{`G}{f}~\andml~U = \typeafn{`G}{u}~\inml & \\
%         & & \letml~(\Pi x : V.W) = \mualgo{F}~\inml & \\
%         & & \letml~\pi = \coerce{`G}{F}{(\Pi x : V.W)} & \\
%         & & \letml~c = \coerce{`G}{U}{V}\inml & \\
%         & & (\pi[\ip{f}{`G}])~(c[\ip{u}{`G}]) & \\
%       \end{array}\]    
%     \end{example}
%   }
  
%   % \item Correspondance code �crit/extrait.
% \end{frame}

% \begin{frame}[t]
%   \frametitle{Theoretical properties}
  
%   \begin{theorem}[Soundness]
%     If $`G \typea t : T$ then $\ipG{`G} \typec \ip{t}{`G} : \ip{T}{`G}$.
%   \end{theorem}
  
%   \begin{itemize}
%   \item<2-> Depart from the coercion inference algorithm.
%   \item<3-> Requires a larger equational theory.   
%   \only<3->{
%     \[\begin{array}{llcll}
%     (\beta) & (\lambda x : X.e)~v & = & e[v/x] & \\
%     (\pi_i) & \pi_i~\pair{T}{e_1}{e_2} & = & e_i & \\
%     (\sigma_i) & \sigma_i~(\elt{E}{P}{e_1}{e_2}) & = & e_i & \\
%     (\eta) & (\lambda x : X.e~x) & = & e & \text{if $x `; FV(e)$} \\ % et $e : \Pi x : X.Y$} \\
%     (\rho) & \pair{\Sigma x : X.Y}{\pi_1~e}{\pi_2~e} & = & e & \text{if $e : \Sigma x : X. Y$} \\
%     & \elt{E}{P}{(\eltpit~e)}{(\eltpip~e)} & = & e & \text{if $e : \mysubset{x}{E}{P}$} \\
%     (\sigma) & \elt{E}{P}{t}{p} & = & \elt{E}{P}{t'}{p'} & \text{if $t
%       `= t'$}
%   \end{array}\]
%   }
% \end{itemize}

% \end{frame}

% \subsection{The prototype}

% \begin{frame}[t]
%   \frametitle{Implementation}
  
%   \begin{block}{Architecture}
%     Wrap around \Coq{}'s vernacular commands (\texttt{Definition},
%     \texttt{Fixpoint}, \ldots).
    
%     \begin{enumerate}
%     \item \Coq{} parser ;
%     \item Typing with coercions ;
%     \item Interactive proving of obligations.
%     \end{enumerate}
%   \end{block}

%   \pause
%   \begin{block}{Existential variables $?_{`t}$}
%     \Refine{} \only<3>{$"=>"$ \eterm{}}
%   \end{block}
% \end{frame}

% \begin{frame}[t]
%   \frametitle{Pratical enhancements}

%   \begin{itemize}
%   \item Pattern-matching and equalities.
%   \item<2-> Well-founded recursion.
%   \end{itemize}

%   \only<1>{
%     \[\matchml~v~\returnml~T~\withml~p_1 "=>" t_1 \cdots{} p_n "=>" t_n \]
%     \[\begin{array}{lcl}
%       (\matchml~v~\asml~t' & \returnml & (t' = t "->" T)~\withml \\
%       p_1 "=>" \funml{}~H & "=>" & t_1 \\
%       & \vdots{} & \\
%       p_n "=>" \funml{}~H & "=>" & t_n) \\
%       (\sref{refl\_equal}~t) & &
%     \end{array}
%     \]
%   }

%   \only<2->{
%     \[ \Fixpoint~f~(a : nat)~\{ \mlkw{wf}~lt~a \} : \nat := t \]
%   }
%   \only<3->{
%     \[\begin{array}{lcl}
%       a & : & \nat \\
%       f & : & \{ x : \nat `| x < a \} "->" \nat \\
%       \hline
%       t & : & \nat
%     \end{array}\]
%   }

% \end{frame}

% % \begin{frame}
% %   \frametitle{Example: Definition}
% %   \small
% %   \input{subtac-euclid}
% % \end{frame}

% % \begin{frame}
% %   \frametitle{Exemple: obligations de preuve}
% %   \small
% %   \input{subtac-euclid-tcc}
% % \end{frame}

% \subsection{R�sultats th�oriques}

% \begin{frame}
%   \frametitle{Work in progress ;)}
%   \begin{itemize}
%   \item Correction \& compl�tude entre les diff�rents syst�mes.
%   \item Lemmes classiques sur le sous-typage.
%   \item Probl�me de la conversion (revoir Gang Chen).
%   \end{itemize}
% \end{frame}

% \begin{frame}
%   \frametitle{Une preuve int�ressante et courte ?}
% \end{frame}

% \section{Conclusion}
% \begin{frame}
%   \frametitle<presentation>{Conclusion}

%   % Keep the summary *very short*.
%   \begin{itemize}
%   \item A more \alert{flexible} programming language, conservative over \CCI.
%   \item Intergration with existing environnment.
%   \item Formal justification of ``\ps{}''.
%   \end{itemize}
  
%   \pause
%   % The following outlook is optional.
%   \vskip0pt plus.5fill
%   \begin{itemize}
%   \item[\bf{$"=>"$}] Further work:
%     \begin{itemize}
%     \item Extension to other \Coq{} constructs ((co-)inductive types,
%       universes, \ldots).
%     \item Integration of proof-irrelevance in \Coq{}.
%     \end{itemize}
%   \end{itemize}
% \end{frame}
% \end{document}

%%% Local Variables: 
%%% mode: latex
%%% TeX-master: "slides"
%%% LaTeX-command: "x=pdf; TEXINPUTS=\"..:../style:../figures:\" ${pdfx}latex"
%%% TeX-PDF-mode: t
%%% End: 
