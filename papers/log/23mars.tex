\subsection*{23 mars}
Lu \cite{Chen:POPL-2003} en d�tail ainsi que
\cite{DBLP:journals/tcs/LuoS99} ou Zhaohui laisse en suspens la question
de la pertinence d'avoir des r�gles pour les produits reliant 
$\Pi x: A. \matht{list}~ `N$ et $\Pi x : A. \{ \matht{list}~ `N `| \ldots \}$ dans notre syst�me.
Une partie de \cite{DBLP:conf/csl/Luo96}. Plus int�ressant est
peut-�tre l'article sur les combinaisons incoh�rentes de coercions pour
les types sommes \cite{DPLB:conf/types/LuoL03}. Les objectifs de
coh�rence et d'�limination de la transitivit� ne sont pas loin des
notres: on veut que 'computationellement' les coercions soit
inessentielles (au contraire on accepte tout dans la partie logique)
et avoir un sous-typage avec de bonnes propri�t�s.


%%% Local Variables: 
%%% mode: latex
%%% TeX-master: "~/research/coq/papers/subset-typing"
%%% End: 
