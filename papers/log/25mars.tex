\subsection*{25 mars}
Question de la transitivit� de notre "sous-typage".
On peut avoir/demander une forme restreinte de transitivit� du genre: 
$`G \subta t : A \sub B "~>" t' `^ `G \subta t' : B \sub C "~>" t'' "=>"
`G \subta t : A \sub C "~>" t''$. Mais dans un sens puisque notre
sous-typage d�pend des termes, il est clair que l'on a
$`G \subta t : A \sub B "~>" t' `^ `G \subta x : B \sub C "~>" t'' "=>"
`G \subta t : A \sub C "~>" t''$ si $x `; `G$ mais pas plus. Il faut
r�flechir � quelle est la solution la plus utile/souhaitable.
Lecture de la 3�me partie de \cite{ChenPhD} sur le sous-typage dans
$\lambda CC_{\leq}$.

Lecture d'articles sur la syntaxe abstraite pour le challenge {\sc
  PoplMark}\ldots


%%% Local Variables: 
%%% mode: latex
%%% TeX-master: "~/research/coq/papers/subset-typing"
%%% End: 
        