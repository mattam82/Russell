\subsection*{16 mars}
Il faut faire du sous-typage dans la sp�cification aussi:
$f : x : \subset{n}{`N}{0 \neq} "->" \subset{n}{`N}{x <}$.

Les deux phases:
\begin{description}
\item[Inf�rence] on donne les types impr�cis, ie: dans $x > n$, 
  $n "~>" `N$.
\item[Typage] on traverse la premi�re d�rivation de typage en ajoutant
  les coercions appropri�es, par exemple:
  $`G \type n : \subset{n}{`N}{0 \neq}$,  $\type_{inf} n "~>" `N$ est
  r�ecrit en: 
  $`G \type \pi_{1}~n : `N$.
\end{description}



%%% Local Variables: 
%%% mode: latex
%%% TeX-master: "~/research/coq/papers/subset-typing"
%%% End: 
