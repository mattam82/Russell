\subsection*{16 mars}
Il faut faire du sous-typage dans la sp�cification aussi:
$f : x : \subset{n}{`N}{0 \neq} "->" \subset{n}{`N}{x <}$.

Les deux phases:
\begin{description}
\item[Inf�rence] on donne les types impr�cis, ie: dans $x > n$, 
  $n "~>" `N$.
\item[Typage] on traverse la premi�re d�rivation de typage en ajoutant
  les coercions appropri�es, par exemple:
  $`G \type n : \subset{n}{`N}{0 \neq}$,  $\type_{inf} n "~>" `N$ est
  r�ecrit en: 
  $`G \type \pi_{1}~n : `N$.
\end{description}

\subsubsection*{Soir!}

Formalisation des trois jugements:
\begin{description}
\item[$\typed$] Typage d�claratif, syst�me ind�cidable, repr�sentant
  exactement ce qu'on veut ajouter comme fonctionnalit�.
\item[$\typei$] Version algorithmique, utilisant le dernier jugement
  pour r�aliser l'ad�quation avec la pr�sentation d�clarative.
\item[$\judgesubi$] ``Sous-typage'', sans obligations de preuves,
  d�cidable et � peu pr�s d�terministe.
\end{description}

Il faudra ensuite faire la traduction dans \Coq, avec de nouveaux
jugements r�ecrivant les d�rivations.

Propri�t�s � montrer:
\begin{itemize}
\item $`G \typed t : T "=>" `E U, `G \typei t : U `^ `G \judgesubi t : U \sub T$
\item $`G \typei t : T "=>" `G \typed t : T$
\end{itemize}

%%% Local Variables: 
%%% mode: latex
%%% TeX-master: "~/research/coq/papers/subset-typing"
%%% End: 
