\subsection*{15 mars}
Distinction inf\'erence et typage. 
\begin{description}
\item[Inf\'erence ($"~>"$)] \`a la ML, on v\'erifie: $`G \typei p "~>" T "=>"
  `G \typed p ":" T$. 
\item[Typage ($:$)]. On a $`G \typed p : T$, on veut $`E U, `G \typei p
  "~>" U `^ `G \typei U "~>" T$. On utilise le sous-typage g\'en\`erant les
  obligations de preuve.
\end{description}

Eliminer \rname{LetSub}, inutilisable en pratique.

Quelques points \`a m\'editer:
\begin{itemize}
\item $`O \type 3 : `N "~>" `O \type 3 : `N^*$ ? D\'ependance envers le
  terme pour le sous-typage. De m\^eme, $2 : `N ``<= `N^*$, on devrait
  parler de renforcement.
\item On peut restreindre le sous-typage aux projections de types
  subsets avec l'\'egalit\'e syntaxique.
\item Je peux garder mes r\^egles de sous-typages, si elles sont syntax-directed!
\end{itemize}

Sous-typage \`a l'application et variable suffisante pour l'ad\'equation ?
On distingue les deux phases, pas de sous-typage \`a l'application.

V\'erifier Sub-{Left, Right}, l'application du sous-typage.

%%% Local Variables: 
%%% mode: latex
%%% TeX-master: "~/research/coq/papers/subset-typing"
%%% End: 
