\subsection*{9 mars}
Le \ps{} dans HOL n'est pas correct, il peut �tre subverti ais�ment �
cause d'une sorte de covariance des domaines de fonctions et de la
fonction de suppression des predicats (similaire � $\mu$):
$inv : `R^{\neq 0} "->" `R `: `R "->" `R$. Le d�veloppement dans \HOL{}
est fait directement dans le language, et il est argu� qu'il n'est pas
possible de le faire correctement � cause de l'exemple
pr�c�dent. L'algorithme de sous-typage est simplement une g�n�ration de
tout les sous-types possibles � partir d'un ensemble de r�gles de
sous-typage pour les constantes et constructions logiques ou fonctionelles.

Trouv� un article \cite{stumpsubset} sur les types sous-ensembles dans PF, logique d'ordre sup�rieur
avec fonctions partielles... Permet de traiter le cas
$\ifml 1 / i > 0 \thenml i \neq 0 \elseml `_$, qui g�n�re une obligation de
preuve $i \neq 0$ dans \PVS{}. Evidemment on ne risque pas de pouvoir
typer ce code en \Coq{} lorsque $`/ : `Z "->" \{ x : `Z `| x \neq 0 \} "->" `Z$ mais
certaines id�es peuvent �tre int�ressantes. 

%%% Local Variables: 
%%% mode: latex
%%% TeX-master: "~/research/coq/papers/subset-typing"
%%% End: 
