\subsection*{17 mars}

\begin{lemma}[Substitutivit� du sous-typage]
  \label{substitutive-subtyping}
  $`G \judgesub t : T \impsub U "=>" `G \typed t : T[u/x] \impsub U[u/x]$
\end{lemma}
\begin{proof}
  Par induction sur la d�rivation de sous-typage:

  \begin{itemize}
  \item[\SubConvRule:] On a $T \eqbi U$ et l'on suppose 
    $x \typed T$. On a donc bien $`G \typed x : U$.
    
  \item[\SubProdRule:] On a
    \begin{prooftree}
      \SubProd
    \end{prooftree}
    
    Par induction, $`G \typed x : U$, $`G \typed x : T$, 
    $`G, x : U \typed v : V$ et $`G, x : U \typed v : W$.
    Par \AbsRule, $`G \typed \lambda x : U.v : \Pi x : U.W$.
    
  \item[\SubSigmaRule:] On a 
    \begin{prooftree}
      \SubSigma
    \end{prooftree}
    
    Par hypoth�se d'induction, $`G \typed t : T, U$, $`G \typed v : V[t/x], W[t/y]$.
    Par \SumRule, $(t, v) : \Sigma y : U, W$.

  \item[\SubLeftRule:] On a
    \begin{prooftree}
      \SubLeft
    \end{prooftree}
    
    Par induction, $`G \typed p : V$.

  \item[\SubRightRule:] On a
    \begin{prooftree}
      \SubRight
    \end{prooftree}
    
    Par induction, $`G \typed p : U$. Par \SubsetRule, $`G \typed p :
    \subset{x}{U}{P}$.
    
  \end{itemize}
\end{proof}

\begin{lemma}[Inversion du sous-typage]
  \label{inversion-subtyping}
\end{lemma}

\begin{lemma}[D�rivabilit� de la r�flexivit�, transitivit� du sous-typage]
  \label{refl-trans-subtyping}
\end{lemma}

\begin{lemma}[Correction du sous-typage]
  \label{correct-subtyping}
  Si $`G \judgesub t : U \impsub V$ alors $`G \typed t : U "=>" `G
  \typed t : V$.
\end{lemma}

\setboolean{displayLabels}{false}

\begin{proof}
  Par induction sur la d�rivation de sous-typage:

  \begin{itemize}
  \item[\SubConvRule:] On a $T \eqbi U$ et l'on suppose 
    $x \typed T$. On a donc bien $`G \typed x : U$.
    
  \item[\SubProdRule:] On a
    \begin{prooftree}
      \SubProd
    \end{prooftree}
    
    Par induction, $`G \typed x : U$, $`G \typed x : T$, 
    $`G, x : U \typed v : V$ et $`G, x : U \typed v : W$.
    Par \AbsRule, $`G \typed \lambda x : U.v : \Pi x : U.W$.
    
  \item[\SubSigmaRule:] On a 
    \begin{prooftree}
      \SubSigma
    \end{prooftree}
    
    Par hypoth�se d'induction, $`G \typed t : T, U$, $`G \typed v : V[t/x], W[t/y]$.
    Par \SumRule, $(t, v) : \Sigma y : U, W$.

  \item[\SubLeftRule:] On a
    \begin{prooftree}
      \SubLeft
    \end{prooftree}
    
    Par induction, $`G \typed p : V$.

  \item[\SubRightRule:] On a
    \begin{prooftree}
      \SubRight
    \end{prooftree}
    
    Par induction, $`G \typed p : U$. Par \SubsetRule, $`G \typed p :
    \subset{x}{U}{P}$.
    
  \end{itemize}
  
\end{proof}


\begin{lemma}[Inversion du typage]
\label{inversion-typing}
\end{lemma}

\begin{lemma}[Correction du typage]
  \label{correct-typing}
  $`G \typei t : T "=>" `G \typed t : T$
\end{lemma}

\begin{proof}
  Par induction sur la d�rivation dans le syst�me algorithmique:

  \begin{description}
  \item[\WfAtomRule,\WfVarRule,\PropSetRule,\VarRule,\ProdRule,\AbsRule,
    \LetInRule, \SigmaRule, \SumRule, \LetSumRule:] r�gles inchang�es.

  \item[\AppRule:] On a
    \def\fCenter{\typei}
    \begin{prooftree}
      \AppI
    \end{prooftree}
    
    Par induction, $`G \typed f : \Pi x : V. W $.
    Par le lemme \ref{correct-sub}, et l'hypoth�se $`G \typed u : U$, 
    $`G \typed u : V$. Donc, par \AppRule, on a bien $`G \typed f u :
    W[u/x]$.
  \end{description}
  
\end{proof}

\begin{lemma}[Compl�tude du typage]
  \label{complete-typing}
  $`G \typed t : T "=>" `E U, `G \typei t : U `^ `G \judgesub t : U \impsub T$
\end{lemma}

\begin{proof}
  Par induction sur la d�rivation dans le syst�me d�claratif:

  \begin{description}
  \item[\WfAtomRule,\WfVarRule,\PropSetRule,\VarRule,\ProdRule,\AbsRule,
    \LetInRule, \SigmaRule, \SumRule, \LetSumRule:] r�gles inchang�es.
    
  \item[\AppRule:] On a 
    \begin{prooftree}
      \App
    \end{prooftree}
    
    Par induction, $`E X, Y, `G \typei f : \Pi x : X. Y `^ 
    `G \judgesub f : \Pi x : X. Y \impsub \Pi x : V. W$ et
    $`E U, `G \typei u : U `^ `G \judgesub u : U \impsub V$.
    
    Si $`G \judgesub f : \Pi X.Y \impsub \Pi x : V.W$, alors 
    $`G \judgesub x : V \impsub X$ (lemme \ref{inversion-subtyping}). 
    Par transitivit� du sous-typage \ref{refl-trans-subtyping}, 
    $`G \judgesub U \impsub X$. On peut donc appliquer \AppRule{} pour
    obtenir $`G \typei f u : Y[u/x]$. Par la covariance du
    produit en son codomaine et la substitivit� du sous-typage,
    on a $`G \judgesub f u : Y[u/x] \impsub W[u/x]$, la propri�t� est
    donc bien v�rifi�e.
    
  \item[\SubsetRule:] On a
    \begin{prooftree}
      \Subset
    \end{prooftree}
    
    Par induction, $`E T, `G \typei x : T `^ `G \judgesub x : T \impsub U$.
    

  \item[\LetSubRule:]
    
  \item[\ConvRule:]

  \end{description}
  
\end{proof}


%%% Local Variables: 
%%% mode: latex
%%% TeX-master: "~/research/coq/papers/subset-typing"
%%% End: 
