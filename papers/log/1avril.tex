\subsection*{1er avril}
On enl�ve les contextes au sous-typage algorithmique d�finitivement.
Pour faciliter les preuves on consid�re que le sous-typage contient la
$\eqbi$ d�s le d�part.
Ecriture de la r�ecriture du sous-typage pour Coq. 
On fait du sous-typage sur les traductions et l'on devra montrer que $U
\suba V "=>" `E t, t : U' \subi V'$.
R�ecriture: $\subimpl{`G}{U}{V}{`G'}{t}{U'}{V'}$. On connait $`G, U, V,
`G', U', V'$ et l'on cherche la coercion $t$.
        
%%% Local Variables: 
%%% mode: latex
%%% TeX-master: "~/research/coq/papers/subset-typing"
%%% End: 